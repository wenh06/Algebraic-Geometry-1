%
\chapter{Schemes}

\newpage

\section{Presheaves and sheaves}

\begin{Def}
Let $X$ be a topological space. A presheaf $\mathscr{F}$ of sets on
$X$ consists of the following data:
\enum
\item[(1)]For every nonempty open subset $U$ of $X,$ we have a set
$\mathscr{F}(U),$ whose elements are called sections of
$\mathscr{F}$ over $U.$
\item[(2)]For every inclusion $V\subseteq U$ of nonempty open
subsets of $X,$ we have a mapping
$\rho_{U\,V}:\mathscr{F}(U)\rightarrow \mathscr{F}(V),$ called
restriction, satisfying

$(a).$ $\rho_{U\,U}=id_{\mathscr{F}(U)}.$

$(b).$ If $W\subseteq V\subseteq U$ are nonempty subsets of $X,$
then $\rho_{U\,W}=\rho_{V\,W}\circ\rho_{U\,V}.$
\end{list}
\end{Def}
\begin{remarks}\
\enum
\item[(1)]$\forall s\in\mathscr{F}(U)$ and let $V\subseteq U$ be open subsets of $X,$ we
often denote $\rho_{U\,V}(s)$ by $\left.s\right|_{V}.$
\item[(2)]Elements in $\mathscr{F}(U)$ are often denoted by $(s,U).$
\item[(3)]By convention, we always set
$\mathscr{F}(\emptyset)=\{0\}.$
\item[(4)]Put $$\mathds{I}_X=\{U\mid U\text{ is a nonempty open subset of
}X\}.$$ Then $(\mathds{I}_X,\subseteq)$ is a partially ordered set.
Then a presheaf $\mathscr{F}$ of sets on $X$ is equivalent to an
inverse system $(\mathscr{F}(U),\rho_{U\,V})_{U,V\in\mathds{I}_X},$
thus equivalent to a contravariant functor from $Cat\mathds{I}_X,$
the category of nonempty subsets of $X,$ to the category $SET.$
\item[(5)]By definition, a presheaf of Abelian groups (resp. rings)
on $X$ is a contravariant functor from $Cat\mathds{I}_X$ to the
category of Abelian groups (resp. rings).
\end{list}
\end{remarks}
\begin{egs}\
Let $X$ be a topological space and denote by $\mathds{I}_X$ the
family of all nonempty open subsets of $X.$
\enum
\item[(1)]Let $A$ be an Abelian group. Define
$\mathscr{F}(U)=A,\forall U\in\mathds{I}_X;\rho_{U\,V}=id_A,\forall
V\subseteq U$ in $\mathds{I}_X.$ Then $\mathscr{F}$ is a presheaf of
Abelian groups on $X,$ called the constant presheaf associated to
$A.$
\item[(2)]Define
$$\mathscr{C}(U)=\{f:U\rightarrow\mathbb{C}\mid f\text{ is continuous
}\},$$ $\forall U\in\mathds{I}_X;$
$\rho_{U\,V}:\mathscr{C}(U)\rightarrow\mathscr{C}(V)$ to be the
restriction of functions, $\forall V\subseteq U$ in $\mathds{I}_X.$
Then $\mathscr{C}$ is a presheaf of rings.
\item[(3)]Let $\Pi: X^{\prime}\rightarrow X$ be a continuous mapping
of topological spaces. Define
$$\mathscr{S}(U)=\{s:U\rightarrow\Pi^{-1}(U)\mid \Pi\circ s=id_U, s\text{ is continuous
}\},$$ the set of all continuous sections of $\Pi$ over $U;$
$\rho_{U\,V}:\mathscr{S}(U)\rightarrow\mathscr{S}(V)$ to be the
restriction of functions, $\forall V\subseteq U$ in $\mathds{I}_X.$
Then $\mathscr{S}$ is a presheaf of sets.
\end{list}
\end{egs}
\begin{Def}
Let $\mathscr{F}$ be a presheaf on a topological space $X.$ We call
$\mathscr{F}$ a sheaf, if the following conditions are satisfied:
\enum
\item[(1)]$\forall s,t\in\mathscr{F}(U),$ if there exists an open
covering $(U_i)_{i\in I}$ of $U$ such that $\left.s\right|_{U_i} =
\left.t\right|_{U_i},\forall i\in I,$ then $s=t.$
\item[(2)]Let $U$ be a nonempty open subset of $X,$ $(U_i)_{i\in I}$
an open covering of $U,$ and $s_i\in \mathscr{F}(U_i)(i\in I)$
satisfying $\left.s_i\right|_{U_i\cap U_j}=\left.s_j\right|_{U_i\cap
U_j},\forall i,j\in I,$ then $\exists s\in \mathscr{F}(U)$ such that
$\left.s\right|_{U_i}=s_i,\forall i\in I.$ Indeed, such an $s$ is
unique by $(1).$
\end{list}
\end{Def}
\begin{eg}
The presheaves $\mathscr{C}$ and $\mathscr{S}$ are both sheaves. But
in general, the constant presheaf associated to $A$ is not a sheaf.
\end{eg}
\begin{remark}
A presheaf $\mathscr{F}$ of Abelian groups is a sheaf iff for any
open covering $(U_i)_{i\in I}$ of every nonempty open subset $U$ of
$X,$ the sequence
$$0\rightarrow \mathscr{F}(U)\rightarrow
\prod\limits_{i\in I}\mathscr{F}(U_i)\rightarrow
\prod\limits_{i,j\in I}\mathscr{F}(U_i\cap U_j),$$ which is defined
by $s\mapsto (\left.s\right|_{U_i})_{i\in I}$ and
$(\left.s\right|_i)_{i\in I}\mapsto (\left.s_j\right|_{U_i\cap
U_j}-\left.s_i\right|_{U_i\cap U_j})_{i,j\in I},$ is exact.
\end{remark}
\begin{Def}
Let $X$ be a topological space, $p\in X,$ and $\mathscr{F}$ be a
presheaf on $X.$ Put $\mathds{I}_X(p)=\{U\in \mathds{I}_X\mid p\in
U\}.$ Then $(\mathds{I}_X(p),\supseteq)$ is a directed set. The
stalk $\mathscr{F}_p$ of $\mathscr{F}$ at $p$ is defined by
$\mathscr{F}_p=\varinjlim\limits_{p\in U}\mathscr{F}(U),$ the direct
limit of the direct system $(\mathscr{F}(U),\rho_{U\,V})_{U,V\in
\mathds{I}_X(p)}.$
\end{Def}
\begin{remark}
Every element of $\mathscr{F}_p$ can be represented by section of
$\mathscr{F}$ over some neighborhoods of $p.$ Two sections $s\in
\mathscr{F}(U)$ and $t\in \mathscr{F}(V)$ define the same element in
$\mathscr{F}_p$ iff $\exists W\subseteq U\cap V$ with $p\in W$ and
$W$ open, such that $\left.s\right|_W = \left.t\right|_W.$

For any $U\in \mathds{I}_X(p),$ we have a canonical mapping
$\mathscr{F}\rightarrow \mathscr{F}_p.$ The image of a section $s\in
\mathscr{F}(U)$ is called the germ of $s$ at $p,$ and is denoted by
$s_p.$
\end{remark}
\begin{Def}
Let $\mathscr{F},\mathscr{G}$ be two presheaves on a topological
space $X.$ A morphism of presheaves $\phi: \mathscr{F}\rightarrow
\mathscr{G}$ is by definition a natural transformation from the
contravariant functor $\mathscr{F}$ to the contravariant functor
$\mathscr{G}.$ In other words, $\phi$ consists of morphisms
$\phi(U)\in Hom(\mathscr{F}(U),\mathscr{G}(U))(\forall U\in
\mathds{I}_X)$ such that $\forall U,V\in \mathds{I}_X,V\subseteq U,$
the following diagram
\[ \xymatrix{
   \mathscr{F}(U) \ar[d]_{\rho_{U\,V}} \ar[r]^-{\phi(U)} &
   \mathscr{G}(U) \ar[d]^{\rho_{U\,V}^{\prime}}         \\
   \mathscr{F}(V) \ar[r]^-{\phi(V)} & \mathscr{G}(V) }  \]
is commutative.
\end{Def}
\begin{remarks}\
\enum
\item[(1)]Let $(I,\leqslant)$ be a directed set, let
$\mathscr{F}=(A_i,\phi_{ij})_{i,j\in I},
\mathscr{G}=(A_i^{\prime},\phi_{ij}^{\prime})_{i,j\in I}$ be two
direct systems in a category $\mathscr{C}.$ A morphism $\phi:
\mathscr{F}\rightarrow \mathscr{G}$ is by definition a natural
transformation from the covariant functor $\mathscr{F}$ to
$\mathscr{G},$ i.e. a family of morphisms $u_i: A_i\rightarrow
A_i^{\prime}(\forall i\in I)$ such that $\forall i\leqslant j$ in
$I,$
\[ \xymatrix{
   A_i \ar[d]_{\phi_{ij}} \ar[r]^-{u_i} & A_i^{\prime}
   \ar[d]^{\phi_{ij}^{\prime}}                      \\
   A_j \ar[r]^-{u_j} & A_j^{\prime} }  \]
is commutative.

Assume $(A,\phi_i)_{i\in I} =
\varinjlim\limits_I(A_i,\phi_{ij})_{i,j\in I},
(A^{\prime},\phi_i^{\prime})_{i\in I} =
\varinjlim\limits_I(A_i^{\prime},\phi_{ij}^{\prime})_{i,j\in I}.$
$\forall i\in I,$ put $v_i=\phi_i^{\prime}\circ u_i\in
Hom(A_i,A^{\prime}),$ then $\forall i,j\in I(i\leqslant j),$ we have
$$v_j\circ\phi_{ij}=\phi_j^{\prime}\circ u_j\circ\phi_{ij}=
\phi_j^{\prime}\circ\phi_{ij}^{\prime}\circ u_i=\phi_i^{\prime}\circ
u_i=v_i.$$ Then $\exists!v\in Hom(A,A^{\prime})$ such that $\forall
i\in I, v_i=v\circ \phi_i:$
\[ \xymatrix{
   & A^{\prime} \\
   & A \ar@{-->}[u]_(.2){\exists! v} \\
   A_i \ar[rr]_-{\phi_{ij}} \ar"1,2"^{v_i} \ar[ur]_-{\phi_i} & & A_j
   \ar"1,2"_{v_j} \ar[ul]^-{\phi_j} }  \]
In this case, we write $v=\varinjlim\limits_I u_i.$

Now let $\mathscr{F}$ and $\mathscr{G}$ be two presheaves on $X,$
and $\phi: \mathscr{F}\rightarrow \mathscr{G}$ be a morphism of
sheaves. $\forall p\in X,$ consider the directed set
$(\mathds{I}_X(p), \supseteq),$ then $(\mathscr{F}(U),
\rho_{U\,V})_{U,V\in \mathds{I}_X(p)}$ and $(\mathscr{G}(U),
\rho_{U\,V}^{\prime})_{U,V\in \mathds{I}_X(p)}$ are two direct
systems, and $\phi$ is a morphism between them. Put
$$\phi_p=\varinjlim\limits_{p\in U}\phi(U)\in Hom(\mathscr{F}_p, \mathscr{G}_p).$$
Then $\forall U\in \mathds{I}_X(p), \forall s\in \mathscr{F}(U),$ we
have $\phi_p(s_p)=(\phi(U)(s))_p.$
\item[(2)]For a presheaf $\mathscr{F},$ we have the identity
morphism $id_{\mathscr{F}}: \mathscr{F}\rightarrow \mathscr{F}.$
\item[(3)]Given two morphisms of presheaves $\phi:
\mathscr{F}\rightarrow \mathscr{G}$ and $\psi:
\mathscr{G}\rightarrow\mathscr{H}.$ By definition, $\psi\circ\phi:
\mathscr{F}\rightarrow\mathscr{H}$ is the composition of natural
transformations. Hence all the presheaves on the topological space
$X,$ together with their morphisms of presheaves form a category. A
morphism of presheaves $\phi: \mathscr{F}\rightarrow \mathscr{G}$ is
called an isomorphism if it is an isomorphism in the category of
presheaves on $X.$ In other words, $\phi$ has a two-sided inverse
$\psi: \mathscr{G}\rightarrow\mathscr{F}$ such that
$\psi\circ\phi=id_{\mathscr{F}}, \phi\circ\psi=id_{\mathscr{G}}.$ It
is equivalent to saying that $\forall U\in \mathds{I}_X, \phi(U):
\mathscr{F}(U)\rightarrow \mathscr{G}(U)$ is an isomorphism.
\item[(4)]We define morphism of sheaves as morphism of presheaves.
Thus the category of sheaves on $X$ is a full subcategory of the
category of presheaves on $X.$
\end{list}
\end{remarks}
\begin{prop}
Let $\phi: \mathscr{F}\rightarrow \mathscr{G}$ be a morphism of
sheaves on a topological space $X.$ Then $\phi$ is an isomorphism
iff $\forall p\in X, \phi_p: \mathscr{F}_p\rightarrow \mathscr{G}_p$
is an isomorphism.
\end{prop}
\begin{proof}
$\mathit{1^{\circ}}$ Necessity:

Suppose that $\phi$ is an isomorphism, then $\psi=\phi^{-1}:
\mathscr{G}\rightarrow \mathscr{F}$ is a morphism, and
$\psi\circ\phi=id_{\mathscr{F}}, \phi\circ\psi=id_{\mathscr{G}}.$
Then $\forall p\in X,$ we have
$$id_{\mathscr{F}_p}=(id_{\mathscr{F}})_p=(\psi\circ\phi)_p=\psi_p\circ\phi_p,$$
$$id_{\mathscr{G}_p}=(id_{\mathscr{G}})_p=(\phi\circ\psi)_p=\phi_p\circ\psi_p.$$
Hence $\phi_p: \mathscr{F}_p\rightarrow \mathscr{G}_p$ is an
isomorphism.

$\mathit{2^{\circ}}$ Sufficiency:

Fix $U$ a nonempty open subset of $X.$

First we show that $\phi(U): \mathscr{F}(U)\rightarrow
\mathscr{G}(U)$ is injective. Take $s,t\in \mathscr{F}(U)$ such that
$\phi(U)(s)=\phi(U)(t).$ Then $\forall p\in U,$ we have
$\phi_p(s_p)=(\phi(U)(s))_p=(\phi(U)(t))_p=\phi_p(t_p).$ But
$\phi_p$ is injective, thus $s_p=t_p.$ Then $\exists U_p\subseteq U$
in $\mathds{I}_X(p)$ such that $\left.s\right|_{U_p} =
\left.t\right|_{U_p}.$ But $U = \bigcup\limits_{p\in U}U_p$ and
$\mathscr{F}$ is a sheaf, thus $s=t.$ Hence $\phi(U)$ is injective.

Second, we show that $\phi(U): \mathscr{F}(U)\rightarrow
\mathscr{G}(U)$ is surjective. Take $t\in \mathscr{G}(U).$ $\forall
p\in U, \phi_p: \mathscr{F}_p\rightarrow \mathscr{G}_p$ is
surjective, thus $\exists s^{(p)}\in \mathscr{F}_p$ such that
$\phi_p(s^{(p)})=t_p.$ By definition, $\exists U_p\subseteq U$ in
$\mathds{I}_X(p),$ and $s_{U_p}\in \mathscr{F}(U_p)$ such that
$(s_{U_p})_p=s^{(p)}$ and
$(\phi(U_p)(s_{U_p}))_p=\phi_p(s^{(p)})=t_p.$ Hence $\exists V_p\in
\mathds{I}_X(p),V_p\subseteq U_p,$ such that
$\left.\phi(U_p)(s_{U_p})\right|_{V_p}=\left.t\right|_{V_p}.$ Note
that $U=\bigcup\limits_{p\in U}V_p$ and $\forall p,q\in U,$ if
$V_p\cap V_q\neq \emptyset,$ we have
$\left.\phi(V_p)(s_{V_p})\right|_{V_p\cap V_q} =
\left.t\right|_{V_p\cap
V_q}=\left.\phi(V_q)(s_{V_q})\right|_{V_p\cap V_q},$ where we put
$s_{V_p} = \left.s_{U_p}\right|_{V_p}.$ Hence $\phi(V_p\cap
V_q)(\left.s_{V_p}\right|_{V_p\cap V_q})=\phi(V_p\cap
V_q)(\left.s_{V_q}\right|_{V_p\cap V_q}).$ But $\phi(V_p\cap V_q)$
is injective, therefore $\left.s_{V_p}\right|_{V_p\cap V_q} =
\left.s_{V_q}\right|_{V_p\cap V_q}.$ However, $\mathscr{F}$ is a
sheaf, thus $\exists s\in \mathscr{F}(U)$ such that $\forall p\in
U,$ we have $\left.s\right|_{V_p}=s_{V_p}.$ Hence $\phi(U)(s)=t.$
\end{proof}
\begin{prop}
Let $X$ be a topological space. Then the category of presheaves of
Abelian groups on $X$ is an Abelian category. More precisely, if
$\phi: \mathscr{F}\rightarrow \mathscr{G}$ is a morphism of
presheaves of Abelian groups on $X,$ then $ker\phi, coker\phi,
im\phi, coim\phi$ are respectively the following presheaves defined
by
\[ \xymatrix@R=0em@C=1em{
   ker\phi(U)=ker(\phi(U)), & coker\phi(U)=coker(\phi(U)),\\
   im\phi(U)=im(\phi(U)), & coim\phi(U)=coim(\phi(U)). }  \]
\end{prop}
\begin{proof}
$\mathit{1^{\circ}}$ $ker\phi: U\mapsto ker(\phi(U))$ is a presheaf
of Abelian groups on $X:$

$\mathit{2^{\circ}}$ $ker\phi$ is the kernel of $\phi:$

$\mathit{3^{\circ}}$ $coker\phi: U\mapsto coker(\phi(U))$ is a
presheaf of Abelian groups on $X:$

$\mathit{4^{\circ}}$ We can prove $coker\phi$ is the cokernel of
$\phi$ like what we have done in $\mathit{2^{\circ}}.$ Then $im\phi$
and $coim\phi$ are well-defined as well.
\end{proof}
\begin{cor}
Let $\phi: \mathscr{F}\rightarrow \mathscr{G}$ be a morphism of
presheaves of Abelian groups on $X.$ Then $\forall p\in X,
(ker\phi)_p=ker\phi_p, (coker\phi)_p=coker\phi_p,
(im\phi)_p=im\phi_p, (coim\phi)_p=coim\phi_p.$
\end{cor}
\begin{proof}
$$(ker\phi)_p = \varinjlim\limits_{p\in U}(ker\phi)(U) = \varinjlim\limits_{p\in
U}ker(\phi(U)) = ker\phi_p,$$
$$(coker\phi)_p = \varinjlim\limits_{p\in U}(coker\phi)(U) = \varinjlim\limits_{p\in
U}coker(\phi(U)) = coker\phi_p.$$ Then $(im\phi)_p=im\phi_p$ and
$(coim\phi)_p=coim\phi_p.$ are clear by their definitions.
\end{proof}
\begin{prop}
Let $\mathscr{F}$ be a presheaf on a topological space $X,$ there
exists a pair $(\mathscr{F}^+,\theta)$ consisting of a sheaf
$\mathscr{F}^+$ and $\theta: \mathscr{F}\rightarrow \mathscr{F}^+$ a
morphism of presheaves such that for any sheaf $\mathscr{G}$ and any
morphism $\phi: \mathscr{F}\rightarrow \mathscr{G},$ there exists a
unique morphism $\psi: \mathscr{F}^+\rightarrow \mathscr{G}$ such
that $\phi=\psi\circ\theta.$
\[ \xymatrix{
   \mathscr{F} \ar[r]^-{\theta} \ar[d]_-{\phi} & \mathscr{F}^+
   \ar@{-->}[dl]^-{\exists! \psi}                           \\
   \mathscr{G} }  \]
The pair $(\mathscr{F}^+,\theta)$ is unique up to unique
isomorphism. $\forall p\in X,$ $\theta_p: \mathscr{F}_p\rightarrow
(\mathscr{F}^+)_p$ is an isomorphism. We call $\mathscr{F}^+$ the
sheaf associated to the presheaf $\mathscr{F}.$
\end{prop}
\begin{proof}
$\mathit{1^{\circ}}$ Existence of $(\mathscr{F}^+,\theta)$:

$\forall U\in \mathds{I}_X,$ define $\mathscr{F}^+(U)$ to be the set
of functions $s: U\rightarrow \coprod\limits_{p\in U}\mathscr{F}_p$
satisfying the following conditions:
\enum
\item[(1)]$\forall p\in U, s(p)\in \mathscr{F}_p.$
\item[(2)]$\forall p\in U, \exists U_p\in \mathds{I}, U_p\subseteq
U,$ and $t\in \mathscr{F}(U_p),$ such that $\forall q\in U_p,
s(q)=t_q.$
\end{list}
$\forall s\in \mathscr{F}(U), \forall p\in U,$ define
$\theta(U)(s)(p)=s_p,$ thus $\theta(U)(s)\in \mathscr{F}^+(U).$
Since $\forall U\in \mathds{I}_X, \forall s\in \mathscr{F}^+(U),$
$s$ is a function, thus it is immediate that $\mathscr{F}^+$ is a
presheaf. Because of the same reason, let $(U_i)_{i\in I}$ be an
open covering of $U, s_i\in \mathscr{F}^+(U_i)$ satisfying
$\left.s_i\right|_{U_i\cap U_j}=\left.s_j\right|_{U_i\cap U_j},$ we
can glue $s_i$ together to get a unique $s\in \mathscr{F}^+(U)$ such
that $\left.s\right|_{U_i}=s_i.$ Therefore $\mathscr{F}^+$ is a
sheaf.

$\forall V\subseteq U$ in $\mathds{I}_X(p),$ we have
$(\left.s\right|_V)_p=s_p,$ thus
\[ \xymatrix{
   \mathscr{F}(U) \ar[r]^-{\theta(U)} \ar[d]_{\rho_{U\,V}} &
   \mathscr{F}^+(U) \ar[d]^{\rho_{U\,V}^{\prime}}         \\
   \mathscr{F}(V) \ar[r]_-{\theta(V)} & \mathscr{F}^+(V) }  \]
commutes, where $\rho_{U\,V}^{\prime}$ is defined by
$$(U\ni p\mapsto s_p)\mapsto (V\ni p\mapsto(\left.s\right|_V)_p=s_p).$$
Hence $\theta: \mathscr{F}\rightarrow \mathscr{F}^+$ is a morphism.

$\mathit{2^{\circ}}$ Universal property of $(\mathscr{F}^+,
\theta):$

$\forall U\in \mathds{I}_X, \forall s\in \mathscr{F}^+(U),$ applying
$(2),$ we have: $\forall p\in U, \exists U_p\in \mathds{I}_X(p),
U_p\subseteq U,$ and $t\in \mathscr{F}(U_p)$ such that $\forall q\in
U_p, s(q)=t_q.$ Then we have the following commutative diagram
\[ \xymatrix{
   \mathscr{F}(U_p) \ar[r]^-{\theta(U_p)} \ar[d]_{\phi(U_p)} &
   \mathscr{F}^+(U_p) \ar@{-->}[dl]^{\exists! \bar{\phi}(U_p)} \\
   \mathscr{G}(U_p) }  \]
where $\theta(U_p): t\mapsto (\left.s\right|_{U_p}: U_p\ni q\mapsto
t_q),$ and $\bar{\phi}(U_p)(\left.s\right|_{U_p})=\phi(U_p)(t).$

$\mathit{3^{\circ}}$ Uniqueness of $(\mathscr{F}^+, \theta):$

Suppose there exists another pair $(\mathscr{F}^+_1, \theta_1)$ that
satisfies the universal property, then we have
\[ \xymatrix{
   \mathscr{F} \ar[r]^-{\theta} \ar[d]_{\theta_1} &
   \mathscr{F}^+ \ar@{-->}@<.6ex>"2,1"^{\exists! \bar{\theta}_1} \\
   \mathscr{F}^+_1 \ar@{-->}@<.6ex>"1,2"^{\exists! \bar{\theta}} }  \]
But we also have
\[ \xymatrix@C=4em@R=4em{
   \mathscr{F} \ar[r]^-{\theta_1} \ar[d]_{\theta_1} &
   \mathscr{F}^+_1 \ar[d]^{\bar{\theta}}
   \ar@{-->}[dl]_{\exists!}^{id_{\mathscr{F}^+_1}}   \\
   \mathscr{F}^+_1 & \mathscr{F}^+ \ar[l]_-{\bar{\theta}_1} }  \]
\[ \xymatrix@C=4em@R=4em{
   \mathscr{F} \ar[r]^-{\theta} \ar[d]_{\theta} & \mathscr{F}^+
   \ar[d]^{\bar{\theta}_1} \ar@{-->}[dl]_{\exists!}^{id_{\mathscr{F}^+}} \\
   \mathscr{F}^+ & \mathscr{F}^+_1 \ar[l]_-{\bar{\theta}} } \]
Hence $\bar{\theta}\circ\bar{\theta}_1=id_{\mathscr{F}^+},
\bar{\theta}_1\circ\bar{\theta}=id_{\mathscr{F}^+_1}.$ Thus
$(\mathscr{F}^+,\theta)$ is unique up to unique isomorphism.

$\mathit{4^{\circ}}$ $\theta_p: \mathscr{F}_p\rightarrow
\mathscr{F}_p^+$ is an isomorphism:

From the following commutative diagram we obtain $\theta_p:$
\[ \xymatrix{
   & (\mathscr{F}^+)_p \\
   \mathscr{F}^+(U) \ar[ur] & \mathscr{F}_p \ar@{-->}[u]_(.3){\exists!
   \theta_p} & \mathscr{F}^+(V) \ar[ul]               \\
   \mathscr{F}(U) \ar[u]^{\theta(U)} \ar[ur] \ar[rr]^-{\rho_{U\,V}} &
   & \mathscr{F}(V) \ar[ul] \ar[u]_{\theta(V)} }  \]
where $U,V\in \mathds{I}_X(p), V\subseteq U.$

$\forall s^{(p)}\in (\mathscr{F}^+)_p,$ $\exists U_p$ a neighborhood
of $p,$ and $s\in \mathscr{F}^+(U_p)$ such that $s^{(p)}=s_p.$ Then
$\exists (U_i)_{i\in I}$ an open covering of $U_p,$ and $s_i\in
\mathscr{F}(U_i)(\forall i\in I),$ such that
$\theta(U_i)(s_i)=\left.s\right|_{U_i}.$ We may assume
$U_p=\bigcup\limits_{i\in I}U_i,$ thus $\exists i\in I,$ such that
$p\in U_i,$ then $(\left.s\right|_{U_i})_p=s_p.$ Define $\varphi:
(\mathscr{F}^+)_p\rightarrow \mathscr{F}_p$ by $s^{(p)}\mapsto
(s_i)_p.$ We can easily check that $\varphi$ is independent of
choices of $s_i$ by taking intersections of $U_i$ and $s_i$'s
restrictions.
$$\theta_p\circ\varphi(s^{(p)}) = \theta_p((s_i)_p) =
(\theta(U_i)(s_i))_p = (\left.s\right|_{U_i})_p = s_p = s^{(p)} =
id_{(\mathscr{F}^+)_p}(s^{(p)}).$$

$\forall t^{(p)}\in \mathscr{F}_p, \exists V_p\in \mathds{I}_X(p),
t\in \mathscr{F}(V_p)$ such that $t_p=t^{(p)}.$ We have
$$\theta_p(s^{(p)})=(\theta(V_p)(t))_p\in (\mathscr{F}^+)_p.$$
Let $s=\theta(V_p)(t)\in \mathscr{F}^+(V_p),$ then we have
$s_p=\theta_p(s^{(p)}).$ There exists an open covering $(U_i)_{i\in
I}$ of $V_p,$ and $s_i\in \mathscr{F}(U_i),$ such that
$\left.s\right|_{U_i}=\theta(U_i)(s_i).$ Then $\exists i\in I$ such
that $p\in U_i.$ Since $s=\theta(V_p)(t),$ then
$$\theta(U_i)(s_i) = \left.s\right|_{U_i} =
\left.\theta(V_p)(s_i)\right|_{U_i} =
\theta(U_i)(\left.t\right|_{U_i}),$$ hence $(s_i)_p =
(\left.t\right|_{U_i})_p.$ Then
$$\varphi\circ\theta_p(t^{(p)}) = \varphi(s_p) = (s_i)_p =
(\left.t\right|_{U_i})_p = t_p = t^{(p)} =
id_{\mathscr{F}_p}(t^{(p)}).$$

Therefore $\theta_p: \mathscr{F}_p\rightarrow (\mathscr{F}^+)_p$ is
an isomorphism.
\end{proof}
\begin{remarks}\
\enum
\item[(1)]$\forall s\in \mathscr{F}^+(U),$ we can find an open
covering $(U_i)_{i\in I}$ of $U$ and $s_i\in \mathscr{F}(U_i),$ such
that $\theta(U_i)(s_i)=\left.s\right|_{U_i}.$ In other words,
sections in $\mathscr{F}^+$ locally come from sections of
$\mathscr{F}.$
\item[(2)]Two sections $s,t\in \mathscr{F}(U)$ have the same image
in $\mathscr{F}^+(U)$ iff $\forall p\in U, s_p=t_p,$ iff there
exists an open covering $(U_i)_{i\in I}$ of $U$ such that
$\left.s\right|_{U_i}=\left.t\right|_{U_i}(\forall i\in I).$
\item[(3)]If $\mathscr{F}$ is a sheaf, then $\theta:
\mathscr{F}\rightarrow \mathscr{F}^+$ is an isomorphism.
\item[(4)]
\[ \xymatrix{
   \mathscr{F} \ar[r]^-{\theta} \ar[d]_{\phi} & \mathscr{F}^+
   \ar@{-->}[dl]^-{\exists! \bar{\phi}}     \\
   \mathscr{G} }  \]
We can define a forgetful functor $SHEAF(X)\stackrel{U}{\rightarrow}
PRESHEAF(X).$ Since
\[ \xymatrix@R=0em{
   Hom_{Presheaf}(\mathscr{F}, U(\mathscr{G})) \ar[r]^-{\sim} &
   Hom_{Sheaf}(\theta(\mathscr{F}), \mathscr{G})  \\
   \phi \ar@{|->}[r] & \bar{\phi} }  \]
then $\theta$ is the left adjoint of $U.$ Both of them are exact.
\end{list}
\end{remarks}
\begin{prop}
Let $X$ be a topological space, then the category of sheaves of
Abelian groups on $X$ is an Abelian category.

Let $\phi: \mathscr{F}\rightarrow \mathscr{G}$ be a morphism of
sheaves of Abelian groups on $X,$ then $ker\phi$ is the sheaf
defined by $U\mapsto ker\phi(U):=ker(\phi(U));$ $coker\phi$ is the
sheaf associated to the presheaf $U\mapsto coker(\phi(U));$ $im\phi$
is the sheaf associated to the presheaf $U\mapsto im(\phi(U));$ and
$coim\phi$ is the sheaf associated to the sheaf $U\mapsto
coim(\phi(U)).$ Moreover, $\forall p\in X,$ we have
\[ \xymatrix@R=0em{
   (ker\phi)_p=ker\phi_p, & (coker\phi)_p=coker\phi_p,  \\
   (im\phi)_p=im\phi_p, & (coim\phi)_p=coim\phi_p. }  \]
\end{prop}
\begin{proof}
$\mathit{1^{\circ}}$ $ker\phi: U\mapsto ker(\phi(U))$ is a sheaf:

Since $\mathscr{F}$ is a sheaf, it is isomorphic to $\mathscr{F}^+,$
thus $\forall U\in \mathds{I}_X, s\in \mathscr{F}(U),$ $s$ can be
identified with a function in $\mathscr{F}^+.$ Let $(U_i)_{i\in I}$
be an open covering of $U,$ we may assume that
$U=\bigcup\limits_{i\in I}U_i.$ Let $s_i\in \mathscr{F}(U_i)(\forall
i\in I)$ be such that $\left.s_i\right|_{U_i\cap U_j} =
\left.s_j\right|_{U_i\cap U_j}(\forall i,j\in I).$ By gluing the
functions $s_i(i\in I)$ we get a unique $s\in \mathscr{F}(U)$ such
that $\left.s\right|_{U_i}=s.$ We want to show that $s\in
ker(\phi(U)).$ $\forall i\in I,$ we have the following commutative
diagram
\[ \xymatrix{
   \mathscr{F}(U) \ar[r]^-{\phi(U)} \ar[d]_{\rho_{U\,U_i}} &
   \mathscr{G}(U) \ar[d]^{\rho_{U\,U_i}^{\prime}}         \\
   \mathscr{F}(U_i) \ar[r]^-{\phi(U_i)} & \mathscr{G}(U_i) }  \]
Hence $\forall i\in I,$ we have
$$\left.\phi(U)(s)\right|_{U_i} = \phi(U_i)(\left.s\right|_{U_i}) =
\phi(U_i)(s_i) = 0.$$ $\mathscr{G}$ is a sheaf, thus $\phi(U)(s)=0,$
i.e. $s\in ker(\phi(U)).$

$\mathit{2^{\circ}}$ $ker\phi$ is the kernel of $\phi:$

Since $ker\phi$ is the kernel of $\phi$ as a morphism of presheaves,
and $ker\phi$ is a sheaf, then $ker\phi$ is the kernel of $\phi$ as
a morphism of sheaves.

$\mathit{3^{\circ}}$ $(U\mapsto coker(\phi(U)))^+$ is the cokernel
of $\phi:$

We have the following commutative diagrams of presheaves
\[ \xymatrix{
   \mathscr{F} \ar[r]^-{\phi} & \mathscr{G} \ar[d]_{\varphi} \ar[r]
   & (U\mapsto coker(\phi(U))) \ar[r]^-{\theta} \ar@{-->}[dl] & (U\mapsto
   coker(\phi(U)))^+ \ar@{-->}"2,2"    \\
   & \mathscr{H} }  \]
where $\mathscr{H}$ is a sheaf and $\varphi\circ \phi=0.$

$\mathit{4^{\circ}}$ Since for any presheaf $\mathscr{F}$ on $X,$
and $\forall p\in X,$ $\mathscr{F}_p$ is isomorphic to
$(\mathscr{F}^+)_p,$ then we have $(ker\phi)_p=ker\phi_p,
(coker\phi)_p=coker\phi_p, (im\phi)_p=im\phi_p,
(coim\phi)_p=coim\phi_p.$
\end{proof}
\begin{cor}
Let $X$ be a topological space.
\enum
\item[(1)]In the category of presheaves of Abelian groups on $X,$ a
morphism $\varphi: \mathscr{F}\rightarrow \mathscr{G}$ is injective
(resp. surjective or bijective) iff $\forall U\in \mathds{I}_X,
\varphi(U): \mathscr{F}(U)\rightarrow \mathscr{G}(U)$ is injective
(resp. surjective or bijective).
\item[(2)]In the category of sheaves of Abelian groups on $X,$ a
morphism $\varphi: \mathscr{F}\rightarrow \mathscr{G}$ is injective
iff $\forall U\in \mathds{I}_X, \varphi(U):
\mathscr{F}(U)\rightarrow \mathscr{G}(U)$ is injective.
\item[(3)]In the category of sheaves of Abelian groups on $X,$ a
morphism $\varphi: \mathscr{F}\rightarrow \mathscr{G}$ is injective
(resp. surjective or bijective) iff $\forall p\in X,$ $\varphi_p:
\mathscr{F}_p\rightarrow \mathscr{G}_p$ is injective(resp.
surjective or bijective).
\item[(4)]In the category of sheaves of Abelian groups on $X,$ a
morphism $\varphi: \mathscr{F}\rightarrow \mathscr{G}$ is surjective
iff $\forall U\in \mathds{I}_X$ and $\forall t\in \mathscr{G}(U),$
there exists an open covering $(U_i)_{i\in I}$ of $U$ and $s_i\in
\mathscr{F}(U_i)(\forall i\in I)$ such that
$\varphi(U_i)(s_i)=\left.t\right|_{U_i}.$
\end{list}
\end{cor}
\begin{proof}\
\enum
\item[(1)]$\varphi$ is injective iff $ker\varphi=0,$ iff $\forall
U\in \mathds{I}_X, ker\varphi(U)=0$ (i.e. $\forall U\in
\mathds{I}_X, ker(\varphi(U))=0$) iff $\forall U\in \mathds{I}_X,$
$\varphi(U)$ is injective.

$\varphi$ is surjective iff $im\varphi=\mathscr{G}$ iff $\forall
U\in \mathds{I}_X, im\varphi(U)=im(\varphi(U))=\mathscr{G}(U)$ iff
$\forall U\in \mathds{I}_X, \varphi(U)$ is surjective.
\item[(2)]Since $U\mapsto ker(\phi(U))$ is a sheaf, then the desired
result is clear by $(1).$
\item[(3)]$\varphi$ is injective iff $ker\varphi=0$ iff $\forall
p\in X, (ker\varphi)_p=0$ iff $\forall p\in X, ker(\varphi)_p=0$ iff
$\forall p\in X, \varphi_p=0.$

Likewise, we can prove $\varphi$ is surjective $\Leftrightarrow$
$\forall p\in X, \varphi_p$ is surjective; and $\varphi$ is
bijective $\Leftrightarrow$ $\forall p\in X, \varphi_p$ is
bijective.
\item[(4)]$\Longrightarrow:$ Suppose that $\varphi: \mathscr{F}\rightarrow
\mathscr{G}$ is surjective. Then $\forall p\in X, \varphi_p:
\mathscr{F}_p\rightarrow \mathscr{G}_p$ is surjective. Fix $U\in
\mathds{I}_X$ and $t\in\mathscr{G}(U),$ $t$ satisfies the following
properties:
\enum
\item[(a)]$\forall p\in U, t(p)\in \mathscr{G}_p.$
\item[(b)]$\forall p\in U, \exists U_p\in \mathds{I}_X(p),
U_p\subseteq U,$ and $t^{(p)}\in \mathscr{G}(U_p),$ such that
$\forall q\in U_p, t(q)=t^{(p)}_q.$
\end{list}
Now $im\varphi=\mathscr{G}$ is the sheaf associated to the presheaf
$V\mapsto im(\varphi(V)).$ Then for $t\in
im\varphi(U)=\mathscr{G}(U),$ $t$ satisfies the following
conditions:
\enum
\item[(a)]$\forall p\in U, t(p)\in \mathscr{G}_p=im\varphi_p.$
\item[(b)]$\forall p\in U, \exists V_p\in \mathds{I}_X(p),
V_p\subseteq U_p\subseteq U,$ and $v^{(p)}\in im\varphi(V_p),$ such
that $\forall q\in V_p, t(q)=v^{(p)}_q=t^{(p)}_q.$ $\exists
s^{(p)}\in \mathscr{F}(V_p)$ such that
$v^{(p)}=\varphi(V_p)(s^{(p)}).$
\end{list}
In particular, $v^{(p)}_p = t^{(p)}_p,$ thus $\exists W_p\in
\mathds{I}_X, W_p\subseteq V_p,$ such that
$\left.v^{(p)}\right|_{W_p} = \left.t^{(p)}\right|_{W_p}.$ Then
$(W_p)_{p\in U}$ is an open covering of $U,$ and $\forall p\in U_p,$
we have
$$\left.t\right|_{W_p} = \left.t^{(p)}\right|_{W_p} =
\left.v^{(p)}\right|_{W_p} =
\left.\varphi(V_p)(s^{(p)})\right|_{W_p} =
\varphi(W_p)(\left.s^{(p)}\right|_{W_p}).$$

$\Longleftarrow:$ It's sufficient to show that $\forall p\in X,
\varphi_p$ is surjective. Fix $p\in X, \forall t^{(p)}\in
\mathscr{G}_p,$ we want to find $s^{(p)}\in \mathscr{F}_p$ such that
$\varphi_p(s^{(p)}) = t^{(p)}.$ Since$t^{(p)}\in \mathscr{G}_p,$
then $\exists U\in \mathds{I}_X(p), t\in \mathscr{G}(U)$ such that
$t^{(p)}=t_p.$ There exists an open covering $(U_i)_{i\in I}$ of $U,
s_i\in \mathscr{F}(U_i)(\forall i\in I),$ such that
$\varphi(U_i)(s_i)=\left.t\right|_{U_i}.$ Since $p\in U,$ hence
$\exists i\in I$ such that $p\in U_i.$ Put $s^{(p)}=(s_i)_p,$ then
we have
$$\varphi_p(s^{(p)}) = \varphi_p((s_i)_p) = (\varphi(U_i)(s_i))_p =
(\left.t\right|_{U_i})_p = t_p = t^{(p)}.$$ So $\varphi_p$ is
surjective, hence $\varphi$ is surjective.
\end{list}
\end{proof}
\begin{cor}
Let $X$ be a topological space.
\enum
\item[(1)]In the category of presheaves of Abelian groups on $X,$ a
sequence $\mathscr{F} \stackrel{\varphi}{\rightarrow} \mathscr{G}
\stackrel{\psi}{\rightarrow} \mathscr{H}$ is exact iff $\forall U\in
\mathds{I}_X, \mathscr{F}(U) \stackrel{\varphi(U)}{\rightarrow}
\mathscr{G}(U) \stackrel{\psi(U)}{\rightarrow} \mathscr{H}(U)$ is
exact.
\item[(2)]In the category of heaves of Abelian groups on $X,$ a
sequence $\mathscr{F} \stackrel{\varphi}{\rightarrow} \mathscr{G}
\stackrel{\psi}{\rightarrow} \mathscr{H}$ is exact iff
$\mathscr{F}_p \stackrel{\varphi_p}{\rightarrow} \mathscr{G}_p
\stackrel{\psi_p}{\rightarrow} \mathscr{H}_p$ is exact, $\forall
p\in X.$
\end{list}
\end{cor}
\begin{proof}\
\enum
\item[(1)]$im\varphi=ker\psi \Longleftrightarrow \forall U\in \mathds{I}_X,
im\varphi(U)=ker\psi(U)(\text{i.e. }im(\varphi(U))=ker(\psi(U)))
\Longleftrightarrow \mathscr{F}(U)\stackrel{\varphi(U)}{\rightarrow}
\mathscr{G}(U)\stackrel{\psi(U)}{\rightarrow} \mathscr{H}(U)$ is
exact.
\item[(2)]$im\varphi=ker\psi \Longleftrightarrow \forall p\in X,
(im\varphi)_p=(ker\psi)_p(\text{i.e. }im\varphi_p=ker\psi_p)
\Longleftrightarrow \mathscr{F}_p\stackrel{\varphi_p}{\rightarrow}
\mathscr{G}_p\stackrel{\psi_p}{\rightarrow} \mathscr{H}_p$ is exact.
\end{list}
\end{proof}
\begin{remarks}\
\enum
\item[(1)]In the category of presheaves of Abelian groups on $X,$ if
$\mathscr{F} \stackrel{\varphi}{\rightarrow} \mathscr{G}
\stackrel{\psi}{\rightarrow} \mathscr{H}$ is exact, then $\forall
U\in \mathds{I}_X, im\varphi(U)=ker\psi(U).$ Hence $\forall p\in X,
im\varphi_p=ker\psi_p,$ thus $\mathscr{F}_p
\stackrel{\varphi_p}{\rightarrow} \mathscr{G}_p
\stackrel{\psi_p}{\rightarrow} \mathscr{H}_p$ is exact. But the
converse is false in general.
\item[(2)]In the category of presheaves of Abelian groups on $X,$ if
$\mathscr{F} \stackrel{\varphi}{\rightarrow} \mathscr{G}
\stackrel{\psi}{\rightarrow} \mathscr{H}$ is such that $\forall U\in
\mathds{I}_X, \mathscr{F}(U) \stackrel{\varphi(U)}{\rightarrow}
\mathscr{G}(U) \stackrel{\psi(U)}{\rightarrow} \mathscr{H}(U)$ is
exact, then $\forall p\in X, \mathscr{F}_p
\stackrel{\varphi_p}{\rightarrow} \mathscr{G}_p
\stackrel{\psi_p}{\rightarrow} \mathscr{H}_p$ is exact, hence
$\mathscr{F} \stackrel{\varphi}{\rightarrow} \mathscr{G}
\stackrel{\psi}{\rightarrow} \mathscr{H}$ is exact. But the converse
is false in general.
\end{list}
\end{remarks}

\newpage

\section{Direct images and inverse images of sheaves}

\begin{Def}
Let $f:X\rightarrow Y$ be a continuous mapping of topological
spaces, $\mathscr{F}$ a sheaf on $X,$ and $\mathscr{G}$ a sheaf on
$Y.$
\enum
\item[(1)]The direct image $f_{\ast}\mathscr{F}$ of $\mathscr{F}$ is
the sheaf on $Y$ defined by
$$\mathds{I}_Y\ni V\mapsto f_{\ast}\mathscr{F}(V):=\mathscr{F}(f^{-1}(V)).$$
\item[(2)]The inverse image $f^{-1}\mathscr{G}$ of $\mathscr{G}$ is
the sheaf associated to the presheaf defined by
$$\mathds{I}_X\ni U\mapsto\varinjlim_{f(U)\subseteq V}\mathscr{G}(V),$$
where the direct limit is the direct limit of the direct system
$(\mathscr{G}(V),\rho_{W\,V})_{W\supseteq V\supseteq f(U)}$ with the
partial ordering $\supseteq.$
\end{list}
\end{Def}
\begin{remarks}\
\enum
\item[(1)]Let $\mathscr{F}_1,\mathscr{F}_2$ be two sheaves on $X$
and $\varphi:\mathscr{F}_1\rightarrow \mathscr{F}_2$ be a morphism
of sheaves on $X.$ Define $f_{\ast}\varphi:
f_{\ast}\mathscr{F}_1\rightarrow f_{\ast}\mathscr{F}_2$ to be the
morphism such that $\forall V\in\mathds{I}_Y,
f_{\ast}\varphi(V)=\varphi(f^{-1}(V)).$ Then $f_{\ast}$ is an
additive covariant functor from the category of sheaves of Abelian
groups on $X$ to that on $Y.$
\item[(2)]Let $\psi:\mathscr{G}_1\rightarrow \mathscr{G}_2$ be a
morphism of sheaves on $Y.$ Define
$f^{-1}\psi:f^{-1}\mathscr{G}_1\rightarrow f^{-1}\mathscr{G}_2$ to
be the morphism in the following way:

Fix $U\in \mathds{I}_X,$ let $\bar{\mathscr{G}_1}$ be the sheaf
$U\mapsto \varinjlim\limits_{f(U)\subseteq V}\mathscr{G}_1(V),$
$\bar{\mathscr{G}_2}$ be the sheaf $U\mapsto
\varinjlim\limits_{f(U)\subseteq V}\mathscr{G}_2(V),$ and
$\bar{\psi}:\bar{\mathscr{G}_1}\rightarrow \bar{\mathscr{G}_2}$ be
the morphism induced by $\psi.$ We have the following commutative
diagram:
\[ \xymatrix{
   \bar{\mathscr{G}_1} \ar[r]^-{\bar{\psi}} \ar[d] & \bar{\mathscr{G}_2}
   \ar[r]^-{\theta} &
   {\bar{\mathscr{G}_2}^{+}=f^{-1}\mathscr{G}_2}               \\
   {\bar{\mathscr{G}_1}^{+}=f^{-1}\mathscr{G}_1} \ar@{-->}"1,3"_-{\exists !
   f^{-1}\psi} }  \]

Then $f^{-1}$ is an additive covariant functor from the category of
sheaves of Abelian groups on $Y$ to that on $X.$ $f^{-1}$ is exact.
\item[(3)]When $f:X\rightarrow Y$ is an embedding, we often denote
$f^{-1}\mathscr{G}$ by $\left.\mathscr{G}\right|_X$ and call it the
restriction of $\mathscr{G}$ on $X.$

For example, fix $p\in X,$ consider $i:\{p\}\hookrightarrow X,$ let
$\mathscr{F}$ be a sheaf on $X.$ Then $i^{-1}\mathscr{F}$ is a sheaf
on $\{p\}.$ Indeed, the only nonempty open subset of $\{p\}$ is
$\{p\},$ thus the presheaf
$$\{p\}\mapsto \varinjlim_{i(p)\subseteq U}\mathscr{F}(U)=\varinjlim_{p\subseteq U}\mathscr{F}(U)=\mathscr{F}_p$$
is a sheaf. Hence $i^{-1}\mathscr{F}(\{p\})=\mathscr{F}_p.$
\end{list}
\end{remarks}
\begin{thm}
Let $f:X\rightarrow Y$ be a continuous mapping of topological
spaces. Then the functor $f^{-1}$ is a left adjoint of $f_{\ast}.$
More precisely, for any sheaf $\mathscr{F}$ on $X$ and any sheaf
$\mathscr{G}$ on $Y,$ we have
\[ \xymatrix{
   Hom(\mathscr{G}, f_{\ast}\mathscr{F})
   \ar@<.6ex>[r]^-{\alpha_{\mathscr{F},\mathscr{G}}} & Hom(f^{-1}\mathscr{G},
   \mathscr{F}) \ar@<.6ex>[l]^-{\beta_{\mathscr{F},\mathscr{G}}} }
\]
where $\alpha_{\mathscr{F},\mathscr{G}}$ and
$\beta_{\mathscr{F},\mathscr{G}}$ are two canonical isomorphisms and
are functorial in $\mathscr{F}$ and $\mathscr{G}.$
\end{thm}
\begin{proof}
For any sheaf $\mathscr{F}$ on $X,$ define $\phi:
f^{-1}f_{\ast}\mathscr{F}\rightarrow \mathscr{F}$ as follows:\\
$f^{-1}f_{\ast}\mathscr{F}$ is the sheaf associated to the presheaf
$\bar{\mathscr{F}}$ defined by
$$U\mapsto \varinjlim\limits_{f(U)\subseteq V} f_{\ast}\mathscr{F}(V)
= \varinjlim\limits_{U\subseteq f^{-1}(V)} \mathscr{F}(f^{-1}(V)).$$
i.e. $f^{-1}f_{\ast}\mathscr{F}=(\bar{\mathscr{F}})^+.$ Then
$\forall U\in \mathds{I}_X,$ we have the following commutative
diagram
\[ \xymatrix{
   & \mathscr{F}(U) \\\\
   & {\varinjlim\limits_{U\subseteq f^{-1}(V)}
   \mathscr{F}(f^{-1}(V))} \ar@{-->}[uu]_(.3){\exists! \varphi(U)} \\
   \mathscr{F}(f^{-1}(V_1)) \ar[ur] \ar"1,2"^{\rho_{f^{-1}(V_1)\,U}}
   \ar[rr]^-{\rho_{f^{-1}(V_1)\,f^{-1}(V_2)}} & &
   \mathscr{F}(f^{-1}(V_2)) \ar[ul] \ar"1,2"_{\rho_{f^{-1}(V_2)\,U}}
   }  \]
We can easily check that $\varphi$ is a morphism from
$\bar{\mathscr{F}}$ to $\mathscr{F}.$ By the universal property of
$(\bar{\mathscr{F}})^+,$ we obtain a commutative diagram
\[ \xymatrix{
   \bar{\mathscr{F}} \ar[r]^-{\varphi} \ar[d]_{\bar{\theta}} &
   \mathscr{F}                   \\
   f^{-1}f_{\ast}\mathscr{F}=(\bar{\mathscr{F}})^+ \ar@{-->}[ur]_{\exists!
   \phi} }  \]

For any sheaf $\mathscr{G}$ on $Y,$ define $\psi:
\mathscr{G}\rightarrow f_{\ast}f^{-1}\mathscr{G}$ as follows:\\
Define $\bar{\mathscr{G}}: V\mapsto
\varinjlim\limits_{f(f^{-1}(V))\subseteq W}\mathscr{G}(W).$
$\bar{\mathscr{G}}$ is a presheaf, and $(\bar{\mathscr{G}})^+ =
f_{\ast}f^{-1}\mathscr{G}.$ $\forall V\in \mathds{I}_Y,$ we have
$f(f^{-1}(V))\subseteq V,$ hence we can define
\[ \xymatrix@R=0em{
   \mathscr{G}(V) \ar[r]^-{\eta(V)} & \bar{\mathscr{G}}(V)=\varinjlim\limits_{f(f^{-1}(V))\subseteq
   W}\mathscr{G}(W)           \\
   s \ar@{|->}[r] & \text{the equivalence class of }s,\text{ noted
   }\bar{s} }  \]
Put $\psi=\tilde{\theta}\circ\eta,$ where $\tilde{\theta}$ is the
canonical morphism $\tilde{\theta}: \bar{\mathscr{G}}\rightarrow
(\bar{\mathscr{G}})^+=f_{\ast}f^{-1}\mathscr{G}.$

Define
\[ \xymatrix{
   \alpha_{\mathscr{F},\mathscr{G}}(\pi)=\phi\circ(f^{-1}\pi), &
   \pi\in Hom(\mathscr{G}, f_{\ast}\mathscr{F});              \\
   \beta_{\mathscr{F},\mathscr{G}}(\tau)=(f_{\ast}\tau)\circ\psi, &
   \tau\in Hom(f^{-1}\mathscr{G}, \mathscr{F}). }  \]

$\mathit{1^{\circ}}$ $\alpha_{\mathscr{F},\mathscr{G}}$ and
$\beta_{\mathscr{F},\mathscr{G}}$ are functorial in $\mathscr{F}$
and $\mathscr{G}:$

Let $\mathscr{F}^{\prime}$ be another sheaf on $X,$ and take $h\in
Hom(\mathscr{F}, \mathscr{F}^{\prime}).$ We obtain a canonical
morphism $\phi^{\prime}:
f^{-1}f_{\ast}\mathscr{F}^{\prime}\rightarrow \mathscr{F}^{\prime},$
and the following commutative diagram
\[ \xymatrix{
   \bar{\mathscr{F}} \ar[ddd]_{\bar{h}} \ar[dr]_-{\bar{\theta}}
   \ar[rr]^-{\varphi} & & \mathscr{F} \ar[ddd]^h              \\
   & f^{-1}f_{\ast}\mathscr{F} \ar[ur]_-{\phi} \ar[d]^-{\exists!
   f^{-1}f_{\ast}h}                                           \\
   & f^{-1}f_{\ast}\mathscr{F}^{\prime} \ar[dr]^-{\phi^{\prime}}\\
   \bar{\mathscr{F}}^{\prime} \ar[ur]^-{\bar{\theta}^{\prime}}
   \ar[rr]^-{\varphi^{\prime}} & & \mathscr{F} }  \]
where $\bar{h}$ is obtained from
\[ \xymatrix{
   & \mathscr{F}^{\prime}(U) \\
   \mathscr{F}^{\prime}(f^{-1}(V_1)) \ar[ur] & \bar{\mathscr{F}}(U)
   \ar@{-->}[u]_(.3){\exists! \bar{h}(U)} &
   \mathscr{F}^{\prime}(f^{-1}(V_2)) \ar[ul]                       \\
   \mathscr{F}(f^{-1}(V_1)) \ar[u]^{h(f^{-1}(V_1))} \ar[ur] \ar[rr] &
   & \mathscr{F}(f^{-1}(V_2)) \ar[ul] \ar[u]_{h(f^{-1}(V_2))} }  \]
Hence we have
$$\phi^{-1}\circ(f^{-1}(f_{\ast}h)\circ\pi)=\phi^{-1}\circ(f^{-1}f_{\ast}h\circ f^{-1}\pi)=h\circ\phi\circ f^{-1}\pi.$$
Therefore the following diagram
\[ \xymatrix{
   Hom(\mathscr{G}, f_{\ast}\mathscr{F}) \ar[d]_{Hom(\mathscr{G},
   f_{\ast}h)} \ar[r]^-{\alpha_{\mathscr{F}, \mathscr{G}}} &
   Hom(f^{-1}\mathscr{G},\mathscr{F}) \ar[d]^{Hom(f^{-1}\mathscr{G},
   h)}   \\
   Hom(\mathscr{G}, f_{\ast}\mathscr{F}^{\prime}) \ar[r]_-{\alpha_{\mathscr{F}^{\prime},
   \mathscr{G}}} & Hom(f^{-1}\mathscr{G},\mathscr{F}^{\prime}) }  \]
commutes. Let $\mathscr{G}^{\prime}$ be another sheaf on $Y,$ take
$k\in Hom(\mathscr{G}, \mathscr{G}^{\prime}).$ Similarly we obtain
that
\[ \xymatrix{
   Hom(\mathscr{G}^{\prime}, f_{\ast}\mathscr{F}) \ar[d]_{Hom(k,
   f_{\ast}\mathscr{F})} \ar[r]^-{\alpha_{\mathscr{F}, \mathscr{G}^{\prime}}}
   & Hom(f^{-1}\mathscr{G}^{\prime}, \mathscr{F})
   \ar[d]^{Hom(f^{-1}k, \mathscr{F})}                             \\
   Hom(\mathscr{G}, f_{\ast}\mathscr{F}) \ar[r]_-{\alpha_{\mathscr{F},
   \mathscr{G}}} & Hom(f^{-1}\mathscr{G},\mathscr{F}) }  \]
is commutative. Hence $\alpha_{\mathscr{F}, \mathscr{G}}$ is
functorial in $\mathscr{F}$ and $\mathscr{G}.$ We can prove
$\beta_{\mathscr{F}, \mathscr{G}}$ is functorial in $\mathscr{F}$
and $\mathscr{G}$ likewise.

$\mathit{2^{\circ}}$ $\alpha_{\mathscr{F},\mathscr{G}}$ and
$\beta_{\mathscr{F},\mathscr{G}}$ are canonical isomorphisms, i.e.
$$\alpha_{\mathscr{F},\mathscr{G}}\circ\beta_{\mathscr{F},\mathscr{G}}
= id_{Hom(f^{-1}\mathscr{G}, \mathscr{F})},
\beta_{\mathscr{F},\mathscr{G}}\circ\alpha_{\mathscr{F},\mathscr{G}}
= id_{Hom(\mathscr{G},f_{\ast}\mathscr{F})}:$$

$\forall V\in \mathds{I}_Y, \forall s\in \mathscr{G}(V),$ we have
\[ \xymatrix@R=0em{
   \mathscr{G}(V) \ar[r]^-{\psi(V)} & f_{\ast}f^{-1}\mathscr{G}(V)
   \ar[r]^-{f_{\ast}f^{-1}\pi(V)} &
   f_{\ast}f^{-1}f_{\ast}\mathscr{F}(V) \ar[r]^-{f_{\ast}\phi(V)} &
   f_{\ast}\mathscr{F}(V)                                       \\
   s \ar@{|->}[r] & (V\ni p\mapsto (\bar{s})_p) \ar@{|->}[r] &
   (V\ni p\mapsto (\overline{\pi(V)(s)})_p) \ar@{|->}[r] & \pi(V)(s) }
\]
So we have $\pi = f_{\ast}\phi\circ f_{\ast}f^{-1}\pi\circ\psi =
\beta_{\mathscr{F},\mathscr{G}}\circ\alpha_{\mathscr{F},\mathscr{G}}(\pi)
= id_{Hom(\mathscr{G},f_{\ast}\mathscr{F})}(\pi).$

Now $\forall p\in X, \forall t^{(p)}\in (f^{-1}\mathscr{G})_p,$
there exists $U_p$ a neighborhood of $p$ and $t\in
f^{-1}\mathscr{G}(U_p),$ such that $t_p=t^{(p)}.$ Then we have
\[ \xymatrix@R=0em@C=3.5em{
   (f^{-1}\mathscr{G})_p \ar[r]^-{(f^{-1}\psi)_p} &
   (f^{-1}f_{\ast}f^{-1}\mathscr{G})_p \ar[r]^-{(f^{-1}f_{\ast}\tau)_p}
   & (f^{-1}f_{\ast}\mathscr{F})_p \ar[r]^-{\phi_p} & \mathscr{F}_p \\
   t^{(p)}=t_p \ar@{|->}[r] & (\bar{t})_p \ar@{|->}[r] &
   (\overline{\tau(U_p)(t)})_p \ar@{|->}[r] & (\tau(U_p)(t))_p } \]
and $(\tau(U_p)(t))_p=\tau_p(t^{(p)}).$ Hence we obtain that
$$\tau_p = \phi_p\circ(f^{-1}f_{\ast}\tau)_p\circ(f^{-1}\psi)_p =
(\phi\circ f^{-1}f_{\ast}\tau\circ f^{-1}\psi)_p.$$ Applying the
lemma below, we have
$$\tau = \phi\circ f^{-1}f_{\ast}\tau\circ f^{-1}\psi =
\alpha_{\mathscr{F},\mathscr{G}}\circ\beta_{\mathscr{F},\mathscr{G}}(\tau)
= id_{Hom(f^{-1}\mathscr{G},\mathscr{F})}(\tau).$$ Therefore
$\alpha_{\mathscr{F},\mathscr{G}}$ and
$\beta_{\mathscr{F},\mathscr{G}}$ are canonical isomorphisms.
\end{proof}
\begin{lemma}
Let $\varphi, \psi$ be two morphisms from $\mathscr{F}$ to
$\mathscr{G},$ which are sheaves on a topological space $X.$ Then
$\varphi=\psi$ iff $\forall p\in X, \varphi_p=\psi_p.$
\end{lemma}
\begin{proof}
$\Longrightarrow:$ is trivial.

$\Longleftarrow:$ $\forall U\in \mathds{I}_X, \forall s\in
\mathscr{F}(U),$ since $\varphi_p=\psi_p,$ then $\varphi_p(s_p) =
\psi_p(s_p),$ i.e. $(\varphi(U)(s))_p = (\psi(U)(s))_p,$ where $p$
is an arbitrary point in $U.$ Then $\exists U_p\in \mathds{I}_X(p),
U_p\subseteq U$ such that
$\left.\varphi(U)(s)\right|_{U_p}=\left.\psi(U)(s)\right|_{U_p}.$
But $(U_p)_{p\in U}$ is an open covering of $U,$ and $\mathscr{G}$
is a sheaf, then we have $\varphi(U)(s)=\psi(U)(s).$ Hence
$\varphi=\psi.$
\end{proof}
\begin{cor}
Let $f: X\rightarrow Y$ and $g: Y\rightarrow Z$ be two continuous
mappings of topological spaces, then $(g\circ f)_{\ast} =
g_{\ast}\circ f_{\ast}, (g\circ f)^{-1} = f^{-1}\circ g^{-1}.$
\end{cor}
\begin{proof}
$\mathit{1^{\circ}}$ Let $\mathscr{F}$ be a sheaf on $X,$ $\forall
W\in \mathds{I}_Z,$ we have
\begin{eqnarray*}
(g\circ f)_{\ast}\mathscr{F}(W) & = & \mathscr{F}((g\circ
f)^{-1}(W)) = \mathscr{F}(f^{-1}(g^{-1}(W)))    \\
& = & f_{\ast}\mathscr{F}(g^{-1}(W)) =
g_{\ast}(f_{\ast}\mathscr{F}(W))     \\
& = & (g_{\ast}\circ f_{\ast})\mathscr{F}(W).
\end{eqnarray*}
Hence $(g\circ f)_{\ast}\mathscr{F} = (g_{\ast}\circ
f_{\ast})\mathscr{F}.$ Let $\phi: \mathscr{F}_1\rightarrow
\mathscr{F}_2$ be a morphism of sheaves on $X,$ then
\begin{eqnarray*}
(g\circ f)_{\ast}\phi(W) & = & \phi((g\circ f)^{-1}(W)) =
\phi(f^{-1}(g^{-1}(W)))    \\
& = & f_{\ast}\phi(g^{-1}(W)) = g_{\ast}(f_{\ast}\phi(W)) \\
& = & (g_{\ast}\circ f_{\ast})\phi(W),
\end{eqnarray*}
hence$(g\circ f)_{\ast}\phi = (g_{\ast}\circ f_{\ast})\phi.$ So
$(g\circ f)_{\ast} = g_{\ast}\circ f_{\ast}.$

$\mathit{2^{\circ}}$ For any sheaf $\mathscr{F}$ on $X$ and any
sheaf $\mathscr{H}$ on $Z,$ we have
\begin{eqnarray*}
& & Hom((g\circ f)^{-1}\mathscr{H}, \mathscr{F})    \\
& = & Hom(\mathscr{H}, (g\circ f)_{\ast}\mathscr{F}) =
Hom(\mathscr{H}, g_{\ast}\circ f_{\ast}\mathscr{F}) \\
& = & Hom(g^{-1}\mathscr{H}, f_{\ast}\mathscr{F}) = Hom(f^{-1}\circ
g^{-1}\mathscr{H}, \mathscr{F}).
\end{eqnarray*}
Hence $(g\circ f)^{-1}\mathscr{H} = f^{-1}\circ g^{-1}\mathscr{H}.$
Let $\phi: \mathscr{H}_1\rightarrow \mathscr{H}_2$ be a morphism of
sheaves on $Z,$
\[ \xymatrix{
   Hom((g\circ f)^{-1}\mathscr{H}_2, \mathscr{F}) \ar[r] \ar[d]_{Hom((g\circ f)^{-1}\phi,
   \mathscr{F})} & Hom(f^{-1}\circ g^{-1}\mathscr{H}_2, \mathscr{F})
   \ar[d]^{Hom(f^{-1}\circ g^{-1}\phi, \mathscr{F})}              \\
   Hom((g\circ f)^{-1}\mathscr{H}_1, \mathscr{F}) \ar[r] & Hom(f^{-1}\circ g^{-1}\mathscr{H}_1,
   \mathscr{F}) }  \]
is commutative. Then $Hom((g\circ f)^{-1}\phi, \mathscr{F}) =
Hom(f^{-1}\circ g^{-1}\phi, \mathscr{F}).$ Applying on
$id_{\mathscr{F}},$ we have $(g\circ f)^{-1}\phi = f^{-1}\circ
g^{-1}\phi.$ So $(g\circ f)^{-1} = f^{-1}\circ g^{-1}.$
\end{proof}
\begin{cor}
Let $f: X\rightarrow Y$ be a continuous mapping of topological
spaces. Let $p\in X$ and $\mathscr{G}$ be any sheaf on $Y.$ Then
$(f^{-1}\mathscr{G})_p = \mathscr{G}_{f(p)}.$
\end{cor}
\begin{proof}
Let $i: \{p\}\hookrightarrow X$ be the canonical embedding, then the
inverse image $i^{-1}\mathscr{F}$ of any sheaf $\mathscr{F}$ on $X$
can be identified with $\mathscr{F}_p.$ So
$$(f^{-1}\mathscr{G})_p = i^{-1}(f^{-1}\mathscr{G}) = (f\circ
i)^{-1}\mathscr{G} = \mathscr{G}_{f(p)}.$$
\end{proof}
\begin{cor}
Let $f: X\rightarrow Y$ be a continuous mapping of topological
spaces. Then $f^{-1}$ is an exact functor from the category of
sheaves of Abelian groups on $Y$ to that on $X.$
\end{cor}
\begin{proof}
Let $\mathscr{F}\rightarrow \mathscr{G}\rightarrow \mathscr{H}$ be
exact in $Y,$ then $\forall q\in Y,$ $\mathscr{F}_q\rightarrow
\mathscr{G}_q\rightarrow \mathscr{H}_q$ is exact. $\forall p\in X,$
take $q=f(p)$ and apply the above corollary, we have
$(f^{-1}\mathscr{F})_p\rightarrow (f^{-1}\mathscr{G})_p\rightarrow
(f^{-1}\mathscr{H})_p$ is exact. Hence $f^{-1}\mathscr{F}\rightarrow
f^{-1}\mathscr{G}\rightarrow f^{-1}\mathscr{H}$ is exact.
\end{proof}

\newpage

\section{Schemes and morphisms}

Fix $A$ a ring, define
$$SpecA=\{\mathfrak{p}\mid \mathfrak{p}\text{ is prime in }A\}.$$
For any ideal $I$ of $A,$ define
$$V(I)=\{\mathfrak{p}\in SpecA\mid I\subseteq\mathfrak{p}\}.$$
\begin{prop}\
\enum
\item[(1)]$V(0)=SpecA,V(A)=\emptyset.$
\item[(2)]Let $I\subseteq J$ be two ideals of $A$, then
$V(I)\supseteq V(J).$
\item[(3)]For any family of ideals of $A,$ denoted by
$(I_i)_{i\in\Lambda},$ we have
$$\bigcap\limits_{i\in\Lambda}V(I_i)=V(\sum\limits_{i\in\Lambda}).$$
\item[(4)]Let $I,J$ be two ideals of $A,$ then $$V(I)\cup
V(J)=V(IJ)=V(I\cap J).$$
\end{list}
\end{prop}
\begin{remark}
$\{V(I)\mid I\text{ is an ideal of }A\}$ is closed under
intersection, finite union and contains the empty set and the whole
space $SpexA,$ thus induced a topology on $SpecA$ for which the
closed sets take the form $V(I)$ with $I$ an ideal of $A.$ Such a
topology is called the Zariski topology.
\end{remark}
\begin{proof}\
\enum
\item[(1)]
\item[(2)]
\item[(3)]
\item[(4)]
\end{list}
\end{proof}
\begin{eg}
Fix $k$ an algebraically closed field, put $A=k[X],$ then
$$SpecA=\{0\}\cup\{(X-\alpha)\mid \alpha\in k\}.$$
For any ideal $I$ of $A,$ we can write $I=(g)$ with $g\in A.$ If
$g\neq 0,$ write $$g=\prod\limits_{i=1}^m(X-\alpha_i)^{p_i}.$$ Then
$V(I)=\{(X-\alpha_i)\mid 1\leqslant i\leqslant m\}.$
\end{eg}
\begin{prop}\
\enum
\item[(1)]$SpecA=\emptyset$ iff $A=0.$
\item[(2)]For any ideal $I$ of $A,$ we have $V(I)=V(\sqrt{I}).$
\item[(3)]$\sqrt{I}=\bigcap\limits_{\mathfrak{p}\in
V(I)}\mathfrak{p}.$
\item[(4)]For any ideals $I,J$ of $A,$ we have
$$V(I)\subseteq V(J)\Longleftrightarrow \sqrt{I}\supseteq\sqrt{J}.$$
\end{list}
\end{prop}
\begin{proof}\
\enum
\item[(1)]
\item[(2)]
\item[(3)]
\item[(4)]
\end{list}
\end{proof}
\begin{Def}
Let $A$ be a nonzero ring. The sheaf of rings $\mathcal {O}_{SpecA}$
on $SpecA$ is defined as follows: for any nonempty open subset $U$
of $SpecA,$ $\mathcal {O}_{SpecA}(U)$ consists of the functions
$s:U\rightarrow\coprod\limits_{p\in U}A_p$ satisfying: \enum
\item[(1)]$\forall p\in U,s(p)\in A_p;$
\item[(2)]$\forall p\in U, \exists U_p\subseteq U$ a neighborhood of
$p,$ and $a,f\in A,$ such that $\forall q\in U_p,$ we have $f\not\in
q,$ and $s(q)=\frac{a}{f}$ in $A_q.$
\end{list}
For any nonempty open subsets $U,V$ of $SpecA,$ if $V\subseteq U,$
define
\[ \xymatrix@R=0em{
   {\rho_{U\,V}:\mathcal {O}_{SpecA}(U)} \ar[r] & {\mathcal
   {O}_{SpecA}(V)}                                       \\
   s \ar@{|->}[r] & {\left.s\right|_V} }  \]
Then $\mathcal {O}_{SpecA}$ is a sheaf on $SpecA,$ called the
structural sheaf on $SpecA.$ The pair $(SpecA,\mathcal {O}_{SpecA})$
is called the spectrum of $A.$ Often, we denote $\mathcal
{O}_{SpecA}$ by $\mathcal {O}$ if no confusion is possible.
\end{Def}
\begin{prop}
Let $A$ be a nonzero ring, then
\enum
\item[(1)]$\forall p\in SpecA,$ we have a canonical isomorphism
$$\mathcal {O}_p\cong A_p.$$
\item[(2)]$\forall f\in A,$ we have a canonical isomorphism
$$\mathcal {O}(D(f))\cong A_f.$$
In particular, take $f=1,$ then $D(1)=SpecA,$ and $\mathcal
{O}(SpecA)\cong A.$
\end{list}
\end{prop}
\begin{proof}\
\enum
\item[(1)]For any neighborhood $U$ of $p,$ define
\[ \xymatrix@R=0em{
   {\varphi_U:\mathcal {O}(U)} \ar[r] & A_p     \\
   s \ar@{|->}[r] & s(p) }  \]
Take direct limit, and then we obtain a ring homomorphism
$$\varphi_p:\varinjlim_{p\in U}\mathcal {O}(U)=\mathcal {O}_p\rightarrow A_p.$$

$\mathit{1^{\circ}}$ Surjectivity:

$\forall y\in A_p,$ write $y=\frac{a}{f}$ with $a\in A,f\in
A\setminus p.$ $\forall q\in D(f),$ define $s(q)=\frac{a}{f}$ in
$A_q.$ Thus $s\in\mathcal {O}(D(f)),$ and $s(p)=\frac{a}{f}$ in
$A_p.$ Hence $s(p)=y,$ and $\varphi_p(s_p)=y.$

$\mathit{2^{\circ}}$ Injectivity:

Let $s_p\in\mathcal {O}_p$ such that $\varphi_p(s_p)=0.$ Then there
exists a neighborhood of $p,$ noted $U,$ and $s\in\mathcal {O}(U)$
such that $s(p)=0.$ By the definition of $s,$ $\exists U_p\subseteq
U$ a neighborhood of $p, a\in A,f\in U_p$ such that $\forall q\in
U_p,f\not\in q,$ and $s(q)=\frac{a}{f}$ in $A_q.$ Hence
$\frac{a}{f}=0$ in $A_p,$ thus $\exists t\in A\setminus p$ such that
$at=0.$ Then $\forall q\in D(f)\cap D(t)\cap U_p,$ we have
$s(q)=\frac{a}{f}=\frac{at}{ft}=0$ in $A_q.$ But $p\in D(f)\cap
D(t),$ hence $D(f)\cap D(t)\cap U_p$ is a neighborhood of $p,$ and
we have $\left.s\right|_{D(f)\cap D(t)\cap U_p}=0,$ hence $s_p=0.$
Therefore $\varphi_p$ is injective.
\item[(2)]Consider
\[ \xymatrix@R=0em{
   {\varphi:A_f} \ar[r] & \mathcal {O}(D(f))   \\
   x \ar@{|->}[r] & \varphi(x) }  \]
where $\varphi(x)\in\mathcal {O}(D(f)),$ we define $\varphi(x)(p)=x$
in $A_p,$ $\forall p\in D(f).$ We shall show that $\varphi$ is
bijective.

$\mathit{1^{\circ}}$ Injectivity:

If $x\in ker\varphi,$ then $\varphi(x)=0.$ Write $x=\frac{a}{f^k}$
with $a\in A,k\geqslant 1.$ Then $\forall p\in D(f),$ we have
$\frac{a}{f^k}=0$ in $A_p.$ Hence $\exists t^{(p)}\in A\setminus p,$
such that $t^{(p)}a=0,$ thus $Ann(a)\nsubseteq p.$ Therefore
$V(Ann(a))\subseteq V(f),$ and $\sqrt{Ann(a)}\supseteq\sqrt{(f)}\ni
f.$ Thus $\exists n\geqslant 1$ such that $f^n\in Ann(a),$ i.e.
$af^n=0.$ Consequently in $A_f,$ we have
$x=\frac{a}{f^k}=\frac{af^n}{f^{n+k}}=0$ in $A_f.$

$\mathit{2^{\circ}}$ Surjectivity:

Fix $s\in\mathcal {O}(D(f)),$ we want to find $\frac{a}{f^k}\in A_f$
such that $s(q)=\frac{a}{f^k}$ in $A_q,$ $\forall q\in D(f).$
$\forall p\in D(f),\exists U_p\subseteq D(f)$ a neighborhood of $p,$
and $a{(p)}\in A,f^{(p)}\in A,$ such that $\forall q\in U_p,$ we
have $f^{(p)}\not\in q,$ and $s(q)=\frac{a^{(p)}}{f^{(p)}}$ in
$A_q.$ Since open subsets of the form $D(f)$ are the basic open
subsets of $SpecA,$ we can suppose that $U_p=D(g^{(p)})$ for some
$g^{(p)}\in A.$ Since $D(f)=\bigcup\limits_{p\in D(f)}D(g^{(p)})$ is
quasi-compact, thus $\exists p_1,\cdots,p_m\in D(f),$ such that
$D(f)=\bigcup\limits_{i=1}^m D(g^{(p_i)}).$

For $1\leqslant i\leqslant m,$ set
$g_i=g^{(p_i)},f_i=f^{(p_i)},a_i=a^{(p_i)}.$ $\forall q\in D(g_i),$
we have $f_i\not\in q,$ thus $q\in D(f_i),$ and $D(g_i)\subseteq
D(f_i).$ Then $V(f_i)\subseteq
V(g_i),g_i\in\sqrt{(g_i)}\subseteq\sqrt{(f_i)}.$ Thus $\exists
h_i\in A$ such that $g_i^{k_i}=f_ih_i$ with $k_i\geqslant 1.$ Note
that $\forall q\in D(g_i),$ we have $g_i\not\in q,$ thus
$h_if_i\not\in q.$ Hence
$s(q)=\frac{a_i}{f_i}=\frac{a_ih_i}{f_ih_i}=\frac{a_ih_i}{g_i^{k_i}}$
in $A_q.$

Put $\tilde{g}_i=g_i^{k_i},\tilde{a}_i=a_ih_i,$ then
$D(\tilde{g}_i)=D(g_i^{k_i})=D(g_i),$ and $\forall q\in
D(\tilde{g}_i),$ we have $s(q)=\frac{\tilde{a}_i}{\tilde{g}_i}$ in
$A_q.$ $\forall q\in D(\tilde{g}_i)\cap D(\tilde{g}_j)(1\leqslant
i,j\leqslant m),$ we have
$\frac{\tilde{a}_i}{\tilde{g}_i}=s(q)=\frac{\tilde{a}_j}{\tilde{g}_j}$
in $A_q.$ Hence $\exists t\in A\setminus q,$ such that
$t(\tilde{a}_i\tilde{g}_j-\tilde{a}_j\tilde{g}_i)=0.$ So
$Ann(\tilde{a}_i\tilde{g}_j-\tilde{a}_j\tilde{g}_i)\subseteq q,$
then $V(Ann(\tilde{a}_i\tilde{g}_j-\tilde{a}_j\tilde{g}_i))\subseteq
V(\tilde{g}_i\tilde{g}_j),$ and then
$\sqrt{(\tilde{g}_i\tilde{g}_j)} \subseteq
\sqrt{Ann(\tilde{a}_i\tilde{g}_j-\tilde{a}_j\tilde{g}_i)}.$ Then
$\exists l\geqslant 1$ large enough such that $\forall 1\leqslant
i,j\leqslant m, (\tilde{g}_i\tilde{g}_j)^l\in
Ann(\tilde{a}_i\tilde{g}_j-\tilde{a}_j\tilde{g}_i).$ i.e.
$(\tilde{g}_i\tilde{g}_j)^l(\tilde{a}_i\tilde{g}_j-\tilde{a}_j\tilde{g}_i)=0,$
hence $\tilde{a}_i\tilde{g}_j^{l+1}\tilde{g}_i^l =
\tilde{a}_j\tilde{g}_j^l\tilde{g}_i^{l+1}.$

Put $G_i=\tilde{g}_i^{l+1}, b_i=\tilde{a}_i\tilde{g}_i^l(1\leqslant
i\leqslant m),$ then $b_iG_j=b_jG_i.$ $\forall q\in
D(G_i)=D(\tilde{g}_i)=D(g_i),$ we have
$s(q)=\frac{\tilde{a}_i}{\tilde{g}_i} =
\frac{\tilde{a}_i\tilde{g}_i^l}{\tilde{g}_i^{l+1}} =
\frac{b_i}{G_i}$ in $A_q.$ Note that
$D(f)=\bigcup\limits_{i=1}^mD(g_i)=\bigcup\limits_{i=1}^mD(G_i),$
then $V(f)=\bigcap\limits_{i=1}^mV(G_i)=V((G_1,\cdots,G_m)),$ hence
$\sqrt{(f)}=\sqrt{(G_1,\cdots,G_m)}.$ Thus $\exists k\geqslant
1,c_i\in A$ such that $f^k=\sum\limits_{i=1}^mc_iG_i.$

Put $a=\sum\limits_{i=1}^mb_ic_i\in A.$ $\forall 1\leqslant
j\leqslant m,$ we have
$$b_jf^k=b_j\sum\limits_{i=1}^mc_iG_i=\sum\limits_{i=1}^mc_iG_ib_j=
\sum\limits_{i=1}^mc_ib_iG_j=aG_j.$$ Now $\forall q\in D(f), \exists
j(1\leqslant j\leqslant m)$ such that $q\in D(G_i).$ Hence in $A_q,$
we have
$s(q)=\frac{b_j}{G_j}=\frac{a}{f^k}=\varphi(\frac{a}{f^k})(q).$ So
$\varphi(\frac{a}{f^k})=s,$ hence $\varphi$ is surjective.
\end{list}
\end{proof}
\begin{Def}
Let $X$ be a topological space, and $\mathcal {O}_X$ be a sheaf of
rings on $X.$
\enum
\item[(1)]The pair $(X,\mathcal {O}_X)$ is called a ringed space.
\item[(2)]If $\forall p\in X,$ the stalk $\mathcal {O}_{X,p}$ of $\mathcal
{O}_X$ at $p$ is a local ring, then we call $(X,\mathcal {O}_X)$ a
locally ringed space. In this case, we denote by $\mathfrak{M}_p$
the maximal ideal of $\mathcal {O}_{X,p},$ and call $k(p)=\mathcal
{O}_{X,p}/\mathfrak{M}_p$ the residue field of $(X,\mathcal {O}_X)$
at $p.$
\end{list}
\end{Def}
\begin{egs}\
\enum
\item[(1)]Let $X$ be a topological space, and $\mathscr{C}_X$ be the
sheaf of continuous functions on $X.$ Then $(X,\mathcal {O}_X)$ is a
locally ringed space. Indeed, $\forall p\in X,$
$$\mathscr{C}_{X,P}=\{f\mid f\text{ is continuous and defined on a neighborhood of }p\}/\sim.$$
We say $f\sim g,$ if $\exists U_p$ a neighborhood of $p$ such that
$\left.f\right|_{U_p}=\left.g\right|_{U_p}.$
$$\mathfrak{M}_p=\{f\mid f\in\mathscr{C}_{X,p}, f(p)=0\}.$$
$\mathfrak{M}_p$ is an ideal and
$\mathscr{C}_{X,P}^{\times}=\mathscr{C}_{X,P}\setminus
\mathfrak{M}_p.$
\item[(2)]Let $A$ be a nonzero ring, then the spectrum $(SpecA,\mathcal
{O}_{SpecA})$ is a locally ringed space. Indeed, $\forall p\in
SpecA,$ we have $\mathcal {O}_{SpecA,p}\cong A_p,$ thus $\mathcal
{O}_{SpecA,p}$ is a local ring.
\end{list}
\end{egs}
\begin{Def}
Let $(X,\mathcal {O}_X)$ and $(Y,\mathcal {O}_Y)$ be two ringed
spaces. A morphism from $(X,\mathcal {O}_X)$ to $(Y,\mathcal {O}_Y)$
is by definition a pair $(f,f^{\sharp})$ where $f:X\rightarrow Y$ is
a continuous mapping and $f^{\sharp}:\mathcal {O}_Y\rightarrow
f_{\ast}\mathcal {O}_X$ is a morphism of sheaves.
\end{Def}
\begin{remarks}\
\enum
\item[(1)]By definition, for any nonempty open subset $V$ of $Y,$ we
have $(f_{\ast}\mathcal {O}_X)(V)=\mathcal {O}_X(f^{-1}(V)).$
$\forall s\in (f_{\ast}\mathcal {O}_X)(V),$
$$s:V\rightarrow\coprod\limits_{p\in V}(f_{\ast}\mathcal {O}_X)_p,$$
$\exists t\in \mathcal {O}_X(f^{-1}(V))$ such that $s\circ f=t.$
Indeed, we only have to define $t=s\circ f.$
\item[(2)]Let
\[ \xymatrix{
   (X,\mathcal {O}_X) \ar[r]^-{(f,f^{\sharp})} & (Y,\mathcal {O}_Y)
   \ar[r]^-{(g,g^{\sharp})} & (Z,\mathcal {O}_Z) }  \]
be morphisms of ringed spaces, and
$$((g\circ f),(g\circ f)^{\sharp}):=(g,g^{\sharp})\circ(f,f^{\sharp}).$$
We have $(g\circ f)^{\sharp}=g_{\ast}(f^{\sharp})\circ g^{\sharp}.$
\item[(3)]Fix $p\in X,$ let $(f,f^{\sharp}):(X,\mathcal {O}_X)\rightarrow (Y,\mathcal
{O}_Y)$ be a morphism. We have a canonical ring homomorphism
$\varphi_{(p)}:$
$$(f_{\ast}\mathcal {O}_X)_{f(p)}=\varinjlim_{f(p)\in V}(f_{\ast}\mathcal {O}_X)(V)
=\varinjlim_{p\in f^{-1}(V)}\mathcal {O}_X(f^{-1}(V))\rightarrow
\varinjlim_{p\in U}\mathcal {O}_X(U)=\mathcal {O}_{X,p}.$$
\[ \xymatrix{
   & \mathcal {O}_{X,p}                                       \\
   & (f_{\ast}\mathcal {O}_X)_{f(p)} \ar@{-->}[u]_(.3){\exists !
   \varphi^{(p)}}                                             \\
   \mathcal {O}_X(f^{-1}(V_1)) \ar[ur] \ar"1,2" \ar[rr] & & \mathcal
   {O}_X(f^{-1}(V_2)) \ar[ul] \ar"1,2" }  \]
Indeed, for any neighborhood $V$ of $f(p),$ we denote by
$$\rho_V:\mathcal {O}_X(f^{-1}(V))\rightarrow (f_{\ast}\mathcal
{O}_X)_{f(p)}$$ the canonical morphism. $\forall s\in
(f_{\ast}\mathcal {O}_X)_{f(p)},\exists
V_p\in\mathds{I}_Y(f(p)),s^{(V_p)}\in \mathcal {O}_X(f^{-1}(V))$
such that $s=\rho_V(s^{(V_p)}).$ Define
$\varphi_{(p)}(s)=s^{(V_p)}_p.$ Put
$$f^{\sharp}_p=\varphi_{(p)}\circ (f^{\sharp})_{f(p)}:\mathcal {O}_{Y,f(p)}\rightarrow \mathcal
{O}_{X,p}.$$ We say that $(f,f^{\sharp})$ is a morphism of locally
ringed space if $\forall p\in X,$ $f^{\sharp}_p$ is a local
homomorphism of local rings, i.e. $\forall p\in X,
(f^{\sharp}_p)^{-1}(\mathfrak{M}_p)=\mathfrak{M}_{f(p)}.$
\end{list}
\end{remarks}
\begin{Def}
Let $(f,f^{\sharp}):(X,\mathcal {O}_X)\rightarrow(Y,\mathcal {O}_Y)$
be a morphism of locally ringed spaces. We call $(f,f^{\sharp})$ an
isomorphism if it has a two-sided inverse. Equivalently, it is to
say that $f:X\rightarrow Y$ is a homeomorphism of topological
spaces, and $f^{\sharp}:\mathcal {O}_Y\rightarrow f_{\ast}\mathcal
{O}_X$ is an isomorphism of sheaves.
\end{Def}
\begin{prop}
Let $f:X\rightarrow Y$ be a homeomorphism of topological spaces.
Then a morphism of locally ringed space $(f,f^{\sharp}):(X,\mathcal
{O}_X)\rightarrow (Y,\mathcal {O}_Y)$ is an isomorphism iff $\forall
p\in X, f^{\sharp}_p=:\mathcal {O}_{Y,f(p)}\rightarrow \mathcal
{O}_{X,p}$ is an isomorphism.
\end{prop}
\begin{proof}
Since $f$ is a homeomorphism, $\forall p\in X,$ the canonical
morphism
$$\varphi_{(p)}:(f_{\ast}\mathcal {O}_X)_{f(p)}=\varinjlim_{p\in f^{-1}(V)}\mathcal {O}_X(f^{-1}(V))=
\varinjlim_{p\in U}\mathcal {O}_X(U)=\mathcal
{O}_{X,p}\rightarrow\mathcal {O}_{X,p}$$ is an isomorphism. But
$f^{\sharp}_p=\varphi^{(p)}\circ(f^{\sharp})_{f(p)},$ hence
\begin{eqnarray*}
& & f^{\sharp}_p \text{ is an isomorphism}             \\
& \Longleftrightarrow & (f^{\sharp})_{f(p)} \text{ is an
isomorphism}                                           \\
& \Longleftrightarrow & (f^{\sharp})_q \text{ is an isomorphism, }
\forall q\in Y                                       \\
& \Longleftrightarrow & f^{\sharp} \text{ is an isomorphism of
sheaves of rings on }Y.
\end{eqnarray*}
\end{proof}
\begin{Def}
Let $(X,\mathcal {O}_X)$ be a locally ringed space.
\enum
\item[(1)]We call $(X,\mathcal {O}_X)$ an affine scheme of it is
isomorphic to $(SpecA,\mathcal {O}_{SpecA})$ for some ring $A.$
\item[(2)]We call $(X,\mathcal {O}_X)$ a scheme if there exists an
open covering $(U_i)_{i\in I}$ of $X$ such that $\forall i\in
I,(U_i, \left.\mathcal {O}_X\right|_{U_i})$ is an affine scheme. In
this case, we call $X$ the underlying topological space, and
$\mathcal {O}_X$ the structural sheaf, and we denote $(X,\mathcal
{O}_X)$ simply by $X.$
\item[(3)]Let $(X,\mathcal {O}_X)$ and $(Y,\mathcal {O}_Y)$ be two
schemes, and $(f,f^{\sharp}):(X,\mathcal {O}_X)\rightarrow
(Y,\mathcal {O}_Y)$ be a morphism of locally ringed space. Then we
call $(f,f^{\sharp})$ a morphism of schemes, and we often denote it
simply by $f.$
\end{list}
\end{Def}
\begin{prop}
Let $A,B$ be two nonzero rings, then
$$Hom(SpecB,SpecA)\cong Hom(A,B).$$
\end{prop}
\begin{proof}
$\mathit{1^{\circ}}$ $\forall\varphi\in Hom(A,B),$ define
\[ \xymatrix@R=0em@C=5em{
   SpecB \ar[r]^-{f=Spec\varphi} & SpecA   \\
   q \ar@{|->}[r] & \varphi^{-1}(q) }  \]
$f$ is continuous, because $\forall a\in A,
f^{-1}(D(a))=D(\varphi(a)).$ Then we shall define
$f^{\sharp}:\mathcal {O}_{SpecA}\rightarrow f_{\ast}\mathcal
{O}_{SpecB}$ as follows. For each nonempty open subset $V$ of
$SpecA,$ and $\forall s\in \mathcal {O}_{SpecA}(V),$ define
$f^{\sharp}(V)(s)\in f_{\ast}\mathcal {O}_{SpecB}(V)=\mathcal
{O}_{SpecB}(f^{-1}(V))$ by $\forall q\in f^{-1}(V),$ put
$f^{\sharp}(V)(s)(q)=\varphi_q(s(f(q))).$ Then $(f,f^{\sharp})$ is a
morphism of schemes if $\forall q\in SpecB,$
$$f^{\sharp}_q:A_{f(q)}\cong\mathcal {O}_{SpecA,f(q)}\rightarrow \mathcal {O}_{SpecB,q}\cong
B_q$$ is local. Indeed, $f^{\sharp}_q=\varphi_q,$ $\varphi_q$ is
local, because $(\varphi_q)^{-1}(qB_q)=pA_p$ with
$p=\varphi^{-1}(q).$

$\mathit{2^{\circ}}$ Let $(f,f^{\sharp}):(SpecB,\mathcal
{O}_{SpecB})\rightarrow (SpecA,\mathcal {O}_{SpecA})$ be a morphism
of schemes. Then we obtain a ring homomorphism
$$f^{\sharp}(SpecA):A\cong\mathcal {O}_{SpecA}(SpecA)\rightarrow
f_{\ast}\mathcal {O}_{SpecB}(SpecA)=\mathcal {O}_{SpecB}(SpecB)\cong
B.$$ Put $\varphi=f^{\sharp}(SpecA)\in Hom(A,B).$

$\mathit{3^{\circ}}$
$\xymatrix@R=0em{
Hom(A,B) \ar[r] & Hom(SpecB,SpecA) \ar[r] & Hom(A,B)           \\
\varphi \ar@{|->}[r] & (f,f^{\sharp})(f=Spec\varphi) \ar@{|->}[r] &
f^{\sharp}(SpecA) } $ is identity: We have the following ring
homomorphism:
\[ \xymatrix@R=0em{
   A \ar[r] & \mathcal {O}_{SpecA}(SpecA)
   \ar[r]^-{f^{\sharp}(SpecA)} & \mathcal {O}_{SpecB}(SpecB) \ar[r]
   & B                                                           \\
   a \ar@{|->}[r] & (p\mapsto a\in A_p) \ar@{|->}[r] & (q\mapsto \varphi_q(a) \in
   B_q) \ar@{|->}[r] & \varphi(a) }  \]
Thus $(Spec\varphi)^{\sharp}(SpecA)(a)=\varphi(a),\forall a\in A.$
Hence $f^{\sharp}(SpecA)=\varphi.$

$\mathit{4^{\circ}}$ $Hom(SpecB,SpecA)\rightarrow
Hom(A,B)\rightarrow Hom(SpecB,SpecA)$ is identity: $\forall q\in
SpecB,$ we have the following canonical commutative diagram
\[ \xymatrix{
   A \ar[r]^-{\sim} \ar"3,2"_-{p_A} & \mathcal {O}_{SpecA}(SpecA)
   \ar[r]^-{f^{\sharp}(SpecA)}_-{\varphi} \ar[d] & \mathcal {O}_{SpecB}(SpecB)
   \ar[r]^-{\sim} \ar[d] & B \ar"3,3"^-{p_B}                  \\
   & \mathcal {O}_{SpecA,f(q)} \ar[d]^-{\sim} \ar[r]^-{f^{\sharp}_q}
   & \mathcal {O}_{SpecB,q} \ar[d]^-{\sim}                    \\
   & A_{f(q)} \ar@{-->}[r]^-{\psi_q} & B_q }  \]
$\varphi^{-1}(q)=\varphi^{-1}(p_B^{-1}(qB_q))=p_A^{-1}(\psi_q^{-1}(qB_q))=p_A^{-1}(f(q)A_{f(q)})=f(q).$
Then we obtain that $f=Spec\varphi$ with
$\varphi=f^{\sharp}(SpecA)$. $\psi_q$ is local, for $^{\sharp}_q$ is
local. And we have $\psi_q=\varphi_q.$ Therefore for any nonempty
open subset $V$ of $SpecA$ and for any $q\in F^{-1}(V),$ we have the
following commutative diagram
\[ \xymatrix{
   \mathcal {O}_{SpecA}(V) \ar[r]^-{f^{\sharp}(V)} \ar[d]_{\tau_1} & \mathcal
   {O}_{SpecB}(f^{-1}(V)) \ar[d]^{\tau_2}                         \\
   \mathcal {O}_{SpecA,f(q)} \ar[r]^-{f^{\sharp}_q} \ar[d]^{\sim} & \mathcal
   {O}_{SpecB,q} \ar[d]_{\sim}                                    \\
   A_{f(q)} \ar[r]^{\varphi_q} & B_q }  \]
where $\tau_1$ is defined by $s\mapsto s(f(q)),$ $\tau_2$ is defined
by $f^{\sharp}(V)(s)\mapsto f^{\sharp}(V)(s)(q).$ But
$\tau_2(f^{\sharp}(V)(s))=\varphi_q(s(f(q)))$ by the commutativity
of the diagram above. Hence
$f^{\sharp}(V)(s)(q)=\varphi_q(s(f(q))).$ Let $(g,g^{\sharp})$ be
the morphism induced by $\varphi,$ we have
$g^{\sharp}(V)(s)(q)=\varphi_q(s(g(q)))$ by definition. But we have
proved that $g=f,$ hence
$$g^{\sharp}(V)(s)(q)=\varphi_q(s(g(q)))=\varphi_q(s(f(q)))=f^{\sharp}(V)(s)(q).$$
Therefore we obtain that $g^{\sharp}=f^{\sharp}(V).$
\end{proof}
\begin{prop}
$\forall a\in A\setminus\{0\},$ there exists a canonical isomorphism
of locally ringed space
$$(D(a), \left.\mathcal {O}_{SpecA}\right|_{D(a)})\cong(SpecA_a,\mathcal {O}_{SpecA_a}).$$
\end{prop}
\begin{proof}
The canonical homomorphism $\varphi:A\rightarrow A_a$ induced a
morphism of schemes $$(f,f^{\sharp}):(SpecA_a,\mathcal
{O}_{SpecA_a})\rightarrow(SpecA,\mathcal {O}_{SpecA}).$$

Put $S=\{a^n\mid n\geqslant 0\},$ $f:SpecA_a\rightarrow SpecA$ is an
open embedding. Indeed, it's immediate that $f$ is continuous. $f$
is injective, because we have a $1-1$ correspondence
\[ \xymatrix@R=0em{
   SpecA_a \ar@{<->}[r] & D(a)   \\
   S^{-1}p \ar@{<->}[r] & p,p\cap S=\emptyset }  \]
$f$ is open, because $\forall t=\frac{b}{a^n}\in A_a,$ we have
$$f(D(t))=\{p\in SpecA\mid S^{-1}p\not\ni\frac{b}{a^n}\}=\{p\in SpecA\mid b\not\in p\}=D(b).$$

For $f^{\sharp}:\mathcal {O}_{SpecA}\rightarrow f_{\ast}\mathcal
{O}_{SpecA},$ we want to prove that
$$\left.f^{\sharp}\right|_{D(a)}:\mathcal {O}_{D(a)}\rightarrow f_{\ast}\mathcal
{O}_{SpecA_a}$$ is an isomorphism. i.e. $\forall q\in SpecA_a,
f^{\sharp}_q$ is an isomorphism, which is equivalent to saying that
$f^{\sharp}_q=\varphi_q$ is an isomorphism. But we have
\[ \xymatrix{
   A \ar[r]^-{\varphi} \ar[d] & A_a \ar[d]                  \\
   {A_p=A_{f(q)}} \ar[r]^-{\varphi_q} &
   {(A_a)_q=(S^{-1}A)_{S^{-1}p}=S^{-1}(A_p)=A_p} }  \]
\end{proof}
\begin{cor}
Let $(X,\mathcal {O}_X)$ be a scheme, and $U$ an nonempty open
subset of $X.$ Then $(U, \left.\mathcal {O}_X\right|_U)$ is a
scheme, called an open subscheme of $(X,\mathcal {O}_X).$
\end{cor}
\begin{remark}
Let $(f,f^{\sharp}):(X,\mathcal {O}_X)\rightarrow (Y,\mathcal
{O}_Y)$ be a morphism of schemes. We say that $(f,f^{\sharp})$ is an
open immersion if $f$ is an open embedding and
$\left.f^{\sharp}\right|{f(X)}: \left.\mathcal
{O}_Y\right|_{f(X)}\rightarrow f_{\ast}\mathcal {O}_X$ is an
isomorphism. In other words, we have
$$(f,f^{\sharp}):(X,\mathcal {O}_X)\cong (f(X), \left.\mathcal {O}_Y\right|_{f(X)}).$$
\end{remark}
\begin{proof}
Cover $X$ by affine open subschemes $U_i=SpecA_i(i\in I).$ Then $U$
can be covered by $U_i\cap U(i\in I).$ $\forall i\in I,$ if $U_i\cap
U\neq\emptyset,$ then $U_i\cap U$ is a nonempty open subset of
$U_i=SpecA_i,$ thus can be covered by $D(f_{ij})$ with $f_{ij}\in
A_i(j\in J_i).$ But $(U, \left.\mathcal {O}_X\right|_U)$ is a
locally ringed space, hence $U$ is a scheme.
\end{proof}
\begin{eg}
Let $A$ be a nonzero ring, $\forall a\in A\setminus \{0\},$
$SpecA_a$ is an open subscheme of $SpecA.$ In particular,
$SpecA_a\rightarrow SpecA$ which is induced by the canonical ring
homomorphism $A\rightarrow A_a$ is an open immersion.
\end{eg}

\newpage

\section{Collage}

\begin{prop}
Let $(U_i)_{i\in I}$ be a family of topological spaces. Suppose that
$\forall i,j\in I(i\neq j),$ there exists an open subset $U_{ij}$ of
$U_i$ and a homeomorphism $\varphi_{ij}: U_{ij}\rightarrow U_{ji}$
with the following properties: \enum
\item[(1)]$\forall i,j\in I(i\neq j),
\varphi_{ij}^{-1}=\varphi_{ji}^{-1}$
\item[(2)]For any distinct $i,j,k\in I,$ we have $\varphi_{ij}(U_{ij}\cap U_{ik})=U_{ji}\cap
U_{jk}, \left.\varphi_{jk}\circ\varphi_{ij}\right|_{U_{ij}\cap
U_{ik}}=\left.\varphi_{ik}\right|_{U_{ij}\cap U_{ik}}$
\end{list}

Then there exists a topological space $X$ together with open
embeddings $\varphi_i: U_i\rightarrow X(i\in I)$ such that: \enum
\item[1]$(\varphi_i(U_i))_{i\in I}$ is an open covering of $X$
\item[2]$\forall i,j\in I(i\neq j), \varphi_i(U_{ij}) = \varphi_i(U_i)\cap
\varphi_j(U_j), \left.\varphi_j\circ\varphi_{ij}\right|_{U_{ij}} =
\left.\varphi_i\right|_{U_{ij}}$
\end{list}
\end{prop}
\begin{proof}
$\forall i\in I,$ define $U_{ii}=U_i, \varphi_{ii}=id_{U_i},$
then$(1),(2)$ hold for any $i,j,k\in I.$
Put$$\Omega=\coprod\limits_{i\in I}U_i=\{(i,x_i)\mid i\in I, x_i\in
U_i\}$$ $\forall x=(i, x_i), y=(j, x_j)\in\Omega,$ we say that
$x\sim y$ if $x_j=\varphi_{ij}(x_i).$ We want to show that $\sim$ is
an equivalent relation.

$\mathit{1^{\circ}}$ If $x\sim y,$ then $x_j=\varphi_{ij}(x_i),$
thus $x_i=\varphi^{-1}_{ij}(x_j)=\varphi_{ji}(x_j),$ hence $y\sim
x.$

$\mathit{2^{\circ}}$ If $x\sim y, y\sim z=(k,x_k),$ then
$x_j=\varphi_{ij}(x_i), x_k=\varphi_{jk}(x_j).$ Since
$x_j=\varphi_{ij}(x_i),$ we have $x_i\in U_{ij}.$
$x_k=\varphi_{jk}(x_j),$ then $x_k\in U_{jk},$ i.e.
$\varphi_{ij}(x_i)\in U_{jk},$ thus $x_i\in\varphi_{ji}(U_{jk}\cap
U_{ji}),$ which is equivalent to saying that $x_i\in U_{ij}\cap
U_{ik}.$ Applying $(2)$ we obtain
$x_k=\varphi_{jk}\circ\varphi_{ij}(x_i)=\varphi_{ik}(x_i),$ hence
$x\sim z$

$\forall x\in\Omega,$ we denote $\widetilde{x}$ the equivalence
class containing $x,$ and put $X=\Omega/\sim=\{\widetilde{x}\mid
x\in\Omega\}.$ $\forall x_i\in U_i,$ define
$\varphi_i(x_i)=\widetilde{(i, x_i)}\in X.$ We endow $X$ with the
finest topology such that all $\varphi_i$ are continuous.
$$\tau(X)=\{V\subseteq X\mid\forall i\in I, \varphi_i^{-1}(V) \text{ is open in }
U_i\}$$ In particular, $\forall V\in\tau(X),$ we have
$V=\bigcup\limits_{i\in I}\varphi_i(\varphi_i^{-1}(V)).$ Indeed
$\forall i\in I, \varphi_i(\varphi_i^{-1}(V))\subseteq V,$ hence
$\bigcup\limits_{i\in I}\varphi_i(\varphi_i^{-1}(V))\subseteq V.$
Conversely, $\forall \widetilde{x}\in V, \exists i\in I, x_i\in
U_i,$ such that $\widetilde{x}=\widetilde{(i,x_i)},$ thus
$x_i\in\varphi_i^{-1}(\widetilde{x}),$ then
$\widetilde{x}\in\varphi_i(\varphi_i^{-1}(V)).$ So $V\subseteq
\bigcup\limits_{i\in I}\varphi_i(\varphi_i^{-1}(V)).$ Therefor we
obtain that $V=\bigcup\limits_{i\in I}\varphi_i(\varphi_i^{-1}(V)).$
Moreover $\forall i\in I, \varphi_i(U_i)$ is open in $X,$ and
$X=\bigcup\limits_{i\in I}\varphi_i(U_i).$ Indeed $\forall j\in I,$
we have $\varphi_j(\varphi_i^{-1}(U_i))=U_{ji}$:
\begin{eqnarray*}
& & \forall x_j\in U_j, x_j\in\varphi^{-1}_j(\varphi_i(U_i)) \\
& \Leftrightarrow & \widetilde{(j, x_j)}\in
\varphi_i(U_i)                                           \\
& \Leftrightarrow & \exists x_i\in U_i \text{ such that }
\widetilde{(j, x_j)}=\widetilde{(i, x_i)}                \\
& \Leftrightarrow & \varphi_{ij}(x_i)                    \\
& \Leftrightarrow & x_j\in U_{ji}
\end{eqnarray*}
hence $\varphi^{-1}_j(\varphi_i(U_i))=U_{ji}.$ $U_{ji}$ is open in
$U_i,$ so $\varphi_i(U_i)$ is open in $X.$

Now we show that $\forall i\in I, \varphi_i: U_i\rightarrow X$ is an
open embedding. First, by construction, $\varphi_i$ is continuous.
Second, $\varphi_i$ is injective. Indeed, $\forall x_i, y_i\in U_i,$
if $\varphi_i(x_i)=\varphi_i(y_i),$ then
$\widetilde{(i,x_i)}=\widetilde{(i,y_i)},$ hence
$$x_i=\varphi_{ii}(y_i)=id_{U_i}(y_i)=y_i.$$ Third, for any open
subset $V\subseteq U_i, \varphi_i(V)$ is open in $X.$ In other
words, $\forall j\in I, \varphi^{-1}_j(\varphi_i(V))$ is open in
$U_j.$ But $\varphi^{-1}_j(\varphi_i(V))=\varphi_{ij}(U_{ij}\cap V)$
is open in $U_{ji},$ for $U_{ij}\cap V$ is open in $U_{ij},$ and
$\varphi_{ij}$ is a homeomorphism. Therefore
$\varphi^{-1}_j(\varphi_i(V))$ is open in $U_j,$ hence
$\varphi_i(V)$ is open in $X.$

$\forall i,j,k\in I, \forall x_{ij}\in U_{ij},
\widetilde{(i,x_{ij})}=\widetilde{(j,\varphi_{ij}(x_{ij}))},$ then
we have$$\varphi_i(x_{ij})=\widetilde{(i,x_{ij})}=
\widetilde{(j,\varphi_{ij}(x_{ij}))}=\varphi_j(\varphi_{ij}(x_{ij})),$$hence
$\left.\varphi_j\circ\varphi_{ij}\right|_{U_{ij}} =
\left.\varphi_i\right|_{U_{ij}}.$
$$\varphi_i(U_{ij})=\varphi_j(\varphi_{ij}(U_{ij}))=\varphi_j(U_{ji})\subseteq \varphi_j(U_j),
\varphi_i(U_{ij})\subseteq\varphi_i(U_i),$$ so we have
$\varphi_i(U_{ij})\subseteq\varphi_i(U_i)\cap\varphi_j(U_j).$
$\forall z\in\varphi_i(U_i)\cap\varphi_j(U_j), \exists x_i\in U_i,
x_j\in U_j$ such that $z=\varphi_i(x_i)=\varphi_j(x_j),$ i.e.
$\widetilde{(i,x_i)}=\widetilde{(j,x_j)},$ hence
$x_j=\varphi_{ij}(x_i)$ and $x_i\in U_{ij}.$ So $z=\varphi_i(x_i)\in
\varphi_i(U_{ij}),$ thus
$\varphi_i(U_i)\cap\varphi_j(U_j)\subseteq\varphi_i(U_{ij}).$
Therefore we obtain that
$\varphi_i(U_{ij})=\varphi_i(U_i)\cap\varphi_j(U_j).$

Uniqueness of $(X,(\varphi_i)_{i\in I})$: If
$(\widetilde{X},(\widetilde{\varphi_i})_{i\in I})$ satisfies the
same properties, then $\forall i\in I,$ we have
\[ \xymatrix{
   & {\varphi_i(U_i)} \ar@{-->}[dd]^-{\exists ! \psi_i}   \\
   U_i \ar[ur]^-{\varphi_i} \ar[dr]_-{\widetilde{\varphi_i}}  \\
   & {\widetilde{\varphi_i}(U_i)} }  \]
$\psi_i$ is a homeomorphism, for $\varphi_i, \widetilde{\varphi_i}$
are homeomorphisms. $\psi_i =
\left.\widetilde{\varphi_i}\circ\varphi_i^{-1}\right|_{\varphi_i(U_i)},$
then $\forall z\in\varphi_i(U_{ij}),$ write $z=\varphi_i(x_{ij}),
x_{ij}\in U_{ij},$ we have
\begin{eqnarray*}
\psi_i(z) & = &
\widetilde{\varphi_i}\circ\varphi_i^{-1}(\varphi_i(x_{ij}))=\widetilde{\varphi_i}(x_{ij})\\
          & = &
          \widetilde{\varphi_i}\circ\varphi_{ji}(\varphi_{ij}(x_{ij}))=\widetilde{\varphi_j}\circ\varphi_{ij}(x_{ij})\\
          & = &
          \widetilde{\varphi_j}\circ\varphi_j^{-1}\circ\varphi_j\circ\varphi_{ij}(x_{ij})=\psi_j(\varphi_i(x_{ij}))\\
          & = & \psi_j(z)
\end{eqnarray*}
Hence $\left.\psi_i\right|_{\varphi_i(U_{ij})} =
\left.\psi_j\right|_{\varphi_i(U_{ij})},$ i.e.
$\left.\psi_i\right|_{\varphi_i(U_i)\cap\varphi_j(U_j)} =
\left.\psi_j\right|_{\varphi_i(U_i)\cap\varphi_j(U_j)}.$ Then we can
define $\psi: X\rightarrow\widetilde{X}$ such that $\forall i\in I,
\left.\psi\right|_{\varphi_i(U_i)}=\psi_i.$ Likewise we can define
$\widetilde{\psi}: \widetilde{X}\rightarrow X$ such that $\forall
i\in I,
\left.\widetilde{\psi}\right|_{\widetilde{\varphi_i}(U_i)}=\psi_i^{-1}.$
Thus $\psi\circ\widetilde{\psi}=id_{\widetilde{X}},
\widetilde{\psi}\circ\psi=id_X.$ Moreover $\psi, \widetilde{\psi}$
are continuous, since $\forall i\in I, \psi_i$ is a homeomorphism.
Therefore $X, \widetilde{X}$ are homeomorphic.
\end{proof}
\begin{prop}
Let $X$ be a topological space, and $\mathscr{F},
\widetilde{\mathscr{F}}$ be two sheaves on $X.$ Let $(U_i)_{i\in I}$
be an open covering of $X$ such that $\forall i\in I,\psi_i:
\left.\mathscr{F}\right|_{U_i}\rightarrow
\left.\widetilde{\mathscr{F}}\right|_{U_i}$ is an isomorphism of
sheaves on $U_i$ such that $\left.\psi_i\right|_{U_i\cap
U_j}=\left.\psi_j\right|_{U_i\cap U_j}.$ Then there exists an
isomorphism $\psi: \mathscr{F}\rightarrow \widetilde{\mathscr{F}}$
such that $\forall i\in I, \left.\psi\right|_{U_i}=\psi_i.$ Moreover
$\psi$ is unique.
\end{prop}
\begin{proof}
$\mathit{1^{\circ}}$ Existence:

For any nonempty open subset $V$ of $X,$ define $V_i=U_i\cap V,
\forall i\in I.$ $\forall s\in\mathscr{F}(V),$ set
$t_i=\psi_i(V_i)(\left.s\right|_{V_i})\in\widetilde{\mathscr{F}}(V_i),
\forall i\in I.$ We want to find a unique
$t\in\widetilde{\mathscr{F}}(V)$ such that
$\left.t\right|_{V_i}=t_i.$ $\forall i,j\in I,$
\begin{eqnarray*}
\left.t_i\right|_{V_i\cap V_j} & = &
\left.\psi_i(V_i)(\left.s\right|_{V_i})\right|_{V_i\cap
V_j}=\psi_i(V_i\cap V_j)(\left.s\right|_{V_i\cap V_j})       \\
& = & \psi_j(V_i\cap V_j)(\left.s\right|_{V_i\cap
V_j})=\left.\psi_j(V_j)(\left.s\right|_{V_j})\right|_{V_i\cap V_j} \\
& = & \left.t_j\right|_{V_i\cap V_j}
\end{eqnarray*}
Since $\widetilde{\mathscr{F}}$ is a sheaf, such a $t$ does exist.

Set $\psi(V)(s)=t,$ then $\psi: \mathscr{F}\rightarrow
\widetilde{\mathscr{F}}$ is a morphism of sheaves such that $\forall
i\in I, \left.\psi\right|_{U_i}=\psi_i.$ Moreover $\forall p\in X,
\exists i\in I$ such that $p\in U_i,$ then $\psi_p=(\psi_i)_p$ is an
isomorphism. Hence $\psi$ is an isomorphism.

$\mathit{2^{\circ}}$ Uniqueness:

Let $\widetilde{\psi}: \mathscr{F}\rightarrow
\widetilde{\mathscr{F}}$ be an isomorphism of sheaves such that
$\left.\widetilde{\psi}\right|_{U_i}=\psi_i, \forall i\in I.$ Then
$\forall p\in X, \exists i\in I$ such that $p\in U_i,$ then
$\psi_p=(\psi_i)_p=\widetilde{\psi}_p.$ Therefore
$\psi=\widetilde{\psi}.$
\end{proof}
\begin{prop}
Let $X$ be a topological space and $(U_i)_{i\in I}$ be an open
covering of $X.$ $\forall i\in I,$ let $\mathscr{F}_i$ be a sheaf on
$U_i.$ Suppose that $\forall i,j\in I, \varphi_{ij}:
\left.\mathscr{F}_i\right|_{U_i\cap U_j}\rightarrow
\left.\mathscr{F}_j\right|_{U_i\cap U_j}$ is an isomorphism of
sheave on $U_i\cap U_j.$ Suppose that the following hold: \enum
\item[(1)]$\forall i\in I, \varphi_{ii}=id_{\mathscr{F}_i}$
\item[(2)]$\forall i,j,k\in
I,\varphi_{ik}=\varphi_{jk}\circ\varphi_{ij}$ on
$\left.\mathscr{F}_i\right|_{U_i\cap U_j\cap U_k}$
\end{list}

Then there exists a sheaf $\mathscr{F}$ on $X$ together with
isomorphisms $\varphi_i:
\left.\mathscr{F}\right|_{U_i}\rightarrow\mathscr{F}_i, \forall i\in
I,$ such that $\varphi_j=\varphi_{ij}\circ\varphi_i$ on
$\left.\mathscr{F}\right|_{U_i\cap U_j}.$ Moreover $(\mathscr{F},
(\varphi_i)_{i\in I})$ is unique up to an isomorphism.
\end{prop}
\begin{proof}
$\forall V\in \mathds{I}_X,$ define
\begin{eqnarray*}
\mathscr{F}(V)=\{(s_i)_{i\in I}\!\! & \mid & \!\!\forall i\in
I,s_i\in\mathscr{F}_i(V\cap U_i)\text{ such that }\forall j\in I,
\text{ we have}                                                \\
& & \varphi_{ij}(V\cap U_i\cap U_j)(\left.s_i\right|_{V\cap U_i\cap
U_j})=\left.s_j\right|_{V\cap U_i\cap U_j}\}.
\end{eqnarray*}
We now show that $\mathscr{F}$ is a sheaf. Let
$\widetilde{V}\subseteq V$ in $\mathds{I}_X,$ put
\[ \xymatrix@R=0em{
   {\rho_{V\,\widetilde{V}}: \mathscr{F}(V)} \ar[r] &
   \mathscr{F}(\widetilde{V})                           \\
   (s_i)_{i\in I} \ar@{|->}[r] & (\left.s_i\right|_{\widetilde{V}\cap U_i})_{i\in I}
   }  \]
It's easily checked that $(\left.s_i\right|_{\widetilde{V}\cap
U_i})_{i\in I}\in \mathscr{F}(\widetilde{V}),$ hence $\mathscr{F}$
is a presheaf in the obvious way. Let $(V_k)_{k\in K}$ be an open
covering of $V.$ For each $k\in K,$ let $(s_i^{(k)})_{i\in I}\in
\mathscr{F}(V_k)$ such that $\forall k,l\in K,$
$(\left.s_i^{(k)}\right|_{V_k\cap V_l\cap U_i})_{i\in
I}=(\left.s_i^{(l)}\right|_{V_k\cap V_l\cap U_i})_{i\in I}.$ Then
for each $i\in I,$ $\left.s_i^{(k)}\right|_{V_k\cap V_l\cap
U_i}=\left.s_i^{(l)}\right|_{V_k\cap V_l\cap U_i}.$ Since each
$\mathscr{F}_i$ is a sheaf, hence $\exists
!s_i\in\mathscr{F}_i(V\cap U_i)$ such that $\forall k\in K,$
$\left.s_i\right|_{V_k\cap U_i}=s_i^{(k)}.$ Now we need only to show
that the unique $(s_i)_{i\in I}\in \mathscr{F}(V).$ $\forall j\in
I,k\in K$ we have
\begin{eqnarray*}
& & \left.\varphi_{ij}(V\cap U_i\cap U_j)(\left.s_i\right|_{V\cap
U_i\cap U_j})\right|_{V_k\cap U_i\cap U_j}                      \\
& = & \varphi_{ij}(V_k\cap U_i\cap U_j)(\left.s_i\right|_{V_k\cap U_i\cap U_j})\\
& = & \varphi_{ij}(V_k\cap U_i\cap
U_j)(\left.s_i^{(k)}\right|_{V_k\cap
U_i\cap U_j})                                                   \\
& = & \left.s_j^{(k)}\right|_{V_k\cap U_i\cap U_j}              \\
& = & \left.s_j\right|_{V_k\cap U_i\cap U_j}                    \\
& = & \left.(\left.s_j\right|_{V\cap U_i\cap U_j})\right|_{V_k\cap
U_i\cap U_j}.
\end{eqnarray*}
Since $(V_k\cap U_i\cap U_j)_{k\in K}$ forms an open covering of
$V\cap U_i\cap U_j,$ hence we obtain that
$$\varphi_{ij}(V\cap U_i\cap U_j)(\left.s_i\right|_{V\cap U_i\cap
U_j})=\left.s_j\right|_{V\cap U_i\cap U_j}.$$ Thus $(s_i)_{i\in
I}\in \mathscr{F}(V).$ Therefore $\mathscr{F}$ is a sheaf.

$\mathit{1^{\circ}}$ $\forall j\in I,\forall V\in\mathds{I}_{U_j},$
define
\[ \xymatrix@R=0em{
   {\varphi_j(V):\mathscr{F}(V)} \ar[r] & {\mathscr{F}_j(V)} \\
   {(s_i)_{i\in I}} \ar@{|->}[r] & s_j }  \]
We shall show that $\varphi_j(V)$ is an isomorphism. Define
\[ \xymatrix@R=0em{
   {\psi_j(V):\mathscr{F}_j(V)} \ar[r] & {\mathscr{F}(V)} \\
   s_j \ar@{|->}[r] &
   {(\varphi_{ji}(V\cap U_i)(\left.s_j\right|_{V \cap U_i}))_{i\in I}} }
\]
We have
\begin{eqnarray*}
& & \varphi_j(V)\circ\psi_j(V)(s_j) \\
& = & \varphi_j(V)((\varphi_{ji}(V\cap U_i)(\left.s_j\right|_{V \cap
U_i}))_{i\in
I})                                                             \\
& = & \varphi_{jj}(V\cap U_j)(\left.s_j\right|_{V \cap U_j})    \\
& = & id_{\mathscr{F}_j}(V)(s_j),
\end{eqnarray*}
\begin{eqnarray*}
& & \psi_j(V)\circ\varphi_j(V)((s_i)_{i\in I}) \\
& = & \psi_j(V)(s_j) =
(\varphi_{ji}(V\cap U_i)(\left.s_j\right|_{V \cap U_i}))_{i\in I} \\
& = & (\left.s_i\right|_{V \cap U_i})_{i\in I}=(s_i)_{i\in I}   \\
& = & id_{\mathscr{F}}(V)((s_i)_{i\in I}).
\end{eqnarray*}
Hence $\psi_j(V)$ is the two-sided inverse of $\varphi_j(V),$ thus
$\varphi_j(V)$ is an isomorphism.

$\mathit{2^{\circ}}$ $\varphi_j:
\left.\mathscr{F}\right|_{U_j}\rightarrow \mathscr{F}_j$ is a
morphism of sheaves on $U_j.$

$\forall V\subseteq \widetilde{V}$ in $\mathds{I}_{U_j},$ we have
the following commutative diagram:
\[ \xymatrix{
   {\mathscr{F}(\widetilde{V})} \ar[d]_-{\rho_{\widetilde{V}\,V}}
   \ar[r]^-{\varphi_j(\widetilde{V})} &
   {\mathscr{F}_j(\widetilde{V})}
   \ar[d]^-{\rho_{\widetilde{V}\,V}^{\prime}}                 \\
   {\mathscr{F}(V)} \ar[r]^-{\varphi_j(V)} & {\mathscr{F}_j(V)} }
\]
Hence $\varphi_j$ is a morphism of sheaves on $U_j,$ thus is an
isomorphism of sheaves on $U_j$ by $\mathit{1^{\circ}}.$

$\mathit{3^{\circ}}$ $\forall i,i\in I,
\varphi_j=\varphi_{ij}\circ\varphi_i$ on
$\left.\mathscr{F}\right|_{U_i\cap U_j}.$

To show this, we should show that $\forall V\in\mathds{I}_{U_i\cap
U_j}, \varphi_j(V)=\varphi_{ij}(V)\circ\varphi_i(V).$ $\forall
s\in\mathscr{F}(V),$ write $s=(s_k)_{k\in I}$ with
$s_k\in\mathscr{F}(V\cap U_k),$ then we have
\begin{eqnarray*}
& & \varphi_{ij}(V)\circ\varphi_i(V)(s)     \\
& = & \varphi_{ij}(V)(s_i) =
\varphi_{ij}(V\cap U_i\cap U_j)(\left.s_i\right|_{V\cap U_i\cap U_j})   \\
& = & \left.s_j\right|_{V\cap U_i\cap U_j} = s_j                        \\
& = & \varphi_j(V)(s).
\end{eqnarray*}

$\mathit{4^{\circ}}$ Uniqueness.

Suppose that $(\widetilde{\mathscr{F}},(\widetilde{\varphi_i})_{i\in
I})$ satisfies the same properties. i.e. $\forall i\in I,
\widetilde{\varphi}_i:
\left.\widetilde{\mathscr{F}}\right|_{U_i}\rightarrow \mathscr{F}_i$
is an isomorphism; $\forall i,j\in I,
\widetilde{\varphi}_i=\varphi_{ij}\circ\widetilde{\varphi}_j$ on
$\left.\widetilde{\mathscr{F}}\right|_{U_i\cap U_j}.$ Then $\forall
i\in I,$ we obtain the following commutative diagram:
\[ \xymatrix{
   & {\left.\mathscr{F}\right|_{U_i}} \ar[dl]_-{\varphi_i}                 \\
   {\mathscr{F}_i} \ar[dr]_-{\widetilde{\varphi}_i^{-1}}          \\
   & {\left.\widetilde{\mathscr{F}}\right|_{U_i}} \ar@{-->}[uu]_-{\psi_i} }
\]
$\psi_i: \left.\widetilde{\mathscr{F}}\right|_{U_i}\rightarrow
\left.\mathscr{F}\right|_{U_i}$ is an isomorphism of sheaves on
$U_i,$ for $\varphi_i,\widetilde{\varphi}_i$ are isomorphisms.
$\forall V\in\mathds{I}_{U_i\cap U_j},$
\begin{eqnarray*}
\psi_i(V) & = & \widetilde{\varphi}_i^{-1}(V)\circ\varphi_i(V) =
\widetilde{\varphi}_j^{-1}(V) \circ \widetilde{\varphi}_j(V) \circ
\widetilde{\varphi}_i^{-1}(V) \circ \varphi_i(V)                \\
& = & \widetilde{\varphi}_j^{-1}(V) \circ \varphi_{ij}(V) \circ
\varphi_i(V) = \widetilde{\varphi}_j^{-1} \circ \varphi_j(V)    \\
& = & \psi_j(V).
\end{eqnarray*}
Hence there exists a unique isomorphism $\psi:\mathscr{F}\rightarrow
\widetilde{\mathscr{F}}$ such that $\left.\psi\right|_{U_i}=\psi_i.$
\end{proof}
\begin{prop}
Let $(U_i)_{i\in I}$ be a family of schemes. For any $i,j\in I(i\neq
j),$ let $U_{ij}$ be an open subscheme of $U_i$ and
$\varphi_{ij}:U_{ij}\rightarrow U_{ji}$ be an isomorphism of schemes
with the following properties:
\enum
\item[(1)]$\forall i,j\in I(i\neq
j), \varphi_{ij}=\varphi_{ji}^{-1}.$
\item[(2)]For any distinct $i,j,k\in I, \varphi_{ij}(U_{ij}\cap U_{ik}) = U_{ji}\cap
U_{jk},$ and $\left.\varphi_{jk}\circ\varphi_{ij}\right|_{U_{ij}\cap
U_{ik}} = \left.\varphi_{ik}\right|_{U_{ij}\cap U_{ik}}.$
\end{list}
Then there exists a scheme $X$ together with open immersions
$\varphi_i:U_i\rightarrow X$ such that $(\varphi_i(U_i))_{i\in I}$
forms an open covering of $X,$ and $\forall i,j\in I(i\neq j),
\varphi_i(U_{ij})=\varphi_i(U_i)\cap \varphi_J(U_j),
\left.\varphi_j\circ \varphi_{ij}\right|_{U_{ij}} =
\left.\varphi_i\right|_{U_{ij}}.$ Moreover $(X,(\varphi_i)_{i\in
I})$ is unique up to an isomorphism of schemes.
\end{prop}
\begin{proof}
$\forall i\in I,$ define $U_{ii}=U_i, \varphi_{ii}=id_{U_i},
\varphi_{ii}^{\sharp}=id_{\mathcal {O}_{U_i}}.$ Then the following
properties hold:
\enum
\item[(a)]$\forall i,j\in I, \varphi_{ij}=\varphi_{ji}^{\prime}.$
\item[(b)]$\forall i,j,k\in I, \varphi_{ij}(U_{ij}\cap U_{ik}) =
U_{ji}\cap U_{jk},
\left.\varphi_{jk}\circ\varphi_{ij}\right|_{U_{ij}\cap U_{ik}} =
\left.\varphi_{ik}\right|_{U_{ij}\cap U_{ik}},
\left.(\varphi_{jk}\circ\varphi_{ij})^{\sharp}\right|_{U_{kj}\cap
U_{ki}} = \left.\varphi_{ik}^{\sharp}\right|_{U_{kj}\cap U_{ki}}.$
\end{list}

\textbf{Topological Part}:

We can find a topological space $X$ together with open embeddings
$\varphi_i: U_i\rightarrow X(i\in I)$ such that the desired
topological properties hold for $(X,(\varphi_i)_{i\in I}),$ which is
unique up to a homeomorphism.

\textbf{Sheaf's Part}:

Definition of $\varphi_i^{\sharp}:\mathcal {O}_X\rightarrow
(\varphi_i)_{\ast}\mathcal {O}_{U_i},$ where
$(\varphi_i,\varphi_i^{\sharp})$ is a morphism of schemes:

$\forall i\in I,$ put $V_i=\varphi_i(U_i),$ which is open in $X.$
Set $\mathscr{F}_i=(\varphi_i)_{\ast}\mathcal {O}_{U_i}$ a sheaf on
$V_i.$ We shall show that we can find an $(\mathcal
{O}_X,(\varphi_i^{\sharp})_{i\in I}),$ which is unique up to
isomorphism, with $\mathcal {O}_X$ a sheaf on $X$ and
$\varphi_i^{\sharp}:\mathcal {O}_X\rightarrow
(\varphi_i)_{\ast}\mathcal {O}_{U_i},$ such that $\left.\mathcal
{O}_X\right|_{U_i}=\mathscr{F}_i.$ For this, it suffices to show

$(a)$ $\forall i,j\in I, \left.\mathscr{F}_i\right|_{V_i\cap
V_j}\cong \left.\mathscr{F}_j\right|_{V_i\cap V_j}.$ $\varphi_{ij}:
U_{ij}\rightarrow U_{ji}$ is an isomorphism of schemes, then
$\varphi_{ij}^{\sharp}:\mathcal {O}_{U_{ji}}\rightarrow
(\varphi_{ij})_{\ast}\mathcal {O}_{U_{ij}}$ is an isomorphism of
sheaves. Put
$$\psi_{ji}^{\sharp}=(\varphi_j)_{\ast}(\varphi_{ij}^{\sharp}):
(\varphi_j)_{\ast}\mathcal {O}_{U_{ji}}\rightarrow
(\varphi_j)_{\ast}(\varphi_{ij})_{\ast}\mathcal {O}_{U_{ij}} =
(\varphi_i)_{\ast}\mathcal {O}_{U_{ij}}.$$ $V_i\cap
V_j=\varphi_i(U_i)\cap
\varphi_j(U_j)=\varphi_i(U_{ij})=\varphi_j(U_{ji}).$ For any
nonempty open subset $V$ of $V_i\cap V_j,$ we have
\begin{eqnarray*}
(\varphi_j)_{\ast}\mathcal {O}_{U_{ji}}(V) & = & \mathcal
{O}_{U_{ji}}(\varphi_j^{-1}(V)) = \mathcal
{O}_{U_j}(\varphi_j^{-1}(V)\cap U_{ji})                 \\
& = & \mathcal {O}_{U_j}(\varphi_j^{-1}(V)) =
(\varphi_j)_{\ast}\mathcal {O}_{U_j}(V)                 \\
& = & \mathscr{F}_j(V),
\end{eqnarray*}
thus $(\varphi_j)_{\ast}\mathcal {O}_{U_{ji}} =
\left.\mathscr{F}_j\right|_{V_i\cap V_j},$
$(\varphi_i)_{\ast}\mathcal {O}_{U_{ij}} =
\left.\mathscr{F}_i\right|_{V_i\cap V_j}.$ Hence
$\psi_{ji}^{\sharp}$ is an isomorphism from
$\left.\mathscr{F}_j\right|_{V_i\cap V_j}$ to
$\left.\mathscr{F}_i\right|_{V_i\cap V_j}.$

To conclude the existence of $(\mathcal
{O}_X,(\varphi_i^{\sharp})_{i\in I}),$ it remains to show:

$(b)$ $\forall i,j,k\in
I,\psi_{ik}^{\sharp}=\psi_{jk}^{\sharp}\circ\psi_{ij}^{\sharp}.$
Indeed,
$$\varphi_{ki}^{\sharp}=(\varphi_{ji}\circ\varphi_{kj})^{\sharp}=
((\varphi_{ji})_{\ast}(\varphi_{kj}^{\sharp}))\circ\varphi_{ji}^{\sharp},$$
then we have
\begin{eqnarray*}
& & \psi_{jk}^{\sharp}\circ\psi_{ij}^{\sharp} \\
& = & [(\varphi_j)_{\ast}(\varphi_{kj}^{\sharp})] \circ
[(\varphi_i)_{\ast}(\varphi_{ji}^{\sharp})] =
[(\varphi_i\circ\varphi_{ji})_{\ast}(\varphi_{kj}^{\sharp})] \circ
[(\varphi_i)_{\ast}(\varphi_{ji}^{\sharp})]                       \\
& = & [(\varphi_i)_{\ast}\circ
(\varphi_{ji})_{\ast}(\varphi_{kj}^{\sharp})] \circ
[(\varphi_i)_{\ast}(\varphi_{ji}^{\sharp})] =
(\varphi_i)_{\ast}[((\varphi_{ji})_{\ast}(\varphi_{kj}^{\sharp}))\circ\varphi_{ji}^{\sharp}]\\
& = & (\varphi_i)_{\ast}(\varphi_{ji}\circ\varphi_{kj})^{\sharp} =
(\varphi_i)_{\ast}(\varphi_{ki}^{\sharp})                 \\
& = & \psi_{ik}^{\sharp}.
\end{eqnarray*}

Hence there exists a sheaf $\mathcal {O}_X$ together with
isomorphisms $$\varphi_i^{\sharp}: \mathcal {O}_{V_i} =
\left.\mathcal {O}_X\right|_{V_i}\rightarrow\mathscr{F}_i =
(\varphi_i)_{\ast}\mathcal {O}_{U_i}$$ such that
$\varphi_i^{\sharp}=\psi_{ij}^{\sharp}\circ\varphi_i^{\sharp}$ on
$\left.\mathcal {O}_X\right|_{U_i\cap U_j},$ which are unique up to
isomorphism. Therefore $(\varphi_i,\varphi_i^{\sharp}):
U_i\rightarrow X$ is an open immersion $\forall i\in I,$ and satisfy
the desired properties.
\end{proof}
\begin{prop}
Let $X,Y$ be two schemes. Let $(U_i)_{i\in I}$ be an open covering
of $X.$ $\forall i\in I,$ let $f_i: U_i\rightarrow Y$ be a morphism
of schemes such that $\forall i,j\in I, \left.f_i\right|_{U_i\cap
U_j}=\left.f_j\right|_{U_i\cap U_j}.$ Then there exists a unique
morphism of schemes $f:X\rightarrow Y$ such that
$\left.f\right|_{U_i}=f_i.$
\end{prop}
\begin{proof}
$\mathit{1^{\circ}}$ There exists a unique continuous mapping $f:
X\rightarrow Y$ such that $\left.f\right|_{U_i}=f_i,\forall i\in I:$

$\forall x\in X,\exists i\in I$ such that $x\in U_i,$ put
$f(x)=f_i(x),$ then we can easily check that $f$ is well-defined and
continuous.

$\mathit{2^{\circ}}$ There exists a unique $f^{\sharp}:\mathcal
{O}_Y\rightarrow f_{\ast}\mathcal {O}_X$ such that
$(\left.f\right|_{U_i})^{\sharp}=f_i^{\sharp}:$

Define $\varphi_i:U_i\hookrightarrow X,$ and define
$\varphi_i^{\sharp}:\mathcal {O}_X\rightarrow
(\varphi_i)_{\ast}\mathcal {O}_{U_i}$ by: $\forall
U\in\mathds{I}_X,$
$$\varphi_i^{\sharp}=\rho_{U\,U\cap U_i}:\mathcal {O}_X(U)\rightarrow
(\varphi_i)_{\ast}\mathcal {O}_{U_i}(U)=\mathcal {O}_{U_i}(U\cap
U_i)=\mathcal {O}_{X}(U\cap U_i).$$ We want to show that
$(f\circ\varphi_i)^{\sharp}=f_i^{\sharp},$ i.e.
$(f_{\ast}(\varphi_i^{\sharp}))\circ f^{\sharp}=f_i^{\sharp}.$ But
\begin{eqnarray*}
& & (f_{\ast}(\varphi_i^{\sharp}))\circ f^{\sharp}=f_i^{\sharp}   \\
& \Longleftrightarrow & \forall V\in\mathds{I}_Y,
f_{\ast}(\varphi_i^{\sharp})(V)\circ f^{\sharp}(V)=f_i^{\sharp}(V)\\
& \Longleftrightarrow & \varphi_i^{\sharp}(f^{-1}(V))\circ
f^{\sharp}(V)=f_i^{\sharp}(V)                                     \\
& \Longleftrightarrow & \forall s\in\mathcal {O}_Y(V),
(\varphi_i^{\sharp}(f^{-1}(V))\circ
f^{\sharp}(V))(s)=f_i^{\sharp}(V)(s)                              \\
& \Longleftrightarrow & \forall s\in\mathcal {O}_Y(V),
\left.f^{\sharp}(V)(s)\right|_{f^{-1}(V)\cap U_i}=f_i^{\sharp}(V)(s)
\end{eqnarray*}
So it suffices to show
$\left.f_i^{\sharp}(V)(s)\right|_{f^{-1}(V)\cap U_i\cap U_j} =
\left.f_j^{\sharp}(V)(s)\right|_{f^{-1}(V)\cap U_i\cap U_j}.$ Define
$$\varphi_{ij}: U_i\cap U_j\hookrightarrow U_j,\quad \varphi_{ji}:U_i\cap U_j\hookrightarrow
U_i,$$ since $(\left.f_i\right|_{U_i\cap U_j})^{\sharp} =
(\left.f_j\right|_{U_i\cap U_j})^{\sharp},$ then we obtain
$(f_i\circ\varphi_{ji})^{\sharp}=(f_j\circ\varphi_{ij})^{\sharp},$
i.e. $(f_i)_{\ast}(\varphi_{ji}^{\sharp})\circ f_i^{\sharp}
=(f_j)_{\ast}(\varphi_{ij}^{\sharp})\circ f_j^{\sharp}.$ Thus
$\forall V\in\mathds{I}_Y,\forall s\in\mathcal {O}_Y(V),$ we have
$$((f_i)_{\ast}(\varphi_{ji}^{\sharp})(V)\circ f_i^{\sharp}(V))(s)
=((f_j)_{\ast}(\varphi_{ij}^{\sharp})(V)\circ f_j^{\sharp}(V))(s),$$
then
$$((\varphi_{ji}^{\sharp})(f_i^{-1}(V))\circ f_i^{\sharp}(V))(s)
=((\varphi_{ij}^{\sharp})(f_j^{-1}(V))\circ f_j^{\sharp}(V))(s),$$
i.e.
$$((\varphi_{ji}^{\sharp})(f^{-1}(V)\cap U_i)\circ f_i^{\sharp}(V))(s)
=((\varphi_{ij}^{\sharp})(f^{-1}(V)\cap U_j)\circ
f_j^{\sharp}(V))(s),$$ hence
$$\left.f_i^{\sharp}(V)(s)\right|_{f^{-1}(V)\cap U_i\cap U_i\cap U_j}=
\left.f_j^{\sharp}(V)(s)\right|_{f^{-1}(V)\cap U_j\cap U_i\cap
U_j},$$ i.e.
$$\left.f_i^{\sharp}(V)(s)\right|_{f^{-1}(V)\cap U_i\cap U_j}=
\left.f_j^{\sharp}(V)(s)\right|_{f^{-1}(V)\cap U_i\cap U_j}.$$
\end{proof}
\begin{prop}
Let $X$ be a scheme and $A$ a ring. Then
\[ \xymatrix@C=5em{
   {Hom_{Scheme}(X,SpecA)} \ar@{<->}[r]^-{1-1} & {Hom_{Ring}(A,\mathcal {O}_X(X))}
}  \]
\end{prop}
\begin{remark}
Let $A,B$ be two rings, we have
\[ \xymatrix@R=0em{
   Hom(A,B) \ar[r]^-{\sim} & Hom(SpecB,SpecA) \\
   \varphi \ar@{|->}[r] & Spec\varphi }  \]
\end{remark}
\begin{proof}
$\mathit{1^{\circ}}$ Define
\[ \xymatrix@R=0em{
   {\alpha:Hom(X,SpecA)} \ar[r] & {Hom(A,\mathcal {O}_X(X))}  \\
   f \ar@{|->}[r] & {\alpha(f)=f^{\sharp}(SpecA)} }  \]
where $f^{\sharp}(SpecA):A\cong \mathcal
{O}_{SpecA}(SpecA)\rightarrow f_{\ast}\mathcal {O}_X(SpecA)=\mathcal
{O}_X(X).$

$\mathit{2^{\circ}}$ Define
\[ \xymatrix@R=0em{
   {\beta:Hom(A,\mathcal {O}_X(X))} \ar[r] & Hom(X,SpecA)  \\
   \varphi \ar@{|->}[r] & {\beta(\varphi)} }  \]
by:

Cover $X$ by affine open subschemes $U_i=SpecA_i(i\in I).$ Let
$\psi_i: U_i\hookrightarrow X$ be the canonical embedding. The
composite
\[ \xymatrix@C=4em{
   A \ar[r]^-{\varphi} & \mathcal {O}_X(X)
   \ar[r]^-{\psi_i^{\sharp}(X)} & {(\psi_i)_{\ast}\mathcal {O}_{U_i}(X) = \mathcal
   {O}_{U_i}(U_i)} }  \]
induces a morphism of affine schemes
\[ \xymatrix@R=0em{
   {f_i:U_i=SpecA_i} \ar[r] & SpecA           \\
   q \ar@{|->}[r] & {(\psi_i^{\sharp}(X)\circ\varphi)^{-1}(q)} } \]
where $\psi_i^{\sharp}(X)=\rho_{X\,U_i}.$

Now we check that $\forall i,j\in I, \left.f_i\right|_{U_i\cap
U_j}=\left.f_j\right|_{U_i\cap U_j}:$ Cover $U_i\cap U_j$ by affine
open subschemes $U_{ijk}=SpecA_{ijk}(k\in K_{ij}).$ Then
$$\left.f_i\right|_{U_{ijk}}: SpecA_{ijk}\rightarrow SpecA,\quad
\left.f_j\right|_{U_{ijk}}: SpecA_{ijk}\rightarrow SpecA$$
are induced by the following compositions:
\[ \xymatrix@C=4em{
   A \ar[r]^-{\varphi} & \mathcal {O}_X(X)
   \ar[r]_-{\psi_i^{\sharp}(X)}^-{\rho_{X\,U_i}} & \mathcal {O}_{U_i}(U_i)
   \ar[r]^-{\rho_{U_i\,U_{ijk}}} & \mathcal {O}_{U_{ijk}}(U_{ijk}), }
\]
and
\[ \xymatrix@C=4em{
   A \ar[r]^-{\varphi} & \mathcal {O}_X(X)
   \ar[r]_-{\psi_j^{\sharp}(X)}^-{\rho_{X\,U_j}} & \mathcal {O}_{U_j}(U_j)
   \ar[r]^-{\rho_{U_j\,U_{ijk}}} & \mathcal {O}_{U_{ijk}}(U_{ijk}). }
\]
\begin{eqnarray*}
& & \left.\psi_i^{\sharp}(X)\circ\varphi\right|_{U_{ijk}} \\
& = & \rho_{U_i\,U_{ijk}}\circ\rho_{X\,U_i}\circ\varphi =
\rho_{X\,U_{ijk}}\circ\varphi                           \\
& = & \rho_{U_j\,U_{ijk}}\circ\rho_{X\,U_j}\circ\varphi \\
& = & \left.\psi_j^{\sharp}(X)\circ\varphi\right|_{U_{ijk}},
\end{eqnarray*}
hence $\forall k\in K_{ij},
\left.f_i\right|_{U_{ijk}}=\left.f_j\right|_{U_{ijk}}.$ Then we
obtain that $\left.f_i\right|_{U_i\cap
U_j}=\left.f_j\right|_{U_i\cap U_j}.$ Therefore there exists a
unique $f: X\rightarrow SpecA$ a morphism of schemes such that
$\forall i\in I, \left.f\right|_{U_i}=f_i.$ We put
$\beta(\varphi)=f.$

$\mathit{3^{\circ}}$ $\beta\circ\alpha=id_{Hom(X,SpecA)},$
$\alpha\circ\beta=id_{Hom(A,\mathcal {O}_X(X))}:$

\enum
\item[(a)]$\forall f\in Hom(X,SpecA),
\beta\circ\alpha(f)=\beta(f^{\sharp}(SpecA)).$ Cover $X$ by affine
open subschemes $U_i=SpecA_i.$
\[ \xymatrix@C=4em{
   A \ar[r]^-{f^{\sharp}(SpecA)} & \mathcal {O}_X(X)
   \ar[r]_-{\psi_i^{\sharp}(X)}^-{\rho_{X\,U_i}} & {\mathcal {O}_X(U_i)=\mathcal
   {O}_{U_i}(U_i)} }  \]
induces a morphism of affine schemes
\[ \xymatrix@R=0em@C=3em{
   {U_i=SpecA_i} \ar[r]^-{\left.\beta(f^{\sharp}(SpecA))\right|_{U_i}} & SpecA\\
   q \ar@{|->}[r] & {(\psi_i^{\sharp}(X)\circ f^{\sharp}(SpecA))^{-1}(q)}
   } \]
But $(\left.f\right|_{U_i})^{\sharp} = (f\circ\psi_i)^{\sharp} =
f_{\ast}(\psi_i^{\sharp})\circ f^{\sharp},$ then
\begin{eqnarray*}
(\left.f\right|_{U_i})^{\sharp}(SpecA) & = &
f_{\ast}(\psi_i^{\sharp})(SpecA)
\circ f^{\sharp}(SpecA)                                           \\
& = & \psi_i^{\sharp}(f^{-1}(SpecA)) \circ f^{\sharp}(SpecA)      \\
& = & \psi_i^{\sharp}(X) \circ f^{\sharp}(SpecA)
\end{eqnarray*}
$(\left.f\right|_{U_i})^{\sharp}(SpecA)$ induces
\[ \xymatrix@R=0em@C=3em{
   {\left.f\right|_{U_i}: U_i=SpecA_i} \ar[r] & SpecA\\
   q \ar@{|->}[r] & {(\psi_i^{\sharp}(X)\circ f^{\sharp}(SpecA))^{-1}(q)}
   } \]
Then we know that
$\left.f\right|_{U_i}=\left.\beta(f^{\sharp}(SpecA))\right|_{U_i},$
thus
$$f=\beta(f^{\sharp}(SpecA))=\beta\circ\alpha(f)=id_{Hom(X,SpecA)}(f).$$
\item[(b)]$\forall \varphi\in Hom(A,\mathcal {O}_X(X)),$ set
$f=\beta(\varphi).$ Then we have
\[ \xymatrix{
   A \ar[r] & \mathcal {O}_{SpecA}(SpecA) \ar[r]^-{f^{\sharp}(SpecA)} &
   \mathcal {O}_X(X) \ar[r]_-{\psi_i^{\sharp}(X)}^-{\rho_{X\,U_i}} &
   {\mathcal {O}_X(U_i)=\mathcal {O}_{U_i}(U_i)=A_i} }  \]
which is defined by:
\begin{eqnarray*}
& & a      \\
& \mapsto & (s:p\mapsto a\in A_p)                   \\
& \mapsto & f^{\sharp}(SpecA)(s)                      \\
& \mapsto & \rho_{X\,U_i}\circ f^{\sharp}(SpecA)(s) =
\psi_i^{\sharp}(X)\circ f^{\sharp}(SpecA)(s)          \\
& & =(f\circ\psi_i)^{\sharp}(SpecA)(s) =
(\left.f\right|_{U_i})^{\sharp}(SpecA)(s).
\end{eqnarray*}
But $(\left.f\right|_{U_i})^{\sharp}(SpecA) = (f_i)^{\sharp}(SpecA)
= \psi_i^{\sharp}(X)\circ\varphi,$ then
$$(\left.f\right|_{U_i})^{\sharp}(SpecA)(s) = (\psi_i^{\sharp}(X)\circ\varphi)(s) =
\psi_i^{\sharp}(X)(\varphi(a))=\left.\varphi(a)\right|_{U_i},$$ i.e.
we have $\left.f^{\sharp}(SpecA)(a)\right|_{U_i} =
\left.\varphi(a)\right|_{U_i}.$ Hence
$f^{\sharp}(SpecA)(a)=\varphi(a),$ then we obtain
$f^{\sharp}(SpecA)=\varphi,$ which is equivalent to saying that
$\alpha(f)=\varphi.$ Therefore
$$\varphi=\alpha(f)=\alpha\circ\beta(\varphi)=id_{Hom(A,\mathcal {O}_X(X))}(\varphi).$$
\end{list}
\end{proof}

\newpage

\section{Graded rings and related schemes}

\begin{Def}
Let $S$ be a ring with a decomposition
$S=\bigoplus\limits_{d\geqslant 0} S_d,$ a direct sum of Abelian
groups $S_d(d\geqslant 0),$ such that $S_d\cdot S_e\subseteq
S_{d+e}, \forall d,e\geqslant 0.$ Then we call $S$ a graded ring.
Every element in $S_d$ is said to be homogeneous of degree $d.$
\end{Def}
\begin{remarks}\
\enum
\item[(1)]The identity element $1\in S_0,$ and $S_0$ is a ring.
\item[(2)]An ideal $I$ of $S$ is said to be homogeneous if
$I=\bigoplus\limits_{d\geqslant 0}I\cap S_d$
\end{list}
\end{remarks}
\begin{proof}
$(1)$ Suppose $1=\sum\limits_{d\geqslant 0}a_d$ with $a_d\in S_d,$
then for any homogeneous element $a,$ we have
$$a=a\cdot 1=\sum\limits_{d\geqslant 0}aa_d,$$
hence $\forall d\geqslant 1, aa_d=0,$ i.e. $a=aa_0.$ Then for any
element $a\in S, a=aa_0.$ In particular, $1=1\cdot a_0,$
i.e.$a_0=1.$ Hence $1\in S_0.$
\end{proof}
\begin{prop}
Let $S$ be a graded ring with the gradings $(S_d)_{d\geqslant 0}.$
Let $I$ be an ideal of $S.$ Then the following are equivalent \enum
\item[(1)]$I$ is a homogeneous ideal of $S.$
\item[(2)]$\forall a\in I$ with $a=\sum\limits_{d\geqslant 0}a_d(a_d\in
S_d),$ then $s_d\in I(\forall d\geqslant 0).$
\item[(3)]$I$ can be generated by homogeneous elements as an
additive subgroup of $S$
\item[(4)]$I$ can be generated by homogeneous elements as an ideal
of $S.$
\end{list}
\end{prop}
\begin{proof}\
\enum
\item[$(1)\Longrightarrow(2)$] If $a=\sum\limits_{d\geqslant
0}a_d(a_d\in S_d),$ then $\forall d\geqslant 0, a_d\in I\cap S_d,$
hence $a_d\in I.$
\item[$(2)\Longrightarrow(3)$] It's immediate.
\item[$(3)\Longrightarrow(4)$] Let $I$ be generated by
$\{b_i\}_{i\in\Lambda},$ where $b_i$ are homogeneous elements, as an
additive subgroup of $S.$ Necessarily
$<\{b_i\}_{i\in\Lambda}>\subseteq I.$ $\forall a\in I,$ we can write
$a=\sum\limits_{j=1}^nb_{i_j},$ where $\{b_{i_j}\}_{1\leqslant
j\leqslant n}\subseteq\{b_i\}_{i\in\Lambda},$ thus
$<\{b_i\}_{i\in\Lambda}>\supseteq I.$ Hence
$<\{b_i\}_{i\in\Lambda}>= I.$ Therefore $I$ can be generated by the
homogeneous elements $\{b_i\}_{i\in\Lambda}$ as an ideal of $S.$
\item[$(4)\Longrightarrow(2)$] Let $I=<\{b_i\}_{i\in\Lambda}>,$
where $b_i$ are homogeneous elements. $\forall a\in I,$ we can write
$a=\sum a_{(i_1\cdots i_n)}^{(p_{i_1}\cdots
p_{i_n})}b_{i_1}^{p_{i_1}}\cdots b_{i_n}^{p_{i_n}},$ where
$a_{(i_1\cdots i_n)}^{(p_{i_1}\cdots p_{i_n})}\in S.$ Since
$\{b_i\}_{i\in\Lambda}$ are homogeneous, hence
$b_{i_1}^{p_{i_1}}\cdots b_{i_n}^{p_{i_n}}$ are homogeneous. Express
$a_{(i_1\cdots i_n)}^{(p_{i_1}\cdots p_{i_n})}$ by homogeneous
elements, then we can see that $a=\sum\limits_{d\in J} a_d,$ where
$J$ is a finite subset of $\mathbb{N}.$ Since $\forall i\in\Lambda,
b_i\in I,$ therefore $b_{i_1}^{p_{i_1}}\cdots b_{i_n}^{p_{i_n}}\in
I,$ hence $\forall d\in J, a_d\in I.$
\item[$(2)\Longrightarrow(1)$] $\forall a\in I,$ we can write
$a=\sum\limits_{d\geqslant 0}a_d,$ where $a_d\in I\cap S_d.$ The
expression is unique since $s=\bigoplus\limits_{d\geqslant 0}S_d.$
Hence $I=\bigoplus\limits_{d\geqslant 0}I\cap S_d,$ then $I$ is a
homogeneous ideal of $S.$
\end{list}
\end{proof}
\begin{cor}
Let $I$ be a homogeneous ideal of $S.$ Then$S/I$ is a graded ring
and the canonical ring homomorphism $S\rightarrow S/I$ preserves the
gradings.
\end{cor}
\begin{egs}
Let $k$ be a field. \enum
\item[(1)]$S=k[x], \forall d\geqslant 0,$ set $S_d=kx^d.$ Then
$S=\bigoplus\limits_{d\geqslant 0}S_d,$ $\forall d,e\geqslant 0,$ we
have $S_d\cdot S_e=S_{d+e}.$ So $S$ is a graded ring.
\item[(2)]$S=k[x_1,\cdots,x_n], \forall d\geqslant 0,$ set
$$S_d=\{f\in S\mid f \text{ is homogeneous of degree d }\}\cup\{0\}$$ Then
$S=\bigoplus\limits_{d\geqslant 0}S_d,$ and $\forall d,e\geqslant
0,$ $S_d\cdot S_e=S_{d+e}.$
\end{list}
\end{egs}
\begin{prop}
Let $S$ be a graded ring and $\mathfrak{p}$ be a homogeneous ideal
of $S.$ Then $\mathfrak{p}$ is prime iff $\forall f,g\in S$ which
are homogeneous elements such that $fg\in\mathfrak{p},$ we have
$f\in\mathfrak{p}$ or $g\in\mathfrak{p}.$
\end{prop}
\begin{proof}
$\Longrightarrow:$ is direct.

$\Longleftarrow:$ Suppose that $\exists f,g\in S$ such that
$fg\in\mathfrak{p}$ and $f\not\in\mathfrak{p},
g\not\in\mathfrak{p}.$ Write $f=\sum\limits_{i\geqslant 0}f_i,
g=\sum\limits_{i\geqslant 0}g_i(f_i, g_i\in S_i).$ Let $m\geqslant
0$ be the smallest integer such that $f_m\not\in\mathfrak{p},$
$n\geqslant 0$ be the smallest integer such that
$g_n\not\in\mathfrak{p}.$ Then
$$fg=\sum\limits_{k=0}^{\infty}(\sum\limits_{i=0}^kf_ig_{k-i})\in\mathfrak{p}$$
But $\mathfrak{p}$ is homogeneous, hence
$\sum\limits_{i=0}^{m+n}f_ig_{m+n-i}\in\mathfrak{p}.$ However
$$\sum\limits_{i=0}^{m+n}f_ig_{m+n-i}=\sum\limits_{i=0}^{m-1}f_ig_{m+n-i}
+f_mg_n+\sum\limits_{i=0}^{n-1}f_{m+n-i}g_i\not\in\mathfrak{p},$$
absurd.
\end{proof}
\begin{remark}
In a graded ring, sums, products, intersections and nilpotent
radicals of homogeneous ideals are homogeneous.
\end{remark}
\begin{prop}
Let $S$ be a graded ring, put $S_+=\bigoplus\limits_{d\geqslant
1}S_d,$
$$ProjS=\{\mathfrak{p}\mid S_+\nsubseteq\mathfrak{p},\,\mathfrak{p}
\text{ is a homogeneous prime ideal of } S\}$$

For any homogeneous ideal $I$ if $S,$ define
$V_+(I)=\{\mathfrak{p}\in ProjS\mid I\subseteq\mathfrak{p}\}.$ Then
the following hold: \enum
\item[(1)]$V_+(0)=ProjS, V_+(S)=\emptyset, V_+(S_+)=\emptyset$
\item[(2)]For any family of homogeneous ideals $I_i$ of $S,$ $\bigcap\limits_{i\in
\Lambda}V_+(I_i)=V_+(\sum\limits_{i\in\Lambda}I_i)$
\item[(3)]$V_+(I)\cup V_+(J)=V_+(IJ)=V_+(I\cap J)$ for any
homogeneous ideals $I, J$ of $S.$
\end{list}
\end{prop}
\begin{proof}
For any homogeneous ideal $I$ of $S,$ we have $V_+(I)=V_+(I)\cap
ProjS,$ hence $(1),(2),(3)$ hold.
\end{proof}
\begin{remark}
Put $$\tau(ProjS)=\{V_+(I)\mid I \ \text{is a homogeneous ideal of}\
S\}$$ Then $\emptyset, ProjS\in \tau(ProjS).$ Moreover $\tau(ProjS)$
is closed under intersection and finite union, hence induces a
topology on $ProjS,$ called the Zariski topology.
\end{remark}
\begin{prop}
Let $S$ be a graded ring and $T$ a multiplicatively closed subset of
$S.$ Put $$\widetilde{T^{-1}S}=\{\frac{a}{t}\in T^{-1}S\mid a\in S,
t\in T, \text{and } a,t \text{ are homogeneous of the same degree}
\}$$ Then $\widetilde{T^{-1}S}$ is a subring of $T^{-1}S.$
\end{prop}
\begin{proof}
It's easily checked that $\widetilde{T^{-1}S}$ is closed under
addition and multiplication, and
$\frac{1}{1}\in\widetilde{T^{-1}S},$ hence $\widetilde{T^{-1}S}$ is
a subring of $T^{-1}S.$
\end{proof}
\begin{remarks}\
\enum
\item[(1)]Let $\mathfrak{p}$ be a homogeneous ideal of $S.$ Set
$T=S\setminus\mathfrak{p}.$ Then $T$ is multiplicatively closed. In
this case, we denote $\widetilde{T^{-1}S}$ by $S_{(\mathfrak{p})}.$
$S_{(\mathfrak{p})}$ is a local ring.
\item[(2)]Let $f\in S$ be homogeneous, put $T=\{f^n\mid n\geqslant
0\}.$ Then $T$ is multiplicatively closed, and we denote
$\widetilde{T^{-1}S}$ by $S_{(f)}.$
\end{list}
\end{remarks}
\begin{proof}
$(1)$ Set
$$\mathfrak{M}=\{\frac{a}{t}\mid a\in\mathfrak{p},t\in T, a,t \text{ are homogeneous of the same degree }\}.$$
It's easily checked that $\mathfrak{M}$ is an ideal of
$S_{(\mathfrak{p})}.$ If $\frac{a}{t}\in
S_{(\mathfrak{p})}\setminus\mathfrak{M},$ then $a\not\in
S_{(\mathfrak{p})},$ thus $a\in T,$ then $\frac{a}{t}$ is a unit in
$S_{(\mathfrak{p})}.$ Hence $\mathfrak{M}$ is the only maximal ideal
of $S_{(\mathfrak{p})}.$ Therefore $S_{(\mathfrak{p})}$ is a local
ring.
\end{proof}
\begin{prop}
Let $S$ be a graded ring. For any nonempty open subset $U$ of
$ProjS,$ let $\mathcal {O}_{ProjS}(U)$ be the set of functions $s:
U\rightarrow \coprod\limits_{p\in U} S_{(p)}$ satisfying the
following properties: \enum
\item[(1)]$\forall p\in U, s(p)\in S_{(p)}$
\item[(2)]$\forall p\in U, \exists U_p\subseteq U$ a neighborhood of
$p,$ $\exists a, f\in S$ homogeneous of the same degree, such that
$\forall q\in U_p,$ we have $f\not\in q$ and $s(q)=\frac{a}{f}$ in
$S_{(q)}$
\end{list}

For any nonempty open subset $U,V$ of $ProjS,\,V\subseteq U,$ define
\[ \xymatrix@R=0em{
   {\rho_{U\,V}:\mathcal {O}_{ProjS}(U)} \ar[r] & {\mathcal
   {O}_{ProjS}(V)}                                            \\
   s \ar@{|->}[r] & \left.s\right|_V }  \]
Then $\mathcal {O}_{ProjS}$ is a sheaf on $ProjS.$
\end{prop}
\begin{proof}
$\forall W\subseteq V\subseteq U$ in $\mathds{I}_{ProjS},$ we have
$\left.(\left.s\right|_V)\right|_W=\left.s\right|_W,$ where $s$ is
an arbitrary function in $\mathcal {O}_{ProjS}(U).$ Hence we have
$\rho_{U\,W}=\rho_{V\,W}\circ\rho_{U\,V}.$ $\left.s\right|_U=s,$
then $\rho_{U\,U}=id_{\mathcal {O}_{ProjS}(U)}.$ Therefore $\mathcal
{O}_{ProjS}$ is a presheaf on $ProjS.$ $\forall
U\in\mathds{I}_{ProjS},$ let $(U_i)_{i\in I}$ be an open covering of
$U,$ and for each $i$ let $s_i\in \mathcal {O}_{ProjS}(U_i)$ be such
that $\forall j\in I, \left.s_i\right|_{U_i\cap
U_j}=\left.s_j\right|_{U_i\cap U_j}.$ Since $s_i$ are functions and
$\forall i,j\in I, \left.s_i\right|_{U_i\cap
U_j}=\left.s_j\right|_{U_i\cap U_j},$ then we can glue them together
to get a unique $s\in\mathcal {O}_{ProjS}(U)$ such that $\forall
i\in I, \left.s\right|_{U_i}=s_i.$ Therefore $\mathcal {O}_{ProjS}$
is a sheaf on $ProjS.$
\end{proof}
\begin{prop}
Let $S$ be a graded ring.
\enum
\item[(1)]$\forall p\in ProjS,$ we have a canonical isomorphism
$$\rho: \mathcal {O}_{ProjS,p}\stackrel{\sim}{\rightarrow} S_{(p)}$$
\item[(2)]For any homogeneous elements $f\in S_+$ of positive
degree, define $$D_+(f)=ProjS\setminus V_+(f)=\{p\in ProjS\mid
f\not\in p\}$$ Then $D_+(f)$ is open in $ProjS.$
\item[(3)]$\{D_+(f)\mid f\in S_+ \text{ is homogeneous of positive degree
}\}$ form a basis for the Zariski topology on $ProjS.$
\item[(4)]For any homogeneous element $f\in S_+$ of positive degree,
we have a canonical isomorphism of locally ringed space:
$$(D_+(f), \left.\mathcal {O}_{ProjS}\right|_{D_+(f)})\stackrel{\sim}{=}(SpecS_{(f)},\mathcal {O}_{SpecS_{(f)}}).$$
In particular, $(ProjS,\mathcal {O}_{ProjS})$ is a scheme.
\end{list}
\end{prop}
\begin{proof}\
\enum
\item[(1)]$\forall x\in \mathcal {O}_{ProjS,p}, \exists U_p$ a
neighborhood of $p$ and $s\in\mathcal {O}_{ProjS}(U_p)$ such that
$x=s_p.$ Then define $\rho(x)=s(p).$ $\rho$ is a ring homomorphism,
we shall show that $\rho$ is an isomorphism.

$\mathit{1^{\circ}}$Injectivity:

If $\rho(x)=0,$ then $s(p)=0.$ By definition, $\exists V_p\subseteq
U_p$ a neighborhood of $p,$ and $\exists a, f$ homogeneous of the
same degree such that $\forall q\in V_p,$ we have $f\not\in q,$ and
$s(q)=\frac{a}{f}$ in $S_{(q)}.$ In particular, $0=\frac{a}{f}$ in
$S_{(p)}.$ Thus $\exists t\in S\setminus p$ such that $at=0,$ and
then $\exists b\in S\setminus p$ a homogeneous element such that
$ab=0.$ Then $\forall q\in V_p\setminus V_+(b)$ a neighborhood of
$p,$ we have in $S_{(q)}, s(q)=\frac{a}{f}=\frac{ab}{fb}=0.$ Thus
$x=s_p=0$ in $\mathcal {O}_{ProjS,p}.$ Hence $\rho$ is injective.

$\mathit{2^{\circ}}$Surjectivity:

$\forall y\in S_{(p)}, \exists a\in S, f\in S\setminus p,$ $a, f$
are homogeneous of the same degree, such that $y=\frac{a}{f}.$
$\forall q\in ProjS\setminus V_+(f)$ a neighborhood of $p,$ define
$s(q)=\frac{a}{f}$ where $s\in \mathcal {O}_{ProjS}(ProjS\setminus
V_+(f)).$ In particular $s(p)=y.$ Thus $\rho(s_p)=s(p)=y.$ Therefore
$\rho$ is surjective.
\item[(2)]$D_+(f)=ProjS\setminus V_+(f),$ $V_+(f)$ is closed, hence
$D_+(f)$ is open.
\item[(3)]Take $U$ an arbitrary nonempty subset of $ProjS,$ we only
need to show that $\forall p\in U, \exists f\in S_+$ homogeneous of
positive degree such that $p\in D_+(f)\subseteq U.$

Since $U$ is open, there exists a homogeneous ideal $I$ of $S$ such
that $U=ProjS\setminus V_+(I).$ $\forall p\in U,$ we have $p\not\in
V_+(I).$ Thus $S_+\nsubseteq p$ and $I\nsubseteq p,$ thus
$IS_+\nsubseteq p,$ for $p$ is prime. Let $f\in IS_+\setminus p$ be
homogeneous. Then $f$ has positive degree, and $f\not\in p, f\in I.$
Thus $V_+(f)\supseteq V_+(I),$ and $p\not\in V_+(f).$ Therefore
$p\in D_+(f)\subseteq U=ProjS\setminus V_+(I).$
\item[(4)]Let $f\in S_+$ be homogeneous of positive degree, we shall
show $$(D_+(f),\left.\mathcal {O}_{ProjS}\right|_{D_+(f)})
\stackrel{\sim}{=}(SpecS_{(f)},\mathcal {O}_{SpecS_{(f)}}).$$ Define
\[ \xymatrix@R=0em{
   {\varphi:D_+(f)} \ar[r] & {SpecS_{(f)}} \\
   p \ar@{|->}[r] & {pS_f\cap S_{(f)}} }  \]

$\mathit{1^{\circ}}$Injectivity:

$\forall p_1,p_2\in D_+(f),$ if $\varphi(p_1)=\varphi(p_2),$ then
$p_1S_f\cap S_{(f)}=p_2S_f\cap S_{(f)}.$ $\forall x\in p_1$ be
homogeneous, we have
$$\frac{x^{degf}}{f^{degx}}\in p_1S_f\cap S_{(f)}=p_2S_f\cap
S_{(f)}.$$ Thus $\exists y\in p_2,$ and $k\geqslant 0$ such
that$\frac{x^{degf}}{f^{degx}}=\frac{y}{f^k}.$ Then $\exists
l\geqslant 0$ such that $x^{degf}\cdot f^l\in p_2.$ But $f\not\in
p_2$ and $p_2$ is prime, hence $x\in p_2$. So we obtain
$p_1\subseteq p_2.$ Similarly we have $p_2\subseteq p_1.$ Therefore
$p_1=p_2.$

$\mathit{2^{\circ}}$Surjectivity:

$\forall q\in SpecS_{(f)}$, define
$$E_q=\{a\in S\mid a \text{ is homogeneous such that }\frac{a}{f^k}\in q,\text{ for some }k\geqslant 0\}$$
Let $J_q$ be the ideal generated by $E_q$ in $S,$ set$p=\sqrt{J_q}.$
We shall show that $p\in D_+(f)$ and $\varphi(p)=q.$
\enum
\item[(a)]$p$ is prime and $p\in D_+(f):$

$\forall a_1,a_2\in S$ homogeneous such that $a_1a_2\in p.$ Then
$\exists n\geqslant 1$ such that $(a_1a_2)^n\in J_q,$ hence
$\frac{(a_1a_2)^{ndegf}}{f^{n(dega_1+dega_2)}}\in q.$ But $q$ is
prime, thus $\frac{a_1^{ndegf}}{f^{ndega_1}}\in q$ or
$\frac{a_2^{ndegf}}{f^{ndega_2}}\in q,$ then $a_1^{ndegf}\in J_q$ or
$a_2^{ndegf}\in J_q,$ hence $a_1\in p$ or $a_2\in p.$ We have
$f\not\in P,$ otherwise $\exists n\geqslant 1$ such that $f^n\in
J_q,$ then $1=\frac{f^n}{f^n}\in q,$ absurd. Thus $p\neq S.$
Therefore $p$ is prime and $p\in D_+(f).$
\item[(b)]$\varphi(p)=q:$

\enum
\item[(i)]$\varphi(p)=pS_f\cap S_{(f)}\subseteq q:$

$\forall x\in pS_f\cap S_{(f)},$ write $x=\frac{a}{f^n},$ where
$a\in p$ is homogeneous, and $dega=ndegf.$ Since $a\in p,$ then
$\exists k\geqslant 1$ such that $a^k\in J_q,$ thus
$\frac{a^{ndegf}}{f^{ndega}}\in q,$ i.e. $(\frac{a}{f^n})^{kdegf}\in
q,$ hence $x=\frac{a}{f^n}\in q.$
\item[(ii)]$\varphi(p)=pS_f\cap S_{(f)}\supseteq q:$

$\forall x\in q,$ write $x=\frac{a}{f^n},$ where $a$ is homogeneous,
$dega=kdegf.$ Then $a\in E_q\subseteq J_q\subseteq p,$ thus
$x=\frac{a}{f^n}\in pS_f\cap S_{(f)}=\varphi(p).$
\end{list}
\end{list}

$\mathit{3^{\circ}}$ $\varphi$ is a homeomorphism:

Let $I$ be a homogeneous ideal of $S$, put $J=IS_f\cap S_{(f)}.$
\enum
\item[(a)]We shall show that $\varphi(D_+(f)\cap V_+(I))=V(J)$:
\enum
\item[(i)]$\forall p\in D_+(f)\cap V_+(I),$ we have $f\not\in p$ and
$I\subseteq p,$ thus $pS_f$ is a prime ideal of $S_f$ and $IS_f\cap
S_{(f)}\subseteq pS_f\cap S_{(f)},$ then $\varphi(p)=pS_f\cap
S_{(f)}\in V(J),$ hence $\varphi(D_+(f)\cap V_+(I))\subseteq V(J).$
\item[(ii)]$\forall q\in V(J),$ set
$$E_q=\{a\in S\mid a \text{ is homogeneous such that }\frac{a}{f^k}\in q\text{ for some }k\geqslant 0\},$$
let $J_q$ be the ideal generated by $E_q$ in $S,$ $p=\sqrt{J_q},$
then $p\in D_+(f)$ and $\varphi(p)=q.$ We want to show $p\in
V_+(I),$ i.e. $I\subseteq p.$ Suppose $\exists a\in I\setminus p,$
since $I, p$ are homogeneous, we may assume $a$ is a homogeneous
element. $a\in I,$ thus $\frac{a^{degf}}{f^{dega}}\in IS_f\cap
S_{(f)}=J.$ Then $a^{degf}\in E_q\subseteq J_q\subseteq p.$ $p$ is
prime, hence $a\in p,$ absurd. So we obtain that $I\subseteq p,$
i.e. $p\in\varphi(D_+(f)\cap V_+(I)).$ Therefore
$\varphi(\varphi(D_+(f)\cap V_+(I)))\supseteq V(J).$
\end{list}

So $\varphi(D_+(f)\cap V_+(I))=V(IS_f\cap S_{(f)})$ holds for any
homogeneous ideal $I$ of $S,$ thus $\varphi$ is closed. Since
$\varphi$ is bijective as well, then $\varphi$ is open.
\item[(b)]$\varphi$ is continuous:

Let $J$ be an ideal of $S_{(f)},$ set
$$E_J=\{a\in S\mid a\text{ is homogeneous, }\frac{a}{f^k}\in J\text{ for some }k\geqslant 0\},$$
let $I$ be the ideal generated by $E_J$ in $S.$ We want to show that
$J=IS_f\cap S_{(f)}.$
\enum
\item[(i)]$\forall x\in J,$ write $x=\frac{a}{f^k},$ with
$dega=kdegf.$ Then $a\in E_J\subseteq I,$ thus $\frac{a}{f^k}\in
IS_f\cap S_{(f)}.$ Hence $J\in IS_f\cap S_{(f)}.$
\item[(ii)]It's easily checked that $E_J$ is closed under
multiplication. Moreover $E_J$ is closed under addition of elements
of the same degree. Hence the only homogeneous elements in $I$ are
in $E_J.$ $\forall x\in IS_f\cap S_{(f)},$ write $x=\frac{a}{f^n},$
with $n\geqslant 0$ and $dega=ndegf.$ Then $\exists b\in I$
homogeneous and $l\geqslant 0$ such that
$\frac{b}{f^l}=\frac{a}{f^n}\in IS_f\cap S_{(f)},$ thus $b\in E_J,$
then $\frac{b}{f^l}=\frac{a}{f^n}\in J,$ hence $IS_f\cap S_{(f)}\in
J.$
\end{list}

Therefore $J=IS_f\cap S_{(f)},$ then $\varphi^{-1}(V(J))=D_+(f)\cap
V_+(I),$ for $\varphi$ is bijective and closed. Hence $\varphi$ is
continuous.
\end{list}

$\mathit{4^{\circ}}$ $\forall p\in D_+(f),$ define
\[ \xymatrix@R=0em@C=4em{
   {(S_{(f)})_{\varphi(p)}} \ar[r]^-{\psi^{(p)}} & {S_{(p)}}      \\
   {\frac{a}{f^k}/\frac{b}{f^n}} \ar@{|->}[r] & {\frac{af^n}{bf^k}} }
   \]
We can easily check that $\psi^{(p)}$ is a well-defined ring
homomorphism. \enum
\item[(a)]$\psi^{(p)}$ is an isomorphism:

Injectivity: If $0=\psi^{(p)}(\frac{a}{f^k}/\frac{b}{f^n}),$ then
$g\in S\setminus p$ homogeneous such that $gaf^n=0,$ thus
$$\frac{a}{f^k}/\frac{b}{f^n}=\frac{gaf^n}{f^{n+k+degg}}/\frac{gb}{f^{n+degg}}=0.$$

Surjectivity: $\forall\frac{a}{b}\in S_{(p)},$ we have $dega=degb,$
then we have
$$\varphi(\frac{ab^{degf-1}}{f^{dega}}/\frac{b^{degf}}{f^{degb}})=\frac{a}{b}.$$

Define $\varphi^{\sharp}: \mathcal {O}_{SpecS_{(f)}}\rightarrow
\varphi_{\ast}(\left.\mathcal {O}_{ProjS}\right|_{D_+(f)})$ by:\\
$\forall U\in \mathds{I}_{ProjS},$ define
\[ \xymatrix@R=0em{
   {\varphi^{\sharp}(U):\mathcal {O}_{SpecS_{(f)}}(U)} \ar[r] & {\mathcal
   {O}_{ProjS}(\varphi^{-1}(U))}                                           \\
   s \ar@{|->}[r] & {\varphi^{\sharp}(U)(s)} }  \]
where $\forall q\in \varphi^{-1}(U),$
$\varphi^{\sharp}(U)(s)(q)=\psi^{(q)}(s(\varphi(q)))$ in $S_{(q)}.$
\item[(b)]We want to show that $\varphi_p^{\sharp}=\psi^{(p)}, \forall p\in D_+(f).$

$\forall p\in D_+(f),$ we have
$$(S_{(f)})_{\varphi(p)}\simeq \mathcal
{O}_{SpecS_{(f)},\varphi(p)}\stackrel{\varphi_p^{\sharp}}{\longrightarrow}
\mathcal {O}_{ProjS,p}\simeq S_{(p)}$$ which is defined by
\begin{eqnarray*}
\frac{a}{f^k}/\frac{b}{f^n}
& \mapsto & (s:D(\frac{b}{f^n})\ni
q\mapsto\frac{a}{f^k}/\frac{b}{f^n}\in(S_{(f)})_q)_{\varphi(p)}  \\
& \stackrel{\varphi_p^{\sharp}}{\longmapsto} &
(\varphi^{-1}(D(\frac{b}{f^n}))\ni
q\mapsto\psi^{(p)}(s(\varphi(q)))\in(S_{(q)}))_p                 \\
& \mapsto & \frac{af^n}{bf^k}
\end{eqnarray*}
for $$\psi^{(q)}(s(\varphi(q)))=\psi^{(q)}(s(qS_f\cap
S_{(f)}))=\psi^{(q)}(\frac{a}{f^k}/\frac{b}{f^n})=\frac{af^n}{bf^k}$$
in $S_{(q)}.$ Hence we obtain that $\varphi_p^{\sharp}=\psi^{(p)},
\forall p\in D_+(f).$ Thus $\varphi_p^{\sharp}$ is an isomorphism of
local ring.
\end{list}

Therefore
$$(\varphi,\varphi^{\sharp}): (D_+(f),\left.\mathcal
{O}_{ProjS}\right|_{D_+(f)})\rightarrow (SpecS_{(f)},\mathcal
{O}_{SpecS_{(f)}})$$ is an isomorphism of locally ringed space. In
particular $$\{D_+(f)\mid f \text{ is homogeneous }\}$$ form a basis
for $ProjS,$ hence $(ProjS, \mathcal {O}_{ProjS})$ is a scheme.
\end{list}
\end{proof}
\begin{eg}[Projective space]
Fix $k$ an algebraically closed field. Let $n\geqslant 1$ be an
integer, put $S=k[x_1,\cdots,x_n].$

$\mathit{1^{\circ}}$ Affine space:

$\mathbb{A}^{n+1}:=k^{n+1},$ let $I$ be an ideal of $S,$ put
$$V(I)=\{x\in \mathbb{A}^{n+1}\mid f(x)=0, \forall f\in I\},$$
called the affine closed subset of $\mathbb{A}^{n+1}.$ Then we have:
\[ \xymatrix@R=0em{
   {\mathbb{A}^{n+1}} \ar@{<->}[r]^-{\text{Zariski Topology}} &
   {SpecS}                                                   \\
   (x_0,x_1,\cdots,x_n) \ar@{<->}[r] & {\mathfrak{M} =
   (X_1-x_0,X_1-x_1,\cdots,X_n-x_n)}                         \\
   V(I) \ar@{<->}[r] & {V(I):=\{p\in SpecS\mid I\subseteq p\}} }  \]

$\mathit{2^{\circ}}$ Projective space:

$\forall x,y\in\mathbb{A}^{n+1}\setminus\{0\},$ we say that $x\sim
y$ if $\exists\lambda\in k\setminus\{0\}$ such that $x=\lambda y.$
Then "$\sim$" is an equivalence relation on
$\mathbb{A}^{n+1}\setminus\{0\}.$ Put
$\mathbb{P}^n=(\mathbb{A}^{n+1}\setminus\{0\})/\sim.$ $\forall
x=(x_0,x_1,\cdots,x_n)\in\mathbb{A}^{n+1}\setminus\{0\},$ write
$\bar{x}=(x_0:x_1:\cdots:x_n)$ for the equivalence class of $x$ and
we call $\{x_0,x_1,\cdots,x_n\}$ the homogeneous coordinates of
$\bar{x}.$ $\forall d\geqslant 0,$ put
$$S_d=\{f\in S\mid f\text{ is homogeneous of degree }d\}.$$ Then
$\forall\lambda\in k,\forall f\in S_d,$ we have $f(\lambda
x_0,\cdots,\lambda x_n)=\lambda^d f(x_0,\cdots,x_n),$ i.e.
$$f=\sum\limits_{i_0+\cdots+i_n=d}a_{i_0i_1\cdots i_n}X_0^{i_0}X_1^{i_1}\cdots X_n^{i_n}.$$

$(S_d)_{d\geqslant 0}$ induces a grading over $S.$ Put
$S_+=\bigoplus\limits_{d\geqslant 1}S_d=(X_0,\cdots,X_n)$ a maximal
ideal of $S.$ Then
\[ \xymatrix@R=0em{
   {\mathbb{A}^{n+1}\setminus\{0\}} \ar@{<->}[r] & {SpecS\setminus
   \{S_+\}}                                                   \\
   {(x_0,\cdots,x_n)} \ar@{<->}[r] &
   {\mathfrak{M}=(X_0-x_0,\cdots,X_n-x_n)} }  \]
Fix $f\in S$ with $f=\sum\limits_{i=0}^df_i,f_i\in S_i(0\leqslant
i\leqslant d).$ Fix $\bar{x}=(x_0:x_1:\cdots:x_n)\in \mathbb{P}^n,$
we say that $\bar{x}$ is a root of $f,$ noted $f(\bar{x})=0,$ if
$\forall y\in\bar{x},$ we have $f(y)=0.$ It's equivalent to saying
that $\forall 0\leqslant i\leqslant d,f_i(x_0,\cdots,x_n)=0.$
Indeed, $f(\bar{x})=0$ iff $\forall\lambda\in k\setminus\{0\},$
$$0=f(\lambda
x_0,\cdots,\lambda x_n)=\sum\limits_{i=0}^d=\lambda^i
f(x_0,\cdots,x_n),$$ which is equivalent to the desired equation for
$k$ is infinite.

Let $I$ be a homogeneous ideal of $S,$ define
$$V_+(I)=\{\bar{x}\in\mathbb{P}^n\mid f(\bar{x})=0,\forall f\in I\}=(V(I)\setminus\{0\})/\sim.$$
\end{eg}
\begin{prop}\
\enum
\item[(1)]Let $I$ be a homogeneous ideal of $S,$ then
$V_+(I)=\emptyset$ iff $\sqrt{I}=S$ or $S_+.$
\item[(2)]$\tau(ProjS):=\{V_+(I)\mid I\text{ is a homogeneous ideal of }S\}$ is
closed under intersection, finite union and contains $\emptyset$ and
the whole space $\mathbb{P}^n=V_+(0).$
\item[(3)]Let $Y\subseteq \mathbb{P}^n,$ define
$$I(Y)=\{f\in S\mid f(y)=0, \forall y\in Y\}.$$
Then $I(Y)$ is a homogeneous ideal of $S,$ and the following
properties hold:
\enum
\item[(a)]If $Y$ is closed in $\mathbb{P}^n,$ then $V_+(I(Y))=Y.$
\item[(b)]If $I$ is a homogeneous ideal of $S,$ then
\[
  I(V_+(I))=\left\{
  \begin{array}{ll}
  \sqrt{I} & \text{if }V_+(I)\neq\emptyset;\\
  S\text{ or }S_+ & \text{if }V_+(I)=\emptyset.
  \end{array}
  \right.
\]
\end{list}
\item[(4)]Let $I$ be a homogeneous ideal of $S,$, we have
$$V_+(I)\longleftrightarrow \{ \,p\in ProjS\mid I\subseteq p\}.$$
\end{list}
\end{prop}
\begin{proof}\
\enum
\item[(1)]If $\sqrt{I}=S.$ $\forall
\bar{x}=(x_0:x_1:\cdots:x_n)\in \mathbb{P}^n, \exists
\bar{y}=(y_0:y_1:\cdots:y_n)\in \mathbb{P}^n$ such that $\bar{x}\neq
\bar{y}.$ Set $f=\sum\limits_{i=0}^n(x_i-y_i)\in S,$ then
$f(\bar{x})\neq 0.$ Since $\sqrt{I}=S,$ thus $\exists g\in
I,r\geqslant 1$ such that $g=f^r.$ Then $g(\bar{x})=f^r(\bar{x})\neq
0.$ Therefore $V_+(I)=\emptyset.$ If $\sqrt{I}=S_+.$
$\forall\bar{x}=(x_0:x_1:\cdots:x_n)\in \mathbb{P}^n,\exists
i(0\leqslant i\leqslant n)$ such that $x_i\neq 0,$ then we only have
to set $f=X_i\in S_+.$

Conversely, if $\exists i(0\leqslant i\leqslant n)$ such that
$f=X_i\not\in\sqrt{I},$ then at least
$\bar{x}=(0:0:\cdots:0:x_i:0:\cdots:0)$ with $x_i\neq 0$ such that
$\bar{x}\in V_+(I).$ Hence if $V_+(I)=\emptyset,$ then
$\sqrt{I}\supseteq S_+=(X_0,\cdots,X_n).$ Since $S_+$ is a maximal
ideal of $S,$ then $\sqrt{I}=S_+$ or $S.$
\item[(2)]It's immediate that $\tau(ProjS)$ contains the empty set and the whole space $\mathbb{P}^n.$ By
definition,
$$\bigcap\limits_{i\in\Lambda}V_+(I_i)=\{\bar{x}\in\mathbb{P}^n\mid f(\bar{x})=0,
\forall
f\in\bigcup\limits_{i\in\Lambda}I_i\}=V_+(<\bigcup\limits_{i\in\Lambda}I_i>).$$
Since $\forall i\in\Lambda, I_i$ is a homogeneous ideal, then
$(\bigcup\limits_{i\in\Lambda}I_i)$ is a homogeneous ideal for it is
generated by the union of each $I_i$'s generators which are
homogeneous. Indeed,
$<\bigcup\limits_{i\in\Lambda}I_i>=\sum\limits_{i\in\Lambda}.$ Thus
$\tau(ProjS)$ is closed under intersection.
$$V_+(I_1)\cup V_+(I_2)=\{\bar{x}\in\mathbb{P}^n\mid
f(\bar{x})=0,\forall f\in I_1\cap I_2\}=V_+(I_1\cap I_2).$$ $I_1\cap
I_2$ is homogeneous for $I_1$ and $I_2$ are homogeneous. Hence
$\tau(ProjS)$ is closed under finite union.
\item[(3)]$I(V)$ is an ideal of $S$ is obvious. $\forall f\in I(Y)$
write $f=\sum\limits_{i=0}^mf_i$ with $f_i\in S_i.$ Since $\forall
\bar{x}\in \mathbb{P}^n,$
$$f(\bar{x})=0\Longleftrightarrow \forall i(1\leqslant i\leqslant
m),f_i(\bar{x}),$$ hence $f_i\in I(Y),$ and therefore $I(Y)$ is
homogeneous.

$(b)$ If $I$ is a homogeneous ideal of $S,$ we want to show that
$I(V_+(I))=\sqrt{I}.$ $I(V_+(I))\supseteq\sqrt{I}$ is immediate. In
affine case, we have $I(V(I))=\sqrt{I},$ hence $I(V_+(I))\subseteq
I(V(I))=\sqrt{I}$ if $I(V_+(I))\neq\emptyset.$ Indeed, let $f\in
I(V_+(I))$ be homogeneous, $\forall x\in V(I)\setminus\{0\},$ we
have $f(\bar{x})=0,$ hence $f(x)=0,$ thus $f\in I(V(I))$ if $f$ is
of positive degree. However, if $degf=0,$ then $f=0$ or
$V_+(I)=\emptyset.$ Therefore $I(V_+(I))=\sqrt{I}$ if
$V_+(I)\neq\emptyset.$

$(a)$ If $Y$ is closed in $\mathbb{P}^n,$ then $Y=V_+(I)$ for some
homogeneous ideal $I$ of $S.$ If $Y\neq\emptyset,$ then
$$V_+(I(Y))=V_+(I(V_+(I)))=V_+(\sqrt{I})=V_+(I)=Y.$$
\item[(4)]
\[ \xymatrix{
   V_+(I) \ar@{<->}[r] & {\sqrt{I}=\bigcap\limits_{I\subseteq p}p}
   \ar@{<->}[r] & {\{p\in ProjS\mid I\subseteq p\}} }  \]
\end{list}
\end{proof}

\newpage

\section{Some properties of schemes}

\begin{Def}
Let $X$ be a nonempty topological space, we say that $X$ is
irreducible if $X$ is not the union of two proper closed subset of
$X.$
\end{Def}
\begin{egs}\
\enum
\item[(1)]A Hausdorff space is irreducible iff it consists of a
single point.
\item[(2)]The affine space $\mathbb{A}^1=k,$ where $k$ is an
algebraically closed field, is irreducible. Because the only closed
subsets of $\mathbb{A}^1$ are $\emptyset, \mathbb{A}^1,$ and subsets
of $\mathbb{A}^1$ that consist of finite points.
\end{list}
\end{egs}
\begin{prop}
Let $X$ be a topological space, then the following are equivalent:
\enum
\item[(1)]$X$ is irreducible.
\item[(2)]The intersection of any two nonempty open subsets of $X$
is not empty.
\item[(3)]Every nonempty open subset of $X$ is dense in $X.$
\end{list}
\end{prop}
\begin{proof}\
\enum
\item[$(1)\Leftrightarrow(2)$]Let $U_1,U_2$ be two nonempty open
subset of $X$ which are not the whole space $X.$ If $U_1\cap
U_2=\emptyset,$ then $U_1^c\cup U_2^c=(U_1\cap U_2)^c=X,$ thus $X$
is the union of two proper closed subsets of $X,$ absurd. We can
prove the converse likewise.
\item[$(2)\Leftrightarrow(3)$]Fix $U$ a nonempty open subset of $X,$
then $U$ intersects with any nonempty open subsets of $X,$ thus is
dense in $X.$

Conversely, if $\exists U_1, U_2$ two nonempty open subsets of $X$
such that $U_1\cap U_2=\emptyset,$ then $U_1, U_2$ cannot be dense
in $X.$
\end{list}
\end{proof}
\begin{remarks}\
\enum
\item[(1)]Irreducible topological spaces are connected.
\item[(2)]Every nonempty open subset of an irreducible space is
irreducible.
\item[(3)]The topological closure of an irreducible subset of a
topological space is irreducible.
\end{list}
\end{remarks}
\begin{prop}
Let $A$ be a nonzero ring, then a closed subset $Y$ of $SpecA$ is
irreducible iff $I(Y)=p$ a prime ideal, i.e. $Y=V(p)$ for some $p\in
SpecA.$
\end{prop}
\begin{proof}
Fix $Y$ a nonempty closed subset of $SpecA,$ define
$I(Y)=\bigcap\limits_{q\in Y}q.$ Then $I(Y)$ is an ideal of $A$ and
$V(I(Y))=Y$ for $Y$ is closed.

$\mathit{1^{\circ}}$ $\Longrightarrow:$ If $Y$ is irreducible, we
shall show that $I(Y)$ is prime: $\forall a,b\in A,$ if $ab\in
I(Y),$ then $(a)(b)\subseteq I(Y),$ and thus $Y=V(I(Y))\subseteq
V(ab)=V(a)\cup V(b),$ hence $Y=(Y\cap V(a))\cup(Y\cap V(b)).$ Then
$Y=Y\cap V(a)$ or $Y=Y\cap V(b)$ for $Y$ is irreducible. Therefore
$Y\subseteq V(a)$ or $Y\subseteq V(b),$ thus
$a\in\bigcap\limits_{q\in V(a)}\subseteq\bigcap\limits_{q\in
Y}=I(Y),$ or $b\in I(Y).$ So $I(Y)$ is prime.

$\mathit{2^{\circ}}$ $\Longleftarrow:$ Suppose that $Y=V(p)$ with
$p$ prime, then $$I(Y)=\bigcap\limits_{q\in
Y}q=\bigcap\limits_{p\subseteq q \atop q \text{ prime}}q=p.$$ We
shall show that $Y$ is irreducible.

Let $J_1,J_2$ be two ideals of $A$ such that
$$Y=(Y\cap V(J_1))\cup(Y\cap V(J_2))=Y\cap V(J_1J_2).$$ Then
$V(p)\subseteq V(J_1J_2),$ thus
$J_1J_2\subseteq\sqrt{J_1J_2}\subseteq p.$ Since $p$ is prime, then
$J_1\subseteq p$ or $J_2\subseteq p.$ Hence $V(p)\subseteq V(J_1)$
or $V(p)\subseteq V(J_2).$ Consequently $Y=Y\cap V(J_1)$ or $Y=Y\cap
V(J_2).$ Therefore $Y$ is irreducible.
\end{proof}
\begin{egs}
Fix $k$ an algebraically closed field.
\enum
\item[(1)]Affine variety:

Let $Y$ be a closed subset of $\mathbb{A}^n.$ Then $Y$ is
irreducible iff $I(Y)$ is prime. An irreducible closed subset of
$\mathbb{A}^n$ is called an affine (algebraic) variety.
\item[(2)]Projective variety:

Let $Y$ be a closed subset of $\mathbb{P}^n.$ An irreducible closed
subset of $\mathbb{P}^n$ is a projective (algebraic) variety. An
open subset of a projective variety is called a quasi-projective
variety. Equivalently, a nonempty subset of $\mathbb{P}^n$ is
quasi-projective iff it is locally closed and irreducible, i.e. the
intersection of an open subset and an irreducible closed subset. In
particular an affine variety is quasi-projective.
\end{list}
\end{egs}
\begin{lemma}
Let $A$ be a ring, and $p\in SpecA,$  then $\overline{\{p\}}=V(p).$
\end{lemma}
\begin{proof}
$\overline{\{p\}}=\bigcap\limits_{p\in V(I), \atop I \text{ an
ideal}} V(I)=\bigcap\limits_{\sqrt{I}\subseteq p, \atop I \text{ an
ideal}} V(I)=\bigcap\limits_{V(I)\supseteq V(p), \atop I \text{ an
ideal}} V(I)=V(p).$
\end{proof}
\begin{prop}
Let $X$ be a scheme, and $Y$ be an irreducible closed subset of $X.$
Then $\exists ! y\in Y$ such that $Y=\overline{\{y\}}.$ We call $y$
the generic point of $Y.$
\end{prop}
\begin{proof}
$\mathit{1^{\circ}}$Existence:

Since $Y$ is irreducible, thus nonempty, then we can find an open
affine subscheme $U=SpecA$ with $A$ a nonzero ring, such that $U\cap
Y\neq\emptyset.$ Since $Y$ is closed and irreducible in $X,$ then
$U\cap Y$ is irreducible and closed in $U=SpecA.$ Then $\exists y\in
U\cap Y$ such that $U\cap Y=V(y)=\overline{\{y\}}\cap SpecA,$ where
$\overline{\{y\}}$ is the topological closure of $y$ in $X.$ However
$Y$ is irreducible, and $U\cap Y$ is nonempty and open in $Y,$ thus
$U\cap Y$ is dense in $Y.$ Hence $\overline{\{y\}}\subseteq
Y=\overline{U\cap Y}\subseteq\overline{\{y\}}$ for $Y$ is closed.
Therefore $Y=\overline{\{y\}}.$

$\mathit{2^{\circ}}$Uniqueness:

Let $y^{\prime}\in Y$ be such that
$Y=\overline{\{y\}}=\overline{\{y^{\prime}\}}.$ Since $U\cap Y$ is
dense in $Y,$ then $y^{\prime}\in U\cap Y.$ Otherwise if
$y^{\prime}\not\in U\cap Y,$ then $y^{\prime}\in Y\setminus(U\cap
Y)=Y\cap U^c$ which is closed in $X,$ thus
$\overline{\{y^{\prime}\}}\subseteq Y\setminus (U\cap Y),$ absurd.
Hence $$V(y)=\overline{\{y\}}\cap U=\overline{\{y^{\prime}\}}\cap
U=V(y^{\prime}),$$ then $y=\sqrt{y}=\sqrt{y^{\prime}}=y^{\prime}.$
\end{proof}
\begin{Def}
Let $X$ be a topological space. If the family of closed subsets of
$X$ satisfies the descending chain condition, then we say that $X$
is Noetherian.
\end{Def}
\begin{prop}
Let $X$ be a topological space, then the following are equivalent:
\enum
\item[(1)]$X$ is Noetherian.
\item[(2)]The family of open subsets of $X$ satisfies the ascending
chain condition.
\item[(3)]Every nonempty family of closed subsets of $X$ has a
minimal element.
\item[(4)]Every nonempty family of open subsets of $X$ has a
maximal element.
\end{list}
\end{prop}
\begin{proof}
The descending chains of closed subsets of $X$ are in one-to-one
correspondence with the ascending chains of open subsets of $X,$
hence the family of closed subsets of $X$ satisfies the d.c.c iff
the family of open subsets of $X$ satisfies the a.c.c. Then we
obtain $(1)\Longleftrightarrow(2).$

If $(3)$ doesn't hold, then we can find a chain $F_1\supset
F_2\supset\cdots\supset F_n\supset\cdots$ of closed subsets of $X.$
$\{F_i\}_{i\geqslant 1}$ doesn't satisfy the descending chain
condition, hence $X$ is not Noetherian. If $(3)$ holds, let
$F_1\supseteq F_2\supseteq\cdots\supseteq F_n\supseteq\cdots$ be a
descending chain of closed subsets of $X,$ then $\{F_i\}_{i\geqslant
1}$ has a minimal ideal $F\in \{F_i\}_{i\geqslant 1},$ hence
$\exists N\geqslant 1$ such that $\forall i\geqslant N, F_i=F$.
Hence it satisfies the descending chain condition, thus $X$ is
Noetherian. We can obtain $(2)\Longleftrightarrow(4)$ similarly.
\end{proof}
\begin{egs}\
\enum
\item[(1)]If $A$ is a Noetherian ring, then $SpecA$ is Noetherian.
\item[(2)]Every affine space and projective space are Noetherian.
\end{list}
\end{egs}
\begin{prop}
Let $X$ be a Noetherian topological space, then
\enum
\item[(1)]Every subspace of $X$ is Noetherian.
\item[(2)]$X$ is quasi-compact.
\end{list}
\end{prop}
\begin{proof}\
\enum
\item[(1)]Let $Y\subseteq X$ be a subspace, $F_1\supseteq
F_2\supseteq\cdots$ be a descending chain of closed subsets of $Y.$
Then $\overline{F_1}\supseteq \overline{F_2}\supseteq\cdots$ is a
descending chain of closed subsets of $X,$ thus $\exists n\geqslant
1$ such that $\overline{F_m}=\overline{F_n}, \forall m\geqslant n.$
Then $$F_m=\overline{F_m}\cap Y=\overline{F_n}\cap Y=F_n.$$
\item[(2)]Let $(U_i)_{i\in\Lambda}$ be an open covering of $X.$ Put
$$\Sigma=\{\bigcup\limits_{i\in J}U_i\mid J\subseteq\Lambda \text{ is a finite subset}\}.$$
We need to show that $X\in\Sigma.$ By contradiction, suppose
$X\not\in\Sigma.$ Since $\Sigma$ is nonempty and consists of open
subsets of $X,$ hence $\Sigma$ has a maximal element
$\bigcup\limits_{i\in J}U_i,$ for $X$ is Noetherian. Necessarily
$\bigcup\limits_{i\in J}U_i\subset X.$ But $\bigcup\limits_{i\in
\Lambda}U_i=X,$ hence $\exists j\in\Lambda$ such that
$U_j\nsubseteq\bigcup\limits_{i\in J}U_i.$ Then
$\bigcup\limits_{i\in J}U_i\subset U_j\cup(\bigcup\limits_{i\in
J}U_i),$ absurd.
\end{list}
\end{proof}
\begin{prop}
Let $X$ be a Noetherian topological space
\enum
\item[(1)]For any closed subset $Y$ of $X,$ we have the
decomposition $Y=\bigcup\limits_{i=1}^mY_i$ for some closed
irreducible subsets $Y_i(1\leqslant i\leqslant m)$ such that
$Y_i\nsubseteq Y_j(i\neq j)$ irredundant. Moreover the decomposition
is unique. Every $Y_i$ is called an irreducible component of $Y.$
\item[(2)]An irreducible closed subset $Y$ is an irreducible
component of $X$ iff $Y$ is maximal in the family of irreducible
(closed) subsets of $X.$
\end{list}
\end{prop}
\begin{proof}\
\enum
\item[(1)]$\mathit{1^{\circ}}$Existence of the decomposition:

Let $\Sigma$ be the class of all closed subsets $F$ of $Y$ which is
not a finite union of irreducible closed subsets of $Y.$ We shall
show that $\Sigma=\emptyset,$ and thus $Y\not\in\Sigma.$ By
contradiction, assume $\Sigma\neq\emptyset.$ Since $X$ is
Noetherian, thus $Y$ is Noetherian. Then $\Sigma$ has a minimal
element, say $F.$ Necessarily $F$ is not irreducible, hence we can
find $F_1\subset F$ and $F_2\subset F$ such that $F_1, F_2$ are
nonempty, closed and that $F=F_1\cup F_2.$ By the minimality of $F,$
we obtain that $F_1,F_2$ are finite unions of irreducible closed
subsets of $Y,$ so is $F,$ absurd.

$\mathit{2^{\circ}}$Uniqueness:

Suppose that
$Y=\bigcup\limits_{i=1}^mY_i=\bigcup\limits_{j=1}^nY_j^{\prime}$
(irredundant). Since $Y_1=Y\cap Y_1=\bigcup\limits_{j=1}^n(Y_1\cap
Y_j^{\prime})$ is irreducible, then $\exists j(1\leqslant j\leqslant
n)$ such that $Y_1=Y_1\cap Y_j^{\prime}\subseteq Y_j^{\prime}.$ By
renumbering, we can suppose $Y_1\subseteq Y_1^{\prime}.$ We can find
$i(1\leqslant i\leqslant m)$ such that $Y_1^{\prime}\subseteq Y_i$
likewise. Then $Y_1\subseteq Y_1^{\prime}\subseteq Y_i,$ thus $i=1,$
by the irredundance of the decomposition. By the same argument for
each $i(1\leqslant i\leqslant m),$ we can see that $m=n,$ and
$Y_i=Y_i^{\prime}$(by renumbering). Hence the decomposition is
unique.
\item[(2)]Let $X_1,\cdots,X_m$ be the irreducible components of $X.$
Let $Y$ be an irreducible closed subset of $X.$ Then $Y=Y\cap
X=\bigcup\limits_{i=1}^mY\cap X_i.$ Hence $\exists i(1\leqslant
i\leqslant m)$ such that $Y=Y\cap X_i,$ i.e. $Y\subseteq X_i.$ Since
the decomposition of $X$ is unique, if $Y$ is not an irreducible
component of $X,$ then it's not maximal in the family of irreducible
(closed) subsets of $X.$ Consequently $i(1\leqslant i\leqslant m)$
we shall show that $X_i$ is maximal. Suppose $X_i\subseteq Y$ with
$Y$ irreducible and closed, then $\exists j(1\leqslant j\leqslant
m)$ such that $X_i\subseteq Y\subseteq X_j.$ Hence $i=j$ and
$X_i=Y.$ Therefore $X_i$ is maximal in the family of irreducible
(closed) subsets of $X.$
\end{list}
\end{proof}
\begin{remark}
A Hausdorff topological space is Noetherian iff it is a finite set.
Indeed the space has only a finite number of irreducible components
which consist of only one single point.
\end{remark}
\begin{Def}
Let $A$ be a nonzero ring. A prime ideal $p$ is called minimal if it
doesn't contain any prime ideal other than itself.
\end{Def}
\begin{prop}
Let $A$ be a nonzero ring, then
\enum
\item[(1)]Every prime ideal of $A$ contains a minimal prime ideal.
\item[(2)]Every irreducible component of $SpecA$ takes the form
$V(p),$ with $p$ a minimal prime ideal of $A,$ i.e. we have
\[ \xymatrix@R=0em{
   {\{\text{irreducible components of }SpecA\}} \ar@{<->}[r] &
   {\{\text{minimal prime ideal of }A\}}                      \\
   {V(p)} \ar@{<->}[r] & p }  \]
\end{list}
\end{prop}
\begin{proof}\
\enum
\item[(1)]Let $p$ be a prime ideal of $A.$ Set
$$\Sigma=\{q\mid q\subseteq p,\text{ and }q\text{ is prime in }A\}.$$
At least $p\in\Sigma,$ hence $\Sigma\neq\emptyset.$ Let
"$\supseteq$" be the partial ordering in $\Sigma.$ Let
$(q_i)_{i\in\Lambda}\subseteq\Sigma$ be a chain in $\Sigma,$ set
$q=\bigcap\limits_{i\in\Lambda}q_i.$ To prove $q\in\Sigma,$ we need
to show that $q$ is prime. $\forall a,b\in A\setminus q,$ $\exists
i\in\Lambda, j\in\Lambda$ such that $a\not\in q_i, b\not\in q_j.$ We
may assume $q_i\supseteq q_j,$ then $a\not\in q_j, b\not\in q_j,$
thus $ab\not\in q_j,$ and then $ab\not\in q.$ Hence $q$ is prime.

By Zorn's lemma, $\Sigma$ has a minimal element, and immediately we
can see that it is a minimal prime ideal.
\item[(2)]Every irreducible closed subset of $SpecA$ takes the form
$V(p)$ with $p\in SpecA.$ Let $p\in SpecA,$ then by definition,
$$V(p)\text{ is maximal}\Leftrightarrow\forall q\in SpecA, V(p)\subseteq V(q) \text{ iff } p=q,$$
which is equivalent to saying that $\forall q\in SpecA, q\subseteq
p$ iff $q=p.$ i.e. $p$ is a minimal ideal of $A.$
\end{list}
\end{proof}
\begin{Def}
Let $(X, \mathcal {O}_X)$ be a scheme.
\enum
\item[(1)]If $X$ is a connected topological space, then we say that $(X, \mathcal
{O}_X)$ is  connected.
\item[(2)]If $X$ is an irreducible topological space, then we say
that $(X, \mathcal {O}_X)$ is irreducible.
\item[(3)]If for any nonempty open subset $U$ of $X,$ the ring $\mathcal
{O}_X(U)$ is reduced(i.e. $\mathfrak{N}_{\mathcal {O}_X(U)}=0$),
then we say that $(X, \mathcal {O}_X)$ is reduced.
\item[(4)]If for any nonempty open subset $U$ of $X,$ the ring $\mathcal
{O}_X(U)$ is an integral domain, then we say that $(X, \mathcal
{O}_X)$ is integral.
\end{list}
\end{Def}
\begin{remarks}\
\enum
\item[(1)]If $(X, \mathcal {O}_X)$ is irreducible, then it is
connected.
\item[(2)]If $(X, \mathcal {O}_X)$ is integral, then it is reduced.
\end{list}
\end{remarks}
\begin{prop}
Let $A$ be a nonzero ring, then the following are equivalent:
\enum
\item[(1)]$SpecA$ is connected.
\item[(2)]$A$ doesn't have any idempotent element other than $0$ and
$1.$
\item[(3)]For any decomposition of rings $A=A_1\times A_2,$ we have
$A_1=0$ or $A_2=0.$
\end{list}
\end{prop}
\begin{proof}\
\enum
\item[$(1)\Longrightarrow(2)$]
Let $e\in A$ be an idempotent element, then $e(1-e)=0.$ Hence
$$SpecA=V(0)=V(e(1-e))=V(e)\cup V(1-e).$$
However, $$\emptyset=V(1)=V(1-e+e)=V(e)\cap V(1-e).$$ But $SpecA$ is
connected, thus $V(e)=\emptyset$ or $V(1-e)=\emptyset.$ Then $e$ is
a unit or $1-e$ is a unit. However $e(1-e)=0,$ thus $e=1$ or $e=0.$
\item[$(2)\Longrightarrow(3)$]
Suppose $A=A_1\times A_2.$ Let $e_1$ be the identity of $A_1$ and
$e_2$ be the identity of $A_2.$ Then $1=(e_1, e_2)=(e_1, 0)+(0,
e_2).$
$$(e_1,0)^2=(e_1^2,0)=(e_1,0)\,,\quad (0,e_2)^2=(0,e_2^2)=(0,e_2).$$
Hence one of $(e_1,0)$ and $(0,e_2)$ is $1,$ the other is $0.$ Thus
$e_1=0$ or $e_2=0,$ and therefore $A_1=0$ or $A_2=0.$
\item[$(3)\Longrightarrow(1)$]
By contradiction, suppose that $SpecA$ is not connected. Then we can
find $V_1,V_2$ nonempty open subsets of $SpecA$ such that
$SpecA=V_1\cup V_2$ and $V_1\cap V_2=\emptyset.$ $\forall p\in
SpecA,$ define
\[
e(p)\left\{
    \begin{array}{ll}
    1\text{ in }A_p, & \text{ if }p\in V_1;\\
    0\text{ in }A_p, & \text{ if }p\in V_2.
    \end{array}
    \right.
\]
Then $e\in\mathcal {O}_{SpecA}(SpecA)=A,$ and $e^2=e,e\neq 0,1.$ Put
$A_1=Ae,A_2=A(1-e),$ then $A_1,A_2$ are rings respectively with $e$
and $(1-e)$ as identity. Moreover we have
\[ \xymatrix@R=0em{
   {\rho:A} \ar[r]^-{\sim} & {A_1\times A_2}\\
   x \ar@{|->}[r] & {(xe,x(1-e))} }  \]
Indeed, we can define
\[ \xymatrix@R=0em{
   {\rho^{\prime}:A_1\times A_2} \ar[r] & {A}  \\
   {(xe,y(1-e))} \ar@{|->}[r] & {xe+y(1-e)} }  \]
It's easily checked that $\rho\circ\rho^{\prime}=id_{A_1\times A_2}$
and $\rho^{\prime}\circ\rho=id_A.$ Hence $A_1=Ae\neq 0$ for $e\neq
0,$ and $A_2=A(1-e)\neq 0$ for $e\neq 0,$ absurd.
\end{list}
\end{proof}
\begin{prop}
Let $(X, \mathcal {O}_X)$ be a scheme. $\forall f\in \mathcal
{O}_X(X),$ define $$X_f=\{p\in X\mid f_p \text{ is a unit in }
\mathcal {O}_{X,p}\}$$Then the following properties hold:
\enum
\item[(1)]$\forall f\in \mathcal {O}_X(X),$ $X_f$ is open.
\item[(2)]$\forall f\in \mathcal {O}_X(X),$ $X_f=\emptyset$ iff
there exists an open covering $(V_i)_{i\in\Lambda}$ of $X$ such that
$\left.f\right|_{V_1}$ is nilpotent, $\forall i\in\Lambda.$
\item[(3)]$\forall f,g\in \mathcal {O}_X(X),$ we have $X_f\cap X_g=X_{fg}.$
\end{list}
\end{prop}
\begin{proof}\
\enum
\item[(1)]Cover $X$ by open affine subschemes $U_j=SpecA_j(j\in J),$
then $$X_f=X_f\cap X=\bigcup\limits_{j\in J}(X_f\cap U_j).$$ However
$\forall j\in J,$ if we put $f_j=\left.f\right|_{U_j}\in\mathcal
{O}_X(U_j)=A_j,$ then
\begin{eqnarray*}
X_f\cap U_j
& = & \{p\in SpecA_j\mid f_p=(f_j)_p \text{ is a unit in} (A_j)_p\} \\
& = & \{p\in SpecA_j\mid f_j\not\in p\}                          \\
& = & D(f_j).
\end{eqnarray*}

$D(f_j)$ is open in $SpecA_j,$ thus open in $X.$ Therefore $X_f$ is
open in $X.$
\item[(2)]$\Longrightarrow:$ Suppose $X_f=\emptyset,$ take $(U_j)_{j\in
J}$ the open affine subscheme covering $X.$ Then $\forall j\in J,
\emptyset=X_f\cap U_j=D(f_j)$ in $SpecA_j.$ Thus $f_j$ is nilpotent
in $A_j.$

$\Longleftarrow:$ Let $(V_i)_{i\in\Lambda}$ be an open covering of
$X$ such that $\left.f\right|_{V_i}$ is nilpotent, $\forall
i\in\Lambda.$ Then $\forall p\in X, \exists i\in\Lambda$ such that
$p\in V_i.$ Thus $f_p$ is nilpotent in $\mathcal {O}_{X,p}$ for
$\left.f\right|_{V_i}$ is. Hence $p\not\in X_f.$ Therefore
$X_f=\emptyset.$
\item[(3)]$\forall p\in X,$ we have
$$p\in X_f\cap X_g\Leftrightarrow f_p\text{ and }g_p\text{ are units in }\mathcal {O}_{X,p},$$
which is equivalent to saying that $(fg)_p=f_pg_p$ is a unit in
$\mathcal {O}_{X,p},$ thus $p\in X_{fg}.$
\end{list}
\end{proof}
\begin{prop}
A scheme $(X, \mathcal {O}_X)$ is integral iff it is irreducible and
reduced.
\end{prop}
\begin{proof}
$\Longrightarrow:$ Since $(X, \mathcal {O}_X)$ is integral, then it
is reduced.

By contradiction, suppose that $(X, \mathcal {O}_X)$ is not
irreducible. Then we can find two nonempty open subsets $V_1, V_2$
of $X$ such that $V_1\cap V_2=\emptyset.$ Then we have
\[ \xymatrix@R=0em{
   {\mathcal {O}_X(V_1\cup V_2)} \ar[r]^-{\sim} & {\mathcal {O}_X(V_1)\times \mathcal
   {O}_X(V_2)}                                      \\
   s \ar@{|->}[r] & {(\left.s\right|_{V_1}, \left.s\right|_{V_2})} }  \]
But $(X, \mathcal {O}_X)$ is integral, hence $\mathcal {O}_X(V_1\cup
V_2), \mathcal {O}_X(V_1), \mathcal {O}_X(V_2)$ are integral,
absurd.

$\Longleftarrow:$ Suppose that $(X, \mathcal {O}_X)$ is irreducible
and reduced. Let $U$ be a nonempty open subset of $X,$ we should
show that $\mathcal {O}_X(U)$ is an integral domain.

Suppose $f^{(1)}, f^{(2)}\in \mathcal {O}_X(U)$ such that
$f^{(1)}f^{(2)}=0.$ $\forall i=1,2$, put
$$U_{f^{(i)}}=\{p\in U\mid f^{(i)}_p\text{ is a unit in }\mathcal
{O}_{U,p}=\mathcal {O}_{X,p}\}.$$ Cover $U$ by open affine
subschemes $V^{(\alpha)}=SpecA^{(\alpha)}, \alpha\in\Lambda.$
$\forall\alpha\in\Lambda,$ put
$f^{(i,\alpha)}=\left.f^{(i)}\right|_{V^{(\alpha)}}\in\mathcal
{O}_X(V^{(\alpha)}),$ then $U_{f^{(i)}}\cap
V^{(\alpha)}=D(f^{(i,\alpha)})$ in $SpecA^{(\alpha)}.$ Since
$f^{(1)}f^{(2)}=0,$ hence
$\emptyset=U_{f^{(1)}f^{(2)}}=U_{f^{(1)}}\cap U_{f^{(2)}}.$ Since
$X$ is irreducible, then $U$ is irreducible as well. But
$U_{f^{(1)}}$ and $U_{f^{(2)}}$ are open in $U$, then
$U_{f^{(1)}}=\emptyset$ or $U_{f^{(2)}}=\emptyset.$ We may assume
$U_{f^{(1)}}=\emptyset,$ then $f^{(1,\alpha)}$ is nilpotent in
$A^{(\alpha)}, \forall\alpha\in\Lambda.$ Hence
$f^{(1,\alpha)}\in\mathfrak{N}_{A^{(\alpha)}}=\mathfrak{N}_{\mathcal
{O}_X(V^{(\alpha)})}=0.$ So $f^{(1,\alpha)}=0,
\forall\alpha\in\Lambda.$ i.e. $\forall\alpha\in\Lambda,
\left.f^{(1)}\right|_{V^{(\alpha)}}=0.$ But
$(V^{(\alpha)})_{\alpha\in\Lambda}$ is an open covering of $U,$ thus
$f^{(1)}=0.$ Therefore $\mathcal {O}_X(U)$ is an integral domain.
\end{proof}
\begin{prop}
Let $(X, \mathcal {O}_X)$ be an integral scheme, then
\enum
\item[(1)]$\exists !\xi\in X$ such that $X=\overline{\{\xi\}}.$ We
call $\xi$ the generic point of $X.$
\item[(2)]$\mathcal {O}_{X,\xi}$ is a field, called the function
field of $X.$
\item[(3)]$\forall p\in X, \mathcal {O}_{X,p}$ is an integral domain
and its field of fractions is isomorphic to $\mathcal {O}_{X,\xi}.$
\item[(4)]For any nonempty open subsets $U, V$ of $X(U\subseteq V),$
the restriction
$$\rho_{V\,U}: \mathcal {O}_X(V)\rightarrow \mathcal {O}_X(U)$$
is injective.
\end{list}
\end{prop}
\begin{proof}\
\enum
\item[(1)]Since $(X, \mathcal {O}_X)$ is integral, thus $X$ is
irreducible, and then $\exists !\xi\in X$ such that
$X=\overline{\{\xi\}}.$
\item[(2)]Let $U=SpecA$ be an open affine subscheme of $X.$ Then
$\xi\in U$ for $U$ is dense in $X,$ and $A=\mathcal {O}_X(U)$ is an
integral domain. But $$SpecA=V(0)=X\cap U=\overline{\{\xi\}}\cap
U=V(\xi),$$ and $U=SpecA$ is irreducible, then $0$ is the prime
ideal $p$ in $A$ which corresponds to $\xi.$ Then $\mathcal
{O}_{X,\xi}\cong A_p$ the field of fractions of $A.$
\item[(3)]Fix $p\in X.$ For any neighborhood $U$ of $p,$ we have
$\xi\in U$ for $U$ is dense in $X,$ and the canonical homomorphism
$\mathcal {O}(U)\rightarrow \mathcal {O}_{X,\xi}$ is injective.
Indeed if $s\in \mathcal {O}_X(U)$ such that $s_{\xi}=0,$ then we
can show that $s=0.$ Cover $U$ by open affine subschemes
$V^{(\alpha)}=SpecA^{(\alpha)}, \alpha\in\Lambda.$
$\alpha\in\Lambda,$ put
$s^{(\alpha)}=\left.s\right|_{V^{(\alpha)}}.$ Then $\xi\in
V^{(\alpha)}$ and $s_{\xi}^{(\alpha)}=s_{\xi}=0.$ But the canonical
homomorphism
$$A^{(\alpha)}\cong\mathcal {O}_X(V^{(\alpha)})\rightarrow \mathcal {O}_{X,\xi} (\text{ the field of fractions of }A^{(\alpha)})$$
is injective. Hence $s^{(\alpha)}=0,$ i.e.
$\left.s\right|_{V^{(\alpha)}}=0,$ thus $s=0.$

Moreover $\mathcal {O}_{X,p}=\varinjlim\limits_{p\in U}\mathcal
{O}_X(U)\rightarrow \mathcal {O}_{X,\xi}$ is injective. Indeed
$\forall s\in \mathcal {O}_{X,p},$ $\exists U_p$ a neighborhood of
$p$ and $s^{(p)}\in \mathcal {O}_X(U_p)$ such that $s=s_p^{(p)}.$ If
the image of $s$ in $\mathcal {O}_{X,\xi}$ is zero, then the image
of $s^{(p)}$ in $\mathcal {O}_{X,\xi}$ is $0,$ and then $s^{(p)}=0,$
thus $s=s_p^{(p)}=0.$ Therefore $\mathcal {O}_{X,p}$ is an integral
domain.

However, for any open affine subscheme $U=SpecA$ of $X$ which
contains $p,$ we have the following commutative diagram
\[ \xymatrix{
   {A=\mathcal {O}_X(U)} \ar[rr]^-{\varphi_U} \ar[dr]_-{\psi_U}
   & & {\mathcal {O}_{X,\xi}}                              \\
   & {\mathcal {O}_{X,p}} \ar[ur]_g }  \]
Since $\varphi_U$ and $g$ are injective, then $\psi_U$ is injective.
$\mathcal {O}_X(U)$ and $\mathcal {O}_{X,p}$ are integral domains,
$\mathcal {O}_{X,\xi}$ is the field of fractions of $\mathcal
{O}_X(U),$ hence it is also the field of fractions of $\mathcal
{O}_{X,p}.$
\item[(4)]Let $U, V$ be two nonempty open subsets of $X$ such that
$U\subseteq V,$ then $\xi\in U, \xi\in V,$ then we obtain the
following commutative diagram:
\[ \xymatrix{
   {\mathcal {O}_X(V)} \ar[rr]^{\rho_{V\,U}} \ar[dr]_{\varphi_V} & &
   {\mathcal {O}_X(U)} \ar[dl]^{\varphi_U}                      \\
   & {\mathcal {O}_{X,\xi}} }  \]
Since $\varphi_U$ and $\varphi_V$ are injective, hence $\rho_{V\,U}$
is injective as well.
\end{list}
\end{proof}
\begin{prop}
A scheme $(X, \mathcal {O}_X)$ is reduced iff $\forall p\in X,
\mathcal {O}_{X,p}$ is reduced.
\end{prop}
\begin{proof}
$\Longleftarrow:$ Let $U$ be a nonempty open subset of $X,$ we shall
show that $\mathcal {O}_X(U)$ is reduced, i.e.
$\mathfrak{N}_{\mathcal {O}_X(U)}=0.$ $\forall s\in
\mathfrak{N}_{\mathcal {O}_X(U)}, \exists n\geqslant 1$ such that
$s^n=0.$ Hence $\forall p\in U,$ we have $s^n_p=0,$ i.e. $s_p\in
\mathfrak{N}_{\mathcal {O}_{X,p}}=0,$ so $s_p=0.$ But $s(p)=s_p=0,$
thus $s=0.$

$\Longrightarrow:$ Fix $p\in X, \forall t\in \mathfrak{N}_{\mathcal
{O}_{X,p}}, \exists n\geqslant 1$ such that $0=t^n\in \mathcal
{O}_{X,p}.$ Hence $\exists U_p\in\mathds{I}_p(X)$ a neighborhood of
$p$ and $s\in \mathcal {O}_X(U_p)$ such that $s_p=t,$ and
$V_p\in\mathds{I}_p(X), V_p\subseteq U_p$ such that
$\left.s^n\right|_{V_p}=0.$ Then
$\left.s\right|_{V_p}\in\mathfrak{N}_{\mathcal {O}_X(V_p)}=0,$ so
$t=0.$
\end{proof}
\begin{Def}
Let $(X, \mathcal {O}_X)$ be a scheme.
\enum
\item[(1)]We say that $(X, \mathcal {O}_X)$ is locally integral if
$\forall p\in X, \mathcal {O}_{X,p}$ is an integral domain.
\item[(2)]We say that $(X, \mathcal {O}_X)$ is normal if
$\forall p\in X, \mathcal {O}_{X,p}$ is an integral domain and is
integrally closed.
\end{list}
\end{Def}
\begin{remarks}\
\enum
\item[(1)]An integral scheme is locally integral.
\item[(2)]A normal scheme is locally integral.
\item[(3)]Let $A$ be a nonzero ring, put $X=SpecA,$ then
\enum
\item[(a)]$X$ is irreducible iff $\mathfrak{N}_A$ is a prime ideal.
\item[(b)]$X$ is reduced iff $\mathfrak{N}_A=0.$
\item[(c)]$X$ is integral iff $A$ is an integral domain.
\item[(d)]$X$ is locally integral iff $\forall p\in X, A_p$ is an
integral domain.
\item[(e)]$X$ is normal iff $\forall p\in X, A_p$ is an integrally closed domain.
\end{list}
\end{list}
\end{remarks}
\begin{eg}
Take $A=\mathbb{Z}/6\mathbb{Z}=\{0,\pm 1,\pm 2,3\}.$ Set
$p_1=\{0,\pm 2\}, p_2=\{0,3\},$ then $p_1, p_2$ are prime ideals of
$A,$ and $SpecA=\{p_1,p_2\}.$

$A_{p_1}=\{0,\pm\frac{1}{3},\pm 1,3\}, A_{p_2}=\{0,\pm
1,\pm\frac{1}{2},\pm 2\}.$ $A_{p_1}, A_{p_2}$ are integral domains,
so $SpecA$ is locally integral but not integral for
$A=\mathbb{Z}/6\mathbb{Z}$ is not an integral domain.
\end{eg}
\begin{prop}
Let $(X, \mathcal {O}_X)$ be a scheme such that $X$ is a Noetherian
topological space. Then $(X, \mathcal {O}_X)$ is locally integral
iff it is reduced and its irreducible components are disjoint.
\end{prop}
\begin{proof}
$\Longrightarrow:$ Suppose that $(X, \mathcal {O}_X)$ is locally
integral. $\forall p\in X, \mathcal {O}_{X,p}$ is an integral
domain, thus is reduced. Hence $(X, \mathcal {O}_X)$ is reduced.

Let $X_1, X_2$ be two irreducible components of $X$ such that
$X_1\cap X_2\neq\emptyset.$ Take $x\in X_1\cap X_2,$ and let
$U=SpecA$ be an affine open subscheme of $X$ containing $x.$ Then
$X_i\cap U(i=1,2)$ is nonempty and thus is an irreducible component
of $U.$ Indeed $x\in X_i\cap U,$ thus $X_i\cap U$ is a nonempty open
subset of $X_i,$ hence is dense in $X_i.$ Let $Y\supseteq X_i\cap U$
be an irreducible component of $U$, then $Y$ is closed in $U$ and
thus $Y=\overline{Y}\cap U$ where $\overline{Y}$ is the topological
closure of $Y$ in $X.$ Hence $X_i=\overline{X_i\cap
U}\subseteq\overline{Y},$ then $X_i=\overline{Y}$ for $X_i$ is an
irreducible component of $X.$ Thus $Y=\overline{Y}\cap U=X_i\cap U.$

Then $\exists p_i\in U$ such that $X_i\cap U=V(p_i)$ in $U=SpecA$
with $p_i$ a minimal prime ideal of $A.$ Let $p_x\in SpecA$ be the
prime ideal corresponding to $x\in X_1\cap X_2\cap U=V(p_1)\cap
V(p_2),$ then $p_x\in V(p_i)(i=1,2),$ thus $p_i\subseteq p_x.$ Then
$p_1A_{p_x}$ and $p_2A_{p_x}$ are two minimal prime ideals in
$A_{p_x}.$ However, $A_{p_x}=\mathcal {O}_{X,x}$ is an integral
domain, hence $p_1A_{p_x}=p_2A_{p_x}=0,$ and then $p_1=p_2.$
Therefore $X_1\cap U=V(p_1)=V(p_2)=X_2\cap U.$ So
$X_1=\overline{X_1\cap U}=\overline{X_2\cap U}=X_2.$ Hence the
irreducible components if $X$ are disjoint.

$\Longleftarrow:$ Let $X_1,\cdots,X_n$ be the disjoint irreducible
components of $X,$ each of which is open and closed. If $(X,\mathcal
{O}_X)$ is reduced, then each $(X_i,\mathcal {O}_{X_i})$ is reduced.
But $X_i$ is irreducible, hence $(X_i,\mathcal {O}_{X_i})$ is
integral. $\forall p\in X,\exists i(1\leqslant i\leqslant n)$ such
that $p\in X_i,$ then $\mathcal {O}_{X,p}=\mathcal {O}_{X_i,p}$ is
an integral domain. So $(X,\mathcal {O}_X)$ is locally integral.
\end{proof}
\begin{cor}
Let $(X, \mathcal {O}_X)$ be a scheme such that $X$ is a Noetherian
topological space. If $(X, \mathcal {O}_X)$ is locally integral,
then the connected components of $X$ coincide with its irreducible
components.
\end{cor}
\begin{proof}
Let $X_1,\cdots,X_n$ be the irreducible components if $X.$ Since all
of them are disjoint with each other, each of them is open, closed
and connected in $X.$ Moreover each $X_i$ is a maximal connected
subset of $X.$ Indeed if $Y\subseteq X$ is connected, then
$Y=\bigcup\limits_{i=1}^n(Y\cap X_i),$ thus $\exists i(1\leqslant
i\leqslant n)$ such that $Y=Y\cap X_i\subseteq X_i.$
\end{proof}
\begin{prop}
Let $(X, \mathcal {O}_X)$ be a scheme, $\forall f\in \mathcal
{O}_X(X),$ define $$X_f=\{p\in X\mid f_p \text{ is a unit in }
\mathcal {O}_{X,p}\}.$$
\enum
\item[(1)]Let $(\varphi,\varphi^{\sharp}): (X, \mathcal {O}_X)\rightarrow (Y, \mathcal
{O}_Y)$ be a morphism of schemes, then $\forall g\in(Y, \mathcal
{O}_Y),$ we have
$\varphi^{-1}(Y_g)=\varphi^{-1}(Y)_{\varphi^{\sharp}(Y)(g)}=X_{\varphi^{\sharp}(Y)(g)}.$
\item[(2)]Suppose that $X$ can be covered by finitely many affine
open subschemes $(V_i=SpecA_i)_{i\in I}$ such that $V_i\cap
V_j(i,j\in I)$ can be covered by finitely many affine subschemes.
Set $A=\mathcal {O}_X(X),$ then $\forall f\in A,$ we have
$$\mathcal {O}_X(X_f)\cong A_f\cong \mathcal {O}_{SpecA}(D(f)).$$
\end{list}
\end{prop}
\begin{remark}
Let $A,B$ be two nonzero rings. Set $X=SpecA, Y=SpecB.$ Let $\alpha:
B\rightarrow A$ be a ring homomorphism, set $\varphi=Spec\alpha,$
then we have that $(\varphi,\varphi^{\sharp}): SpecA\rightarrow
SpecB$ is a morphism of affine schemes, and
$\alpha=\varphi^{\sharp}(SpecB).$ $\forall Y_g=D_B(g),$
$$\varphi^{-1}(D_B(g))=X_{\alpha(g)}=D_A(\alpha(g)).$$
\end{remark}
\begin{proof}\
\enum
\item[(1)]$\forall p\in X,$ we have
$\varphi^{\sharp}_p(g_{\varphi(p)})=(\varphi^{\sharp}(Y)(g))_p,$ for
\[ \xymatrix@R=0em@C=0.5em{
   {\mathcal {O}_{Y,\varphi(p)}} \ar[r]^-{(\varphi^{\sharp})_{\varphi(p)}} &
   {\varinjlim\limits_{p\in \varphi^{-1}(V)}\mathcal {O}_X(f^{-1}(V))=(f_{\ast}\mathcal
   {O}_X)_{f(p)}} \ar[r] & {\varinjlim\limits_{p\in U}\mathcal {O}_X(U)=\mathcal
   {O}_{X,p}}                                    \\
   {g_{\varphi(p)}} \ar@{|->}[r] & {\text{the equivalence class of }\varphi^{\sharp}(Y)(g)\text{ in }(f_{\ast}\mathcal
   {O}_X)_{f(p)}} \ar@{|->}[r] & {(\varphi^{\sharp}(Y)(g))_p} }  \]
Since $\varphi^{\sharp}_p$ is local, then $g_{\varphi(p)}$ is a unit
in $\mathcal {O}_{Y,\varphi(p)}$ iff $(\varphi^{\sharp}(Y)(g))_p$ is
a unit in $\mathcal {O}_{X,p}.$ Hence
$\varphi^{-1}(Y_g)=X_{\varphi^{\sharp}(Y)(g)}.$ Indeed $\forall p\in
X,$
\begin{eqnarray*}
p\in \varphi^{-1}(Y_g) & \Leftrightarrow & \varphi(p)\in
Y_g\Leftrightarrow g_{\varphi(p)} \text{ is a unit in }\mathcal
{O}_{Y,\varphi(p)}                                           \\
& \Leftrightarrow & (\varphi^{\sharp}(Y)(g))_p \text{ is a unit in
}\mathcal {O}_{X,p}                                          \\
& \Leftrightarrow & p\in X_{\varphi^{\sharp}(Y)(g)}.
\end{eqnarray*}
\item[(2)]Fix $f\in A\setminus \mathfrak{N}_A.$ The restriction $A=\mathcal
{O}_X(X)\stackrel{\rho}{\rightarrow} \mathcal {O}_X(X_f)$ maps $f$
to a unit, because $\forall p\in X_f, \left.f\right|_{X_f}(p)=f_p$
is a unit in $\mathcal {O}_{X,p},$ then we can define $g\in \mathcal
{O}_X(X_f): g(p)=f_p^{-1}$ in $\mathcal {O}_{X,p}.$ It's easily
checked that $g=(\left.f\right|_{X_f})^{-1},$ hence
$\left.f\right|_{X_f}$ is a unit in $\mathcal {O}_X(X_f).$ Then
$\rho$ induces a homomorphism if rings:
$A_f\stackrel{\alpha}{\rightarrow} \mathcal {O}X(X_f).$ We shall
show that $\alpha$ is an isomorphism. $\forall i\in I,$ put
$f_i=\left.f\right|_{V_i}\in \mathcal {O}_X(V_i),$ then $X_f\cap
V_i=D(f_i)$ in $V_i=SpecA_i.$

$\mathit{1^{\circ}}$ Injectivity of $\alpha:$

Take $a\in A_f,$ write $a=\frac{a}{f^n}, s\in A, n\geqslant 1.$
Suppose $\alpha(a)=0,$ i.e.
$\frac{\left.s\right|_{X_f}}{\left.f^n\right|_{X_f}}=0,$ thus
$\left.s\right|{X_f}=0$ for $\left.f\right|_{X_f}$ is a unit in
$\mathcal {O}_X(X_f).$ To show $a=0,$ it suffices to show $\exists
k\geqslant 1$ such that $f^ks=0.$ $\forall i\in I,
\left.s\right|_{V_i}\in ker(\mathcal {O}_X(V_i)\rightarrow\mathcal
{O}_X(X_f\cap V_i))$ for
$\left.(\left.s\right|_{V_i})\right|_{V_i\cap
X_f}=\left.(\left.s\right|_{X_f})\right|_{V_i\cap X_f}=0,$ i.e.
$$\left.s\right|_{V_i}\in ker(A_i\rightarrow \mathcal
{O}_{SpecA_i}(D(f_i)))=ker(A_i\rightarrow (A_i)_{f_i}).$$ Then
$\exists k_i\geqslant 1$ such that
$(\left.s\right|_{V_i})f_i^{k_i}=0,$ so
$\left.(f^{k_i}s)\right|_{V_i}=0.$ Put $k=\max\limits_{i\in I}k_i,$
then $\left.(f^ks)\right|_{V_i}=0.$ Thus $f^ks=0$ for $(V_i)_{i\in
I}$ is an open covering of $X.$

$\mathit{2^{\circ}}$ Surjectivity of $\alpha:$

$\forall t\in \mathcal {O}_X(X_f),$ we want to find $s\in\mathcal
{O}_X(X)=A$ and an integer $n\geqslant 1$ such that
$t=\frac{\left.s\right|_{X_f}}{\left.f^n\right|_{X_f}}.$ $\forall
i\in I,$ consider the restriction
\[ \xymatrix{
   {\mathcal {O}_X(X)} \ar[r] & {\mathcal {O}_X(V_i)} \ar[r]
   \ar@{=}[d]^-{\wr} & {\mathcal {O}_X(V_i\cap X_f)} \ar@{=}[d]^-{\wr}\\
   & A_i \ar[r] & {(A_i)_{f_i}} }  \]
$\left.t\right|_{V_i\cap X_f}\in (A_i)_{f_i},$ then $\exists t_i\in
A_i\cong \mathcal {O}_X(V_i)$ and $n_i\geqslant 1$ such that
$\left.t\right|_{V_i\cap X_f}=\frac{\left.t_i\right|_{V_i\cap
X_f}}{\left.f_i^{n_i}\right|_{V_i\cap X_f}}.$ Put
$n=\max\limits_{i\in I}n_i$, then $\left.t\right|_{V_i\cap
X_f}=(\left.t_i\right|_{V_i\cap X_f})(\left.f_i^n\right|_{V_i\cap
X_f})^{-1},$ hence $\left.t_i\right|_{V_i\cap
X_f}=(\left.t\right|_{V_i\cap X_f})(\left.f_i\right|_{V_i\cap
X_f})^n.$ Moreover $\forall i,j\in I,$ we have
$$\left.(\left.t_i\right|_{V_i\cap V_j})\right|_{V_i\cap V_j\cap X_f}=
(\left.t\right|_{V_i\cap V_j\cap X_f})(\left.f_i\right|_{V_i\cap
V_j\cap X_f})^n= \left.(\left.t_j\right|_{V_i\cap
V_j})\right|_{V_i\cap V_j\cap X_f}.$$ Thus
$\left.(\left.t_i\right|_{V_i\cap V_j}-\left.t_j\right|_{V_i\cap
V_j})\right|_{V_i\cap V_j\cap X_f}=0.$ Since $V_i\cap V_j$ can be
covered by finitely many affine open subschemes, then applying
$\mathit{1^{\circ}}$ by taking $X=V_i\cap V_j,$ we obtain that
$\exists m\geqslant 1$ such that $(\left.f\right|_{V_i\cap
V_j})^m(\left.t_i\right|_{V_i\cap V_j}-\left.t_j\right|_{V_i\cap
V_j})=0.$ Then
$$\left.((\left.f^m\right|_{V_i})t_i)\right|_{V_i\cap
V_j}=\left.((\left.f^m\right|_{V_j})t_j)\right|_{V_i\cap V_j},$$
hence $\exists !s\in \mathcal {O}_X(X)$ such that
$\left.s\right|_{V_i}=\left.f^m\right|_{V_i}t_i.$ Then
$$\left.t\right|_{V_i\cap X_f}\left.f^n\right|_{V_i\cap X_f} = \left.t_i\right|_{V_i\cap
X_f}=\frac{\left.s\right|_{V_i\cap X_f}}{\left.f^m\right|_{V_i\cap
X_f}}.$$ Therefore
$t=\frac{\left.s\right|_{X_f}}{\left.f^{m+n}\right|_{X_f}}$ because
$(X_f\cap V_i)_{i\in I}$ forms an open covering of $X_f.$
\end{list}
\end{proof}
\begin{prop}
A scheme $(X, \mathcal {O}_X)$ is affine iff we can find
$f_1,\cdots,f_n\in \mathcal {O}_X(X)$ such that:
\enum
\item[(1)]$(f_1,\cdots,f_n)_{\mathcal {O}_X(X)}=\mathcal {O}_X(X);$
\item[(2)]Each open subscheme $(X_{f_i},\left.\mathcal
{O}_X\right|_{X_{f_i}})$ is affine.
\end{list}
\end{prop}
\begin{proof}
$\Longrightarrow:$ is direct by taking $n=1$ and $f_1=1\in \mathcal
{O}_X(X).$

$\Longleftarrow:$ Put $A=\mathcal {O}_X(X),$ then
\[ \xymatrix@R=0em{
   A \ar[r]^-{\alpha} & {\mathcal {O}_X(X)} \\
   x \ar@{|->}[r] & x }  \]
induced a morphism of schemes $\varphi: X\rightarrow SpecA.$ We
shall show that $\varphi$ is an isomorphism of schemes.

By hypothesis, in the ring $A,$ we have $(f_1,\cdots,f_n)_A=(1),$
thus
$\emptyset=V(1)=V((f_1,\cdots,f_n))=\bigcap\limits_{i=1}^nV(f_i),
SpecA=\bigcup\limits_{i=1}^nD(f_i).$ We have
$$X_{f_i}=X_{\alpha(f_i)}=X_{\varphi^{\sharp}(SpecA)(f_i)}=\varphi^{-1}(SpecA)_{f_i}=\varphi^{-1}(D(f_i)),$$
then $X=\varphi^{-1}(SpecA)$ can be covered by
$(X_{f_i})_{1\leqslant i\leqslant n}$ which are affine open
subschemes of $X.$ Write $X_{f_i}=SpecB_i(1\leqslant i\leqslant n),$
put $\psi_i: X_{f_i} \hookrightarrow X,$ then we have
$\psi_i^{\sharp}: \mathcal {O}_X\rightarrow (\psi_i)_{\ast}\mathcal
{O}_{X_{f_i}}.$ $\forall j(1\leqslant j\leqslant n),$ put
$f_j^{\prime}=\psi_i^{\sharp}(X)(f_j)=\left.f_j\right|_{X_{f_i}}\in\mathcal
{O}_X(X_{f_i})=B_i.$ Then
$$X_{f_i}\cap X_{f_j}=\psi_i^{-1}(X_{f_j})=(X_{f_i})_{\psi_i^{\sharp}(X)(f_j)}
=(X_{f_i})_{f_j^{\prime}}=(SpecB_i)_{f_j^{\prime}}=D(f_j^{\prime})$$
which is affine in $SpecB_i.$

In other words, we have
\enum
\item[(a)]$X$ can be covered by affine open subschemes
$X_{f_1},\cdots,X_{f_n}.$
\item[(b)]$\forall 1\leqslant i,j\leqslant n, X_{f_i}\cap X_{f_j}$
is affine.
\end{list}
Hence $\forall i(1\leqslant i\leqslant n), \mathcal
{O}_X(X_{f_i})\cong A_{f_i}.$ Set $\varphi_i=\varphi_{X_{f_i}}.$
Since $X_{f_i}$ is affine, then we obtain the isomorphism of schemes
$\varphi_i: X_{f_i}\rightarrow SpecA_{f_i}.$ Since
$\varphi_i=\varphi_{X_{f_i}}, X=\bigcup\limits_{i=1}^nX_{f_i},$
therefore $\varphi$ is an isomorphism.
\end{proof}
\begin{Def}
Let $(X, \mathcal {O}_X)$ be a scheme.
\enum
\item[(1)]It is called locally Noetherian if it can be covered by
affine open subschemes $U_i=SpecA_i(i\in I)$ such that each $A_i$ is
Noetherian.
\item[(2)]It is called Noetherian if it is locally Noetherian and is
quasi-compact. Equivalently, $(X, \mathcal {O}_X)$ can be covered by
finitely many affine open subschemes $U_i=SpecA_i$ with $A_i$ a
Noetherian ring.
\end{list}
\end{Def}
\begin{remarks}\
\enum
\item[(1)]If $(X, \mathcal {O}_X)$ is a Noetherian scheme, then $X$
is a Noetherian topological space. Indeed $X$ is a finite union of
$U_i=SpecA_i,$ each of which is a Noetherian topological space. The
converse is false in general.
\item[(2)]If $A$ is a nonzero Noetherian ring, then $SpecA$ is a
Noetherian scheme.
\end{list}
\end{remarks}
\begin{prop}
Let $(X, \mathcal {O}_X)$ be a locally Noetherian scheme, then for
any affine open subscheme $V=SpecA,$ necessarily $A$ is a Noetherian
ring.
\end{prop}
\begin{proof}
By definition, we can find a covering of $X$ by affine open
subschemes $X_i=SpecA_i(i\in I)$ with each $A_i$ a Noetherian ring.
Let $V=SpecA$ be a nonempty affine open subscheme of $X.$ We shall
show that $A$ is a Noetherian ring.

Each $V\cap X_i$ is open in $X_i=SpecA_i,$ thus can be covered by
$D(f_{ij})(j\in J_i)$ in $X_i.$ But $D(f_{ij})=Spec(A_i)_{ij},$ and
each $(A_i)_{ij}$ is a Noetherian ring. Up to notation, we may
assume that $V=SpecA$ is covered by affine open subschemes
$V_i=SpecA_i(i\in I)$ with each $A_i$ a Noetherian ring. Since each
$V_i=SpecA_i$ is open in $V=SpecA,$ then $V_i$ can be covered by
$D(f_{ij})(j\in J_i)$ with $f_{ij}\in A.$ Define $\varphi_i:
V_i\hookrightarrow V,$ $\varphi_i^{\sharp}:\mathcal {O}_V\rightarrow
(\varphi_i)_{\ast}\mathcal {O}_{V_i}=\left.\mathcal
{O}_V\right|_{V_i}.$ Then
\[ \xymatrix@R=0em{
   {\varphi_i^{\sharp}(V):A=\mathcal {O}_V(V)} \ar[r] & {\mathcal
   {O}_{V_i}(V_i)=A_i}                                            \\
   f_{ij} \ar@{|->}[r] & {\left.f_{ij}\right|_{V_i}=:f_{ij}^{\prime}} }  \]
then we have $D_A(f_{ij})=D_A(f_{ij})\cap
V_i=D_{A_i}(f_{ij}^{\prime}).$ Hence $V_i$ can be covered by
$$D_A(f_{ij})=SpecA_{f_{ij}}=D_{A_i}(f_{ij}^{\prime})=Spec(A_i)_{f_{ij}^{\prime}}.$$
But each $(A_i)_{f_{ij}^{\prime}}$ is Noetherian, so is
$A_{f_{ij}}.$ Hence up to notation, we may assume $V=SpecA$ can be
covered by $D(f_i)(i\in I)$ with $A_{f_i}$ a Noetherian ring. Note
that $V=SpecA$ is quasi-compact, we may assume $I=\{1,\cdots,n\}.$
Fix $J$ an ideal of $A.$ For each $i(1\leqslant i\leqslant n),$
$A_{f_i}$ is Noetherian, so $JA_{f_i}$ is a finitely generated ideal
of $A_{f_i}.$ Then we can write
$JA_{f_i}=(\frac{a_{ij}}{f_i^{k_{ij}}})_{j\in J_i}$ with $J_i$ a
finite set.

We shall show that $J=(a_{ij})_{i\in I \atop j\in J_i}$ in $A.$
$\forall x\in J,\forall i(1\leqslant i\leqslant n), \frac{x}{1}\in
JA_{f_i},$ so we can write $\frac{x}{1}=\sum\limits_{j\in
J_i}\frac{b_{ij}}{f_i^{l_{ij}}}\cdot\frac{a_{ij}}{f_i^{k_{ij}}}(k_{ij}\in
A).$ Then we can find an integer $k_i$ large enough such that
$f_i^{k_i}x=\sum\limits_{j\in J_i}c_{ij}a_{ij}$ with $c_{ij}\in A.$
However $SpecA=V=\bigcup\limits_{i\in I}D(f_i)=\bigcup\limits_{i\in
I}D(f_i^{k_i}),$ hence $\emptyset=\bigcap\limits_{i\in
I}V(f_i^{k_i})=V((f_1^{k_1},\cdots,f_n^{k_n})),$ thus
$(1)=(f_1^{k_1},\cdots,f_n^{k_n}).$ Then we can write
$1=\sum\limits_{i\in I}b_if_i^{k_i}(b_i\in A).$ Therefore
$$x=\sum\limits_{i\in I}b_if_i^{k_i}x=\sum\limits_{i\in I \atop j\in
J_i}(b_ic_{ij})a_{ij}.$$ Hence $J=(a_{ij})_{i\in I \atop j\in J_i}.$
\end{proof}
\begin{cor}
Let $X$ be a locally Noetherian scheme, then every nonempty open
subscheme of $X$ is a locally Noetherian scheme.
\end{cor}
\begin{proof}
Let $V$ be an open subscheme of $X,$ and $U$ an affine open
subscheme of $V.$ Then $U$ is an affine open subscheme of $X.$ Let
$U=SpecA,$ then $A$ is a Noetherian ring by virtue of the
proposition above. Therefore $V$ is a locally Noetherian scheme.
\end{proof}
\begin{cor}
Let $A$ be a nonzero ring, then the affine scheme $X=SpecA$ is
(locally) Noetherian iff $A$ is Noetherian.
\end{cor}
\begin{proof}\

$\Longleftarrow:$ If $A$ is a Noetherian ring, then $SpecA$ is
locally Noetherian by definition. Moreover it is Noetherian because
$SpecA$ is quasi-compact.

$\Longrightarrow:$ If $SpecA$ is Noetherian, then for any affine
open subscheme $V=SpecB$ of $SpecA,$ necessarily $B$ is a Noetherian
ring. In particular, $SpecA$ is an affine open subscheme of itself.
\end{proof}
\begin{Def}
Let $\varphi: X\rightarrow Y$ be a morphism of schemes.
\enum
\item[(1)]We say that $\varphi$ is quasi-compact, if there exists an
open covering of $Y$ by affine open subschemes $Y_i=SpecA_i(i\in I)$
such that each $\varphi^{-1}(Y_i)$ is quasi-compact.
\item[(2)]We say that $\varphi$ is affine, if $Y$ can be covered by
affine open subschemes $Y_i=SpecA_i(i\in I)$ such that each
$\varphi^{-1}$ is affine.
\item[(3)]We say that $\varphi$ is locally of finite type, if $Y$
can be covered by affine open subschemes $Y_i=SpecA_i(i\in I)$ such
that each $\varphi^{-1}(Y_i)$ can be covered by affine open
subschemes $X_{ij}=SpecA_{ij}(j\in J_i)$ with each $A_{ij}$ a
finitely generated $A_i$-algebra.
\item[(4)]We say that $\varphi$ is of finite type, if it is quasi-compact and is locally of finite
type.
\item[(5)]We say that $\varphi$ is finite, if $Y$ can be covered by
affine open subschemes $Y_i=SpecA_i(i\in I)$ such that $\forall i\in
I,$ we have $\varphi^{-1}(Y_i)=SpecB_i$ with each $B_i$ an
$A_i$-algebra and also a finitely generated $A_i$-module.
\end{list}
\end{Def}
\begin{prop}
Let $\varphi: X\rightarrow Y$ be a morphism of schemes, then
\enum
\item[(1)]$\varphi$ is quasi-compact iff for any open quasi-compact
subset $V$ of $Y,$ $\varphi^{-1}(V)$ is quasi-compact.
\item[(2)]$\varphi$ is affine iff for any affine open subscheme
$V=SpecA$ of $Y,$ $\varphi^{-1}(V)$ is affine.
\item[(3)]$\varphi$ is locally of finite type iff for any open affine
subscheme $V=SpecA$ of $Y,$ $\varphi^{-1}(V)$ can be covered by
affine open subschemes $U_i=SpecA_i(i\in I)$ such that each $A_i$ is
a finitely generated $A$-algebra.
\item[(4)]$\varphi$ is of finite type iff for any open affine
subscheme $V=SpecA$ of $Y,$ $\varphi^{-1}(V)$ can be covered by
finitely many affine open subschemes $U_i=SpecA_i(i\in I)$ such that
each $A_i$ is a finitely generated $A$-algebra.
\item[(5)]$\varphi$ is finite iff for any open affine subscheme
$V=SpecA$ of $Y,$ $\varphi^{-1}(V)=SpecB$ for some $A$-algebra $B$
which is finitely generated as an $A$-module.
\end{list}
\end{prop}
\begin{proof}
The "if" parts are direct by definition, so we only prove the "only
if" parts.
\enum
\item[(1)]Suppose that $Y$ can be covered by affine open subschemes
$Y_i=SpecA_i(i\in I)$ such that each $\varphi^{-1}(Y_i)$ is
quasi-compact. Let $V$ be an open quasi-compact subset of $Y.$ For
any $i\in I, V\cap Y_i$ is open in $Y_i,$ hence $V\cap Y_i$ can be
covered by $D(f_{ij})(j\in J_i)$ with $f_{ij}\in A_i.$ Since
$\varphi^{-1}(Y_i)$ is open and quasi-compact, we can cover
$\varphi^{-1}(Y_i)$ by finitely many affine open subschemes
$X_{ik}=SpecA_{ik}(k\in K_i).$ Put
$f_{ijk}=\left.\varphi^{\sharp}(Y_i)(f_{ij})\right|_{X_{ik}},$ then
we have
\begin{eqnarray*}
& & \varphi^{-1}(D(f_{ij}))\cap X_{ik} \\
& = & (\left.\varphi\right|_{X_{ik}})^{-1}(D(f_{ij}))                  \\
& = & (X_{ik})_{(\left.\varphi\right|_{X_{ik}})^{\sharp}(Y_i)(f_{ij})} \\
& = & (X_{ik})_{\left.\varphi^{\sharp}(Y_i)(f_{ij})\right|_{X_{ik}}}   \\
& = & (X_{ik})_{f_{ijk}}=D(f_{ijk})
\end{eqnarray*}
in $X_{ik},$ where $\left.\varphi\right|_{X_{ik}}: X_{ik}\rightarrow
Y_i$ is the restriction of $\varphi.$ Since $V$ is quasi-compact, we
can find a finite subset $\widetilde{I}\subseteq I,$ and for each
$i\in\widetilde{I},$ a finite subset $\widetilde{J_i}\subseteq J_i$
such that $V=\bigcup\limits_{i\in\widetilde{I} \atop
j\in\widetilde{J_i}}D(f_{ij}).$ Then
$$\varphi^{-1}(V)=\bigcup\limits_{i\in\widetilde{I} \atop j\in\widetilde{J_i}}\varphi^{-1}(D(f_{ij}))
=\bigcup\limits_{i\in\widetilde{I} \atop j\in\widetilde{J_i}, k\in
K_i}\varphi^{-1}(D(f_{ij}))\cap
X_{ik}=\bigcup\limits_{i\in\widetilde{I} \atop j\in\widetilde{J_i},
k\in K_i}D(f_{ijk})$$ is quasi-compact, because
$\widetilde{I},\widetilde{J_i},K_i$ are finite sets and each
$D(f_{ijk})$ is quasi-compact.
\item[(2)]Suppose that $Y$ can be covered by
affine open subschemes $Y_i=SpecA_i(i\in I)$ such that each
$\varphi^{-1}$ is affine. Let $V=SpecA$ be an affine open subscheme
of $Y,$ we shall show that $\varphi^{-1}(V)$ is affine.

$\forall i\in I,$ $V\cap Y_i$ is open in $Y_i=SpecA_i,$ hence it can
be covered by $D(f_{ij})(j\in J_i)$ with $f_{ij}\in A_i.$ Then
$$\varphi^{-1}(D(f_{ij}))=\varphi^{-1}((Y_i)_{f_{ij}})=\varphi^{-1}(Y_i)_{\varphi^{\sharp}(Y_i)(f_{ij})}.$$
Set $f_{ij}^{\prime}=\varphi^{\sharp}(Y_i)(f_{ij}),$ then
$\varphi^{-1}(D(f_{ij}))=Spec(B_i)_{f_{ij}^{\prime}},$ where
$\varphi^{-1}(Y_i)=SpecB_i.$

Up to notation, we can suppose that $V$ can be covered by affine
open subschemes $V_i=SpecA_i(i\in I)$ such that each
$\varphi^{-1}(V_i)=SpecB_i.$ Each $V_i=SpecA_i$ is open in
$V=SpecA,$ hence can be covered by $D(f_{ij})(j\in J_i)$ with
$f_{ij}\in A.$ Put $f_{ij}^{\prime}=\left.f_{ij}\right|_{V_i},$ then
$$D_A(f_{ij})=D_A(f_{ij})\cap V_i=D_A(f_{ij}^{\prime}),$$ and
\begin{eqnarray*}
\varphi^{-1}(D_A(f_{ij})) & = &
\varphi^{-1}(D_A(f_{ij}^{\prime}))=\varphi^{-1}((V_i)_{f_{ij}^{\prime}})\\
& = & \varphi^{-1}(V_i)_{\varphi^{\sharp}(V_i)(f_{ij}^{\prime})}  \\
& = & Spec(B_i)_{\widetilde{f_{ij}}},
\end{eqnarray*}
where $\widetilde{f_{ij}}:=\varphi^{\sharp}(V_i)(f_{ij}^{\prime}).$

Up to notation, we can suppose that $V=SpecA$ can be covered by
$D(f_i)(i\in I)$ such that $\varphi^{-1}(D(f_i))=SpecB_i.$ But
$V=SpecA$ is affine, so it is quasi-compact, and then we can suppose
$I=\{1,\cdots,n\}.$ $V=SpecA=\bigcup\limits_{i=1}^nD(f_i),$
$\varphi^{-1}(D(f_i))=SpecB_i(1\leqslant i \leqslant n),$ thus
$\varphi^{-1}(V)=\bigcup\limits_{i=1}^nSpecB_i.$ $\forall i\in I,$
put $g_i=\varphi^{\sharp}(V)(f_i)\in \mathcal
{O}_X(\varphi^{-1}(V)).$ $\emptyset=\bigcap\limits_{i=1}^nV(f_i),$
hence $f_1,\cdots,f_n$ generate $A=\mathcal {O}_Y(V).$ We have
\[ \xymatrix@R=0em{
   {\varphi^{\sharp}(V):A=\mathcal {O}_Y(V)} \ar[r] & {\varphi_{\ast}\mathcal {O}_X(V)=\mathcal
   {O}_X(\varphi^{-1}(V))=:B}                                  \\
   f_i \ar@{|->}[r] & g_i }  \]
Then we know that $g_1,\cdots,g_n$ generate $B.$ $\forall i\in I,$
we have that
$$SpecB_i=\varphi^{-1}(D(f_i))=\varphi^{-1}(V_{f_i})=(\varphi^{-1}(V))_{\varphi^{\sharp}(V)(f_i)}=(\varphi^{-1}(V))_{g_i}$$
is an open affine scheme.

In conclusion, $\exists g_1,\cdots,g_n\in \left.\mathcal
{O}_X\right|_{\varphi^{-1}(V)}(\varphi^{-1}(V))=B$ such that \enum
\item[(a)]$(g_1,\cdots,g_n)_B=B;$
\item[(b)]Each open subscheme $\varphi^{-1}(V)_{g_i}$ is affine.
\end{list}

Therefore $\varphi^{-1}(V)$ is an affine scheme.
\item[(3)]By definition, $Y$ can be covered by affine open subschemes $Y_i=SpecA_i(i\in I)$ such
that each $\varphi^{-1}(Y_i)$ can be covered by affine open
subschemes $X_{ij}=SpecA_{ij}(j\in J_i)$ with each $A_{ij}$ a
finitely generated $A_i$-algebra. Let $V=SpecA$ be an open affine
subscheme of $Y.$ $\forall i\in I, Y_i\cap V$ is open in $Y_i,$ thus
can be covered by $D(f_{ik})=Spec(A_i)_{f_{ik}}(k\in K_i).$
$$\varphi^{-1}(D(f_{ik}))=\varphi^{-1}((Y_i)_{f_{ik}})=
\varphi^{-1}(Y_i)_{\varphi^{\sharp}(Y_i)(f_{ik})}=
\varphi^{-1}(Y_i)_{f_{ik}^{\prime}},$$ where we put
$f_{ik}^{\prime}=\varphi^{\sharp}(Y_i)(f_{ik}).$
$$\varphi^{-1}(Y_i)_{f_{ik}^{\prime}}\cap X_{ij}=D(f_{ijk}^{\prime})=Spec(A_{ij})_{f_{ijk}^{\prime}},$$
where
$f_{ijk}^{\prime}=\left.f_{ik}^{\prime}\right|_{X_{ij}}=\left.\varphi^{\sharp}(Y_i)(f_{ik})\right|_{X_{ij}}.$
Since $\rho_{\varphi^{-1}(Y_i)\,X_{ij}}\circ\varphi^{\sharp}(Y_i)$
maps $f_{ik}$ to $f_{ijk}^{\prime},$ and $A_{ij}$ is a finitely
generated $A_i$-algebra, hence we obtain that
$(A_{ij})_{f_{ijk}^{\prime}}$ is a finitely generated
$(A_i)_{f_{ik}}$-algebra.

Up to notation, we can suppose that $V=SpecA$ can be covered by
$V_i=SpecA_i(i\in I)$ such that each $\varphi^{-1}(V_i)$ can be
covered by $X_{ij}=SpecA_{ij}(j\in J_i)$ with each $A_{ij}$ a
finitely generated $A_i$-algebra. Each $V_i$ is open in $V,$ thus
can be covered by $D(f_{ik})=Spec(A_i)_{f_{ik}}(k\in K_i).$
$$SpecA_{f_{ik}}=D_A(f_{ik})=D_A(f_{ik})\cap V_i=D_{A_i}(\widetilde{f_{ik}})=Spec(A_i)_{\widetilde{f_{ik}}},$$
where $\widetilde{f_{ik}}:=\left.f_{ik}\right|_{V_i}.$
\begin{eqnarray*}
& & \varphi^{-1}(D(\widetilde{f_{ik}}))\cap X_{ij}  \\
& = & \varphi^{-1}((V_i)_{\widetilde{f_{ik}}})\cap X_{ij} \\
& = &
\varphi^{-1}(V_i)_{\varphi^{\sharp}(V_i)(\widetilde{f_{ik}})}\cap
X_{ij}                                              \\
& = & D(f_{ijk}^{\prime})                           \\
& = & Spec(A_{ij})_{f_{ijk}^{\prime}},
\end{eqnarray*}
where
$f_{ijk}^{\prime}=\left.\varphi^{\sharp}(V_i)(\widetilde{f_{ik}})\right|_{X_{ij}}.$
Again, we can check that $(A_{ij})_{f_{ijk}^{\prime}}$ is a finitely
generated $(A_i)_{\widetilde{f_{ik}}}$-algebra. Since
$SpecA_{f_{ik}}=Spec(A_i)_{\widetilde{f_{ik}}},$ then
$A_{f_{ik}}\cong (A_i)_{\widetilde{f_{ik}}},$ hence
$(A_i)_{\widetilde{f_{ik}}}$ is a finitely generated
$A_{f_{ik}}$-algebra, and therefore is a finitely generated
$A$-algebra.

Up to notation, we have proved that $\varphi^{-1}(V)$ can be covered
by affine open subschemes $U_i=SpecA_i(i\in I)$ such that each $A_i$
is a finitely generated $A$-algebra.
\item[(4)]It's direct by $(3)$ and $(1).$
\item[(5)]By definition, $Y$ can be covered by
affine open subschemes $Y_i=SpecA_i(i\in I)$ such that $\forall i\in
I,$ we have $\varphi^{-1}(Y_i)=SpecB_i$ for some $A_i$-algebra $B_i$
which is also a finitely generated $A_i$-module. Let $V=SpecA$ be an
open affine subscheme of $Y.$ $\forall i\in I, Y_i\cap V$ is open in
$Y_i,$ thus can be covered by $D(f_{ij})=Spec(A_i)_{f_{ij}}(j\in
J_i).$
$$\varphi^{-1}(D_{A_i}(f_{ij}))=\varphi^{-1}(Y_i)_{\varphi^{\sharp}(Y_i)(f_{ij})}=
D_{B_i}(f_{ij}^{\prime})=Spec(B_i)_{f_{ij}^{\prime}},$$ where
$f_{ij}^{\prime}:=\varphi^{\sharp}(Y_i)(f_{ij}).$
$$\varphi^{\sharp}(Y_i): A_i=\mathcal {O}_Y(Y_i)\rightarrow\mathcal {O}_X(\varphi^{-1}(Y_i))=B_i$$
maps $f_{ij}$ to $f_{ij}^{\prime},$ and $B_i$ is a finitely
generated $A_i$-module, thus $(B_i)_{f_{ij}^{\prime}}$ is an
$(A_i)_{f_{ij}^{\prime}}$-algebra and finitely generated as an
$(A_i)_{f_{ij}^{\prime}}$-module.

Up to notation, we may assume that $V=SpecA$ can be covered by
affine open subschemes $V_i=SpecA_i(i\in I)$ such that each
$\varphi^{-1}(V_i)=SpecB_i$ for some $A_i$-algebra $B_i$ which is
also a finitely generated $A_i$-module. $\forall i\in I, V_i$ is
open in $V,$ thus can be covered by
$D(f_{ij})=Spec(A_i)_{f_{ij}}(j\in J_i).$ Put
$f_{ij}^{\prime}=\left.f_{ij}\right|_{V_i},$ then
$$D_A(f_{ij})=D_A(f_{ij})\cap V_i=D_{A_i}(f_{ij}^{\prime})=Spec(A_i)_{f_{ij}^{\prime}},$$
thus $A_{f_{ij}}\cong (A_i)_{f_{ij}^{\prime}}.$
\begin{eqnarray*}
\varphi^{-1}(D_A(f_{ij})) & = &
\varphi^{-1}(D_{A_i}(f_{ij}^{\prime}))                          \\
& = & \varphi^{-1}(V_i)_{\varphi^{\sharp}(V_i)(f_{ij}^{\prime})}\\
& = & D_{B_i}(\widetilde{f_{ik}})                               \\
& = & Spec(B_i)_{\widetilde{f_{ik}}},
\end{eqnarray*}
where we put
$\widetilde{f_{ik}}=\varphi^{\sharp}(V_i)(f_{ij}^{\prime}).$ Again,
we can check that each $(B_i)_{\widetilde{f_{ik}}}$ is an
$(A_i)_{f_{ij}^{\prime}}$-algebra and finitely generated as an
$(A_i)_{f_{ij}^{\prime}}$-module. Thus $(B_i)_{\widetilde{f_{ik}}}$
is an $(A_i)_{f_{ij}}$-algebra and finitely generated as an
$(A_i)_{f_{ij}}$-module.

Up to notation, we can suppose that $V=SpecA$ can be covered by
$D(f_i)=SpecA_{f_i}(i\in I)$ such that each
$\varphi^{-1}(D(f_i))=SpecB_i$ for some $A_{f_i}$-algebra $B_i$
which is also a finitely generated $A_{f_i}$-module. But $V=SpecA$
is affine, thus is quasi-compact. Then we can suppose that
$I=\{1,\cdots,n\}.$ $V=SpecA=\bigcup\limits_{i=1}^nD(f_i),
\varphi^{-1}(D(f_i))=SpecB_i(1\leqslant i\leqslant n),$ thus
$\varphi^{-1}(V)=\bigcup\limits_{i=1}^nSpecB_i.$ $\forall i\in I,$
put $g_i=\varphi^{\sharp}(V)(f_i)\in \mathcal
{O}_X(\varphi^{-1}(V)).$ Since
$\emptyset=\bigcap\limits_{i=1}^nV(f_i),$ hence $f_1,\cdots,f_n$
generate $A=\mathcal {O}_Y(V).$ Consider
\[ \xymatrix@R=0em{
   {A=\mathcal {O}_Y(V)} \ar[r]^-{\varphi^{\sharp}} & {\varphi_{\ast}\mathcal {O}_X(V)=\mathcal
   {O}_X(\varphi^{-1}(V)):=B}                                 \\
   f_i \ar@{|->}[r] & g_i }  \]
then we know that $(g_1,\cdots,g_n)_B=B.$ $\forall i\in I,$
$$SpecB_i=\varphi^{-1}(D(f_i))=\varphi^{-1}(V_{f_i})=\varphi^{-1}(V)_{\varphi^{\sharp}(V)(f_i)}=\varphi^{\sharp}(V)_{g_i}$$
is an open affine subscheme. Hence $\varphi^{-1}(V)$ is an affine
scheme and $\varphi^{-1}=B.$ $\forall i\in I, B_{g_i}=B_i$ is an
$A_{f_i}$-algebra and finitely generated as an $A_{f_i}$-module.

Put $\psi=\varphi^{\sharp},$ $\forall i\in I,$ we have
\[ \xymatrix{
   A \ar[r]^-{\psi} \ar[d] & B \ar[d]            \\
   {A_{f_i}} \ar[r]^-{\psi_i} & {B_{g_i}=B_i} }  \]
Suppose that $\{\frac{b_{ij}}{g_i^{k_{ij}}}\}_{j\in J_i}$ are the
generators of $B_{g_i}$ as an $A_{f_i}$-module with each $J_i$ a
finite set, and $b_{ij}\in B.$

$\forall g\in B,$ $\frac{g}{1}$ is in $B_{g_i},$ hence $\exists
a_{ij}\in A_{f_i}$ such that $\frac{g}{1}=\sum\limits_{j\in
J_i}\frac{b_ij}{g_i^{k_{ij}}}\psi_i(a_{ij}).$
$\frac{1}{g_i^{k_{ij}}}=\psi_i(\frac{1}{f_i^{k_j}}),$ hence we can
write $\frac{g}{1}=\sum\limits_{j\in
J_i}b_{ij}\psi_i(\frac{c_{ij}}{f_i^{t_{ij}}})=\sum\limits_{j\in
J_i}b_{ij}\frac{\psi(c_{ij})}{\psi(f_i^{t_{ij}})}$ with $c_{ij}\in
A$ and $t_{ij}\geqslant 0$ being integers. We can find an $m$ large
enough such that $g=\sum\limits_{j\in
J_i}b_{ij}\frac{\psi(d_{ij})}{\psi(f_i^m)}$ with $d_{ij}\in A.$
Since $D(f_i^m)=D(f_i),$ hence $\exists x_1,\cdots,x_n\in A$ such
that $1=\psi(\sum\limits_{i=i}^nx_if_i^m).$ Then
\begin{eqnarray*}
g& = & g\cdot 1=\sum\limits_{i\in I}\psi(x_i)\psi(f_i^m)g     \\
 & = & \sum\limits_{i\in I}\psi(x_i)(\sum\limits_{j\in J_i}b_{ij}\psi(d_{ij}))\\
 & = & \sum\limits_{i\in I}\sum\limits_{j\in J_i}b_{ij}[\psi(d_{ij})\psi(x_i)].
\end{eqnarray*}
Hence $\{b_{ij}\}_{i\in I \atop j\in J_i}$ generate $B$ as an
$A$-module. Therefore $B$ is a finitely generated $A$-module.
\end{list}
\end{proof}
\begin{prop}
Let $p: X\rightarrow S$ and $q: Y\rightarrow S$ be two morphisms of
schemes, where $q$ is locally of finite type. Let $f,g: X\rightarrow
Y$ be two morphisms of schemes such that the following diagram is
commutative
\[ \xymatrix{
   X \ar[rr]^-{f,g} \ar[dr]_p & & Y \ar[dl]^q \\
   & S }  \]
Let $x\in X$ such that $y=f(x)=g(x)$ and $f^{\sharp}_x,g^{\sharp}_x:
\mathcal {O}_{Y,y}\rightarrow \mathcal {O}_{X,x}$ coincide. Then
there exists a neighborhood $U$ of $x$ in $X$ such that
$\left.f\right|_U=\left.g\right|_U.$
\end{prop}
\begin{proof}
Set $s=p(x)=q(y)\in S,$ and $W=SpecC$ an affine neighborhood of $s$
in $S.$ Since $q$ is locally of finite type, then $\exists
V=SpecB\subseteq q^{-1}(W)$ an affine neighborhood of $y$ in $Y$
such that $B$ is a finitely generated $C$-algebra, and $\exists
U=SpecA\subseteq f^{-1}(V)\cap g^{-1}(V)$ an affine neighborhood of
$x$ in $X.$ Set $$\varphi=\rho_{f^{-1}(V)\,U}\circ f^{\sharp}(V),
\psi=\rho_{g^{-1}(V)\,U}\circ g^{\sharp}(V).$$ Let $P, Q, R$ be
respectively the prime ideals corresponding to $x, y, s$ in $A, B,
C.$ Since $y=f(x)=g(x),$ $\left.f\right|_U$ and $\left.g\right|_U$
are induced by $\varphi$ and $\psi,$ hence
$Q=\varphi^{-1}(p)=\psi^{-1}(p).$ Since $f^{\sharp}_x=g^{\sharp}_y,$
then $\varphi_p=\psi_p: B_Q\rightarrow A_P.$

Set $B=C[b_1,\cdots,b_n], b_i\in B(1\leqslant i\leqslant n).$
$\forall i(1\leqslant i\leqslant n),
\varphi_p(\frac{b_i}{1})=\psi_p(\frac{b_i}{1}),$ then
$\frac{\varphi(b_i)}{1})=\frac{\psi(b_i)}{1})$ in $A_p.$ Thus
$\exists s_i\in A\setminus p,$ such that
$s_i(\varphi(b_i)-\psi(b_i))=0.$ Put $t=\prod\limits_{i=1}^ns_i,$
then $t(\varphi(b_i)-\psi(b_i))=0, \forall 1\leqslant i\leqslant n.$
Hence in $A_t,$ $ \varphi_p(\frac{b_i}{1})=\psi_p(\frac{b_i}{1}),
\forall i(1\leqslant i\leqslant n).$ Therefore
$B\stackrel{\varphi,\psi}{\longrightarrow}
A\stackrel{\tau}{\hookrightarrow} A_t,$ we have $\tau\circ\varphi$
and $\tau\circ\psi$ are the same. $\left.f\right|_{D(t)}$ and
$\left.g\right|_{D(t)}$ are induced respectively by
$\tau\circ\varphi$ and $\tau\circ\psi,$ hence
$\left.f\right|_{D(t)}=\left.g\right|_{D(t)},$ and $D(t)$ is the
desired neighborhood of $x.$
\end{proof}
\begin{prop}
Let $p: X\rightarrow S$ and $q: Y\rightarrow S$ be two morphisms of
schemes, where $q$ is locally of finite type. Suppose that $S$ is a
locally Noetherian scheme. Let $x\in X$ and $y\in Y$ be such that
$s=p(x)=q(y).$ Let $\varphi:\mathcal {O}_{Y,y}\rightarrow \mathcal
{O}_{X,x}$ be a homomorphism such that the following diagram is
commutative
\[ \xymatrix{
   {\mathcal {O}_{X,x}} & & {\mathcal {O}_{Y,y}} \ar[ll]_{\varphi}\\
   & {\mathcal {O}_{S,s}} \ar[ul]^{p^{\sharp}_x}
   \ar[ur]_{q^{\sharp}_y} }  \]
Then there exists a neighborhood $U$ of $x$ in $X$ and a morphism of
schemes $f: U\rightarrow Y$ such that $y=f(x),
\varphi=f^{\sharp}_x,$ and
\[ \xymatrix{
   U \ar[rr]^f \ar[dr]_p & & Y \ar[dl]^q \\
   & S }  \]
is commutative.
\end{prop}
\begin{proof}
Let $W=SpecC$ be an affine neighborhood of $s$ in $S.$ Since $q$ is
locally of finite type and $q(y)=s,$ then we can find an affine
neighborhood $V=SpecB$ of $y$ in $Y$ such that $V\subseteq
q^{-1}(W),$ and $B$ is finitely generated as a $C$-algebra. Since
$p(x)=s,$ thus there exists an affine neighborhood $U=SpecA$ of $x$
in $X$ such that $U\subseteq p^{-1}(W).$ Let $P,Q,R$ be the prime
ideals corresponding to $x,y,s$ in $A,B,C$ respectively. Then
$A_P=\mathcal {O}_{X,x},B_Q=\mathcal {O}_{Y,y},C_R=\mathcal
{O}_{S,s}.$ And we have the following commutative diagram
\[ \xymatrix{
   A_P & & B_Q \ar[ll]_{\varphi}                          \\
   & C_R \ar[ul]^{p_x^{\sharp}} \ar[ur]_{q_y^{\sharp}} }  \]

Write $B=C[b_1,\cdots,b_n]$ with $B_I(1\leqslant i\leqslant n)\in
B.$ Since $S$ is a locally Noetherian scheme, so $C$ is a Noetherian
ring, and so is $B.$ Let $x_1,\cdots,x_n$ be some indeterminant over
$C,$ consider
\[ \xymatrix@R=0em{
   {g: C[x_1,\cdots,x_n]} \ar[r] & {B=C[b_1,\cdots,b_n]}  \\
   \alpha \ar@{|->}[r] & {\alpha(b_1,\cdots,b_n)} }  \]
$ker g$ is an ideal of $C[x_1,\cdots,x_n],$ thus can be generated by
$f_1,\cdots,f_m.$ Write $f_i=\sum
c_{i_{i_1,\cdots,i_n}}x_1^{i_1}\cdots x_n^{i_n}(1\leqslant
i\leqslant m).$ $\forall j(1\leqslant j\leqslant n),$ set
$\frac{a_j}{s_j}=\varphi(\frac{b_j}{1})$ with $a_j\in A, s_j\in
A\setminus P.$ Then $\sum
c_{i_{i_1,\cdots,i_n}}(\frac{a_1}{s_1})^{i_1}\cdots
(\frac{a_n}{s_n})^{i_n}=0$ in $A_P.$ Therefore $\exists t_i\in
A\setminus P$ and $d$ an integer large enough such that
$$t_i\sum c_{i_{i_1,\cdots,i_n}}a_1^{i_1}\cdots a_n^{i_n}\cdot s^{d-i_1}
\cdots S^{d-i_n}=0,\forall i(1\leqslant i\leqslant m).$$ Put
$t=s_1\cdots s_n\cdot t_1\cdots t_m\in A\setminus P,$ then
$$t\sum c_{i_{i_1,\cdots,i_n}}a_1^{i_1}\cdots a_n^{i_n}\cdot s^{d-i_1}
\cdots S^{d-i_n}=0,\forall i(1\leqslant i\leqslant m).$$ $\forall
j(1\leqslant j\leqslant n),$ set $\rho(x_j)=\frac{a_js_1\cdots
s_{j-1}\widehat{s_j}s_{j+1}\cdots s_nt_1\cdots t_m}{t}$ in $A_t.$
Then
\begin{eqnarray*}
\rho(f_i) & = & \sum c_{i_{i_1,\cdots,i_n}}\rho(x_1)^{i_1}\cdots
\rho(x_n)^{i_n}                                                \\
& = & \sum c_{i_{i_1,\cdots,i_n}}(\frac{a_1s_2\cdots s_nt_1\cdots
t_m}{t})^{i_1}\cdots (\frac{a_ns_1\cdots s_{n-1}t_1\cdots
t_m}{t})^{i_n}                                                 \\
& = & 0
\end{eqnarray*}
in $A_t,$ thus $\rho(ker g)=0$ in $A_t.$

Therefore $\exists ! \widetilde{\varphi}: B\rightarrow A_t$ such
that the following diagram is commutative
\[ \xymatrix{
   B_Q \ar[rr]^{\varphi} & & A_P                                \\
   B \ar[u]^{\tau_B} \ar@{-->}[rr]^{\widetilde{\varphi}} & & A_t
   \ar[u]_{\tau_{A_t}}                                          \\
   & {C[x_1,\cdots,x_n]} \ar[ul]^g \ar[ur]_p }  \]
And we know that $\varphi=\widetilde{\varphi}_{PA_t}.$
\begin{eqnarray*}
\widetilde{\varphi}^{-1}(PA_t) & = &
\widetilde{\varphi}^{-1}\tau_{A_t}^{-1}(PA_P)=
\tau_B^{-1}\widetilde{\varphi}^{-1}_{PA_t}(PA_P)      \\
& = & \tau_B^{-1}\varphi^{-1}(PA_P)=\tau_B^{-1}(QB_Q) \\
& = & Q.
\end{eqnarray*}
$\widetilde{\varphi}$ induces a morphism of schemes $f:
D(A_t)\rightarrow Y.$ $f(x)=y,$ because
$\widetilde{\varphi}^{-1}(PA_t)=Q.$
$f^{\sharp}_x=\widetilde{\varphi}_{PA_t}=\varphi.$ Since $g,\rho$
are $C$-algebra homomorphisms, and so is $\widetilde{\varphi},$ then
we have the following commutative diagram
\[ \xymatrix{
   {D(A_t)} \ar[rr]^-{f} \ar[dr]_p & & {SpecB\subseteq Y}
   \ar[dl]^q                                           \\
   & {SpecC\subseteq S} }  \]
\end{proof}
\begin{prop}
Let $p: X\rightarrow S$ and $q: Y\rightarrow S$ be two morphisms of
scheme which are locally of finite type. Let $f: X\rightarrow Y$ be
a morphism of schemes such that the following diagram is commutative
\[ \xymatrix{
   X \ar[rr]^f \ar[dr]_p & & Y \ar[dl]^q \\
   & S }  \]
Let $x\in X, y=f(x)$ such that $f_x^{\sharp}:\mathcal
{O}_{Y,y}\rightarrow \mathcal {O}_{X,x}$ is an isomorphism. Suppose
$S$ is a locally Noetherian scheme. Then there exists a neighborhood
$U$ of $x$ in $X,$ and a neighborhood $V$ of $y$ in $Y$ such that
$\left.f\right|_U: U\rightarrow V$ is an isomorphism of schemes.
\end{prop}
\begin{proof}
Put $\varphi=(f_x^{\sharp})^{-1}:\mathcal {O}_{X,x}\rightarrow
\mathcal {O}_{Y,y},$ then there exists a neighborhood $V$ of $y$ in
$Y$ and $g: V\rightarrow X$ a morphism of schemes such that $g(y)=x,
g_y^{\sharp}=\varphi=(f_x^{\sharp})^{-1},$ and the following diagram
commutes:
\[ \xymatrix{
   V \ar[rr]^g \ar[dr]_q & & X \ar[dl]^p \\
   & S }  \]

Now we have the following commutative diagram
\[ \xymatrix{
   {f^{-1}(V)} \ar[rr]^-{g\circ f}_-{id_{f^{-1}(V)}} \ar[dr]_p & & X \ar[dl]^p \\
   & S }  \]
and $x\in f^{-1}(V), g\circ f(x)=g(y)=x=id_{f^{-1}(V)}(x).$
$$(g\circ f)_x^{\sharp}=(g_{\ast}f^{\sharp}\circ g^{\sharp})_x=
f_x^{\sharp}\circ g^{\sharp}_{f(x)}=f_x^{\sharp}\circ
g^{\sharp}_y=id_{\mathcal {O}_{X,x}},$$ for we can check that the
following diagram is commutative
\[ \xymatrix@C=0em{
   & {\varinjlim\limits_{x\in (g\circ f)^{-1}(U)}\mathcal {O}_X((g\circ
   f)^{-1}(U))} \ar[dr]                                         \\
   {\mathcal {O}_{X,g\circ f(x)}} \ar[ur]^-{((g\circ f)^{\sharp})_{g\circ
   f(x)}} \ar[d]_-{(g^{\sharp})_{g(f(x))}} \ar"4,2"^-{g^{\sharp}_y}
   \ar[rr]^{(g\circ f)_x^{\sharp}} & & {\varinjlim\limits_{x\in W}\mathcal
   {O}_X(W)}                                                     \\
   {\varinjlim\limits_{f(x)\in g^{-1}(U)}\mathcal {O}_Y(g^{-1}(U))}
   \ar[dr] & & {\varinjlim\limits_{x\in f^{-1}(V)}\mathcal
   {O}_X(f^{-1}(V))} \ar[u]                                      \\
   & {\varinjlim\limits_{f(x)\in V}\mathcal {O}_Y(V)=\mathcal
   {O}_{Y,f(x)}} \ar[ur]_-{(f^{\sharp})_{f(x)}} \ar"2,3"^-{f^{\sharp}_x} } \]
Therefore there exists a neighborhood $U$ of $x$ in $f^{-1}(V)$ such
that $g\circ \left.f\right|_U=id_U,$ i.e. $g\circ
(\left.f\right|_U)=id_U.$

Similarly, we have
\[ \xymatrix{
   V \ar[rr]^-{f\circ g}_-{id_V} \ar[dr]_q & & Y \ar[dl]^q \\
   & S }  \]
and $(f\circ g)^{\sharp}_y=g_y^{\sharp}\circ
f^{\sharp}_{g(y)}=id_{\mathcal {O}_{Y,y}}.$ Thus $\exists
V^{\prime}\subseteq V$ a neighborhood of $y$ such that $f\circ
\left.g\right|_{V^{\prime}}=id_{V^{\prime}},$ i.e.
$f\circ(\left.g\right|_{V^{\prime}})=id_{V^{\prime}}.$ Since
$g\circ(\left.f\right|_U)=id_U,$ then $f(u)\subseteq g^{-1}(U),$
thus $f(U\cap f^{-1}(V^{\prime}))\subseteq g^{-1}(U)\cap
V^{\prime}.$ Similarly we have $g((V^{\prime})\cap g^{-1}U)\subseteq
f^{-1}(V^{\prime})\cap U.$ Moreover
$$\left.(g\circ f)\right|_{f^{-1}(V^{\prime})\cap U}=
\left.(\left.(g\circ f)\right|_U)\right|_{f^{-1}(V^{\prime})\cap U}=
\left.id_U\right|_{f^{-1}(V^{\prime})\cap
U}=id_{f^{-1}(V^{\prime})\cap U},$$
$$\left.(f\circ g)\right|_{g^{-1}(U)\cap V^{\prime}}=
\left.(\left.(f\circ g)\right|_{V^{\prime}})\right|_{g^{-1}(U)\cap
V^{\prime}}= \left.id_{V^{\prime}}\right|_{g^{-1}(U)\cap
V^{\prime}}=id_{g^{-1}(U)\cap V^{\prime}}.$$ Therefore
$$\left.f\right|_{f^{-1}(V^{\prime})\cap U}: f^{-1}(V^{\prime})\cap U\rightarrow g^{-1}(U)\cap V^{\prime}$$
is an isomorphism of schemes with
$$\left.g\right|_{g^{-1}(U)\cap V^{\prime}}: g^{-1}(U)\cap V^{\prime}\rightarrow f^{-1}(V^{\prime})\cap U$$
being its two-sided inverse.
\end{proof}
\begin{Def}
Let $X$ and $Y$ be two integral schemes. A morphism $f: X\rightarrow
Y$ is called dominant if $\overline{f(X)}=Y.$
\end{Def}
\begin{prop}
Let $X$ and $Y$ be two integral schemes, let $\xi$ be the generic
point of $X,$ $\eta$ be the generic point of $Y.$ Then a morphism
$f: X\rightarrow Y$ is dominant iff $f(\xi)=\eta.$
\end{prop}
\begin{proof}\

$\Longleftarrow:$
$Y\supseteq\overline{f(X)}\supseteq\overline{\{f(\xi)\}}=\overline{\{\eta\}}=Y,$
thus $Y=\overline{f(X)}.$

$\Longrightarrow:$ $\{f(\xi)\}\subseteq\overline{\{f(\xi)\}},$ then
$\{\xi\}\subseteq f^{-1}(\overline{\{f(\xi)\}}).$ Since $f$ is
continuous, then $f^{-1}(\overline{\{f(\xi)\}})$ is closed, thus
$\overline{\{\xi\}}\subseteq f^{-1}(\overline{\{f(\xi)\}}),$ and
then $$f(\overline{\{\xi\}})\subseteq
f(f^{-1}(\overline{\{f(\xi)\}}))\subseteq \overline{\{f(\xi)\}},$$
so $\overline{f(\overline{\{\xi\}})} \subseteq
\overline{\{f(\xi)\}}.$ Therefore
$\overline{f(\overline{\{\xi\}})}=\overline{\{f(\xi)\}}.$ Since $f$
is dominant, then we have
$$\overline{\{\eta\}}=Y=\overline{f(X)}=\overline{f(\overline{\{\xi\}})}=\overline{\{f(\xi)\}},$$
hence $\eta=f(\xi).$
\end{proof}
\begin{Def}
Let $X$ and $Y$ be two integral schemes, $\xi$ and $\eta$
respectively  be the generic point of $X$ and $Y.$ Let $f:
X\rightarrow Y$ be a dominant morphism of schemes. If
$f^{\sharp}_{\xi}: \mathcal {O}_{Y,\eta}\rightarrow\mathcal
{O}_{X,\xi}$ is an isomorphism, then we say that $f$ is birational.
\end{Def}
\begin{prop}
Let $S$ be a locally Noetherian scheme. Let $X, Y$ be two integral
schemes. Suppose that the diagram
\[ \xymatrix{
   X \ar[rr]^f \ar[dr]_p & & Y \ar[dl]^q \\
   & S }  \]
is commutative, with $f$ birational, $p, q$ locally of finite type.
Then there exists a nonempty open subset $U$ of $X,$ and a nonempty
open subset $V$ of $Y$ such that $\left.f\right|_U: U\rightarrow V$
is an isomorphism.
\end{prop}
\begin{proof}
Since $f$ is birational, then $f$ is dominant, i.e. $f(\xi)=\eta.$
Moreover $f^{\sharp}_{\xi}: \mathcal {O}_{Y,\eta}\rightarrow\mathcal
{O}_{X,\xi}$ is an isomorphism. Then $\exists U$ a neighborhood of
$\xi,$ and $V$ a neighborhood of $\eta,$ such that
$\left.f\right|_U: U\rightarrow V$ is an isomorphism.
\end{proof}
\subsection*{Interpretation in classical algebraic geometry}
Fix $k$ an algebraically closed field. Let $V$ be an affine variety
in $\mathbb{A}^n=k^n.$ Set $A=k[x_1,\cdots,x_n].$ Then $I(V)$ is a
prime ideal of $A,$ and $V=V(I(V)),$
\[ \xymatrix@C=5em{ V \ar@{<->}[r]^-{1-1} & {SpecA/I(V)} }  \]
Set $k[V]=\{f\mid \exists g\in A=k[x_1,\cdots,x_n],\forall a\in
V,f(a)=g(a)\},$ where $f$ is a polynomial function defined on $V.$
Put
\[ \xymatrix@R=0em{
   {\varphi:k[x_1,\cdots,x_n]} \ar[r] & k{V}  \\
   f \ar@{|->}[r] & {\left.f\right|_V} }  \]
then $ker\varphi=\{f\in A\mid f(x)=0,\forall x\in V\}=I(V),$ thus
$k[V]\cong A/I(V).$ Let $V_1\subseteq k^m,V_2\subseteq k^n$ be
affine varieties. Set $f:V_1\rightarrow V_2, f=(f_1,\cdots,f_n)$
with $f_i\in k[x_1,\cdots,x_n].$ $f$ is an isomorphism if $\exists
g:V_2\rightarrow V_1$ defined by polynomials such that $f\circ
g=id_{V_2}, g\circ f=id_{V_1}.$
\begin{thm}
$V_1$ and $V_2$ are isomorphic iff $k[V_1]\cong k[V_2].$
\end{thm}
\begin{proof}
$\Longrightarrow:$ Since $V_1$ and $V_2$ are isomorphic, hence
$\exists \varphi:V_1\rightarrow V_2,\psi:V_2\rightarrow V_1$ such
that $\varphi\circ\psi=id_{V_2},\psi\circ\varphi=id_{V_1}.$ Define
\[ \xymatrix@R=0em{
   {\varphi^{\ast}:k[V_2]} \ar[r] & k[V_1]   \\
   f \ar@{|->}[r] & {f\circ\varphi} } \qquad
   \xymatrix@R=0em{
   {\psi^{\ast}:k[V_1]} \ar[r] & k[V_2]   \\
   g \ar@{|->}[r] & {g\circ\varphi} }
\]
Then $\forall f\in k[V_2],$
$$\psi^{\ast}\varphi^{\ast}(f)=\psi^{\ast}(f\circ\varphi)=f\circ\varphi\circ\psi=f\circ id_{V_2}=f=id_{k[V_2]}(f);$$
$\forall g\in k[V_1],$
$$\varphi^{\ast}\psi^{\ast}(g)=\varphi^{\ast}(g\circ\psi)=g\circ\psi\circ\varphi=g\circ id_{V_1}=g=id_{k[V_1]}(g).$$
Hence $k[V_1]\cong k[V_2]$ as $k$-algebras.

$\Longleftarrow:$ Since $k[V_1]\cong k[V_2],$ then $\exists
\alpha:k[V_1]\rightarrow k[V_2], \beta:k[V_2]\rightarrow k[V_1]$
with $\alpha$ and $\beta$ being each other's two-sided inverse.
Define
\[ \xymatrix@R=0em{
   {\varphi:V_1} \ar[r] & V_2                           \\
   a \ar@{|->}[r] & {(\beta(x_1),\cdots,\beta(x_n))(a)} }
\]
\[   \xymatrix@R=0em{
   {\psi^:V_2} \ar[r] & V_1                              \\
   b \ar@{|->}[r] & {(\alpha(x_1),\cdots,\alpha(x_m))(b)} }
\]
$\forall a\in I(V_2),
g(\beta(x_1),\cdots,\beta(x_n))(a)=\beta(g)(a)=0(a)=0,$ hence
$(\beta(x_1),\cdots,\beta(x_n))(a)\in V_2,$ thus $\varphi$ is
well-defined. Similarly we can prove that $\psi$ is well-defined.
$\forall a\in V_1,$ we have
\begin{eqnarray*}
\psi\circ\varphi(a) & = &
(\alpha(x_1),\cdots,\alpha(x_m))(\beta(x_1),\cdots,\beta(x_n))(a)\\
& = & (\beta\circ\alpha(x_1),\cdots,\beta\circ\alpha(x_m))(a)    \\
& = & (x_1,\cdots,x_m)(a)\,=\,a                                  \\
& = & id_{V_1}(a)
\end{eqnarray*}
Similarly we can prove that $\varphi\circ\psi=id_{V_2}.$ Therefore
$V_1,V_2$ are isomorphic.
\end{proof}
Let $V\subseteq k^m$ be an affine variety. Then $I(V)$ is prime and
$k[V]\cong A/I(V),$ thus $k[V]$ is an integral domain. We denote by
$k(V)$ the field of fractions of $k[V],$ whose elements are called
rational functions. Let $V_1\subseteq k^m,V_2\subseteq k^n$ be
affine varieties. $V_1$ and $V_2$ are called birationally equivalent
if $\exists \varphi: V_1\rightarrow V_2$ and $\psi:V_2\rightarrow
V_1$ with $\varphi$ a rational map on $V_1$ and $\psi$ a rational
map in $V_2,$ and $\varphi$ and $\psi$ are each other's two-sided
inverse.
\begin{thm}
$V_1$ and $V_2$ are birationally equivalent iff $k(V_1)\cong
k(V_2).$
\end{thm}
\begin{proof}

\end{proof}

\newpage

\section{Immersions}

\begin{Def}
Let $f:X\rightarrow Y$ be a morphism of schemes. \enum
\item[(1)] If $f$ induces an isomorphism of $(X,\mathcal {O}_{X})$
with an open subscheme of $(Y,\mathcal {O}_{Y}),$ then we say that
$f$ is an open immersion.
\item[(2)] If $f$ induces an homeomorphism of $X$
with a closed subset of $Y,$ and $f^{\sharp}:\mathcal {O}_{Y}
\rightarrow f_{\ast }\mathcal {O}_{X}$is surjective, then we say
that $f$ is a closed immersion.
\item[(3)] We say that $f$ is an immersion if there exists a scheme $S$
such that
\[ \xymatrix{
  X \ar[rr]^{f} \ar[dr]_{p}
                &  &    Y  \\
                & S\ar[ur]_{q} } \]
where $q$ is an open immersion, and $p$ is a closed immersion.
\end{list}
\end{Def}
\begin{remarks}\
\enum
\item[(1)]$f$ is an open immersion iff $f$ is an open embedding such
that $\left.f^{\sharp}\right|_{f(X)}: \mathcal {O}_{f(X)}\rightarrow
f_{\ast}\mathcal {O}_X$ is an isomorphism of sheaves on $f(X).$
\item[(2)]$f$ is a closed immersion iff $f$ is a closed embedding
such that $f^{\sharp}: \mathcal {O}_Y\rightarrow f_{\ast}\mathcal
{O}_X$ is surjective.
\item[(3)]Two closed immersions $X_1\rightarrow Y, X_2\rightarrow
Y$ are called equivalent, if $\exists$ an isomorphism of schemes
$\varphi: X_1\rightarrow X_2$ such that the following diagram
commutes:
\[ \xymatrix{
  X_1 \ar[rr]^{f} \ar[dr]_{\varphi}
                &  &    Y  \\
                & X_2\ar[ur]_{g} } \]
$\varphi$ is unique if $\varphi$ does exist, for immersions are
monomorphisms in the category of schemes. The equivalence classes of
closed immersions into $Y$ are called closed subschemes of $Y.$

Consider the closed immersion $f: X\rightarrow Y.$ Set $F=f(X),$ let
$V$ be an open subset of $F,$ put $\mathcal {O}_F(V)=\mathcal
{O}_X(f^{-1}(V)).$ Then we can define $$(h,h^{\sharp}):(X, \mathcal
{O}_X)\rightarrow (F, \mathcal {O}_F)$$ as follows: put $h=f:
X\rightarrow F=f(X); \forall V\in \mathds{I}_F,$ put
\begin{eqnarray*}
h^{\sharp}(V)=id_{\mathcal {O}_F(V)}: \mathcal {O}_F(V)\rightarrow
h_{\ast}\mathcal {O}_X(V) & = & \mathcal {O}_X(h^{-1}(V))=\mathcal
{O}_X(f_{-1}(V))        \\
& = & \mathcal {O}_F(V).
\end{eqnarray*}
Then $h^{\sharp}$ is an isomorphism of sheaves on $F$ and $h$ is a
homeomorphism of $X$ and $F.$ Therefore $(X, \mathcal {O}_X)
\stackrel{h}{\cong} (F, \mathcal {O}_F).$ Define $i: F
\hookrightarrow  Y, \forall V\in \mathds{I}_Y,$ we have that
\begin{eqnarray*}
i_{\ast}\mathcal {O}_F(V) & = & \mathcal
{O}_F(i_{-1}(V))\,=\,\mathcal {O}_V(V\cap F)                       \\
& = & \mathcal {O}_X(f_{-1}(V\cap F))\,=\,\mathcal {O}_X(f_{-1}(V))\\
& = & f_{\ast}\mathcal {O}_X(V),
\end{eqnarray*}
then we can put $i^{\sharp}(V)=f^{\sharp}(V),$ thus $i^{\sharp}:
\mathcal {O}_Y \rightarrow i_{\ast}\mathcal {O}_F$ is surjective.
Therefore $i: F \rightarrow Y$ is a closed immersion. Then for any
$X^{\prime} \stackrel{g}{\rightarrow} Y$ which is in the same
equivalence class of the closed immersion $X
\stackrel{f}{\rightarrow} Y,$we have
\[ \xymatrix{
   X^{\prime} \ar"1,3"^{g} \ar[d]_{\varphi} & & Y\\
   X \ar[d]_{h} \ar"1,3"^{f} \\
   F \ar"1,3"_{i}} \]
i.e. $(X^{\prime} \stackrel{g}{\rightarrow} Y) \sim (F
\stackrel{i}{\rightarrow } Y)$
\item[(4)]Open immersions and closed immersions are immersions.
Because we can put $S=X, p=id_X, q=f$ and $S=Y, p=g, q=id_y$
respectively.
\end{list}
\end{remarks}
\begin{prop}
Let $f: X \rightarrow Y$ be a morphism of schemes. \enum
\item[(1)]$f$ is an open immersion iff $f$ is an open embedding and
$\forall p\in X, f^{\sharp}_p:\mathcal {O}_{Y, f(p)}\rightarrow
\mathcal {O}_{X,p}$ is an isomorphism of sheaves.
\item[(2)]$f$ is a closed immersion iff $f$ is a closed embedding and
$\forall p\in X, f^{\sharp}_p:\mathcal {O}_{Y, f(p)}\rightarrow
\mathcal {O}_{X,p}$ is surjective.
\item[(3)]$f$ is an  immersion iff $f$ induces a homeomorphism of $X$ with a locally closed subset of $Y,$ and
$\forall p\in X, f^{\sharp}_p:\mathcal {O}_{Y, f(p)}\rightarrow
\mathcal {O}_{X,p}$ is surjective.
\item[(4)]Immersions are monomorphisms in the category of schemes.
Moreover, the composite of two immersions(resp. open or closed
immersions) ia an immersion(resp. open or closed immersion).
\end{list}
\end{prop}
\begin{proof}\
\enum
\item[(1)]By definition,we know that $f$ is an open immersion iff
$f$ is an open embedding and $\left.f^{\sharp}\right|_{f(X)}:
\mathcal {O}_{f(X)}\rightarrow f_{\ast}\mathcal {O}_X$ is an
isomorphism. Set $V=f(X),$ then $V$ is open in $Y,$ $\mathcal
{O}_V=\mathcal {O}_{f(X)}=\left.\mathcal {O}_Y\right|_V,$ and $f:
X\rightarrow V$ is a homeomorphism. Then $\left.f^{\sharp}\right|_V:
\mathcal {O}_V\rightarrow f_{\ast}\mathcal {O}_X$ is an isomorphism
$\Longleftrightarrow$ $\forall p \in X,
(\left.f^{\sharp}\right|_V)_{f(p)}=(f^{\sharp})_{f(p)}=f^{\sharp}_p:
\mathcal {O}_{V,f(p)}\rightarrow (f_{\ast}\mathcal
{O}_X)_{f(p)}=\mathcal {O}_{X,p}$ is an isomorphism, for $f:
X\rightarrow V$ is a homeomorphism.
\item[(2)]$f$ is a closed immersion iff $f$ is a closed embedding
and $f^{\sharp}: \mathcal {O}_Y\rightarrow f_{\ast}\mathcal {O}_X$
is surjective. $f^{\sharp}$ is surjective iff $\forall y\in Y,$
$(f^{\sharp})_y: \mathcal {O}_{Y,y}\rightarrow (f_{\ast}\mathcal
{O}_X)_y$ is surjective. If $y\in Y,$ since $f: X\rightarrow f(X)$
is a homeomorphism, then there exists a unique $p\in X,$ such that
$f(p)=y,$ and the following diagram commutes:
\[ \xymatrix{
   {\mathcal {O}_{Y,f(p)}=\mathcal {O}_{Y,y}}   \ar"1,2"^-{f^{\sharp}_p}
   \ar"2,1"_{(f^{\sharp})_{f(p)}}  & {\mathcal {O}_{X,p}}    \\
   {(f_{\ast}\mathcal {O}_X)_{f(p)}} \ar@{=}"1,2"} \]
Then $(f^{\sharp})_{f(p)}$ is surjective iff $f^{\sharp}_p$ is.

Therefore \mbox{$\forall y\in Y, (f^{\sharp})_y$ is surjective
$\Longleftrightarrow$ $\forall p\in X, f^{\sharp}_p$ is surjective}.
\item[(3)]Let $X$ be a topological space, $F$ be a subset of $X.$ We
say that $F$ is locally closed if $F=V\cap E$ with $V\subseteq X$
open in $X$ and $E\subseteq X$ closed in $X.$ suppose that $F$ is
locally closed, then:
\enum
\item[(a)] $F=V\cap \overline{F},$ where $\overline{F}$ is the
topological closure of $F$ in $X$:

Let $F=V\cap E$ with $V\subseteq X$ open in $X$ and $E\subseteq X$
closed in $X.$ Then $F\subseteq V,$ hence $F\subseteq
\overline{F}\cap V.$ However $\overline{F}\cap V=\overline{(V\cap
E)}\cap V\subseteq \overline{V}\cap E\cap V.$ Since
$\overline{V}\cap E\cap V\subseteq E\cap V=F,$ then
$\overline{F}\cap V\subseteq F.$ Therefore $F=\overline{F}\cap V.$
\item[(b)] Put $U=(X\setminus \overline{F})\cup F,$ then $U$ is the
largest open subset of $X$ such that $F$ is closed in $U$:

$U^c=(X\setminus \overline{F})^c\cap F^c=\overline{F}\cap (V\cap
\overline{F})^c=\overline{F}\cap (V^c\cup
\overline{F}^c)=(\overline{F}\cap V^c)\cup (\overline{F}\cap
\overline{F}^c)=\overline{F}\cap V^c$ is closed in $X,$ hence $U$ is
open in $X.$ Obviously $F$ is closed in $U$ for $U\cap
\overline{F}=(\overline{F}^c\cup F)\cap
\overline{F}=\overline{F}\cap F=F.$ Let $W$ be an open subset of $X$
such that $F$ is closed in $W.$ By $(a)$ we have $F=W\cap
\overline{F},$ thus $U^c=\overline{F}\cap W^c.$ Then $W^c\supseteq
U^c,$ i.e. $U\supseteq W.$ Hence $U$ is the largest open subset of
$X$ such that $F$ is closed in $U.$
\end{list}

$\Longrightarrow$: Suppose $f$ is an immersion, then we have
\[ \xymatrix{
  X \ar[rr]^{f} \ar[dr]_{h}&  &  Y \\
                & S \ar[ur]_{g}}  \]
with $S$ a scheme , $g$ an open immersion, and $h$ a closed
immersion.

Put $F=f(X)=g(h(X)),$ then $F$ is a closed subset of $f(S),$ thus
locally closed in $Y,$ for $g(S)$ is open in $Y.$ Hence $f:
X\rightarrow F$ is a homeomorphism of $X$ with a locally closed
subset $F$ of $Y.$ $\forall p \in X, f^{\sharp}_p=(g\circ
h)^{\sharp}_p=h^{\sharp}_p\circ g^{\sharp}_{h(p)}.$ By definition,
$g$ is an open immersion $\Longrightarrow g^{\sharp}_{h(p)}$ is an
isomorphism. By $(b),$ $h$ is a closed immersion $\Longrightarrow
h^{\sharp}_p$ is surjective. Hence $f^{\sharp}_p$ is surjective.

$\Longleftarrow:$Put $F=f(X),$ then $F$ is locally closed in $Y, f:
X\rightarrow F$ is a homeomorphism, and $\forall p\in X,
f^{\sharp}_p: \mathcal {O}_{Y,f(p)}\rightarrow \mathcal {O}_{X,p}$
is surjective. Put$V=(Y\setminus \overline{F})\cup F,$ then $V$ is
the largest open subset of $Y$ such that $F$ is closed in $V,$ and
we have
\[ \xymatrix{
  X \ar[rr]^{f} \ar[dr]_{h}&  &  Y \\
                & S \ar[ur]_{g}}  \]
where $g$ is the canonical inclusion, and $h$ coincides with $f$ on
$X$ to $F.$

Therefore $g$ is an open immersion. $h$ induces a homeomorphism of
$X$ with a closed subset $F$ of $V,$ and $\forall p\in X,
f^{\sharp}_p=(g\circ h)^{\sharp}_p=h^{\sharp}_p\circ
g^{\sharp}_{h(p)}.$ $f^{\sharp}_p$ is surjective,
$g^{\sharp}_{h(p)}$ is an isomorphism, so $h^{\sharp}_p$ is
surjective. Hence $h$ is a closed immersion. Therefore $f:
X\rightarrow Y$ is an immersion.
\item[(4)]$\mathit{1^{\circ}}$ Let
\[ \xymatrix{
    X_1 \ar"2,2"^{g_1} \\
     & X \ar[r]^{f} & Y\\
    X_2 \ar"2,2"^{g_2}} \]
be such that $f\circ g_1=f\circ g_2$ with $f$ a closed or an open
immersion, we want to show that $g_1=f_2.$

Topological part is clear, for $f$ is injective.

Sheaf's part: We have $(f\circ g_1)^{\sharp}=(f\circ g_2)^{\sharp},$
then $f_{\ast}(g_1)^{\sharp}\circ f^{\sharp}=
f_{\ast}(g_2)^{\sharp}\circ f^{\sharp}.$ Since for the open or
closed immersion $f,$ $f^{\sharp}$ is surjective, then
$f_{\ast}(g_1)^{\sharp}= f_{\ast}(g_2)^{\sharp}.$ $\forall V$ open
in $X,$ since $f: X\rightarrow f(X)$ is a homeomorphism, thus
$\exists W$ open in $Y$ such that $V=f^{-1}(W).$ Then
$g_1^{\sharp}(V)=f_{\ast}(g_1^{\sharp})(W)=f_{\ast}(g_2^{\sharp})(W)=g_2^{\sharp}(V),$
hence $g_1^{\sharp}=g_2^{\sharp}.$ Therefore $g_1=g_2$ if $f$ is an
open or a closed immersion. Since the composite of monomorphisms is
a monomorphism, hence immersions are monomorphisms.

$\mathit{2^{\circ}}$ Take $S \stackrel{g}{\rightarrow} X
\stackrel{f}{\rightarrow} Y$ with $f$ and $g$ immersions. Let
$h=f\circ g,$ then $h(S)$ is a locally closed subset of $Y,$ for
$f(X)$ is locally closed in $Y,$ and $g(S)$ is locally closed in
$X.$ Hence $h$ induces a homeomorphism of $S$ with a locally closed
subset $h(S)$ of $Y.$ $\forall s\in S, h^{\sharp}_s=(f\circ
g)^{\sharp}_s=g^{\sharp}_s\circ f^{\sharp}_{g(s)}$ is surjective,
since $g^{\sharp}_s$ and $f^{\sharp}_{g(s)}$ are surjective. By
$(3),$ we obtain that $h$ is an immersion.
\end{list}
\end{proof}
\begin{remark}
If $f:X\rightarrow Y$ is a closed immersion and $g: S\rightarrow X$
is an open immersion, then $f\circ g: S\rightarrow Y$ is an
immersion.
\end{remark}
\begin{lemma}
Let $f: A\rightarrow B$ be a homomorphism of rings, let $\varphi:
SpecB\rightarrow SpecA$ be the morphism induced by $f,$ then
$\overline{(\varphi (SpecB)}=V(kerf)$ in $SpecA.$
\end{lemma}
\begin{proof}
For any ideal $b$ of $B,$ we have
$$\overline{V(B)}=V(b^c)=V(f^{-1}(b)).$$ Hence $$\overline{(\varphi
(SpecB)}=\overline{V(0)}=V(f^{-1}(0))=V(kerf)$$ in $SpecA.$
\end{proof}
\begin{cor}
Let $f: X\rightarrow Y$ be an immersion,then \enum
\item[(1)]$f$ is a closed immersion iff $f(X)$ is closed in $Y.$
\item[(2)]$f$ is an open immersion iff $f(X)$ is open in $Y$ and $\forall p\in X, f^{\sharp}_p$ is injective.
\end{list}
\end{cor}
\begin{proof}
Since $f: X\rightarrow Y$ is an immersion, then $\forall p\in X,
f^{\sharp}_p$ is surjective. And $f: X\rightarrow f(X)$ is a
homeomorphism with $f(X)$ locally closed in $Y.$ \enum
\item[(1)]$f$ is a closed immersion $\Longleftrightarrow f: X\rightarrow
Y$ is a homeomorphism with $f(X)$ closed in $Y,$ and $\forall p\in
X, f^{\sharp}_p$ is surjective. Since $f$ is an immersion, hence $f$
is a closed immersion $\Longleftrightarrow f(X)$ is closed in $Y.$
\item[(2)]$f$ is an open immersion $\Longleftrightarrow f: X\rightarrow
f(X)$ is homeomorphism with $f(X)$ open in $Y,$ and $\forall p\in X,
f^{\sharp}_p$ is an isomorphism. Since $f$ is an open immersion,
then $f$ is an open immersion $\Longleftrightarrow f(X)$ is open in
$Y$ and $\forall p\in X, f^{\sharp}_p$ is injective.
\end{list}
\end{proof}
\begin{prop}
Let $A$ be a nonzero ring,
\enum
\item[(1)]For any ideal $I$ of $A,$ the canonical morphism $\varphi:
SpecA/I \rightarrow SpecA$ is a closed immersion.
\item[(2)]Any closed immersion into $SpecA$ is isomorphic to the
canonical morphism $SpecA/I \rightarrow SpecA$ with $I$ an ideal of
$A.$
\end{list}
\end{prop}
\begin{proof}\
\enum
\item[(1)]$\mathit{1^{\circ}}$ We have
\[ \xymatrix@R=0em{
   SpecA \ar@{<->}[r] & SpecA/I               \\
   {q,q\supseteq I} \ar@{<->}[r] & q/I }  \]
with $q$ prime in $A.$ Hence for any ideal $J$ of $A$ such that
$J\supseteq I,$ we have $$\varphi(V(J/I))=V(J),\quad
\varphi^{-1}(V(J))=V(J/I).$$ Therefore $\varphi$ induces a
homeomorphism from $SpecA/I$ to $V(I).$

$\mathit{2^{\circ}}$ $\forall q\in SpecA/I,$ we can write $q=p/I$
with $p\in V_A(I).$ Then we obtain the following commutative diagram
\[ \xymatrix{
   {\mathcal {O}_{SpecA,\varphi(q)}} \ar[r]^-{\varphi^{\sharp}_q}
   \ar@{=}[d]^-{\wr} & {\mathcal {O}_{SpecA/I,q}} \ar@{=}[d]^-{\wr}\\
   A_p \ar[r]^-{\varphi^{\sharp}_q} & {(A/I)_q} }  \]
$\varphi^{\sharp}_q$ is induced by the canonical morphism
$\varphi^{\sharp}(SpecA):A\rightarrow A/I,$ and we have
$\varphi^{\sharp}(SpecA)(p)=p/I=q.$ Hence $\varphi^{\sharp}_q$ is
surjective.
\item[(2)]Let $\varphi:X\rightarrow SpecA$ be a closed immersion.

$\mathit{1^{\circ}}$ $X$ is affine:

We assume that the empty set is an affine scheme. $\forall p\in
SpecA,$ we can find a neighborhood $U_p$ of $p$ in $SpecA$ such that
$\varphi^{-1}(U_p)$ is an affine open subscheme of $X.$ Take $f\in
A$ such that $D(f)$ is a neighborhood of $p$ in $SpecA$ contained in
$U_p.$ Then
$\varphi^{-1}(D(f))=\varphi^{-1}((U_p)_f)=\varphi^{-1}(U_p)_{\varphi^{\sharp}(U_p)(f)}$
is affine for $\varphi^{-1}(U_p)$ is affine.

Cover $U_p$ by open subsets of the form $D(f),$ then we can find a
covering $(D(f_i))_{i\in\Lambda}$ of $SpecA$ such that $\forall
i\in\Lambda, \varphi^{-1}(D(f_i))$ is an affine subscheme of $X,$
because $SpecA=\bigcup\limits_{p\in SpecA}U_p,$ and $(D(f))_{f\in
SpecA}$ are the basic open subsets of $SpecA.$ Since $SpecA$ is
quasi-compact, then we can suppose that $\Lambda=\{1,\cdots,n\}.$
Then $SpecA=\bigcup\limits_{i=1}^nD(f_i),$ and thus $f_1,\cdots,f_n$
generate $A.$ Put $g_i=\varphi^{\sharp}(SpecA)(f_i)\in \mathcal
{O}_X(X).$ Then $g_1,\cdots,g_n$ generate $\mathcal {O}_X(X).$
However, $\forall i(1\leqslant i\leqslant n),
X_{g_i}=\varphi^{-1}(D(f_i))$ is an open affine subscheme of $X.$
Therefore $X$ is an affine scheme.

$\mathit{2^{\circ}}$ Write $X=SpecB.$ Then the closed immersion
$\varphi: SpecB\rightarrow SpecA$ can be induced by
$\alpha=\varphi^{\sharp}(SpecA): A\rightarrow B.$ Put $I=ker\alpha,$
since
\[ \xymatrix{
   A/I \ar@^{(->}[rr] & & B                 \\
   & A \ar[ul] \ar[ur]_-{\alpha} }  \]
is commutative, then we obtain
\[ \xymatrix{
   SpecB \ar[rr]^-{\theta} \ar[dr]_{\varphi} & & SpecA/I
   \ar[dl]^{\psi}                                  \\
   & SpecA }  \]
where $\psi: SpecA/I\rightarrow SpecA$ is the canonical closed
immersion. $\psi\circ\theta(SpecB)=\varphi(SpecB)$ is closed in
$SpecA,$ then $\theta(SpecB)=\psi^{-1}(\varphi(SpecB))$ is closed in
$SpecA/I,$ for $\psi^{-1}(\varphi(SpecB))$ is a closed subset of
$\psi^{-1}(SpecA)=SpecA/I.$ Then
$$\theta(SpecB)=\overline{\theta(SpecB)}=V(ker(\theta^{\sharp}(SpecA/I)))=V(0)=SpecA/I.$$
Hence $\theta$ is surjective. We have $\varphi=\psi\circ\theta$ with
$\varphi$ a monomorphism, then $\theta$ is injective. Since
$\theta=\psi^{-1}\circ\varphi$ with $\varphi,\psi$ homeomorphisms,
thus $\theta$ is a homeomorphism from $SpecB$ to $SpecA/I.$

$\forall p\in
B,\varphi^{\sharp}_p=(\psi\circ\theta)^{\sharp}_p=\theta^{\sharp}_p\circ\psi^{\sharp}_{\theta(p)},$
then $\theta^{\sharp}_p$ is surjective, for $\varphi^{\sharp}_p$ is
surjective. Since $\theta$ is a homeomorphism, then $\forall q\in
SpecA/I, (\theta^{\sharp})_q=\theta^{\sharp}_{\theta^{-1}(q)}$ is
surjective. Hence $\theta^{\sharp}:\mathcal {O}_{SpecA/I}\rightarrow
\theta_{\ast}\mathcal {O}_{SpecB}$ is surjective. Put
$\beta=\theta^{\sharp}(SpecA/I):A/I\hookrightarrow B,$ then $\beta$
is injective. $\forall f\in A,$ set $\bar{f}=f+I,$ then we obtain
\[ \xymatrix{
   A/I\, \ar@^{(->}[r]^-{\beta} \ar[d] & B \ar[d]\\
   {(A/I)_{\bar{f}}} \ar[r]^-{\beta_{\bar{f}}} &
   {B_{\alpha(f)}=B_{\beta(\bar{f})}} }  \]
Hence $\beta_{\bar{f}}$ is injective. But we have
\[ \xymatrix@C=4em{
   {\mathcal {O}_{SpecA/I}(D(\bar{f}))}
   \ar[r]^-{\theta^{\sharp}(D(\bar{f}))} \ar@{=}[d]^-{\wr} &
   {\theta_{\ast}\mathcal {O}_{SpecB}(D(\bar{f}))=\mathcal
   {O}_{SpecB}(D(\alpha(f)))} \ar@{=}[d]^-{\wr}           \\
   {(A/I)_{\bar{f}}} \ar[r]^-{\beta_{\bar{f}}} &
   {B_{\alpha(f)}} }  \]
Thus $\theta^{\sharp}(D(\bar{f}))$ is injective, and then $\forall
p\in SpecA/I, (\theta^{\sharp})_p$ is injective, because
$(D(\bar{f}))_{\bar{f}\in SpecA/I}$ are the basic open subsets of
$SpecA/I.$ Therefore $\theta^{\sharp}:\mathcal
{O}_{SpecA/I}\rightarrow \theta_{\ast}\mathcal {O}_{SpecB}$ is
injective.

In conclusion, $\theta$ is a homeomorphism of $SpecB$ and $SpecA/I,$
$\theta^{\sharp}$ is an isomorphism of $\mathcal {O}_{SpecA/I}$ and
$\theta_{\ast}\mathcal {O}_{SpecB}.$ Hence $\theta$ is an
isomorphism of schemes.
\end{list}
\end{proof}
\begin{cor}
Let $\varphi: A\rightarrow B$ be a morphism of rings, then
$Spec\varphi: SpecB \rightarrow SpecA$ is a closed immersion iff
$\varphi$ is surjective.
\end{cor}
\begin{proof}
$\Longrightarrow$: Since $Spec\varphi: SpecB \rightarrow SpecA$ is a
closed immersion, then we have
\[ \xymatrix{
   SpecB \ar"1,3"^-{Spec\varphi} \ar"2,2"_{\theta} & & SpecA\\
   &    SpecA/I \ar"1,3"_{\psi}   }  \]
where $I$ is an ideal of $A,$ $\theta$ is an isomorphism of schemes,
and $\psi$ is the canonical closed immersion. Then
\[ \xymatrix{
   B & & A \ar"1,1"_-{\varphi} \ar"2,2"^-{\theta^{\sharp}(SpecA/I)}\\
   &    A/I \ar"1,1"^-{\psi^{\sharp}(SpecA)}   }  \]
$\theta^{\sharp}(SpecA/I)$ is an isomorphism, $\psi^{\sharp}(SpecA)$
is surjective. $\varphi =
(\psi_{\ast}\theta^{\sharp}\circ\psi^{\sharp})(SpecA) =
\psi_{\ast}\theta^{\sharp}(SpecA)\circ\psi^{\sharp}(SpecA) =
\theta^{\sharp}(\psi^{-1}(SpecA))\circ\psi^{\sharp}(SpecA) =
\theta^{\sharp}(SpecA/I)\circ\psi^{\sharp}(SpecA).$ Therefore
$\varphi$ is surjective.

$\Longleftarrow:$ is direct for $B\cong A/ker\varphi.$
\end{proof}
\begin{prop}
Let $X$ be a scheme and $Y$ be a closed subset of $X.$ Then $\exists
!$ reduced scheme structure $(Y, \mathcal {O}_Y)$ on $Y$ such that
$Y$ becomes a closed subscheme of $X.$ Such a structure on $Y$ is
called the reduced induced closed subscheme structure on $Y.$
\end{prop}
\begin{proof}
$\mathit{1^{\circ}}$ The affine case:

Suppose that $X=SpecA,$ then we can write $Y=V(I)$ with $I$ an ideal
of $A.$ Set $I_Y=\bigcap\limits_{p\in Y}p,$ then $Y=V(I_Y),$ and
$SpecA/I_Y$ is reduced. $I_Y$ is unique for it's uniquely determined
by $Y.$ Let $\varphi:Y\rightarrow SpecA$ be a closed immersion, then
$Y\cong SpecA/I$ for some ideal $I$ of $A.$ $Y$ is reduced, hence
$I=\sqrt{I}=\bigcap\limits_{p\in Y}p=I_Y.$ Therefore such a
$(Y,\mathcal {O}_Y)$ is unique up to isomorphism.

$\mathit{2^{\circ}}$ General case:

Cover $X$ by affine open subschemes $V_i=SpecA_i(i\in\Lambda).$
$\forall i\in\Lambda,$ the closed subset $Y\cap V_i$ of $V_i$ has a
unique reduced induced closed subscheme structure $(Y\cap
V_i,\mathcal {O}_{Y\cap V_i})$ by $\mathit{1^{\circ}}$ Set
$Y_i=Y\cap V_i.$ Then for each $i\in\Lambda,$ we have a unique
closed immersion $\varphi_i:Y_i\rightarrow V_i$ with $Y_i$ reduced,
and $(Y_i)_{i\in\Lambda}$ is an open covering of $Y.$

Cover $V_i\cap V_j$ by affine open subschemes $V_{ijk}(k\in
K_{ij}),$ then $Y_i\cap V_{ijk}$ and $Y_j\cap V_{ijk}$ are closed in
$V_{ijk}.$ But
$$Y_i\cap V_{ijk}=Y\cap V_{ijk}=Y_j\cap V_{ijk}=Y_i\cap Y_j\cap V_{ijk}.$$
$\mathcal {O}_{Y_i\cap V_{ijk}}=\left.\mathcal {O}_{Y\cap
V_i}\right|_{Y_i\cap V_{ijk}}$ and $\mathcal {O}_{Y_j\cap
V_{ijk}}=\left.\mathcal {O}_{Y\cap V_j}\right|_{Y_j\cap V_{ijk}}$
are reduced, because $\mathcal {O}_{Y_i}$ and $\mathcal {O}_{Y_j}$
are reduced. Since the reduced induced closed subscheme structure in
affine case is unique, hence we obtain a commutative diagram
\[ \xymatrix{
   {(Y_i\cap Y_j\cap V_{ijk},\left.\mathcal {O}_{Y_i}\right|_{Y_i\cap
   Y_j\cap V_{ijk}})} \ar[dd]_{\psi_{ijk}} \ar[dr]^-{\left.\varphi_i\right|_{Y_i\cap
   Y_j\cap V_{ijk}}}                                      \\
   & {(V_{ijk},\mathcal {O}_{V_{ijk}})}                   \\
   {(Y_i\cap Y_j\cap V_{ijk},\left.\mathcal {O}_{Y_j}\right|_{Y_i\cap
   Y_j\cap V_{ijk}})} \ar[ur]_-{\left.\varphi_j\right|_{Y_i\cap Y_j\cap
   V_{ijk}}} }  \]
where $\psi_{ijk}$ is an isomorphism of schemes.

Cover $V_{ijk}\cap V_{ijl}$ by affine open subschemes
$V_{ijklh}(h\in H_{ijkl}),$ then $(Y_i\cap Y_j\cap V_{ijk}\cap
V_{ijl})\cap V_{ijklh}$ is a closed subset in $V_{ijklh}.$ Since the
reduced induced closed subscheme structure in the affine case is
unique, hence
$$\left.\mathcal {O}_{Y_i}\right|_{Y_i\cap Y_j\cap V_{ijk}\cap V_{ijl}\cap V_{ijklh}}\cong
\left.\mathcal {O}_{Y_j}\right|_{Y_i\cap Y_j\cap V_{ijk}\cap
V_{ijl}\cap V_{ijklh}},$$ i.e.
$$\left.\psi_{ijk}\right|_{Y_i\cap Y_j\cap V_{ijk}\cap V_{ijl}\cap V_{ijklh}}=
\left.\psi_{ijl}\right|_{Y_i\cap Y_j\cap V_{ijk}\cap V_{ijl}\cap
V_{ijklh}}.$$

Therefore we can glue the closed subschemes $(Y_i,\mathcal
{O}_{Y_i})_{i\in\Lambda}$ along $(Y_i\cap Y_j)_{i,j\in\Lambda}$
together to get a unique scheme structure on $Y.$ Since
$$\left.\varphi_i\right|_{Y_i\cap Y_j\cap V_{ijk}}=\left.\varphi_j\right|_{Y_i\cap
Y_j\cap V_{ijk}}$$ up to isomorphism, then we obtain a unique
$\varphi:Y\rightarrow X$ such that $\forall i\in\Lambda,
\left.\varphi\right|_{Y_i}=\varphi_i.$ $\varphi:Y\rightarrow X$ is
an immersion and the image $Y=\varphi(Y)$ is closed in $X,$ hence
$Y$ becomes a closed subscheme of $X.$ $Y$ is reduced, because it is
locally reduced.
\end{proof}

\newpage

\section{Fibred product and base change}

\begin{Def}
Let $S$ be a scheme, then the category of $S$-scheme is defined as
follows:
\enum
\item[(1)]Object: a scheme $X$ together with a morphism of
schemes: $X\rightarrow S$ (called an $S$-scheme or a scheme over
$S$).
\item[(2)]morphism: an $S$-morphism from an $S$-scheme $X$ to an
$S$-scheme $Y$ is by definition a morphism $X\rightarrow Y$ such
that
\[ \xymatrix{
  X \ar[rr] \ar[dr]
                &   &    Y\ar[dl]    \\
                & S                 } \]
\end{list}
\end{Def}
\begin{egs}\
\enum
\item[(1)]Every scheme $X$ is a $Spec\mathbb{Z}$-scheme, for the
canonical ring homomorphism
\[ \xymatrix@R=0em{
   \mathbb{Z} \ar[r] & \mathcal {O}_X(X)  \\
   n \ar@{|->}[r] & {n\cdot 1} }  \]
induces a morphism of schemes $X\rightarrow Spec\mathbb{Z}.$
\item[(2)]Let $k$ be an algebraically closed field. Every affine
variety over $k$ is a $Speck$-scheme.
\end{list}
\end{egs}

Let $\mathscr{C}$ be a category. Let $X, Y\in Obj\mathscr{C}.$ The
product $X \times Y$ is defined (if it does exist) by the following
universal property
\[ \xymatrix{
   & {X \times Y} \ar"2,1"_{p} \ar"2,3"^{q}            \\
   X            &           &          Y               \\
   & T \ar"2,1" \ar"2,3" \ar@{-->}"1,2"^{\exists !f} } \]
\begin{Def}
Let $X$ and $Y$ be two $S$-schemes. Its fibred product $X \times_S
Y$ (if it does exist) is defined by the following universal
property: \enum
\item[(1)] $X \times_S Y$ is an $S$-scheme;
\item[(2)] We have the following commutative diagram of $S$-schemes:
\[ \xymatrix{
   &           {X \times_S Y} \ar"2,1"_{p} \ar"2,3"^{q}       \\
   X \ar"3,2" & T \ar[l] \ar[r] \ar@{-->}[u]_{\exists !f} \ar[d]
   & Y   \ar"3,2"\\
   &       S   }     \]
\end{list}
\end{Def}
\begin{remark}
If $X \times_S Y$ does exist, then it is unique up to unique
isomorphism of $S$-schemes. If $\widetilde{X \times_S Y}$ satisfies
the universal property, then we put $T=\widetilde{X \times_S Y}$ and
then interchange $X \times_S Y$ and $\widetilde{X \times_S Y},$ we
can obtain a unique isomorphism of $S$-schemes.
\end{remark}
\begin{egs}\
\enum
\item[(1)]In the category of sets. Let $f: X\rightarrow S, g:Y
\rightarrow S$ be two morphisms, then $X \times_S Y=\{(x,y)\mid x\in
X, y\in Y, f(x)=g(y)\}$
\item[(2)]Affine schemes:

Let $S=SpecA, X=SpecB, Y=SpecC$ with $A,B,C$ rings. Then $SpecB
\times_{SpecA} SpecC = SpecB\otimes_A C,$ for we have
\[ \xymatrix{
   &           {SpecB\otimes_A C} \ar"2,1" \ar"2,3"       \\
   SpecB \ar"3,2" & T \ar[l] \ar[r] \ar@{-->}[u]_{\exists !f} \ar[d]
   & SpecC  \ar"3,2"                                      \\
   &       SpecA   }     \]
which is induced by
\[ \xymatrix{
   &            {B\otimes_A C} \ar@{-->}[d]_{\exists !g}           \\
   B \ar"1,2" \ar[r] & \mathcal {O}_T(T) & C  \ar"1,2" \ar[l]      \\
   & A \ar"2,1" \ar[u] \ar"2,3"  }     \]
The existence of $g$ comes from the universal property of tensor
product.
\end{list}
\end{egs}
\begin{remark}
The product of two schemes $X$ and $Y$ in the category of schemes is
denoted by $X \times Y,$ and we have $X \times Y=X
\times_{Spec\mathbb{Z}} Y.$
\end{remark}
\begin{prop}
For any $S$-scheme $X$ and $Y,$ their fibred product $X \times_S Y$
over $S$ does exist and it is unique up to unique isomorphism of
$S$-schemes.
\end{prop}
\begin{proof}
$\mathit{1^{\circ}}$ Affine case:

If $X=SpecA, Y=SpecB, S=SpecC,$ then $X \times_S Y=SpecA\otimes_C
B,$ where the projections $p: X \times_S Y \rightarrow X$ and $q:X
\times_S Y\rightarrow Y$ are induced by
\[ \xymatrix@R=-0.25em{
   A \ar[r]  & {A \otimes_C B} & \text{and} & B \ar[r]  & {A \otimes_C B} &
   \text{respectively.}\\
   a \ar@{|->}[r] & {a \otimes_C 1} &     & b \ar@{|->}[r] & {1 \otimes_C
   b} }  \]

$\mathit{2^{\circ}}$ Let $X$ and $Y$ be $S$-schemes such that $X
\times_S Y$ exists, then for any open subscheme $U$ of $X,$
$p_{-1}(U)=U \times_S Y$:

We have the following commutative diagram
\[ \xymatrix@R=5em{
   &  & {X \times_S Y} \ar"2,1"_{p} \ar"2,4"^{q}          \\
   X \ar"3,3" & \,U \ar@{_{(}->}[l]_{i} \ar"3,3"
   & T \ar[l]_{f_1} \ar[r]^{f_2} \ar@{-->}[u]^{\exists !f} \ar[d]
   & Y \ar"3,3"                                               \\
   & & S    }   \]
$p\circ f=f,$ then $p\circ f(T)=f_1(T)\subseteq U,$ thus
$f(T)\subseteq p^{-1}(U).$

Then we obtain the following commutative diagram
\[ \xymatrix{
   & {p^{-1}(U)} \ar"2,1"_{p} \ar"2,3"^{q}           \\
   U \ar"3,2" & T \ar@{-->}[u]_{\exists !f} \ar[l]_{f_1}
   \ar[r]^{f_2} \ar[d] & Y  \ar"3,2"                 \\
   & S  }     \]
Hence $p^{-1}(U)=U \times_S Y.$ Likewise for any open subscheme $V$
of $Y,$ we have $q^{-1}(V)=X \times_S V.$

$\mathit{3^{\circ}}$ Suppose that $X$ can be covered by open
subschemes $V_{i}(i\in I)$ such that each $V_{i} \times_S Y$ exists,
then $X \times_S Y$ does exist:

By $\mathit{2^{\circ}},$ since each $V_{i} \times_S Y$ exists, thus
we have that: $(V_i\cap V_j)\times_S Y = p_i^{-1}(V_i\cap V_j)$ is
an open subscheme of $V_i\times_S Y,$ $(V_i\cap V_j)\times_S Y =
p_j^{-1}(V_i\cap V_j)$ is an open subscheme of $V_j\times_S Y.$
Since fibred product is unique up to unique isomorphism of
$S$-schemes, then $\exists!$ isomorphism $\varphi_{ij}:
p_i^{-1}(V_i\cap V_j)\rightarrow p_j^{-1}(V_i\cap V_j)$ of
$S$-schemes for each $i,j\in I.$ Since each $\varphi_{ij}$ is
unique, hence satisfies the following properties: \enum
\item[(a)]$\forall i,j\in I(i\neq j), \varphi_{ij}=\varphi_{ji}^{-1}$;
\item[(b)] For any distinct $i,j,k\in I,$ $$\varphi_{ij}(p_i^{-1}(V_i\cap
V_j)\cap p_i^{-1}(V_i\cap V_k)) = p_j^{-1}(V_i\cap V_j)\cap
p_j^{-1}(V_k\cap V_j)$$ and $$\left.\varphi_{jk}\circ
\varphi_{ij}\right|_{p_i^{-1}(V_i\cap V_j)\cap p_i^{-1}(V_i\cap
V_k)} = \left.\varphi_{ik}\right|_{p_i^{-1}(V_i\cap V_j)\cap
p_i^{-1}(V_i\cap V_k)}.$$
\end{list}

Then we can glue $(V_i\times_S Y)_{i\in I}$ together along
$((V_i\cap V_j)\times_S Y)_{i,j\in I}$ to obtain an $S$-scheme $K.$
We want to show that $K=X\times_S Y.$ Indeed, by gluing $(p_i)_{i\in
I}$ and $(q_i)_{i\in I},$ we obtain the projections $$p:
K\rightarrow X,\quad q: K\rightarrow Y$$ such that
$\left.p\right|_{V_i\times_S Y}=p_i, \left.q\right|_{V_i\times_S
Y}=q_i.$ Take $f: T\rightarrow X, g:T\rightarrow Y$ with $T$ an
$S$-scheme and $f,g$ two $S$-morphisms. For any $i,i\in I,$ we have
the following commutative diagrams:
\[ \xymatrix{
   {V_i\times_S Y}\, \ar[d]_{p_i} \ar@^{(->}"1,3" \ar"2,4"_{q_i} & & K  \\
   V_i \ar"3,2" & {\left.T\right|_{f^{-1}(V_i)}} \ar[l]^{\left.f\right|_{f^{-1}(V_i)}}
   \ar"2,4"_{\left.g\right|_{f^{-1}(V_i)}} \ar@{-->}"1,1"^{\exists! F_i}
   \ar@{-->}"1,3"^{\exists! G_i} \ar[d] & & Y \ar"3,2"            \\
   & S}  \]
\[ \xymatrix{
   {(V_i\cap V_J)\times_S Y}\, \ar[d] \ar@^{(->}"1,3" \ar"2,3" & & K \\
   {V_i\cap V_j} \ar"3,2" & {\left.T\right|_{f^{-1}(V_i\cap V_j)}}
   \ar[l] \ar[r] \ar@{-->}"1,1"^{\exists! F_ij} \ar@{-->}"1,3"^{\exists!
   G_ij} \ar[d] & Y \ar"3,2"            \\
   & S}    \]
In other words, we find an open covering $(f^{-1}(V_i))_{i\in I}$ of
$T,$ and $S$-schemes $G_i: \left.T\right|_{f^{-1}(V_i)}\rightarrow
K.$

Again, by the uniqueness of  fibred product and its universal
property, we obtain $$F_{ij} = \left.F_i\right|_{f^{-1}(V_i)\cap
f^{-1}(V_j)} = \left.F_j\right|_{f^{-1}(V_i)\cap f^{-1}(V_j)}$$
hence $$G_{ij} = \left.G_i\right|_{f^{-1}(V_i)\cap f^{-1}(V_j)} =
\left.G_j\right|_{f^{-1}(V_i)\cap f^{-1}(V_j)}$$ Then we obtain a
unique $S$-morphism $G: T\rightarrow K$ such that
$\left.G\right|_{f^{-1}(V_i)} = G_i$ and the following diagram
commutes:
\[ \xymatrix{
   & K \ar"2,1"_p \ar"2,3"^q \\
   X \ar"3,2" & T \ar[l] \ar[r] \ar@{-->}[u]_{\exists !G} \ar[d] & Y \ar"3,2"\\
   & S}     \]
Therefore $K = X\times_S Y.$

$\mathit{4^{\circ}}$ Suppose $S$ and $Y$ are affine, then $X$ can be
covered by affine open subschemes $(U_i)_{i\in I}.$ By
$\mathit{1^{\circ}}$ we know that $\forall i\in I, U_i\times Y$
exists and by $\mathit{3^{\circ}} X\times_S Y$ exists.

$\mathit{5^{\circ}}$ Suppose that $S$ is affine, cover $Y$ by affine
open subschemes, and apply $\mathit{3^{\circ}}$ and
$\mathit{4^{\circ}}$ with $X$ and $Y$ interchanged, we obtain that
$X\times_S Y$ does exist.

$\mathit{6^{\circ}}$ General case:

Cover $S$ by affine open subschemes $S_i = SpecA_i(i\in I).$ Write
$$l_X: X\rightarrow S,\quad l_Y: Y\rightarrow S,$$ put $X_i =
l_X^{-1}(S_i), Y_i = l_Y^{-1}(S_i),$ then $X_i$ is an open subscheme
of $X,$ $Y_i$ is an open subscheme of $Y.$ By $\mathit{5^{\circ}},$
$X_i\times_{S_i} Y_i$ exists, $\forall i\in I.$ But $X_i\times_{S_i}
Y_i$ is the fibred product of $X_i$ and $Y_i$ over $S.$ Indeed we
have the following commutative diagram:
\[ \xymatrix{
   & & {X_i\times_{S_i} Y_i} \ar"2,1" \ar"2,4"                   \\
   X_i \ar"3,2" \ar"4,2"_{l_X} & & T \ar@{-->}"1,3"_{\exists !
   \widetilde{F}} \ar[r]^g \ar"2,1"_f \ar"4,2"^F \ar@{-->}"3,2"_{\exists G}
   & Y_i \ar"4,2"^{l_Y} \ar"3,2"                       \\
   & S_i \ar"4,2"                                               \\
   & S}    \]
$F(T) = l_X\circ f(T), f(T)\subseteq X_i,$ thus $F(T) =
l_X(f(T))\subseteq S_i,$ therefore $\exists G: T\rightarrow S_i$
such that $G$ coincides with $F$ on $T$ to $S_i.$ Hence
$X_i\times_{S_i} Y_i = X_i\times_S Y_i.$ Applying
$\mathit{3^{\circ}},$ we obtain that $X\times_S Y$ does exist.
\end{proof}
\begin{prop}\
\enum
\item[(1)]$X\times_S S=S$
\item[(2)]$X\times_S Y=Y\times_S X$
\item[(3)]$X\times_S (Y\times_T Z)=(X\times_S Y)\times_T Z$
\end{list}
\end{prop}
\begin{proof}\
\enum
\item[(1)]For any $S$-scheme $T$ and $S$-morphisms $f: T \rightarrow
X$ and $g: T\rightarrow S,$ we have the following commutative
diagram:
\[ \xymatrix{
   & X \ar"2,1"_{id_X} \ar"2,3"^{l_X}                             \\
   X \ar"3,2"_{l_X} & T \ar[l]_f \ar[r]^{l_X} \ar@{-->}[u]_{\exists
   ! f} \ar[d]^{l_X} & S \ar"3,2"^{id_S}                          \\
   & S}    \]
Hence $X\times_S S=X.$
\item[(2)]Obviously we have the following commutative diagram:
\[ \xymatrix{
   & {X\times_S Y} \ar"2,1"_q \ar"2,3"^p                          \\
   Y \ar"3,2" & T \ar[l] \ar[r] \ar@{-->}[u]_{\exists ! f} \ar[d]
   & X \ar"3,2"                                                   \\
   & S}    \]
Therefore $X\times_S Y=Y\times_S X.$
\item[(3)]We have the following two commutative diagram:
\[ \xymatrix{
   &  {X\times_S Y} \ar"2,1" \ar"2,3"                             \\
   X & & Y & {Y\times_T Z} \ar[l]                    \\
   & {X\times_S (Y\times_T Z)} \ar"2,1" \ar"2,4" \ar@{-->}[uu] } \]
\[   \xymatrix{
   & & {Y\times_T Z} \ar"2,1" \ar"2,4"                             \\
   Y & & & Z                                                     \\
   & {X\times_S Y} \ar"2,1" & M \ar"2,4" \ar[l] \ar@{-->}"1,3"}  \]
Hence we obtain
\[ \xymatrix{
   & & {X\times_S (Y\times_T Z)} \ar"2,1" \ar"2,5" \ar@{-->}"3,2"   \\
   X & & & Y & {Y\times_T Z} \ar[l] \ar[d]                          \\
   & {X\times_S Y} \ar"2,1" \ar"2,4" \ar"4,3" & M \ar[l] \ar"3,5" \ar@{-->}"1,3"
   \ar@{-->}"2,5" \ar"4,3" & & Z \ar"4,3"                         \\
   & & T}   \]
which indicates that $X\times_S (Y\times_T Z)=(X\times_S Y)\times_T
Z.$
\end{list}
\end{proof}
\begin{Def}
A diagram is called cartesian if it is of the following form:
\[ \xymatrix{
   & {X\times_S Y} \ar"2,1"_p \ar"2,3"^q      \\
   X \ar"3,2" & & Y \ar"3,2"                  \\
   & S } \]
\end{Def}
\begin{Def}
Let $X$ and $Y$ be two $S$-schemes, let $S^{\prime}$ be an
$S$-scheme. For any $S$-scheme $X,$ the projection $X\times_S
S^{\prime} \rightarrow S^{\prime}$ defines an $S^{\prime}$-scheme
structure on$X\times_S S^{\prime}.$ For any $S$-morphism $f:
X\rightarrow Y,$ we have
\[ \xymatrix@C=5em{
   {X\times_S S^{\prime}} \ar[d] \ar[r]^{f\times_S id_{S^{\prime}}} &
   {Y\times_S S^{\prime}} \ar[d]                                    \\
   X \ar[r]^f &Y } \]
Then we get a functor from $SCH_S$(the category of $S$-schemes) to
$SCH_{S^{\prime}}.$ The functor is called the base change from $S$
to $S^{\prime}.$
\end{Def}
\begin{remark}
$f\times_S id_{S^{\prime}}$is defined by the following commutative
diagram:
\[ \xymatrix{
   & {Y\times_S S^{\prime}} \ar"2,1" \ar"2,4"                  \\
   Y \ar"3,2" & X \ar[l]_f \ar"3,2"
   & {X\times_S S^{\prime}} \ar[l] \ar[r] \ar@{-->}"1,2"^{\exists !g:=f\times_S id_{S^{\prime}}}
   \ar"3,2" & S^{\prime} \ar"3,2"                                               \\
   & S  }   \]
\end{remark}
\begin{prop}
Base change and cartesian diagram coincide.
\end{prop}
\begin{proof}
$\Longrightarrow:$ $X\times_S S^{\prime}=(X\times_Y Y)\times_S
S^{\prime}=X\times_Y (Y\times_S S^{\prime}).$ Then
\end{proof}
\begin{prop}
The base change of immersions(resp. open or closed immersions) are
immersions(resp. open or closed immersions).
\end{prop}
\begin{proof}
$\mathit{1^{\circ}}$ Open immersions:

We have
\[ \xymatrix@C=5em{
   {X\times_S S^{\prime}} \ar[d] \ar[r]^{f\times_S id_{S^{\prime}}} &
   {Y\times_S S^{\prime}} \ar[d]                                    \\
   X \ar[r]^f &Y } \]
Since $f$ is an open immersion, then $X\cong f(X)$ which is an open
subscheme of $Y,$ then $X\times_S S^{\prime}\cong f(X)\times_S
S^{\prime}$ which is an open subscheme of $Y\times_S S^{\prime},$
for we have the following commutative diagram:
\[ \xymatrix@C=5em{
   {X\times_S S^{\prime}} \ar[d] \ar[r]^{f\times_S id_{S^{\prime}}} &
   {f(X)\times_S S^{\prime}} \ar[d] \ar[r]^{f^{-1}\times_S id_{S^{\prime}}}
   & {X\times_S S^{\prime}} \ar[d] \ar[r]^{f\times_S id_{S^{\prime}}} &
   {f(X)\times_S S^{\prime}} \ar[d]                                \\
   X \ar[r]^f &f(X) \ar[r]^{f^{-1}} &X \ar[r]^f & f(X) } \]
Hence $f\times_S id_{S^{\prime}}$ is an open immersion.

$\mathit{2^{\circ}}$ Closed immersions:

Let $SpecA$ be an affine open subscheme of $S,$ $SpecB$ be an affine
open subscheme of $l^{-1}_Y(SpecA),$ $SpecC$ be an open affine
subscheme of $f^{-1}(SpecB),$ $SpecA^{\prime}$ be an affine open
subscheme of $l^{-1}_{S^{\prime}}(SpecA).$ Then $SpecC\times_{SpecA}
SpecA^{\prime}$ is an open affine subscheme of $X\times_S
S^{\prime},$ and we have:
\[ \xymatrix{
   {Spec\frac{B\otimes_A A^{\prime}}{(I\otimes_A A^{\prime})^e}=SpecC\times_{SpecA}
   SpecA^{\prime}} \ar[d] \ar[r]^g &
   {SpecB\times_{SpecA} SpecA^{\prime}=SpecB\otimes_A A^{\prime}} \ar[d]\\
   SpecC \ar[r]^f & SpecB } \]
$C\cong B/I$ with $I$ an ideal of $B,$ for $f$ is a closed
immersion. Hence $g$ is a closed immersion, and therefore $f\times_S
id_{S^{\prime}}$ is.

$\mathit{3^{\circ}}$ General case:

Let $f:X\rightarrow Y$ be an immersion, then we have
\[ \xymatrix{
   X \ar"1,3"^f \ar"2,2"_p & & Y\\
   & Z \ar"1,3"_q} \]
where $Z$ is an $S$-scheme, and $p$ is a closed immersion, $q$ is an
open immersion. Then
\[ \xymatrix{
   {X\times_S S^{\prime}} \ar"1,3"^{f\times_S id_{S^{\prime}}} \ar"2,2"_{p\times_S id_{S^{\prime}}}
   & & {Y\times_S S^{\prime}}\\
   & {Z\times_S S^{\prime}} \ar"1,3"_{q\times_S id_{S^{\prime}}}} \]
Since $p$ is a closed immersion and $q$ is an open immersion, then
$p\times_S id_{S^{\prime}}$ is a closed immersion and $q\times_S
id_{S^{\prime}}$ is an open immersion, hence $f\times_S
id_{S^{\prime}}$ is an immersion by definition.
\end{proof}
\begin{prop}
Let $f: X\rightarrow Y$ be a morphism of schemes, $y\in Y,$ and
$k(y)$ be the residue field of the local ring $\mathcal {O}_{Y,y}.$
The projection $X\times_Y Speck(y)\rightarrow X$ induces an
embedding $\rho$ of $X\times_Y Speck(y)$ onto $f^{-1}(y)$ on the
underlying topological space. In this way, we can endow $f^{-1}(y)$
a scheme structure such that $\rho: X\times_Y Speck(y)\rightarrow
f^{-1}(y)$ is an isomorphism of schemes. We call $X\times_Y
Speck(y)$ the scheme-theoretic fibre of $f$ over $y.$
\end{prop}
\begin{proof}

\end{proof}
\begin{Def}
Let $f:X\rightarrow Y$ be a morphism of schemes. The diagonal
morphism $\Delta_{X/Y}: X\rightarrow X\times_Y X$ is by definition
the unique morphism satisfying
\[ \xymatrix{
   & {X\times_Y X} \ar"2,1"_{p} \ar"2,3"^{q}           \\
   X \ar"3,2" & X \ar@{-->}[u]_{\exists !\Delta_{X/Y}} \ar[l]_{id_X}
   \ar[r]^{id_X} \ar[d] & X  \ar"3,2"                 \\
   & Y  }     \]
We often denote $\Delta_{X/Y}$ by $\Delta.$ We say that $f:
X\rightarrow Y$ is separated or $X$ is separated over $Y$ if
$\Delta_{X/Y}$ is a closed immersion. A scheme $X$ is called
separated if it is separated over $Spec\mathbb{Z},$ i.e. the
canonical morphism $X\rightarrow Spec\mathbb{Z}$ is separated, which
is equivalent to saying that $\Delta: X\rightarrow X\times X$ is a
closed immersion.
\end{Def}
\begin{remarks}\
\enum
\item[(1)]Let $X$ and $Y$ be two schemes, let $U$ be an open subset
of $X$ and $V$ an open subset of $Y.$ Then $p^{-1}(U)\cap
q^{-1}(V)=U\times_S V.$
\item[(2)]Let $f:X\rightarrow Y$ be a morphism of schemes. Let
$U\subseteq X$ and $V\subseteq X$ be two open subsets. Then
$\Delta^{-1}(U\times_Y V)=U\cap V.$
\end{list}
\end{remarks}
\begin{proof}\
\enum
\item[(1)]We have the following commutative diagram:
\[ \xymatrix{
   & & {X\times_S Y} \ar"2,1"_p \ar"2,5"^q          \\
   X & \,U \ar@_{(->}[l] & M \ar[l] \ar[r] \ar@{-->}[u]_{\exists !F} &
   V\, \ar@^{(->}[r] &Y } \]
Since $F(M)\subseteq p^{-1}(U)\cap q^{-1}(V),$ hence we obtain:
\[ \xymatrix{
   & {p^{-1}(U)\cap q^{-1}(V)} \ar"2,1"_p \ar"2,3"^q          \\
   U & M \ar[l] \ar[r] \ar@{-->}[u]_{\exists !F} & V } \]
which indicates that $p^{-1}(U)\cap q^{-1}(V)=U\times_S V.$
\item[(2)]By $(1),$ $\Delta^{-1}(U\times_Y V)=\Delta^{-1}(p^{-1}(U)\cap
q^{-1}(V))=\Delta^{-1}(p^{-1}(U))\cap
\Delta^{-1}(q^{-1}(V))=id^{-1}_X(U)\cap id^{-1}_X(V)=U\cap V.$
\end{list}
\end{proof}
\begin{prop}
Let $f: SpecB\rightarrow SpecA$ be a morphism of affine schemes,
then $f$ is separated.
\end{prop}
\begin{proof}
\[ \xymatrix{
   & {SpecB\times_{SpecA} SpecB} \ar[dl]_p \ar[dr]^q      \\
   SpecB & SpecB \ar[l]_-{id_{SpecB}} \ar[r]^-{id_{SpecB}}
   \ar@{-->}[u]_-{\exists !\Delta} & SpecB }  \]
induces
\[ \xymatrix{
   & {B\otimes_A B} \ar@{-->}[d]^{\alpha}                       \\
   B \ar[ur] \ar[r]^-{id_B} & B & B \ar[l]_-{id_B} \ar[ul] }  \]
where
\[ \xymatrix@R=0em{
   {\alpha=\Delta^{\sharp}(SpecB\otimes_A B):B\otimes_A B} \ar[r] &
   B                                                 \\
   {(b_1\otimes_A 1)\cdot(1\otimes_A b_2)=b_1\otimes b_2}
   \ar@{|->}[r] & b_1b_2 }  \]

Since $\alpha$ is surjective and $\Delta=Spec\alpha,$ then $\Delta$
is a closed immersion. Therefore $f$ is separated.
\end{proof}
\begin{prop}
Let $f: X\rightarrow Y$ be a morphism of schemes,
\enum
\item[(1)]$\Delta_{X/Y}:X\rightarrow X\times_Y X$ is an immersion.
\item[(2)]$f:X\rightarrow Y$ is separated iff $\Delta(X)$ is a
closed subset of $X\times_Y X.$
\end{list}
\end{prop}
\begin{proof}\
\enum
\item[(1)]Cover $Y$ by affine open subschemes $V_i(i\in I),$ cover
$f^{-1}(V_i)$ by affine open subschemes $V_{ij}(j\in J_i).$ Then
$$p^{-1}(V_{ij})\cap q^{-1}(V_{ij})=V_{ij}\times_Y V_{ij}=V_{ij}\times_{V_i}
V_{ij}.$$ Put $V=\bigcup\limits_{i\in I \atop j\in
J_i}p^{-1}(V_{ij})\cap q^{-1}(V_{ij}),$ then $\Delta(X)\subseteq V,$
an open subscheme of $X\times_Y X.$ Indeed $X=\bigcup\limits_{i\in I
\atop j\in J_i}V_{ij}.$ Since $\Delta^{-1}(V_{ij}\times_{V_i}
V_{ij})=V_{ij}\cap V_{ij}=V_{ij},$ thus
$$\Delta(V_{ij})=\Delta(\Delta^{-1}(V_{ij}\times_{V_i} V_{ij}))\subseteq
V_{ij}\times_{V_i} V_{ij}=p^{-1}(V_{ij})\cap q^{-1}(V_{ij}),$$ hence
$$\Delta(X)=\Delta(\bigcup\limits_{i\in I \atop j\in J_i}V_{ij})=\bigcup\limits_{i\in I
\atop j\in J_i}\Delta(V_{ij})\subseteq \bigcup\limits_{i\in I \atop
j\in J_i}p^{-1}(V_{ij})\cap q^{-1}(V_{ij})=V.$$

We want to show that $\Delta:X\rightarrow V$ is a closed immersion,
which is equivalent to saying that $\forall i\in I,j\in J_i$,
$$\Delta:\Delta^{-1}(p^{-1}(V_{ij})\cap
q^{-1}(V_{ij}))=V_{ij}\rightarrow V_{ij}\times_{V_i}
V_{ij}=p^{-1}(V_{ij})\cap q^{-1}(V_{ij})$$ is a closed immersion.
Since $V_{ij}$ and $V_i$ are affine, hence
$\left.f\right|_{V_{ij}}:V_{ij}\rightarrow V_i$ is separated, then
$\left.\Delta\right|_{V_{ij}}:V_{ij}\rightarrow V_{ij}\times_{V_i}
V_{ij}$ is a closed immersion. Therefore $\Delta:X\rightarrow
X\times_Y X$ is an immersion, for we have
\[ \xymatrix{
   X \ar[rr]^-{\Delta} \ar[dr]_{\text{closed immersion}} & & {X\times_Y
   X}                                              \\
   & V \ar@^{(->}[ur]_{\text{open immersion}} }  \]
\item[(2)]Note that $\Delta$ is a closed immersion iff $\Delta(X)$
is closed in $X\times_Y X,$ when $\Delta$ is an immersion. Hence
$(2)$ comes directly from $(1).$
\end{list}
\end{proof}
\begin{prop}[Base Change]
Let
\[ \xymatrix{
    X \ar"2,2"_{f_1} & & Y \ar"2,2"^{f_2}  \\
    & T \ar[d]^f  \\
    & S } \]
be a family of morphisms. Then $$X\times_T Y = T\times_{T\times_S T}
(X\times_S Y)$$ Or equivalently
\[ \xymatrix{
   {X\times_T Y} \ar[d] \ar[r]^g & {X\times_S Y} \ar[d]^{f_1
   \times_S f_2}                                          \\
   T \ar[r]^{\Delta_{T/S}} & {T\times_S T} }    \]
is cartesian or a base change.
\end{prop}
\begin{remark}
$g$ is induced by the following commutative diagram:
\[ \xymatrix{
   & {X\times_S Y} \ar"2,1"_{p_S} \ar"2,3"^{q_S}          \\
   X \ar"3,2" \ar"4,2" & {X\times_T Y} \ar[l]^{p_T} \ar[r]_{q_T}
   \ar[d] \ar@{-->}[u]_{\exists ! g} & Y \ar"3,2" \ar"4,2" \\
   & T \ar[d]                                             \\
   & S} \]
\end{remark}
\begin{proof}
$\mathit{1^{\circ}}$Affine proof:

Cover $S$ by affine open subschemes $(S_i)_{i\in I},$ cover
$f^{-1}(S_i)$ by affine open subschemes $(T_{ij})_{j\in J_i},$ cover
$f_1^{-1}(T_{ij})$ by affine open subschemes $(X_{ijk})_{k\in
K_{ij}},$ and cover $f_2^{-1}(T_{ij})$ by affine open subschemes
$(Y_{ijl})_{l\in L_{ij}}.$ Write $S_i=SpecA, T_{ij}=SpecB,
X_{ijk}=SpecC, Y_{ijl}=SpecD.$ Since
\[ \xymatrix{
    X_{ijk} \ar"2,2" & & Y_{ijl} \ar"2,2"  \\
    & T_{ij} \ar[d]  \\
    & S_i } \]
then we obtain the following commutative diagram:
\[ \xymatrix{
   {X_{ijk}\times_{T_{ij}} Y_{ijl}} \ar[d] \ar[r] &
   {X_{ijk}\times_{S_i} Y_{ijl}} \ar[d]            \\
   T_{ij} \ar[r] & {T_{ij}\times_{S_i} T_{ij}} }  \]
for it is induced by the canonical commutative diagram:
\[ \xymatrix{
   {C\otimes_B C} & {C\otimes_A D} \ar[l]   \\
   B \ar[u]  & {B\otimes_A B} \ar[u] \ar[l] }  \]
Hence we get the desired result.

$\mathit{2^{\circ}}$Proof by universal property of fibred product:

We have the following commutative diagram:
\[ \xymatrix@R=4em{
   & & {T\times_S T} \ar"2,1"_-{p_T} \ar"2,5"^-{q_T}    \\
   T \ar"3,3" & X \ar[l]_-{f_1} & {X\times_S Y} \ar[l]_-{p_S}
   \ar[r]^-{q_S} \ar[d] \ar[u]_-{\exists !\varphi} & Y \ar[r]^-{f_2} &
   T \ar"3,3"                                         \\
   & & S} \]
then $p_T\circ\varphi=f_2\circ q_S=id_T\circ f_2 \circ q_S.$ But
$id_T=p_T\circ \Delta_{T/S},$ thus $p_T\circ \varphi=p_T\circ
\Delta_{T/S}\circ f_2\circ q_S,$ hence $\varphi=\Delta_{T/S}\circ
f_2\circ q_S,$ for $p_T$ is surjective. Then we obtain the following
commutative diagram:
\[ \xymatrix{
   &   {T \times_{T \times_S T} (X \times_S Y)} \ar"5,2" \ar"5,4"
   \\\\\\
   & & X \ar"5,2"^{f_1}                                               \\
   S & T \ar"5,1" \ar"7,3"_{\bigtriangleup}
   &Z \ar"5,2" \ar"4,3" \ar"6,3" \ar@{-->}"5,4"^-{\exists !G}
   \ar@{-->}"1,2"^{\exists !F}
   &{X \times_S Y} \ar"4,3"^{p_S} \ar"6,3"_{q_S} \ar"7,3"^{\varphi}     \\
   & & Y \ar"5,2"_{f_2}                                               \\
   & &  {T \times_S T}            }     \]
Therefore $X\times_S Y=T\times_{T\times_S T} (X\times_S Y).$
\end{proof}
\begin{prop}
Let $f:X\rightarrow Y$ be a morphism of $S$-schemes. \enum
\item[(1)]There exists a unique morphism $\Gamma_f: X\rightarrow
X\times_S Y$ such that
\[ \xymatrix{
   & {X\times_S Y} \ar"2,1"_p \ar"2,3"^q               \\
   X \ar"3,2" & X \ar[l]_{id_X} \ar[r]^f \ar[d] \ar[u]_{\exists !\Gamma_f}
   & Y \ar"3,2"   \\
   &  S} \]
i.e. $p\circ \Gamma_f=id_X, q\circ \Gamma_f=f.$ Moreover
\[ \xymatrix@C=4em{
   X \ar[d] \ar[r]^-f & {X\times_S Y} \ar[d]^{f\times_S id_Y}  \\
   Y \ar[r]^-{\Delta_{Y/S}} & {Y\times_S Y} }    \]
is cartesian. We call $\Gamma_f$ the graph of the $S$-morphism $f:
X\rightarrow Y.$
\item[(2)]$\Gamma_f$ is an immersion.
\item[(3)]If $Y$ is separated over $S,$ then $\Gamma_f$ is a closed
immersion.
\end{list}
\end{prop}
\begin{proof}\
\enum
\item[(1)]Consider the family of morphisms:
\[ \xymatrix{
    X \ar"2,2"_f & & Y \ar"2,2"^{id_Y}  \\
    & Y \ar[d]  \\
    & S } \]
then
\[ \xymatrix{
   {X\times_Y Y=X} \ar[d] \ar[r]^-f & {X\times_S Y} \ar[d]^{f\times_S id_Y}\\
   Y \ar[r]^-{\Delta_{Y/S}} & {Y\times_S Y} }    \]
is cartesian.
\item[(2)]Since$\Delta_{Y/S}$ is an immersion and $\Gamma_f$ is a base change
of $\Delta_{Y/S},$ hence $\Gamma_f$ is an immersion.
\item[(3)]Since $Y$ is separated over $S,$then $\Delta_{Y/S}$ is a closed
immersion. $\Gamma_f$ is a base change of $\Delta_{Y/S},$ hence
$\Gamma_f$ is a closed immersion as well.
\end{list}
\end{proof}
\begin{Def}
A morphism of schemes $f: X\rightarrow Y$ is called quasi-separated
if $\Delta_{X/Y}: X\rightarrow X\times_Y X$ is quasi-compact. A
scheme $X$ is called quasi-separated if the canonical morphism
$X\rightarrow Spec\mathbb{Z}$ is quasi-separated.
\end{Def}
\begin{remark}
If $f: X\rightarrow Y$ is separated, then it is quasi-separated.
Indeed if $f$ is separated, then $\Delta_{X/Y}$ is a closed
immersion, hence quasi-compact. i.e. for any affine open subscheme
$V$ of $X\times_Y X,$ $\Delta_{X/Y}^{-1}(V)$ is quasi-compact. Write
$V=SpecA.$ But $\Delta: \Delta^{-1}(V)\rightarrow V$ is a closed
immersion, hence $\Delta^{-1}(V)\cong SpecA/I$ with $I$ an ideal of
$A,$ therefore$\Delta^{-1}(V)$ is quasi-compact.
\end{remark}
\begin{prop}
Let $f: X\rightarrow SpecA$ be a morphism of schemes. Let $U$ and
$V$ be two affine open subschemes of $X.$
\enum
\item[(1)]If $f$ is separated, then $U\cap V$ is affine and the ring
$\mathcal {O}_X(U\cap V)$ is generated by the image of $\rho_{U\,
U\cap V}: \mathcal {O}_X(U)\rightarrow \mathcal {O}_X(U\cap V)$ and
$\rho_{V\, U\cap V}: \mathcal {O}_X(V)\rightarrow \mathcal
{O}_X(U\cap V).$
\item[(2)]If $f$ is quasi-separated, then $U\cap V$ can be covered
by finitely many affine open subschemes.
\end{list}
\end{prop}
\begin{proof}\
\enum
\item[(1)]Suppose that $f": X\rightarrow SpecA$ is separated, then
$\Delta:X\rightarrow X\times_{SpecA} X$ is a closed immersion, thus
$$\Delta: \Delta^{-1}(p^{-1}(U)\cap q^{-1}(V))=U\cap
V\rightarrow U\times_{SpecA} V=p^{-1}(U)\cap q^{-1}(V)$$ is a closed
immersion. But $U$ and $V$ are affine, so $U\times_{SpecA} V$ is
affine, and thus $U\cap V$ is affine as well.

Put $U=SpecB$ with $B=\mathcal {O}_X(U), V=SpecC$ with $C=\mathcal
{O}_X(V), U\cap V=SpecD$ with $D=\mathcal {O}_X(U\cap V).$ Then
$\Delta: SpecD\rightarrow SpecB\otimes_A C$ is induced by $g:
B\otimes_A C\rightarrow D\cong {B\otimes_A C}/I$ with $I$ an ideal
of $B\otimes_A C.$ But $g$ is induced by $\rho_{U\, U\cap V}:
B\rightarrow D$ and $\rho_{V\, U\cap V}: C\rightarrow D.$ So
$D=\mathcal {O}_X(U\cap V)$ is generated by the images of $\rho_{U\,
U\cap V}$ and $\rho_{V\, U\cap V}.$
\item[(2)]If $f$ is quasi-separated, then $\Delta$ is
quasi-compact. Hence $U\cap V=\Delta^{-1}(U\times_{SpecA} V)$ is
quasi-compact for $U\times_{SpecA} V$ is an open affine subscheme of
$X\times_{SpecA} X.$ But $U\cap V$ is open in $X,$ hence can be
covered by finitely many affine open subschemes.
\end{list}
\end{proof}
\begin{Def}
Let $f: X\rightarrow Y$ be a morphism of schemes.We say that $f$ is
proper or $X$ is proper over $Y$ if $f$ is of finite type and
separated such that for any morphism $Y^{\prime}\rightarrow Y,$ the
base change
\[ \xymatrix{
   {X\times_Y Y^{\prime}} \ar[d] \ar[r]^-{f^{\prime}} &
   {Y^{\prime}=Y\times_Y Y^{\prime}} \ar[d]                 \\
   X \ar[r]^-{f} & Y}  \]
$(f^{\prime}=f\times_Y id_{Y^{\prime}})$ of $f$ is a closed mapping
of topological space.(In this case, we say that f is universally
closed).
\end{Def}
\begin{prop}
Let $f: X\rightarrow Y$ be a morphism of schemes. Let $(V_i)_{i\in
I}$ be an open covering of $Y.$ \enum
\item[(1)]$f$ is separated iff
$\left.f\right|_{f^{-1}(V_i)}:f^{-1}(V_i)\rightarrow V_i$ is
separated, $\forall i\in I.$
\item[(2)]$f$ is universally closed iff
$\left.f\right|_{f^{-1}(V_i)}:f^{-1}(V_i)\rightarrow V_i$ is
universally closed, $\forall i\in I.$
\item[(3)]$f$ is proper iff
$\left.f\right|_{f^{-1}(V_i)}:f^{-1}(V_i)\rightarrow V_i$ is proper,
$\forall i\in I.$
\end{list}
\end{prop}
\begin{proof}\
\enum
\item[(1)]$f$ is separated $\Longleftrightarrow \Delta_{X/Y}:
X\rightarrow X\times_Y X$ is a closed immersion. Consider
$\Delta_i:=\left.\Delta_{X/Y}\right|_{f^{-1}(V_i)}:f^{-1}(V_i)\rightarrow
p^{-1}(f^{-1}(V_i))\cap q^{-1}(f^{-1}(V_i))=f^{-1}(V_i)\times_{V_i}
f^{-1}(V_i),$ note the fact that $\Delta_{X/Y}$ is a closed
immersion iff $\Delta_{X/Y}$ is a closed embedding and $\forall p\in
X, \Delta_{X/Y,p}^{\sharp}$ is surjective, which is equivalent to
saying that $\forall i\in I, \Delta_i$ is a closed embedding and
$\forall p\in f^{-1}(V_i), \Delta_{i,p}^{\sharp}$ is surjective.
Hence $\Delta$ is a closed immersion $\Leftrightarrow \forall i\in
I, \Delta_i$ is a closed immersion, but $\Delta_i$ is a closed
immersion $\Leftrightarrow \left.f\right|_{f^{-1}(V_i)}$ is
separated.
\item[(2)]$\Longrightarrow:$ Put $f_i=\left.f\right|_{f^{-1}(V_i)}:f^{-1}(V_i)\rightarrow
V_i.$ We need to  show that $f_i$ is universally closed, $\forall
i\in I.$ Consider $g:Y^{\prime}\rightarrow V_i.$ We have the
following commutative diagram:
\[ \xymatrix{
   & {f^{-1}(V_i)} \ar@_{(->}"2,1" \ar"2,3"^{f_i}               \\
   X \ar"3,2"^f & T \ar[l]_{g_1} \ar[r]^{g_2} \ar[d] \ar@{-->}[u]_{\exists !g_1}
   & V_i \ar@_{(->}"3,2"   \\
   &  Y} \]
for we have $f\circ g_1=g_2,$ hence $g_1(T)\subseteq f^{-1}(V_i),$
thus $f^{-1}(V_i)=X\times_Y V_i.$ Or we can see that the following
diagram is cartesian
\[ \xymatrix{
   & {X=X\times_Y Y} \ar"2,1"_{id_X} \ar"2,3"^f   \\
   X \ar"3,2" & & Y \ar"3,2"                        \\
   & Y } \]
and $f^{-1}(V_i)=X\times_Y V_i.$ Therefore the following diagram is
cartesian, thus is a base change:
\[ \xymatrix@C=5em{
   {f^{-1}(V_i)\times_{V_i} Y^{\prime}} \ar[d] \ar[r]^-{f_i^{\prime}} &
   {Y^{\prime}} \ar[d]                                          \\
   {f^{-1}(V_i)} \ar[r]^-{f_i} \ar@^{(->}[d] & V_i \ar@^{(->}[d] \\
   X \ar[r]^f & Y}  \]
Since $f: X\rightarrow Y$ is universally closed, so the base change
$f_i^{\prime}: f^{-1}(V_i)\times_{V_i} Y^{\prime}\rightarrow
Y^{\prime}$ is a closed mapping of topological space. Hence $f_i$ is
universally closed.

$\Longleftarrow:$

\begin{lemma}
Let $X$ be a topological space, and $(X_i)_{i\in I}$ an open
covering of $X.$ Let $F\subseteq X,$ then $F$ is closed in $X$ iff
each $F\cap X_i$ is closed in $X_i.$
\end{lemma}
$\star$\ If $F\cap X_i$ is closed in $X_i,$ then $X_i\setminus
F=X_i\setminus (F\cap X_i)$ is open in $X_i,$ hence open in $X,$ for
$X_i$ is open in $X.$ But $X\setminus F=\bigcup\limits_{i\in
I}(X_i\setminus F)$ is open, hence $F$ is closed in $X.$ The
converse is direct. \hspace*{\fill} $\star$

Suppose that each $f_i:=\left.f\right|_{f^{-1}(V_i)}:
f^{-1}(V_i)\rightarrow V_i$ is universally closed. Let $f:
Y^{\prime}\rightarrow Y$ be a morphism of schemes. Consider the base
change
\[ \xymatrix@C=5em{
   {X\times_Y Y^{\prime}} \ar[d] \ar[r]^-{f^{\prime}} &
   {Y^{\prime}=Y\times_Y Y^{\prime}} \ar[d]^g                    \\
   X \ar[r]^-{f} & Y }  \]
We need to show that $f^{\prime}$ is a closed mapping of topological
space. Put
$f_i^{\prime}:=\left.f^{\prime}\right|_{f^{-1}(V_i)\times_{V_i}
g^{-1}(V_i)},$ then
\[ \xymatrix@C=5em{
   {f^{-1}(V_i)\times_{V_i} g^{-1}(V_i)} \ar[d] \ar[r]^-{f^{\prime}_i} &
   {g^{-1}(V_i)} \ar[d]                          \\
   {f^{-1}(V_i)} \ar[r]^-{f_i} & V_i }  \]
is a base change and $f_i$ is universally closed, hence
$f^{\prime}_i$ is a closed mapping. Then $\forall F$ closed in
$X\times_Y Y^{\prime},$ $F\cap (f^{-1}(V_i)\times_{V_i}
g^{-1}(V_i))$ is closed in $f^{-1}(V_i)\times_{V_i} g^{-1}(V_i),$
thus $f_i^{\prime}(F)=f_i^{\prime}(F\cap (f^{-1}(V_i)\times_{V_i}
g^{-1}(V_i)))$ is closed in $g^{-1}(V_i).$ Since
$f^{\prime}(F)=\bigcup\limits_{i\in I}f_i^{\prime}(F)$ and
$Y^{\prime}=\bigcup\limits_{i\in I}g^{-1}(V_i),$ then $f^{\prime}$
is closed in $Y^{\prime}$ by the lemma. Therefore $f^{\prime}$ is a
closed mapping.
\item[(3)]comes from $(1)$ and $(2)$ and the fact that the property
to be of finite type is a local property.
\end{list}
\end{proof}
\begin{prop}
Immersions are separated, closed immersions are proper.
\end{prop}
\begin{proof}
$\mathit{1^{\circ}}$Immersions are separated:

Let $f: X\rightarrow Y$ be an immersion, then $$X\times_Y X\cong
X\times_Y f(X)=X\times_{f(X)} f(X)=X$$ and then $\Delta_{X/Y}=id_X.$
Or we can see that
\[ \xymatrix{
   & X \ar"2,1"_{id_X} \ar"2,3"^{id_X}               \\
   X \ar"3,2"^f & T \ar[l]_{f_1} \ar[r]^{f_2} \ar[d] \ar@{-->}[u]_{\exists !g_1}
   & X \ar"3,2"^f   \\
   &  Y} \]
for $f$ is an immersion, thus a monomorphism in the category of
schemes, then $$f\circ f_1=f\circ f_2 \Longrightarrow f_1=f_2$$
Therefore $X\times_Y X,$ and then $\Delta_{X/Y}=id_X.$
\[ \xymatrix{
   & {X\times_Y X=X} \ar"2,1"_{id_X} \ar"2,3"^{id_X}               \\
   X \ar"3,2"^f & X \ar[l]_{id_X} \ar[r]^{id_X} \ar[d]^f \ar@{-->}[u]_{\exists !\Delta_{X/Y}}
   & X \ar"3,2"^f   \\
   &  Y} \]
Thus $\Delta_{X/Y}$ is an isomorphism of schemes, hence is a closed
immersion. Therefore $f: X\rightarrow Y$ is separated.

$\mathit{2^{\circ}}$Closed immersions are proper:

Let $f: X\rightarrow Y$ be a closed immersion.
\enum
\item[(a)]$f$ is of finite type:

Indeed $f$ is finite. Let $V=SpecA$ be an affine open subscheme of
$Y.$ Then $\left.f\right|_{f^{-1}(V)}: f^{-1}(V)\rightarrow V$ is a
closed immersion, thus is isomorphic to the canonical morphism
$SpecA/I\rightarrow SpecA$ with $I$ an ideal of $A.$ It is finite,
for the canonical ring homomorphism $A\rightarrow A/I$ induces an
$A$-algebra structure on $A/I,$ and $A/I$ is a finitely generated
$A$-module($A/I=A\cdot\bar{1}$ with $\bar{1}=1+I$). Hence $f$ is
finite.
\item[(b)]$f$ is separated, for $f$ is an immersion.
\item[(c)]$f$ is universally closed:

Let $g: Y^{\prime}\rightarrow Y$ be a morphism, then the base change
$f^{\prime}: X\times_Y Y^{\prime}\rightarrow Y^{\prime}$ of $f$ is a
closed immersion, in particular, a closed mapping of topological
spaces.
\end{list}

By $(a),(b),(c),$ we obtain that $f$ is proper.
\end{proof}
\begin{prop}
The composite of two separated morphism is separated.
\end{prop}
\begin{proof}
Let $f: X_1\rightarrow X_2$ and $g: X_2\rightarrow X_3$ be two
separated morphisms. Then $\Delta_f:=\Delta_{X_1/X_2}:
X_1\rightarrow X_1\times_{X_2} X_1$ and $\Delta_g:=\Delta_{X_2/X_3}:
X_2\rightarrow X_2\times_{X_3} X_2$ are closed immersions. We need
to show that $\Delta_{g\circ f}:=\Delta_{X_1/X_3}: X_1\rightarrow
X_1\times_{X_3} X_1$ is a closed immersion.

Since
\[ \xymatrix{
   {X_2\times_{X_2}(X_1\times_{X_2}X_1)} \ar@{=}[d] &
   {(X_2\times_{X_3}X_2)\times_{X_2}(X_1\times_{X_2}X_1)} \ar@{=}[d] \\
   {X_1\times_{X_2}X_1} \ar[r]^-{\Delta_g^{\prime}} \ar[d] &
   {X_1\times_{X_3}X_1} \ar[d]                              \\
   X_2 \ar[r]^-{\Delta_g} & {X_2\times_{X_3}X_2} }  \]
is a base change, hence $\Delta_g^{\prime}$ is a closed immersion.
Moreover we can check that
$\Delta_g^{\prime}=\Delta_g\times_{X_2}id_{X_1\times_{X_2}X_1}$.
Then we obtain the following commutative diagram
\[ \xymatrix@R=3em{
   & {X_1\times_{X_3} X_1} \ar"3,1" \ar"3,3"                  \\
   & {X_1\times_{X_2} X_1} \ar"3,1" \ar"3,3" \ar@{-->}[u]_(.3){\exists
   !\Delta^{\prime}_g}                                     \\
   X_1 \ar"4,2" \ar"5,2" & X_1 \ar[l] \ar[r] \ar[d] \ar@{-->}[u]_-{\exists
   !\Delta_f} & X_1 \ar"4,2" \ar"5,2"                      \\
   & X_2 \ar[d]                                               \\
   & X_3 }  \]
Therefore $\Delta_{g\circ f}=\Delta_g^{\prime}\circ\Delta_f.$ Hence
$\Delta_{g\circ f}$ is a closed immersion, for $\Delta_g^{\prime}$
and $\Delta_f$ are closed immersions.
\end{proof}
\begin{prop}
The composite of two universally closed morphism is universally
closed.
\end{prop}
\begin{proof}
Let $X_1\stackrel{f}{\rightarrow}X_2\stackrel{g}{\rightarrow}X_3$ be
two universally closed morphism. Let $Y^{\prime}$ be an
$X_3$-scheme. Since $X_2\stackrel{g}{\rightarrow}X_3$ is universally
closed, thus the base change $X_2\times_{X_3}
Y^{\prime}\stackrel{g^{\prime}}{\rightarrow}Y^{\prime}$ of $g$ is a
closed mapping. Since $X_1\stackrel{f}{\rightarrow}X_2$ is
universally closed, so the base change $X_1\times_{X_2}
(X_2\times_{X_3}Y^{\prime})=X_1\times_{X_3}
Y^{\prime}\stackrel{f^{\prime}}{\rightarrow}X_2\times_{X_3}Y^{\prime}$
of $f$ is a closed mapping. Then we obtain a base change of $g\circ
f$:
\[ \xymatrix{
   {X_1\times_{X_2}(X_2\times_{X_3}Y^{\prime})=X_1\times_{X_3}
   Y^{\prime}} \ar[d] \ar[r]^-{f^{\prime}} &
   {X_2\times_{X_3}Y^{\prime}} \ar[d] \ar[r]^-{g^{\prime}} &
   Y^{\prime} \ar[d]                                         \\
   X_1 \ar[r]^f & X_2 \ar[r]^g & X_3} \]
So the base change of $g\circ f$ is $g^{\prime}\circ f^{\prime},$
thus is a closed mapping. Therefore $g\circ f$ is universally
closed.
\end{proof}
\begin{prop}
The composite of two proper morphism is proper.
\end{prop}
\begin{proof}
We only have to prove that the composite of two morphisms which are
of finite type is of finite type. Let
$X_1\stackrel{f}{\rightarrow}X_2\stackrel{g}{\rightarrow}X_3$ be of
finite type. Let $V=SpecA$ be an affine open subscheme of $X_3.$
Since $g$ is of finite type, hence $g^{-1}(V)$ can be covered by
finitely many affine open subschemes $V_i=SpecA_i(i\in I)$ with
$A_i$ finitely generated $A$-algebra. Since $f$ is of finite type,
then each $f^{-1}(V_i)$ can be covered by finitely many affine open
subschemes $V_{ij=}SpecA_{ij}(j\in J_i)$ with $A_{ij}$ finitely
generated $A_i$-algebra. Then $$(g\circ
f)^{-1}(V)=\bigcup\limits_{i\in I \atop j\in
J_i}V_{ij}=\bigcup\limits_{i\in I \atop j\in J_i}SpecA_{ij}$$ with
$I$ and $J_i$ being finite sets. Since each $A_{ij}$ is finitely
generated as an $A-i$-algebra, and each $A_i$ is finitely generated
as an $A$-algebra, thus $A_{ij}$ is a finitely generated
$A$-algebra. Therefore $g\circ f$ is of finite type.
\end{proof}
\begin{prop}
Let $f: X\rightarrow Y$ and $g:Y^{\prime}\rightarrow Y$ be two
morphisms of schemes. Let $f^{\prime}: X\times_Y
Y^{\prime}\rightarrow Y^{\prime}$ be the base change of $f.$ \enum
\item[(1)]If $f$ is separated, then $f^{\prime}$ is separated.
\item[(2)]If $f$ is proper, then $f^{\prime}$ is proper.
\end{list}
\end{prop}
\begin{proof}\
\enum
\item[(1)]Put $X^{\prime}=X\times_Y Y^{\prime},$ then we have the
following commutative diagram:
\[ \xymatrix@R=5em@C=3em{
   & {(X\times_Y Y^{\prime})\times_{Y^{\prime}} (X\times_Y Y^{\prime})}
   \ar"2,1" \ar "2,4" \ar@{-->}"4,2"^{\exists ! \varphi}          \\
   {X\times_Y Y^{\prime}} \ar[d] & & {X\times_Y Y^{\prime}}
   \ar"2,1"_(.3){id_{X\times_Y Y^{\prime}}} \ar[r]_{id_{X\times_Y
   Y^{\prime}}} \ar@{-->}"1,2"^{\exists !
   \Delta_{X^{\prime}/Y^{\prime}}} \ar[d] & {X\times_Y Y^{\prime}} \ar[d] \\
   X & & X \ar"3,1"_(.3){id_X} \ar[r]^{id_X} \ar@{-->}"4,2"_{\exists !
   \Delta_{X/Y}} & X                                          \\
   & {X\times_Y X} \ar"3,1" \ar"3,4" }  \]
Hence
\[ \xymatrix{
   X^{\prime} \ar[d] \ar[r]^-{\Delta_{X^{\prime}/Y^{\prime}}} &
   {X^{\prime}\times_{X^{\prime}} Y^{\prime}=(X\times_Y X)\times_Y Y^{\prime}} \ar[d]^{\varphi}\\
   X \ar[r]^{\Delta_{X/Y}} & X\times_Y X } \]
is a base change. If $f$ is separated, then $\Delta_{X/Y}$ is a
closed immersion, so is $\Delta_{X^{\prime}/Y^{\prime}},$ and thus
$f^{\prime}: X^{\prime}\rightarrow Y^{\prime}$ is separated.
\item[(2)]If $f$ is proper, then it is of finite type, separated,
and universally closed.

Let
\[ \xymatrix{
   {X\times_Y Y^{\prime}} \ar[d]_p \ar[r]^-{f^{\prime}}_-{q} &
   Y^{\prime} \ar[d]^g                                     \\
   X \ar[r]^f & Y } \]
be a base change of $f.$ Let $V_i=SpecA_i(i\in I)$ be an open
covering of $Y,$ then for each $i\in I,$ $f^{-1}(V_i)$ is covered by
$(SpecB_{ik})_{k\in K_i}$ with $K_i$ a finite set and each $B_{ik}$
finitely generated as an $A_i$-algebra, and $g^{-1}(V_i)$ is covered
by $(SpecA_{ij})_{j\in J_i}.$ Then we obtain
\[ \xymatrix{
   {SpecB_{ik}\otimes_{A_i} A_{ij}=SpecB_{ik}\times_{SpecA_i} SpecA_{ij}}
   \ar[d]_p \ar[r]^-{f^{\prime}}_-{q} & SpecA_{ij} \ar[d]^g      \\
   SpecB_{ik} \ar[r]^f & SpecA_i } \]
$(SpecA_{ij})_{i\in I \atop j\in J_i}$ forms an affine open covering
of $Y^{\prime}.$ $\forall j\in J_i, (f^{\prime})^{-1}(SpecA_{ij})$
is covered by
$SpecB_{ik}\times_{SpecA_i}SpecA_{ij}=SpecB_{ik}\otimes_{A_i}A_{ij}(k\in
K_i),$ with $K_i$ a finite set and $B_{ik}$ a finitely generated
$A_i$-algebra. Hence $\forall k\in K_i,$ $B_{ik}\otimes_{A_i}
A_{ij}$ is a finitely generated $A_{ij}$-algebra. Therefore
$f^{\prime}$ is of finite type.

Put $X^{\prime}=X\times_Y Y^{\prime},$ and let
\[ \xymatrix{
   {X^{\prime}\times_{Y^{\prime}} Y^{\prime\prime}}
   \ar[r]^-{f^{\prime\prime}} \ar[d] & {Y^{\prime\prime}}
   \ar[d]^{g^{\prime}}                                 \\
   {X^{\prime}} \ar[r]^-{f^{\prime}} & {Y^{\prime}} }  \]
be a base change of $f^{\prime}.$ Then
\[ \xymatrix{
   {X^{\prime}\times_{Y^{\prime}} Y^{\prime\prime}}
   \ar[r]^-{f^{\prime\prime}} \ar[d] & {Y^{\prime\prime}}
   \ar[d]^{g^{\prime}}                                            \\
   {X^{\prime}} \ar[r]^-{f^{\prime}} \ar[d] & {Y^{\prime}} \ar[d] \\
   X \ar[r]^-{f} & Y }  \]
is a base change.

Since $f$ is universally closed, thus $f^{\prime\prime}$ is a closed
mapping of topological spaces, and then $f^{\prime}$ is universally
closed.

By $(1)$ we know that $f^{\prime}$ is separated.

Therefore $f^{\prime}$ is proper.
\end{list}
\end{proof}
\begin{prop}
Let $f: X\rightarrow Y$ and $f^{\prime}: X^{\prime}\rightarrow
Y^{\prime}$ be two $S$-morphisms. Let $f\times_S f^{\prime}$ be the
unique morphism such that
\[ \xymatrix{
   & {X\times_S X^{\prime}} \ar"2,1" \ar"2,3"
   \ar@{-->}"4,2"^{\exists ! f\times_S f^{\prime}}              \\
   X \ar[d]_f & & X^{\prime} \ar[d]^{f^{\prime}}                \\
   Y & & Y^{\prime}                                             \\
   & {Y\times_S Y^{\prime}} \ar"3,1" \ar"3,3"} \]
\enum
\item[(1)]If $f$ and $f^{\prime}$ are separated, so is $f\times_S
f^{\prime}.$
\item[(2)]If $f$ and $f^{\prime}$ are proper, so is $f\times_S
f^{\prime}.$
\end{list}
\end{prop}
\begin{proof}
We claim that $f\times_S f^{\prime}=(f\times_S id_{Y^{\prime}})\circ
(id_X\times_S f^{\prime}).$ Indeed we have the following commutative
diagram:
\[ \xymatrix{
   & {X\times_S X^{\prime}} \ar"2,1" \ar"2,4"
   \ar@{-->}"5,2"_(.3){\exists ! f\times_S f^{\prime}}
   \ar@{-->}"3,3"^{\exists ! id_X\times_S f^{\prime}}            \\
   X \ar[d]_{id_X} & & & X^{\prime} \ar[d]^{f^{\prime}}          \\
   X \ar[d]_f & & {X\times_S Y^{\prime}} \ar"3,1" \ar[r]
   \ar@{-->}"5,2"^{\exists ! f\times_S id_{Y^{\prime}}} & Y^{\prime}
   \ar[d]^{id_{Y^{\prime}}}                                      \\
   Y & & & Y^{\prime}                                            \\
   & {Y\times_S Y^{\prime}} \ar"4,1" \ar"4,4" }   \]
Since
\[ \xymatrix@C=3em{
   {X\times_S Y^{\prime}} \ar[d] \ar[r]^-{f\times_S id_{Y^{\prime}}}
   & {Y\times_S Y^{\prime}} \ar[d]                                \\
   X \ar[r]^f & Y }
\qquad and\qquad
   \xymatrix@C=3em{
   {X\times_S X^{\prime}} \ar[d] \ar[r]^-{id_X\times_S f^{\prime}}
   & {X\times_S Y^{\prime}} \ar[d]                           \\
   X^{\prime} \ar[r]^{f^{\prime}} & Y^{\prime} } \]
are base changes, and $f, f^{\prime}$ are proper(resp. separated),
hence $id_X\times_S f^{\prime}$ and $f\times_S id_{Y^{\prime}}$ are
proper(resp. separated). Since the composite of two proper(resp.
separated) morphism is proper(resp. separated), thus $f\times_S
f^{\prime}$ is proper(resp. separated).
\end{proof}
\begin{prop}
Let $f: X\rightarrow Y$ and $g: Y\rightarrow S$ be two morphisms of
schemes.
\enum
\item[(1)]If $g\circ f$ is separated, then $f$ is separated.
\item[(2)]If $g\circ f$ is proper, and $g$ is separated, then $f$ is proper.
\end{list}
\end{prop}
\begin{proof}
We can treat $f$ as an $S$-morphism for we have the commutative
diagram
\[ \xymatrix{
   X \ar[rr]^f \ar[dr]_{g\circ f} & & Y \ar[dl]^g \\
   & S }  \]
Since the following diagram
\[ \xymatrix{
   & {X\times_S Y} \ar[dl]_p \ar[dr]^q                   \\
   X \ar[dr]_{g\circ f} & X \ar[l]_{id_X} \ar[r]^f \ar[d]
   \ar@{-->}[u]_{\exists ! \Gamma_f} & Y \ar[dl]^g       \\
   & S}  \]
is cartesian, hence $q$ is the base change of $g\circ f,$ and we
obtain that $f=q\circ\Gamma_f.$
\enum
\item[(1)]If $g\circ f$ is separated, then $q$ is separated, for it
is the base change of $g\circ f.$ $\Gamma_f$ is an immersion, thus
separated. Then $f=q\circ\Gamma_f$ is separated.
\item[(2)]
\[ \xymatrix{
   X \ar[r]^-{\Gamma_f} \ar[d]_f & {X\times_S Y} \ar[d]  \\
   Y \ar[r]^-{\Delta_{Y/S}} & {Y\times_S Y} } \]
is cartesian. Since $g$ is separated, then $\Delta_{Y/S}$ is a
closed immersion, hence $\Gamma_f$ is a closed immersion as well,
and therefore $\Gamma_f$ is proper. If $g\circ f$ is proper, then
$q,$ the base change of $g\circ f,$ is proper. Hence
$f=q\circ\Gamma_f$ is proper.
\end{list}
\end{proof}
\begin{Def}
Put $\mathbb{P}_{\mathbb{Z}}^n=Proj\mathbb{Z}[x_0,\cdots,x_n].$ For
any scheme $Y,$ define
$$\mathbb{P}_Y^n=\mathbb{P}_{\mathbb{Z}}^n\times
Y=\mathbb{P}_{\mathbb{Z}}^n\times_{Spec\mathbb{Z}} Y$$ and call it
the projective space over $Y.$ Let $f: X\rightarrow Y$ be a morphism
of schemes. \enum
\item[(1)]We say that $f$ is projective or $X$ is projective over
$Y,$ if
\[ \xymatrix{
   X \ar[rr]^f \ar[dr]_{\bar{f}} & & Y \\
   & {\mathbb{P}_Y^n} \ar[ur] }  \]
is commutative with $\bar{f}$ a closed immersion.
\item[(2)]We say that $f$ is quasi-projective if
\[ \xymatrix{
   X \ar[rr]^f \ar[dr]_{\bar{f}} & & Y \\
   & {\mathbb{P}_Y^n} \ar[ur] }  \]
is commutative with $\bar{f}$ an immersion.
\end{list}
\end{Def}
\begin{egs}\
\enum
\item[(1)]$\mathbb{P}_{\mathbb{Z}}^0=Proj\mathbb{Z}[x_0]=\{ p\mid
\mathbb{Z}[x_0]_+\nsubseteq p, p \ \text{is a homogeneous prime
ideal of} \mathbb{Z}[x_0]\}=\{ p\in Proj\mathbb{Z}[x_0]\mid
x_0\not\in p\}\cong Spec\mathbb{Z}[x_0]_{x_0}=Spec\mathbb{Z}$
\item[(2)]Let $Y$ be a scheme, then $Y$ is projective over $Y.$
Indeed $$\mathbb{P}_Y^0=\mathbb{P}_{\mathbb{Z}}^0\times
Y=Spec\mathbb{Z}\times_{\mathbb{Z}} Y=Y$$ and then we obtain
\[ \xymatrix{
   Y \ar[rr]^{id_Y} \ar[dr]_{id_Y} & & Y \\
   & {\mathbb{P}_Y^0=Y} \ar[ur] }  \]
$id_Y$ is a closed immersion, for it is an isomorphism.
\item[(3)]Let $A$ be a ring, then
$$\mathbb{P}_{SpecA}^n=Proj\mathbb{Z}[x_0,\cdots,x_n]\times
SpecA=ProjA[x_0,\cdots,x_n].$$
\begin{eqnarray*}
D_+(x_i) & \cong &
Spec\mathbb{Z}[x_0,\cdots,x_n]_{(x_i)}=Spec\mathbb{Z}[\frac{x_0}{x_i},\cdots,\frac{x_n}{x_i}]\\
& \cong & Spec\mathbb{Z}[x_0,\cdots,\hat{x_i},\cdots,x_n]
\end{eqnarray*}
$(0\leqslant i\leqslant n)$ forms an affine open covering of
$\mathbb{P}_{\mathbb{Z}}^n,$ then
\begin{eqnarray*}
D_+(x_i)\times SpecA & = &
Spec\mathbb{Z}[x_0,\cdots,\hat{x_i},\cdots,x_n]\otimes_{\mathbb{Z}} A \\
& = & SpecA[x_0,\cdots,\hat{x_i},\cdots,x_n]
\end{eqnarray*}
$(0\leqslant i\leqslant n)$ forms an affine open covering of
$\mathbb{P}_{SpecA}^n.$ Then
\begin{eqnarray*}
\mathbb{P}_{SpecA}^n & = & \bigcup\limits_{i=0}^n(D_+(X_i)\times
SpecA)=\bigcup\limits_{i=0}^nSpecA[x_0,\cdots,\widehat{x_i},\cdots,x_n]\\
& = & ProjA[x_0,\cdots,x_n].
\end{eqnarray*}
\item[(4)]Let $k$ be an algebraically closed field.
\end{list}
\end{egs}
\begin{prop}
Projective morphisms are proper.
\end{prop}
\begin{proof}
Let $f: X\rightarrow Y$ be a projective morphism of schemes. Then
\[ \xymatrix{
   X \ar[rr]^f \ar[dr]_{\bar{f}} & & Y \\
   & {\mathbb{P}_Y^n} \ar[ur] }  \]
with $n\geqslant 0$ an integer, and $\bar{f}$ a closed immersion.

Since closed immersions are proper and the composite of two proper
morphisms is proper, so it suffices to show that the canonical
projection $\mathbb{P}_Y^n\rightarrow Y$ is proper. But
\[ \xymatrix{
   {\mathbb{P}_Y^n=\mathbb{P}_{\mathbb{Z}}^n\times Y} \ar[d]
   \ar[r]^-{\varphi} & {Y=Spec\mathbb{Z}\times_{Spec\mathbb{Z}} Y} \ar[d] \\
   \mathbb{P}_{\mathbb{Z}}^n \ar[r] & Spec\mathbb{Z} }  \]
is a base change, hence we can suppose that $Y=Spec\mathbb{Z}.$

$\mathit{1^{\circ}}$ $\varphi$ is of finite type:

Cover $\mathbb{P}_{\mathbb{Z}}^n$ by $D_+(x_i)(0\leqslant i\leqslant
n).$ Note that $Spec\mathbb{Z}$ is affine, and for each
$i(0\leqslant i\leqslant n),$
$$D_+(x_i)=Spec\mathbb{Z}[x_0,\cdots,x_n]_{(x_i)}=Spec\mathbb{Z}[\frac{x_0}{x_i},\cdots,\frac{x_n}{x_i}]$$
is affine with $\mathbb{Z}[\frac{x_0}{x_i},\cdots,\frac{x_n}{x_i}]$
a finitely generated $\mathbb{Z}$-algebra. Hence $\varphi$ is of
finite type.

$\mathit{2^{\circ}}$ $\varphi$ is separated:

We should show that
$$\Delta:=\Delta_{\mathbb{P}_{\mathbb{Z}}^n/Spec\mathbb{Z}}:
\mathbb{P}_{\mathbb{Z}}^n\rightarrow \mathbb{P}_{\mathbb{Z}}^n\times
\mathbb{P}_{\mathbb{Z}}^n$$ is a closed immersion. Cover
$\mathbb{P}_{\mathbb{Z}}^n\times \mathbb{P}_{\mathbb{Z}}^n$ by
$$D_+(x_i)\times
D_+(x_j)=Spec\mathbb{Z}[\frac{x_0}{x_i},\cdots,\frac{x_n}{x_i}]\otimes_{\mathbb{Z}}
Spec\mathbb{Z}[\frac{x_0}{x_j},\cdots,\frac{x_n}{x_j}](0\leqslant
i,j\leqslant n).$$ Note that
\begin{eqnarray*}
\Delta^{-1}(D_+(x_i)\times D_+(x_j)) & = & D_+(x_i)\cap
D_+(x_j)=D_+(x_ix_j)                                 \\
& = & Spec\mathbb{Z}[x_0,\cdots,x_n]_{(x_ix_j)}      \\
& = & Spec\mathbb{Z}[\frac{x_kx_l}{x_ix_j}]_{0\leqslant k,l\leqslant
n}
\end{eqnarray*}
hence it suffices to show that
$\Delta:\Delta^{-1}(D_+(x_i)\times D_+(x_j))=D_+(x_i)\cap
D_+(x_j)=Spec\mathbb{Z}[\frac{x_kx_l}{x_ix_j}]_{0\leqslant
k,l\leqslant n}\rightarrow D_+(x_i)\times
D_+(x_j)=Spec\mathbb{Z}[\frac{x_0}{x_i},\cdots,\frac{x_n}{x_i}]\otimes_{\mathbb{Z}}
Spec\mathbb{Z}[\frac{x_0}{x_j},\cdots,\frac{x_n}{x_j}]$ is a closed
immersion. But
\[ \xymatrix{
   & {D_+(x_i)\times D_+(x_j)} \ar[dl] \ar[dr]               \\
   {D_+(x_i)} & {D_+(x_ix_j)} \ar@_{(->}[l] \ar@^{(->}[r]
   \ar@{-->}[u]_{\Delta} & {D_+(x_j)} }  \]
i.e
\[ \xymatrix{
   & {Spec\mathbb{Z}[\frac{x_0}{x_i},\cdots,\frac{x_n}{x_i}]\otimes_{\mathbb{Z}}Spec\mathbb{Z}
   [\frac{x_0}{x_j},\cdots,\frac{x_n}{x_j}]} \ar[dl] \ar[dr]               \\
   {Spec\mathbb{Z}[\frac{x_0}{x_i},\cdots,\frac{x_n}{x_i}]} &
   {Spec\mathbb{Z}[\frac{x_kx_l}{x_ix_j}]_{0\leqslant k,l\leqslant n}} \ar@_{(->}[l] \ar@^{(->}[r]
   \ar@{-->}[u]_{\Delta} & {Spec\mathbb{Z}[\frac{x_0}{x_j},\cdots,\frac{x_n}{x_j}]} }  \]
induces the canonical commutative diagram:
\[ \xymatrix{
   & {\mathbb{Z}[\frac{x_0}{x_i},\cdots,\frac{x_n}{x_i}]\otimes_{\mathbb{Z}}\mathbb{Z}
   [\frac{x_0}{x_j},\cdots,\frac{x_n}{x_j}]} \ar[d]                  \\
   {\mathbb{Z}[\frac{x_0}{x_i},\cdots,\frac{x_n}{x_i}]} \ar[ur] \ar[r] &
   {\mathbb{Z}[\frac{x_kx_l}{x_ix_j}]_{0\leqslant k,l\leqslant n}} &
   {\mathbb{Z}[\frac{x_0}{x_j},\cdots,\frac{x_n}{x_j}]} \ar[l] \ar[ul] }  \]
Hence the associated ring homomorphism of
$\left.\Delta\right|_{D_+(x_ix_j)}$:
\[ \xymatrix@R=-0.25em{
   {\mathbb{Z}[\frac{x_0}{x_i},\cdots,\frac{x_n}{x_i}]\otimes_{\mathbb{Z}}\mathbb{Z}
   [\frac{x_0}{x_j},\cdots,\frac{x_n}{x_j}]} \ar[r] &
   {\mathbb{Z}[\frac{x_kx_l}{x_ix_j}]_{0\leqslant k,l\leqslant n}}  \\
   {\frac{x_k}{x_i}\otimes\frac{x_l}{x_j}=(\frac{x_k}{x_i}\otimes1)\cdot(1\otimes\frac{x_l}{x_j})} \ar@{|->}[r] &
   {\frac{x_kx_j}{x_ix_j}\cdot\frac{x_ix_l}{x_ix_j}=\frac{x_kx_l}{x_ix_j}} }  \]
is surjective, thus $\Delta$ is a closed immersion. Therefore
$\varphi$ is separated.

$\mathit{2^{\circ}}$ $\varphi$ is universally closed:

\begin{lemma}
Let $A$ be a ring, and $M$ be a finitely generated $A$-module.
Then$$SuppM:=\{p\in SpecA\mid M_p\neq 0\}=V(Ann_A(M))$$ where
$Ann_A(M)=\{a\in A\mid aM=0\}$
\end{lemma}
$\star$\ Indeed, let $x_0,\cdots,x_n$ generate $M.$ $\forall p\in
SpecA,$
\begin{eqnarray*}
& & M_p=0                                                        \\
& \Longleftrightarrow & \frac{x_i}{1}=0\text{ in
}M_p(0\leqslant i\leqslant n)                                    \\
& \Longleftrightarrow & \exists s_i\in A\setminus p\text{ such that
}s_ix_i=0(0\leqslant i\leqslant n)                               \\
& \Longleftrightarrow & \exists s\in A\setminus p\text{ such that
}sx_i=0(0\leqslant i\leqslant n)                                 \\
& \Longleftrightarrow & \exists s\in A\setminus p\text{ such that
}sM=0                                                            \\
& \Longleftrightarrow & \exists s\in A\setminus p\text{ such that
}s\in Ann_A(M)                                                   \\
& \Longleftrightarrow & Ann_A(M)\nsubseteq p.
\end{eqnarray*}
Hence $SuppM=V(Ann_A(M)).$\hspace*{\fill}$\star$

Let $g:Y^{\prime}\rightarrow Spec\mathbb{Z}$ be a morphism of
schemes, we should show that
$\varphi^{\prime}:\mathbb{P}^n_{Y^{\prime}}\rightarrow Y^{\prime},$
the base change of $\varphi,$ is a closed mapping of topological
spaces.

Let $V=SpecA$ be an affine open subscheme of $Y^{\prime},$ we need
only to show that
$$\varphi^{\prime}:(\varphi^{\prime})^{-1}(SpecA)=\mathbb{P}^n_{\mathbb{Z}}\times SpecA=\mathbb{P}^n_{SpecA}\rightarrow SpecA$$
is a closed mapping. But
$\mathbb{P}^n_{SpecA}=ProjA[x_0,\cdots,x_n],$ so it suffices to show
$$ProjA[x_0,\cdots,x_n]\rightarrow SpecA$$
is a closed mapping. Let $I$ be an ideal of $A[x_0,\cdots,x_n],$ we
should show that $\varphi^{\prime}(V_+(I))$ is closed in $SpecA.$
$$V_+(I)\cong ProjA[x_0,\cdots,x_n]/I,$$ then we must show the image
of the canonical morphism
$$\theta:ProjA[x_0,\cdots,x_n]/I\rightarrow SpecA$$
is closed in $SpecA.$

We claim that $Im\theta=V(\bigcup\limits_{l=0}^{\infty}Ann_A(S_l)),$
where we put
$$S=A[x_0,\cdots,x_n]/I=A[\bar{x}_0,\cdots,\bar{x}_n]$$
which is a graded ring. $\forall p\in SpecA,$
$$p\not\in Im\theta\Longleftrightarrow\theta^{-1}(p)=\emptyset
\Longleftrightarrow\forall i(1\leqslant i\leqslant
n),\theta^{-1}(p)\cap D_+(\bar{x}_i)=\emptyset,$$ because
$ProjS=\bigcup\limits_{i=0}^nD_+(\bar{x}_i).$ $\mathcal
{O}_{SpecA,p}\cong A_p,$ and we set $$k(p)=\mathcal
{O}_{SpecA,p}/\mathfrak{M}_{\mathcal {O}_{SpecA,p}}=A_p/pA_p$$ the
residue field of $A_p.$ For each $i(0\leqslant i\leqslant n),$
$D_+(\bar{x}_i)\cong
SpecA[\frac{\bar{x}_0}{\bar{x}_i},\cdots,\frac{\bar{x}_n}{\bar{x}_i}]$
is affine, hence
\begin{eqnarray*}
\theta^{-1}(p)\cap D_+(\bar{x}_i) & = &
(\theta_{D_+(\bar{x}_i)})^{-1}(p)\cong
SpecA[\frac{\bar{x}_0}{\bar{x}_i},\cdots,\frac{\bar{x}_n}{\bar{x}_i}]\times_{SpecA}Speck(y)\\
& = &
SpecA[\frac{\bar{x}_0}{\bar{x}_i},\cdots,\frac{\bar{x}_n}{\bar{x}_i}]\otimes_A
A_p/pA_p.
\end{eqnarray*}
Therefore we obtain that
\begin{eqnarray*}
& & p\not\in Im\theta                                             \\
& \Longleftrightarrow & \forall i(0\leqslant i\leqslant n),
SpecA[\frac{\bar{x}_0}{\bar{x}_i},\cdots,\frac{\bar{x}_n}{\bar{x}_i}]
\otimes_A A_p/pA_p=\emptyset                                      \\
& \Longleftrightarrow & \forall i(0\leqslant i\leqslant n),
A[\frac{\bar{x}_0}{\bar{x}_i},\cdots,\frac{\bar{x}_n}{\bar{x}_i}]
\otimes_A A_p/pA_p=0                                              \\
& \Longleftrightarrow & \forall i(0\leqslant i\leqslant n),
\bar{x}_i\otimes_A 1 \text{ is nilpotent in }S\otimes_A A_p/pA_p  \\
& & (\text{for }
A[\frac{\bar{x}_0}{\bar{x}_i},\cdots,\frac{\bar{x}_n}{\bar{x}_i}]
\otimes_A A_p/pA_p=(S\otimes_A A_p/pA_p)_{\bar{x}_i\otimes_A 1}.)  \\
& \Longleftrightarrow & \exists N\geqslant 1 \text{ such that }
S_N\otimes_A A_p/pA_p=0                                           \\
& & (\text{If } (\bar{x}_i\otimes_A 1)^{n_i}=0,\text{ we can put }
N=(\max\limits_{0\leqslant i\leqslant n}n_i)^{n+1}+1.)             \\
& \Longleftrightarrow & \exists N\geqslant 1 \text{ such that }
S_N\otimes_A A_p=0                                                \\
& & (0=S_N\otimes_A A_p/pA_p\cong (S_N)_p/p(S_N)_p,\text{ hence }
(S_N)_p=p(S_N)_p.                                                 \\
& & \, S_N\otimes_A A_p\cong (S_N)_p.\text{ By Nakayama's lemma,
we have }                                                         \\
& & \, S_N\otimes_A A_p\cong (S_N)_p=0.)                          \\
& \Longleftrightarrow & \exists N\geqslant 1 \text{ such that }
p\not\in V(Ann_A(S_N))                                            \\
& & \text{ by the lemma we have proved.}                          \\
& \Longleftrightarrow & \exists N\geqslant 1 \text{ such that }
Ann_A(S_N)\nsubseteq p                                            \\
& \Longleftrightarrow & \bigcup_{l\geqslant0}Ann_A(S_l)\nsubseteq p\\
& \Longleftrightarrow & p\not\in V(\bigcup_{l\geqslant0}Ann_A(S_l)).
\end{eqnarray*}
Hence we obtain that $p\in Im\theta\Longleftrightarrow p\in
V(\bigcup_{l\geqslant0}Ann_A(S_l)),$ thus
$Im\theta=V(\bigcup_{l\geqslant0}Ann_A(S_l))$ is closed in $SpecA.$
Therefore $\varphi$ is universally closed.
\end{proof}
\begin{prop}[Segre embedding]
Let $S$ be a scheme, then there exists a closed immersion $\varphi:
\mathbb{P}_S^m\times\mathbb{P}_S^n\rightarrow
\mathbb{P}_S^{(m+1)(n+1)-1},$ which is an $S$-morphism.
\end{prop}
\begin{proof}
\[ \xymatrix{
   {\mathbb{P}_S^m\times\mathbb{P}_S^n} \ar[dr] \ar[rr]^{\varphi} &
   & {\mathbb{P}_S^{(m+1)(n+1)-1}} \ar[dl]                        \\
   & {S=Spec\mathbb{Z}\times S} }  \]
can be induced by the base change of
\[ \xymatrix{
   {\mathbb{P}_{\mathbb{Z}}^m\times\mathbb{P}_{\mathbb{Z}}^n} \ar[dr] \ar[rr] &
   & {\mathbb{P}_{\mathbb{Z}}^{(m+1)(n+1)-1}} \ar[dl]                        \\
   & {Spec\mathbb{Z}} }  \]
Hence it suffices to consider the case $S=Spec\mathbb{Z}.$ Write
$\mathbb{P}_{\mathbb{Z}}^m=Proj\mathbb{Z}[x_i]_{0\leqslant
i\leqslant m},
\mathbb{P}_{\mathbb{Z}}^n=Proj\mathbb{Z}[y_j]_{0\leqslant j\leqslant
n},
\mathbb{P}_{\mathbb{Z}}^{(m+1)(n+1)-1}=Proj\mathbb{Z}[z_{ij}]_{0\leqslant
i\leqslant m \atop 0\leqslant j\leqslant n}.$ $\forall (k,
l)(0\leqslant k\leqslant m, 0\leqslant l\leqslant n),$
\[ \xymatrix@R=0em{
   {\psi_{kl}:\mathbb{Z}[z_ij]_{z_{kl}}} \ar[r] & {\mathbb{Z}[x_i]_{x_k}
   \otimes \mathbb{Z}[y_j]_{y_l}}               \\
   {\frac{z_{ij}}{z_{kl}}} \ar@{|->}[r] & {\frac{x_i}{x_k}\otimes
   \frac{y_j}{y_l}} }  \]
is surjective, thus
$$\varphi_{kl}:Spec\mathbb{Z}[z_ij]_{z_{kl}}\cong
D_+(z_{kl})\rightarrow D_+(x_k)\times
D_+(y_l)=Spec\mathbb{Z}[x_i]_{x_k}\otimes \mathbb{Z}[y_j]_{y_l}$$
which is induced by $\psi_{kl}$ is a closed immersion.

To glue $\varphi_{kl}(0\leqslant k\leqslant m, 0\leqslant l\leqslant
n)$ together, we should show that
$$\left.\varphi_{kl}\right|_{(D_+(x_k)\times D_+(y_l))\cap(D_+(x_{k^{\prime}})\times
D_+(y_{l^{\prime}}))}=\left.\varphi_{k^{\prime}l^{\prime}}\right|_{(D_+(x_k)\times
D_+(y_l))\cap(D_+(x_{k^{\prime}})\times D_+(y_{l^{\prime}}))}$$
$\forall 0\leqslant k, k^{\prime}\leqslant m\,,\,0\leqslant l,
l^{\prime}\leqslant n.$ It is easily checked that $(D_+(x_k)\times
D_+(y_l))\cap(D_+(x_{k^{\prime}})\times
D_+(y_{l^{\prime}}))=D_+(x_kx_{k^{\prime}})\times
D_+(y_ly_{l^{\prime}}).$ Indeed we have the following commutative
diagram:
\[ \xymatrix@C=2em{
   & & {\mathbb{P}_{\mathbb{Z}}^m\times\mathbb{P}_{\mathbb{Z}}^n}
   \ar"2,1"_p \ar"2,5"^q                                       \\
   {\mathbb{P}_{\mathbb{Z}}^m} & {D_+(x_kx_{k^{\prime}})}
   \ar@_{(->}[l] & M \ar[l]_-{f_1} \ar[r]^-{f_2} \ar@{-->}[u]_{\exists
   !g} & {D_+(y_ly_{l^{\prime}})} \ar@^{(->}[r] &
   {\mathbb{P}_{\mathbb{Z}}^n} }  \]
We have $$p\circ g(M)=f_1(M)\subseteq
D_+(x_kx_{k^{\prime}})=D_+(x_k)\cap D_+(x_{k^{\prime}})\subseteq
D_+(x_k), $$then $g(M)\subseteq p^{-1}(D_+(x_k))$ and $$q\circ
g(M)=f_2(M)\subseteq D_+(y_ly_{l^{\prime}})=D_+(y_l)\cap
D_+(y_{l^{\prime}})\subseteq D_+(y_l),$$ then $g(M)\subseteq
q^{-1}(D_+(y_l)).$ Thus $$g(M)\subseteq p^{-1}(D_+(x_k))\cap
q^{-1}(D_+(y_l))=D_+(x_k)\times D_+(y_l).$$ Likewise we can prove
that $g(M)\subseteq D_+(x_{k^{\prime}})\times D_+(y_{l^{\prime}}).$
Hence $$g(M)\subseteq (D_+(x_k)\times
D_+(y_l))\cap(D_+(x_{k^{\prime}})\times D_+(y_{l^{\prime}}))$$ and
then we obtain
\[ \xymatrix@C=1em{
   & {(D_+(x_k)\times D_+(y_l))\cap(D_+(x_{k^{\prime}})\times
   D_+(y_{l^{\prime}}))} \ar"2,1"_p \ar"2,3"^q                 \\
   {D_+(x_kx_{k^{\prime}})} & M \ar[l]_-{f_1} \ar[r]^-{f_2} \ar[u]_{\exists ! g} &
   {D_+(y_ly_{l^{\prime}})} }  \]
Therefore $(D_+(x_k)\times D_+(y_l))\cap(D_+(x_{k^{\prime}})\times
D_+(y_{l^{\prime}}))=D_+(x_kx_{k^{\prime}})\times
D_+(y_ly_{l^{\prime}}).$ Hence we want to show that
$$\left.\varphi_{kl}\right|_{D_+(x_kx_{k^{\prime}})\times
D_+(y_ly_{l^{\prime}})}=\left.\varphi_{k^{\prime}l^{\prime}}\right|_{D_+(x_kx_{k^{\prime}})\times
D_+(y_ly_{l^{\prime}})}$$ We have the following canonical
commutative diagram:
\[ \xymatrix{
   {\mathbb{Z}[z_{ij}]_{z_{kl}}} \ar[r]_{\rho_1} \ar[d] &
   {\mathbb{Z}[x_i]_{x_k}\otimes\mathbb{Z}[y_j]_{y_l}} \ar[d]   \\
   {\mathbb{Z}[z_{ij}]_{z_{kl}z_{k^{\prime}l^{\prime}}}}
   \ar[r]^-{\rho} &
   {\mathbb{Z}[x_i]_{x_kx_{k^{\prime}}}\otimes\mathbb{Z}[y_j]_{y_ly_{l^{\prime}}}}\\
   {\mathbb{Z}[z_{ij}]_{z_{k^{\prime}l^{\prime}}}} \ar[r]^{\rho_2} \ar[u] &
   {\mathbb{Z}[x_i]_{x_{k^{\prime}}}\otimes\mathbb{Z}[y_j]_{y_{l^{\prime}}}}
   \ar[u] }  \]
where $\rho_1:\frac{z_{ij}}{z_{kl}} \mapsto
\frac{x_i}{x_k}\otimes\frac{y_j}{y_l},
\rho_2:\frac{z_{i^{\prime}j^{\prime}}}{z_{k^{\prime}l^{\prime}}}
\mapsto
\frac{x_{i^{\prime}}}{x_{k^{\prime}}}\otimes\frac{y_{j^{\prime}}}{y_{l^{\prime}}}.$
$\left.\varphi_{kl}\right|_{D_+(x_kx_{k^{\prime}})\times
D_+(y_ly_{l^{\prime}})}$ and
$\left.\varphi_{k^{\prime}l^{\prime}}\right|_{D_+(x_kx_{k^{\prime}})\times
D_+(y_ly_{l^{\prime}})}$ are induced by $\rho,$ hence
$$\left.\varphi_{kl}\right|_{D_+(x_kx_{k^{\prime}})\times
D_+(y_ly_{l^{\prime}})}=\left.\varphi_{k^{\prime}l^{\prime}}\right|_{D_+(x_kx_{k^{\prime}})\times
D_+(y_ly_{l^{\prime}})}$$ Therefore we obtain the desired results.
\end{proof}
\begin{prop}\
\enum
\item[(1)]Closed immersions are projective.
\item[(2)]The composite of projective morphisms is projective.
\item[(3)]The base change of a projective morphism is projective.
\item[(4)]The fibred product of two projective morphisms is
projective.
\item[(5)]Let $f: X\rightarrow Y$ and $g: Y\rightarrow S$ be two
morphisms of schemes. If $g\circ f$ is projective and $g$ is
separated, then $f$ is projective
\end{list}
\end{prop}
\begin{proof}\
\enum
\item[(1)]Let $f: X\rightarrow Y$ be a closed immersion, then we
have
\[ \xymatrix{
   X \ar[rr]^f \ar[dr]_f & & Y         \\
   & {\mathbb{P}_y^0=Y} \ar@{=}[ur] }  \]
Hence $f$ is projective.
\item[(2)]Let $X\stackrel{f}{\rightarrow} Y\stackrel{g}{\rightarrow}
S$ be two projective morphisms, then we have
\[ \xymatrix{
   X \ar[rr]^f \ar[dr]_{\bar{f}} & & Y   \\
   & \mathbb{P}_Y^m \ar[ur] }\qquad
   \xymatrix{
   Y \ar[rr]^g \ar[dr]_{\bar{g}} & & S   \\
   & \mathbb{P}_S^n \ar[ur] }  \]
with $\bar{f}, \bar{g}$ being closed immersions and $m, n$
nonnegative integers. We have the following commutative diagram:
\[ \xymatrix{
   X \ar[r]^-{\bar{f}} \ar"2,2"_f & {\mathbb{P}_Y^m=Y\times
   \mathbb{P}_{\mathbb{Z}}^m} \ar[d] \ar[r]^-{\bar{g}\times
   id_{\mathbb{P}_{\mathbb{Z}}^m}} & {\mathbb{P}_S^n\times\mathbb{P}_{\mathbb{Z}}^m=\mathbb{P}_S^n\times_S
   \mathbb{P}_S^m} \ar[d] \ar[r]^-{\varphi} &
   {\mathbb{P}_S^{(m+1)(n+1)-1}} \ar"3,3"                   \\
   & Y \ar[r]^{\bar{g}} \ar"3,3"_g & {\mathbb{P}_S^n} \ar[d]\\
   & & S }  \]
where $\varphi$ is the Segre embedding, an $S$-morphism.

$\bar{g}\times id_{\mathbb{P}_{\mathbb{Z}}^m}$ is the base change of
$\bar{g},$ thus is a closed immersion. Since $\bar{f}$ and $\varphi$
are closed immersions, then $\varphi\circ (\bar{g}\times
id_{\mathbb{P}_{\mathbb{Z}}^m})\circ\bar{f}$ is a closed immersion.
Therefore $g\circ f$ is projective.
\item[(3)]Let $f: X\rightarrow Y$ be a projective morphism,
$\varphi: Y^{\prime}\rightarrow Y$ be a morphism of schemes. Then we
have
\[ \xymatrix{
   X \ar[rr]^f \ar[dr]_g & & Y                 \\
   & {\mathbb{P}_Y^n} \ar[ur]_q }  \]
with $g$ a closed immersion, and $n\geqslant 0$ an integer. Then we
obtain the following commutative diagram
\[ \xymatrix@C=1em{
   {X\times_Y Y^{\prime}} \ar"1,7"^-{f\times_Y id_{Y^{\prime}}}
   \ar"2,3" \ar"5,4"_-{g\times_Y id_{Y^{\prime}}} & & & & & & {Y^{\prime}=Y\times_Y
   Y^{\prime}} \ar"2,5"^-{\varphi} \ar"5,4"^-{q\times_Y id_{Y^{\prime}}} \\
   & & X \ar[rr]^f \ar"4,4"_g & & Y \ar"4,4"^q                \\\\
   & & & {\mathbb{P}_Y^n}                                       \\
   & & & {\mathbb{P}_{Y^{\prime}}^n=\mathbb{P}_Y^n\times
   Y^{\prime}} \ar[u] }  \]
$g\times_Y id_{Y^{\prime}}$ is the base change of $g,$ which is a
closed immersion, thus $g\times_Y id_{Y^{\prime}}$ is a closed
immersion. Hence $f\times_Y id_{Y^{\prime}},$ the base change of
$f,$ is projective.
\item[(4)]Let $f: X\rightarrow Y$ and $f^{\prime}: X^{\prime}\rightarrow
Y^{\prime}$ be two projective morphisms, then $$f\times
f^{\prime}=(f\times id_{Y^{\prime}})(id_X\times f^{\prime})$$ hence
$f\times f^{\prime}$ is projective, for $f\times id_{Y^{\prime}}$
and $id_X\times f^{\prime}$ are projective by $(3).$
\item[(5)]Since
\[ \xymatrix{
   X \ar[rr]^f \ar[dr]_{g\circ f} & & Y \ar[dl]^g  \\
   & S }  \]
then we can treat $f$ as an $S$-morphism.
\[ \xymatrix{
   X \ar[r]^-{\Gamma_f} \ar[d]_f & {X\times_S Y} \ar[d] \\
   Y \ar[r]^-{\Delta_{Y/S}} & {Y\times_S Y} }  \]
is a base change. If $g$ is separated, then $\Delta_{Y/S}$ is a
closed immersion by definition, and then $\Gamma_f$ is a closed
immersion as well. But we have the following commutative diagram
\[ \xymatrix{
   & {X\times_S Y} \ar[dl]_p \ar[dr]^q                        \\
   X \ar[dr]_{g\circ f} & X \ar[l]_-{id_X} \ar[r]^f \ar[d]
   \ar@{-->}[u]_{\exists ! \Gamma_f} & Y \ar[dl]^g            \\
   & S }  \]
Then $q$ is the base change of $g\circ f,$ thus is projective, for
$g\circ f$ is projective by $(3).$ If $g$ is separated, then
$\Gamma_f$ is a closed immersion, thus $\Gamma_f$ is projective by
$(1).$ Therefore $f=q\circ \Gamma_f$ is projective by $(2).$
\end{list}
\end{proof}

