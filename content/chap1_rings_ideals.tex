%
\chapter{Rings \& Ideals}

\newpage

\section{Definitions,notation and basic properties}
We shall fix $A$ a commutative ring with an identity $1.$

\begin{eg}
$A=\{0\}$ is a zero ring. A ring is a zero ring iff $1=0.$
\end{eg}

Ring homomorphism:$f:A\rightarrow B$ satisfying
\enum
\item[(1)]$f(x+y)=f(x)+f(y)$
\item[(2)]$f(xy)=f(x)f(y)$
\item[(3)]$f(1)=1$
\end{list}

Subring: Let $S\subseteq A$ be a subset of $A,$ if $S$ is closed
under addition$(+)$ and multiplication$(\times),$ and $1\in S,$ then
$S$ is called a subring of $A.$

Ideal: Let $I\subseteq A$ be an additive subgroup of $A,$ if
$IA\subseteq I,$ then $I$ is called an ideal of $A.$

Quotient ring: Let$I\subseteq A$ be an ideal, put $A/I=\{\ x+I\mid
x\in A\}.$$\forall x\in A, \bar{x}:=x+I,
\bar{x}+\bar{y}:=\overline{x+y}, \bar{x}\cdot
\bar{y}:=\overline{xy}.$

\begin{prop}
Let $f:A\rightarrow B$ be a ring homomorphism. \enum
\item[(1)]$kerf=f^{-1}(0),$ it is an ideal of $A$;
\item[(2)]$Imf=f(A),$ it is a subring of $B$;
\item[(3)]We have the following commutative diagram:
\[ \begin{CD}
A  @>f>>   B\\
@VVV     @VVV\\
A/kerf @>\cong >> Imf\\
\end{CD} \]
\item[(4)]$I\subseteq J$(an ideal of $A$) is in one-to-one
correspondence to $f(J)$ which is an ideal of $Imf.$
\end{list}
\end{prop}
\begin{proof}

\end{proof}
Zero-divisor: $\forall x\in A$ such that $\exists y\in
A\setminus\{0\},xy=0.$

Integral domain: A ring without nonzero zero-divisors.

Nilpotent element: $x\in A$ such that $\exists n\geqslant 1,x^n=0.$
$x$ is a nilpotent element $\Longrightarrow$ $x$ is a zero-divisor.

Units: $x\in A$ such that $\exists y\in A, xy=1.$ Define
$$A^{\times}=\{x\in A\mid x\text{ is a unit }\}.$$ It is called the
unit group of $A.$

Principle ideals: $I=(x)=Ax.$
\begin{prop}
Let $A$ be a nonzero ring, then the following are equivalent:
\enum
\item[(1)]$A$ is a field,
\item[(2)]The only ideals of $A$ are $0$ and $A,$
\item[(3)]Every ring homomorphism from $A$ into a nonzero ring is
injective.
\end{list}
\end{prop}
\begin{proof}

\end{proof}
Prime ideal: $\mathfrak{p}$ an ideal of $A$ such that
$\mathfrak{p}\neq A,$ and if $x,y\in A,xy\in\mathfrak{p},$ then
$x\in\mathfrak{p}$ or $y\in\mathfrak{p}.$

Maximal ideal: $\mathfrak{M}$ an ideal of $A$ such that
$\mathfrak{M}\neq A,$ and if $I\supset \mathfrak{M}$ is an ideal,
then $I=A.$
\begin{cor}
If $\mathfrak{M}$ is a maximal ideal, then $\mathfrak{M}$ is prime.
\end{cor}
\begin{proof}

\end{proof}
\begin{thm}
Every nonzero ring has at least one maximal ideal.
\end{thm}
\begin{proof}

\end{proof}
\begin{cor}
Let $I$ be an ideal of $A$ such that $I\neq A,$ then there exists a
maximal ideal $\mathfrak{M}\supseteq I.$
\end{cor}
\begin{proof}

\end{proof}
\begin{cor}
$A\setminus A^{\times} = \bigcup\limits_{\text{maximal
ideals}}\mathfrak{M}.$
\end{cor}
\begin{proof}

\end{proof}
Local ring: A ring is called local if it has only one maximal ideal
$\mathfrak{M}.$
\begin{prop}
Let $A$ be a ring, and $\mathfrak{M}\neq A$ an ideal of $A,$ then:
\enum
\item[(1)]If $A\setminus\mathfrak{M}\subseteq A^{\times},$ then $A$
is local, and $\mathfrak{M}$ is its only maximal ideal.
\item[(2)]If $\mathfrak{M}$ is maximal such that
$1+\mathfrak{M}\subseteq A^{\times},$ then $A$ is local.
\end{list}
\end{prop}
\begin{proof}

\end{proof}

\newpage

\section{Rings of fractions}

\begin{Def}
Let $A$ be a ring, and $S\subseteq A,$ we say that $S$ is
multiplicatively closed if $1\in S$ and $S$ is closed for
multiplication.
\end{Def}
\begin{prop}
If $g:A\rightarrow B$ is a ring homomorphism satisfying:
\enum
\item[(1)]$f(S)\subseteq B^{\times},$
\item[(2)]$g(a)=0\Longleftrightarrow \exists s\in S$ such that
$as=0,$
\item[(3)]$B=\{g(a)(g(s))^{-1}\mid a\in A,s\in S\}.$
\end{list}
Then there exists a unique isomorphism $h:S^{-1}A\rightarrow B$ such
that $g=h\circ i_S,$ where
\[ \xymatrix@R=0em{
   {i_S:\ A} \ar[r] & S^{-1}A      \\
   a \ar@{|->}[r] & \frac{a}{1} }  \]
\end{prop}
\begin{proof}

\end{proof}
\begin{egs}\
\enum
\item[(1)]Let $\mathfrak{p}$ be a prime ideal of $A,$ set
$S=A\setminus \mathfrak{p}.$ Then $S$ is multiplicatively closed. We
define $S^{-1}A$ by $A_{\mathfrak{p}},$ called the localization of
$A$ at $\mathfrak{p}.$ Then $A_{\mathfrak{p}}$ is local with
$S^{-1}\mathfrak{p}$ its only maximal ideal.
\item[(2)]$S^{-1}A=0$ iff $0\in S.$
\item[(3)]Let $f\in A,$ and $S=\{f^n\mid n\geqslant 0\}.$ Then $S$ is
multiplicatively closed, and we denote $S^{-1}A$ by $A_f.$
\item[(4)]Put $A=k[t_1,\cdots,t_n]$ with $k$ an infinite field.
Let $\mathfrak{p}$ be a prime ideal of $A.$ Put
$$V = \{x=(x_1,\cdots,x_n)\in k^n\mid f(x)=0, \forall f\in
\mathfrak{p} \},$$
$$A_{\mathfrak{p}} = \{\frac{f}{g}\mid f\in A, g\not\in
\mathfrak{p} \}.$$ $A_{\mathfrak{p}}$ is the ring of rational
functions on $k^n$ which are defined at almost all points of $V.$
\end{list}
\end{egs}

\newpage

\section{Nilradical and Jacobson radical}

\begin{prop}
Let $A$ be a ring, set $N = \{x\in A\mid x \text{ is nilpotent} \}.$
Then $N$ is an ideal of $A$ and $A/N$ has no nonzero nilpotent
elements. More precisely, $N = \bigcap\limits_{prime}\mathfrak{p}.$
\end{prop}
\begin{proof}
$\mathit{1^{\circ}}$ $N\subseteq \bigcap\limits_{prime}\mathfrak{p}$
is obvious because $0\in \mathfrak{p}.$

$\mathit{2^{\circ}}$ $N\supseteq
\bigcap\limits_{prime}\mathfrak{p}:$

By contradiction, suppose that $\exists f\in
(\bigcap\limits_{prime}\mathfrak{p})\setminus N,$ then $\forall
n\geqslant 1,$ we have $f^n\neq 0.$ So $A_f\neq 0,$ and $A_f$ has at
least one maximal ideal $\bar{\mathfrak{M}}.$ Then $\mathfrak{M}:=
\{x\in A\mid \frac{x}{1}\in \bar{\mathfrak{M}}\}$ is a prime ideal
in $A.$ Note that we have $f\not\in \mathfrak{M},$ otherwise,
$\frac{f}{1}\in \bar{\mathfrak{M}}$ while $\frac{f}{1}$ is a unit in
$A_f.$ However, $f\in \bigcap\limits_{prime}\mathfrak{p},$ absurd.
\end{proof}
The radical of an ideal $I$:
$$\sqrt{I}:=\{x\in A\mid \exists n\geqslant 1\text{ such that }x^n\in I\}.$$
$\sqrt{I}$ is an ideal of $A,$ and we have
$I\subseteq\sqrt{I}\subseteq A.$
\begin{prop}\
\enum
\item[(1)]$\mathfrak{p}$ is a prime ideal $\Longrightarrow$
$\mathfrak{p}=\sqrt{\mathfrak{p}}.$
\item[(2)]$\exists k\geqslant 1$ such that $N^k\subseteq H$
$\Longrightarrow$ $\sqrt{N}\subseteq \sqrt{H}.$
\item[(3)]$\sqrt{NH}=\sqrt{N\cap H}=\sqrt{N}\cap \sqrt{H}.$ In
particular, $\forall N,$ and $k\geqslant 1,$ $\sqrt{N^k}=\sqrt{N}.$
\item[(4)]$\sqrt{\sqrt{N}}=\sqrt{N}.$
\item[(5)]$\sqrt{N+H}=\sqrt{\sqrt{N}+\sqrt{H}}.$
\end{list}
\end{prop}
\begin{proof}

\end{proof}
Jacobson radical:
$$\mathfrak{R}:=\bigcap\limits_{\text{maximal ideals}}\mathfrak{M}=
\{x\in A\mid\forall y\in A,1+xy\in A^{\times}\}.$$

\newpage

\section{Operations on ideals}

Addition:
$$I+J=\{x+y\mid x\in I,y\in J\},$$
$$\sum\limits_iI_i=\{\sum\limits_ix_i\mid x_i\in I_i,x_i=0 \text{ for almost all }i\}.$$

Intersection:
$$I\cap J=\{x\in A\mid x\in I, x\in J\},$$
$$\sum\limits_i I_i = \bigcap\limits_{\forall i, I\supset I_i} I.$$
Let $E$ be a subset of $A,$ $<E>:= \bigcap\limits_{I\supset E} I =
\{\sum\limits_{i<\infty} x_iy_i\mid x_i\in E, y_i\in A\}.$

Direct product: $A=\prod\limits_{i=1}^n A_i,$ with $A_i$ rings.
\begin{Def}
Let $A$ be a ring, $I,J$ be two ideals of $A.$ We say $I,J$ is
coprime, if $I+J=A.$
\end{Def}
\begin{prop}\
\enum
\item[(1)]Let $\mathfrak{p}_1,\cdots,\mathfrak{p}_n$ be prime ideals
of $A,$ and $I$ be an ideal of $A,$ such that $I\subseteq
\bigcup\limits_{i=1}^n \mathfrak{p}_i.$ Then $\exists i$ such that
$I\subseteq \mathfrak{p}_i.$
\item[(2)]Let $I_1,\cdots,I_n$ be ideals of $A$ and $\mathfrak{p}$ a
prime ideal. If $\bigcup\limits_{i=1}^n I_i\subseteq \mathfrak{p},$
then $\exists i$ such that $I_i\subseteq \mathfrak{p}.$ In
particular, if $\mathfrak{p}=\bigcup\limits_{i=1}^n I_i,$ then
$\exists i$ such that $\mathfrak{p}=I_i.$
\end{list}
\end{prop}
\begin{proof}
We shall prove by induction on $n$ that
$$I\nsubseteq \mathfrak{p}_i(1\leqslant i\leqslant n)\Longrightarrow
I\nsubseteq \bigcup\limits_{i=1}^n \mathfrak{p}_i.$$

$n=1,$ the case is trivial.

Suppose $n\geqslant 2$ and the result holds for $n-1.$ Then $\forall
i(1\leqslant i\leqslant n), I\nsubseteq \bigcup\limits_{j\neq i}
\mathfrak{p}_j,$ so $\exists x_i\in I\setminus \bigcup\limits_{j\neq
i} \mathfrak{p}_j.$ If $\exists i$ such that $x_i\not\in
\mathfrak{p}_i,$ then we get the desired result. Otherwise, $x_i\in
\mathfrak{p}_i, \forall 1\leqslant i\leqslant n,$ and then
$$y = \sum\limits_{i=1}^n x_1\cdots x_{i-1}\hat{x}_ix_{i+1}\cdots x_n\in
I.$$ But $\forall j(1\leqslant j\leqslant n), y\not\in
\mathfrak{p}_j,$ so $y\in I\setminus \bigcup\limits_{j=1}^n
\mathfrak{p}_j.$
\end{proof}

\newpage

\section{Extension and contraction}

\begin{Def}
Let $f: A\rightarrow B$ be a ring homomorphism, $q$ a (prime) ideal
of $B.$ Then $q^c:=f^{-1}(q)$ is a (prime) ideal of $A.$

Let $I$ be an ideal of $A,$ then
$$I^e:=<f(I)>=f(I)B=\{\sum\limits_{i<\infty}f(a_i)b_i\mid a_i\in I, b_i\in
B\}$$ is an ideal of $B.$
\end{Def}
\begin{prop}
Let $I$ be an ideal of $A,$ $J$ be an ideal of $B.$ Then
\enum
\item[(1)]$I\subseteq I^{ec}, J\supseteq J^{ce}.$
\item[(2)]$J^c=J^{cec}, I^e=I^{ece}.$
\item[(3)]Put $C=\{J^c\mid J \text{ an ideal of } B\}, E=\{I^e\mid I \text{ an ideal of }
A\}.$ Then $C=\{I\mid I^{ec}=I\}, E=\{J\mid J^{ce}=J\}.$
\end{list}
\end{prop}
\begin{proof}

\end{proof}
\begin{prop}
Let $A$ be a ring, $S$ a multiplicatively closed set of $A.$ Let
$i_S: A\rightarrow S^{-1}A$ be the canonical ring homomorphism, and
put $B=S^{-1}A.$
\enum
\item[(1)]Every ideal in $S^{-1}A$ is an extension ideal. i.e. if
$J$ is an ideal of $S^{-1}A,$ then $J=s^{-1}(i_S^{-1}(J)).$
\item[(2)]If $I$ is an ideal in $A,$ then $I^{ec} =
\bigcup\limits_{s\in S}(I:S).$ Then, $I^{e}=(1)$ iff $I\cap
S=\emptyset.$
\item[(3)]$I$ is a contracted ideal $\Longleftrightarrow$ no element
of $S$ is a zero-divisor in $A/I.$
\item[(4)]There is a one-to-one correspondence:
\[ \xymatrix@R=0em{
   \mathfrak{p} \ar@{<->}[r]^{1-1} & S^{-1}\mathfrak{p}  \\
   {\mathfrak{p}\ prime, \mathfrak{p}\cap S=\emptyset} & \text{prime ideals in
   } S^{-1}A }  \]
\item[(5)]The operation $S^{-1}$ commute with the formation of
finite sums, products, intersections, and radicals.
\end{list}
\end{prop}
\begin{proof}

\end{proof}
\begin{remark}
$(I:s):=\{x\in A\mid sx\in I\}.$ $(I:J):={x\in A\mid xJ\subseteq
I},$ called the ideal quotient.
\end{remark}
\begin{cor}\
\enum
\item[(1)]$S^{-1}(\sqrt{A})=\sqrt{S^{-1}A}.$
\item[(2)]Let $\mathfrak{p}$ be a prime ideal of $A,$ then the prime
ideals of $A_{\mathfrak{p}}$ are in one-to-one correspondence with
the prime ideals contained in $\mathfrak{p}.$
\end{list}
\end{cor}
\begin{proof}

\end{proof}
