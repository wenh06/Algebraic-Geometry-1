%
\chapter{Abelian Category}

\newpage

\section{Classes and sets(G\"odel-Bernays-Von Neumann)}

A class is a set iff it is a member of another class. It is
equivalent to saying that the only members of a class are sets.
\begin{egs}Let $S$ be the class of all sets.
\enum
\item[(1)]$S$ is a class, but not a set. Indeed, set
$$A=\{X\in S\mid X\not\in X\}.$$
Assume that $S$ is a set, then $A\in S.$ $\forall X\in S,$ we have
$X\in A\Longleftrightarrow X\not\in X.$ In particular, $A\in
A\Longleftrightarrow A\not\in A,$ absurd.
\item[(2)]Let $\mathscr{P}(S)$ be the collection of all subsets of
$S,$ then $\mathscr{P}(S)=S.$ Indeed, $\mathscr{P}(S)\subseteq S$ is
direct by definition. The only members of a class are set, so we
obtain $\mathscr{P}(S)\supseteq S.$
\end{list}
\end{egs}

\newpage

\section{Basic notions about category}

\begin{Def}
A category $\mathscr{C}$ consists of the following data:
\enum
\item[(1)]A class of objects, noted $Obj\mathscr{C}.$
\item[(2)]$\forall M,N\in Obj\mathscr{C},$ there exists a set
$Hom_{\mathscr{C}}(M,N)$ of which every element is called a morphism
from $M$ to $N.$
\item[(3)]$\forall M,N,P\in Obj\mathscr{C},$ there exists a mapping
\[ \xymatrix@R=0em{
   {\mu:Hom_{\mathscr{C}}(M,N)\times Hom_{\mathscr{C}}(N,P)} \ar[r]
   & {Hom_{\mathscr{C}}(M,P)}                                    \\
   (f,g) \ar@{|->}[r] & {\mu(f,g)=:g\circ f} }  \]
called composition satisfying

$(a)$ Composition is associative: $\forall f\in
Hom_{\mathscr{C}}(M,N),g\in Hom_{\mathscr{C}}(N,P),h\in
Hom_{\mathscr{C}}(P,Q),$ we have
$$h\circ(g\circ f)=(h\circ g)\circ f.$$

$(b)$ $\forall M\in Obj\mathscr{C},\exists id_m\in
Hom_{\mathscr{C}}(M,M),$ called identity element, such that $\forall
N\in Obj\mathscr{C},\forall f\in Hom_{\mathscr{C}}(M,N),\forall g\in
Hom_{\mathscr{C}}(N,M),$ we have
$$f\circ id_M=f,\quad id_m\circ g=g.$$
\end{list}
\end{Def}
\begin{remarks}\
\enum
\item[(1)]A morphism $f\in Hom_{\mathscr{C}}(M,N)$ is often denoted
$f:M\rightarrow N$ or $M\stackrel{f}{\rightarrow}N.$
\item[(2)]$id_M$ is unique for we have
$$id_M=id_M\circ\widetilde{id}_M=\widetilde{id}_M.$$
And we often identify $M$ and $id_M.$
\item[(3)]$f\in Hom_{\mathscr{C}}(M,N)$ is called an isomorphism if
$\exists g\in Hom_{\mathscr{C}}(N,M)$ such that $g\circ f=id_M,$ and
$f\circ g=id_N.$
\end{list}
\end{remarks}
\begin{egs}\
\enum
\item[(1)]Trivial category: $Obj\mathscr{C}=\emptyset,
Hom=\emptyset.$
\item[(2)]$SET:$ $ObjSET=\text{the class of all sets,}$
$$Hom_{SET}(A,B)=\{f:A\rightarrow B\mid f\text{ is a mapping }\}.$$
\item[(3)]Group: $ObjGr=\text{the class of all groups,}$
$$Hom_{Gr}(A,B)=\{f:A\rightarrow B\mid f\text{ is a group homomorphism }\}.$$
\item[(4)]Topological spaces: $ObjTOP=\text{the class of all topological
spaces},$
$$Hom_{TOP}(A,B)=\mathscr{C}(A;B)=\{f:A\rightarrow B\mid f\text{ is continuous }\}.$$
\item[(5)]Abelian groups: $ObjAb=\text{the class of all Abelian
groups},$
$$Hom_{Ab}(A,B)=\{f:A\rightarrow B\mid f\text{ is a group homomorphism }\}.$$
\item[(6)]$A$-modules: Let $A$ be a ring, $ObjMod(A)=\text{the class of}$
$A$-modules,
$$Hom_{Mod(A)}(M,N)=\{f:M\rightarrow N\mid f\text{ is an $A$-module homomorphism }\}.$$
\item[(7)]Opposite category $\mathscr{C}^{op}:$ Given a category
$\mathscr{C}$, define $Obj\mathscr{C}^{op}=Obj\mathscr{C},
Hom_{\mathscr{C}^{op}}(A,B)=Hom_{\mathscr{C}}(B,A),$ the composite
denoted by $\ast$ in $\mathscr{C}^{op}$ is defined by $f\ast
g=g\circ f.$
\item[(8)]Let $X$ be a topological space, define a category $\mathscr{C}$ as
follows:
$$Obj\mathscr{C}=\text{the set of all nonempty open subset of }X,$$
$\forall U,V\in Obj\mathscr{C},$
\[
  Hom_{\mathscr{C}}(U,V)=\left\{
  \begin{array}{ll}
  \emptyset, & \text{if } U\nsubseteq V \\
  \{U\hookrightarrow V\}, & \text{if } U\subseteq V
  \end{array}
  \right.
\]
\item[(9)]Let $(I,\leqslant)$ be a partially ordered set, we call it
directed if $\forall i,j\in I, \exists k$ such that $i\leqslant k,
j\leqslant k.$ We suppose $(I,\leqslant)$ is directed in the
following. Define a category $CatI$ as follows:
$$ObjCatI=I,$$
$\forall i,j\in I,$
\[
  Hom(i,j)=\left\{
  \begin{array}{ll}
  \emptyset, & \text{if } i\nleqslant j \\
  \{i\hookrightarrow j\}, & \text{if } i\leqslant j
  \end{array}
  \right.
\]
\end{list}
\end{egs}
\begin{Def}
Let $\mathscr{C}$ and $\mathscr{D}$ be two categories. A (covariant)
functor $F:\mathscr{C}\rightarrow \mathscr{D}$ is a function from
$Obj\mathscr{C}$ to $Obj\mathscr{D}$ (also called $F$), as well as
functions (also called $F$) from $Hom_{\mathscr{C}}(A,B)$ to
$Hom_{\mathscr{D}}(F(A),F(B))$ that satisfy:
\enum
\item[(1)]$F(id_A)=id_{F(A)}.$
\item[(2)]$\forall \varphi\in Hom_{\mathscr{C}}(A,B), \forall
\psi\in Hom_{\mathscr{C}}(B,C), F(\psi\circ\varphi)=F(\psi)\circ
F(\varphi).$
\end{list}
A contravariant functor $F$ from $\mathscr{C}$ to $\mathscr{D}$ is
by definition a covariant functor from $\mathscr{C}^{op}$ to
$\mathscr{D}.$ A functor is defined as a covariant functor or a
contravariant functor. But often when we say that $F$ is a functor,
we mean implicitly that $F$ is a covariant functor.
\end{Def}
\begin{eg}
Let $\mathscr{C},\mathscr{D}$ be two categories such that:
\enum
\item[(1)]$Obj\mathscr{C}\subseteq Obj\mathscr{D},$
\item[(2)]$\forall A,B\in Obj\mathscr{C},
Hom_{\mathscr{C}}(A,B)\subseteq Hom_{\mathscr{D}}(A,B).$
\end{list}
In this case, we say that $\mathscr{C}$ is a subcategory of
$\mathscr{D}.$ If further, $\forall A,B\in Obj\mathscr{C},$ we have
$Hom_{\mathscr{C}}(A,B)=Hom_{\mathscr{D}}(A,B),$ then we say that
$\mathscr{C}$ is a full subcategory of $\mathscr{D}.$

For example, $Ab$ is a full subcategory of $Gr,$ which is a
subcategory of $SET$ but not full.
\end{eg}
\begin{egs}\
\enum
\item[(1)]Let $\mathscr{C}$ be a subcategory of $\mathscr{D},$
\[ \xymatrix@R=0em{
   F: & Obj\mathscr{C} \ar[r] & Obj\mathscr{D}  \\
   & A \ar@{|->}[r] & A                         \\
   & Hom_{\mathscr{C}}(A,B) \ar[r] & Hom_{\mathscr{D}}(A,B) \\
   & \varphi \ar@{|->}[r] & \varphi }  \]
Then $F$ is a covariant functor.
\item[(2)]Let $\mathscr{C}$ be a category, the identity functor
$id_{\mathscr{C}}:$
\[ \xymatrix@R=0em{
   id_{\mathscr{C}}: & Obj\mathscr{C} \ar[r] & Obj\mathscr{C} \\
   & M \ar@{|->}[r] & M            \\
   & Hom(M,N) \ar[r] & Hom(M,N)    \\
   & f \ar@{|->}[r] & f }  \]
\item[(3)]Forgetful functor $U:$
\[ \xymatrix@R=0em{
   U: & Gr \ar[r] & SET            \\
   & M \ar@{|->}[r] & M            \\
   & Hom_{Gr}(M,N) \ar[r] & Hom_{SET}(M,N)    \\
   & f \ar@{|->}[r] & f }  \]
\item[(4)]Let $\mathscr{C}$ be a category and $X\in Obj\mathscr{C},$
functor $Hom_{\mathscr{C}}(X,-)$ is a covariant functor from
$\mathscr{C}$ to $SET:$
\[ \xymatrix@R=0em{
   Hom_{\mathscr{C}}(X,-): & Obj\mathscr{C} \ar[r] & ObjSET  \\
   & M \ar@{|->}[r] & Hom_{\mathscr{C}}(X,M)                 \\
   & Hom_{Gr}(M,N) \ar[r] & Hom_{SET}(Hom_{\mathscr{C}}(X,M),Hom_{\mathscr{C}}(X,N)) \\
   & g \ar@{|->}[r] & Hom_{\mathscr{C}}(X,g) }  \]
$Hom_{\mathscr{C}}(X,g): Hom_{\mathscr{C}}(X,M)\rightarrow
Hom_{\mathscr{C}}(X,N),$ which is defined by $f\mapsto g\circ f,$ is
a mapping from $Hom_{\mathscr{C}}(X,M)$ to $Hom_{\mathscr{C}}(X,N).$
Then $Hom_{\mathscr{C}}(X,-)$ is a covariant functor.

$Hom_{\mathscr{C}}(-,X)$ is a contravariant functor from
$\mathscr{C}$ to $SET:$ $\forall M\in Obj\mathscr{C},
Hom_{\mathscr{C}}(-,X)(M)=Hom_{\mathscr{C}}(M,X); \forall f\in
Hom_{\mathscr{C}}(M,N),
Hom_{\mathscr{C}}(-,X)(f)=Hom_{\mathscr{C}}(f,X):
Hom_{\mathscr{C}}(N,X)\rightarrow Hom_{\mathscr{C}}(M,X)$ is defined
by $g\mapsto g\circ f.$
\end{list}
\end{egs}
\begin{Def}[Natural transformation or Functorial morphism]
Let $\mathscr{C}$ and $\mathscr{D}$ be two categories, $F,G:
\mathscr{C}\rightarrow \mathscr{D}$ be two functors. A natural
transformation $\Phi:F\rightarrow G$ is by definition a function
which assigns to each $M\in Obj\mathscr{C}$ a morphism $\Phi_M\in
Hom_{\mathscr{D}}(F(M),G(M))$ such that $\forall f\in
Hom_{\mathscr{C}}(M,N)$ holds
\[ \xymatrix{
   F(M) \ar[r]^-{\Phi_M} \ar[d]_{F(f)} & G(M) \ar[d]^-{G(f)} \\
   F(N) \ar[r]^-{\Phi_N} & G(N) }  \]
\end{Def}

\newpage

\section{Direct systems and inverse systems}

Let $(I,\leqslant)$ be a directed set, $\mathscr{C}$ be a category.
\begin{Def}
A direct system $(A_i,\Phi_{ij})_{i,j\in I}$ in $\mathscr{C}$
consists of a family of objects $A_i(i\in I)$ and $\Phi_{ij}\in
Hom_{\mathscr{C}}(A_i,A_j)(i\leqslant j)$ such that:
\enum
\item[(1)]$\Phi_{ii}=id_{A_i}, \forall i\in I.$
\item[(2)]$\Phi_{jk}\Phi_{ij}=\Phi_{ik},$ whenever $i\leqslant
j\leqslant k.$
\end{list}
Consider the functor $F: CatI\rightarrow \mathscr{C}:$
\[ \xymatrix@R=0em{
   ObjCatI \ar[r] & Obj\mathscr{C}; \\
   i \ar@{|->}[r] & A_i }  \]
$\forall i\leqslant j,$
\[ \xymatrix@R=0em{
   Hom_{CatI}(i,j) \ar[r] & Hom_{\mathscr{C}}(A_i,A_j)    \\
   (i\hookrightarrow j) \ar@{|->}[r] & \Phi_{ij} }  \]
$\forall i\leqslant j\leqslant k,$ we have the composition
\[ \xymatrix@R=0em{
   Hom_{CatI}(i,j)\times Hom_{CatI}(j,k) \ar[r]^-{\mu} & Hom_{CatI}(i,k)\\
   (i\hookrightarrow j, j\hookrightarrow k) \ar@{|->}[r] &
   (i\hookrightarrow k) }  \]
So we have
$$F((j\hookrightarrow k)\circ(i\hookrightarrow j)) = F(j\hookrightarrow
k)\circ F(i\hookrightarrow j) = f(i\hookrightarrow k),$$ i.e.
$\Phi_{jk}\Phi_{ij}=\Phi_{ik}.$
\end{Def}
\begin{Def}
An inverse system $(A_i, \Phi_{ji})_{j,i\in I}$ in $\mathscr{C}$
consists of objects $A_i(i\in I)$ in $\mathscr{C}$ and $\Phi_{ji}\in
Hom_{\mathscr{C}}(A_j,A_i)(i\leqslant j)$ such that: \enum
\item[(1)]$\forall i\in I, \Phi_{ii}=id_{A_i},$
\item[(2)]$\Phi_{ji}\Phi_{kj}=\Phi_{ki},$ whenever $i\leqslant
j\leqslant k.$
\end{list}
As above, an inverse system is equivalent to a contravariant functor
from $CatI$ to $\mathscr{C}.$
\end{Def}
\begin{Def}
The direct limit of a direct system $(A_i,\Phi_{ij})_{i,j\in I}$ is
an object $A\in Obj\mathscr{C}$ (noted $\varinjlim\limits_{i\in
I}A_i$) together with $\Phi_i\in Hom_{\mathscr{C}}(A_i,A)$
satisfying $\Phi_j\Phi_{ij}=\Phi_i$ whenever $i\leqslant j,$ and
having the following universal property: $\forall C\in
Obj\mathscr{C}$ and $\psi_i\in Hom(A_i,C)(i\in I)$ satisfying
$\psi_j\Phi_{ij}=\psi_i(i\leqslant j),$ there exists a unique
$\psi\in Hom(A,C)$ such that $\psi\Phi_i=\psi_i(\forall i\in I).$
i.e. $\forall i\leqslant j,$
\[ \xymatrix{
   & C \\
   & A \ar@{-->}[u]_(.2){\exists!\psi} \\
   A_i \ar"1,2"^{\psi_i} \ar[ur]_{\Phi_i} \ar[rr]_{\Phi_{ij}} & &
   A_j \ar"1,2"_{\psi_j} \ar[ul]^{\Phi_j} }  \]
is commutative.

For the inverse limit of an inverse system, we need only to reverse
all the arrows in the above definition.
\end{Def}
\begin{egs}\
\enum
\item[(1)]$SET:$ Let $I$ be a nonempty set. $\forall i\in I,$ take
$A_i$ a set. Define a partial order on $I:$ $\forall i,j\in I,$ we
say $i\leqslant j$ if $i=j.$ $\forall i,j\in I, i\leqslant j,$ set
$\Phi_{ij}: A_i\rightarrow A_j$ to be $id_{A_i}.$ Then
$(A_i,\Phi_{ij})_{i,j\in I}$ is a direct system. Set
$$A=\coprod\limits_{i\in I}A_i=\{(i,x_i)\mid i\in I,x_i\in A_i\},$$
then $\varinjlim\limits_{i\in I}A_i=A.$

Indeed, $\forall i\in I,$ put
\[ \xymatrix@R=0em{
   A_i \ar[r]^-{\Phi_i} & A \\
   x_i \ar@{|->}[r] & (i,x_i) }  \]
$\forall x\in A, \exists!i\in I,x_i\in A$ such that $x=(i,x_i),$
define $\psi(x)=\psi_i(x_i),$ then we have
\[ \xymatrix{
   A_i \ar[rr]^-{\Phi_i}_-{x_i\mapsto(i,x_i)} \ar[dr]_{\psi_i} & &
   {A=\coprod\limits_{i\in I}A_i} \ar@{-->}[dl]^{\exists!\psi}  \\
   & C }  \]
with $\psi((i,x_i))=\psi_i(x_i).$
\item[(2)]Inverse limit:

Let $(A_i,\Phi_{ji})_{i,j\in I}$ be an inverse system. Define
$$A=\varprojlim\limits_{i\in I}A_i=\prod\limits_{i\in I}A_i =
\{(x_i)_{i\in I}\mid \forall i\in I,x_i\in A_i\}.$$ $\forall i\in
I,$ define
\[ \xymatrix@R=0em{
   A \ar[r]^-{\Phi_i} & A_i \\
   x=(x_j)_{j\in I} \ar@{|->}[r] & x_i }  \]
Let $C$ be a nonempty set,
\item[(3)]Direct sum in a category $\mathscr{C}:$

Let $I$ be a nonempty set. $\forall i\in I,$ let $A_i\in
Obj\mathscr{C}.$ Its direct limit (called a direct sum) is an object
$A$ in $\mathscr{C}$ with morphisms $k_i\in Hom(A_i,A)$ such that
\[ \xymatrix{
   A_i \ar[rr]^-{k_i} \ar[dr]_{\psi_i} & & A \ar@{-->}[dl]^{\exists!\psi}\\
   & C }  \]
commutes. $(A,k_i)_{i\in I}$ is the direct limit, and we denote $A$
by $\bigoplus\limits_{i\in I}A_i.$
\item[(4)]Direct product in a category $\mathscr{C}:$

Let $(A_i,\Phi_{ji})_{i,j\in I}$ be an inverse system. Its inverse
limit is by definition an object $A\in Obj\mathscr{C}$ (denoted by
$\prod\limits_{i\in I}A_i$), together with $p_i\in Hom(A,A_i)$ with
the following universal property:
\[ \xymatrix{
   A_i & & {A=\prod\limits_{i\in I}A_i} \ar[ll]_-{p_i}  \\
   & C \ar[ul]^{\psi_i} \ar@{-->}[ur]_{\exists!\psi} }  \]
\item[(5)]Let $(I, \leqslant)$ be a directed set and $(A_i,
\Phi_{ij})_{i,j\in I}$ be a direct system. Define
$A=\coprod\limits_{i\in I}A_i/\sim=\varinjlim\limits_{i\in I}A_i.$
$\forall (j,x_j),(k,x_k)\in \coprod\limits_{i\in I}A_i,$ we say that
$(j,x_j)\sim(k,x_k)$ if $\exists l\in I$ such that $j,k\leqslant l$
and $\Phi_{jl}(x_j)=\Phi_{kl}(x_k).$ $\forall i\in I,x_i\in A_i,$
define $\Phi_i(x_i)=\overline{(i,x_i)}$ the equivalence class
containing $(i,x_i).$ Then $(A,\Phi_i)_{i\in I}$ gives the direct
limit of $(A_i,\Phi_{ij})_{i,j\in I}.$
\[ \xymatrix{
   & C \\
   & A \ar@{-->}[u]_(.2){\exists!\psi} \\
   A_i \ar"1,2"^{\psi_i} \ar[ur]_{\Phi_i} \ar[rr]_{\Phi_{ij}} & &
   A_j \ar"1,2"_{\psi_j} \ar[ul]^{\Phi_j} }  \]
Define $\psi(x)=\psi_i(x_i).$ Indeed, $\forall x\in A, \exists
i,j\in I,$ such that $x=\overline{(i,x_i)}=\overline{(j,x_j)}.$ By
definition, $\exists k\geqslant i,j$ such that
$\Phi_{ik}(x_i)=\Phi_{jk}(x_j).$ Then
$\psi_k\Phi_{ik}(x_i)=\psi_k\Phi_{jk}(x_j),$ thus
$\psi_i(x_i)=\psi_j(x_j).$ Hence
$\psi(x)=\psi\circ\Phi_i(x_i)=\psi_i(x_i)=\psi_j(x_j).$ So $\psi$ is
well-defined.

In the category $SET,$ let $(A_i,\Phi_{ji})_{i,j\in I}$ be an
inverse system. Set
$$A=\{(x_i)_{i\in I}\mid \Phi_{ji}(x_j)=x_i,\forall i\leqslant j\}\subseteq \prod\limits_{i\in I}A_i.$$
$\forall x\in A$ with $x=(x_i)_{i\in I},$ and $\forall j\in I,$
define $\Phi_j(x)=x_j,$ similarly we can show that
$A=\varprojlim\limits_{i\in I}A_i.$
\[ \xymatrix{
   & C \ar"3,1"_{\psi_i} \ar@{-->}[d]^(.8){\exists!\psi}
   \ar"3,3"^{\psi_j}\\
   & A \ar[dl]^{\Phi_i} \ar[dr]_{\Phi_j}  \\
   A_i & & A_j \ar[ll]^{\Phi_{ji}} }  \]
\item[(6)]Let $A$ be a ring, consider the category $Mod(A).$
Let $(I, \leqslant)$ be a directed set and $(M_i, \Phi_{ij})_{i,j\in
I}$ be a direct system in $Mod(A).$ Put
$$C=\bigoplus\limits_{i\in I}M_i=\{(x_i)_{i\in I}\mid \forall i\in I, x_i\in M_i, x_i=0 \text{ for almost all }i\}.$$
$\forall i\in I,$ consider
\[ \xymatrix@R=0em{
   M_i \ar[r]^-{\psi_i} & C  \\
   x_i \ar@{|->}[r] & (x_i\delta_{ij})_{j\in I} }  \]
In this way, we can identify $M_i$ with a submodule of $C.$

Set $R$ be the submodule of $C$ generated by the elements
$x_i-\Phi_{ij}(x_i)(i\leqslant j, x_i\in M_i).$ Put $M=C/R,\forall
i\in I,x_i\in M_i,$ define $\Phi_i(x_i)=\bar{x}_i=x_i+R\in M,$ we
can show that $M=\varinjlim\limits_{i\in I}M_i.$ $\forall x\in
M=\bigoplus\limits_{i\in I}M_i/R,$
$$x=\bar{x}_{i_1}+\cdots+\bar{x}_{i_k}=\Phi_{i_1}(x_{i_1})+\cdots+\Phi_{i_k}(x_{i_k}).$$
$\exists l\geqslant i_1,\cdots,i_k,$ put
$x_l=\Phi_{i_1l}(x_{i_1})+\cdots+\Phi_{i_kl}(x_{i_k})\in M_l.$ Then
\begin{eqnarray*}
\Phi_l(x_l) & = &
\Phi_l\Phi_{i_1l}(x_{i_1})+\cdots+\Phi_l\Phi_{i_kl}(x_{i_k}) \\
& = & \Phi_{i_1}(x_{i_1})+\cdots+\Phi_{i_k}(x_{i_k})         \\
& = & x.
\end{eqnarray*}
i.e. we have proved that $\forall x\in M, \exists i\in I, x_i\in
M_i,$ such that $x=\Phi_i(x_i)=\bar{x}_i.$

Then we can define $\psi(x)=\psi(\Phi_i(x_i))=\psi_i(x_i).$ If
$\exists j\in I, x_j\in M_j$ such that $x=\Phi_i(x_i)=\Phi_j(x_j),$
since $I$ is directed, then $\exists k\geqslant i,j,$ such that
$\Phi_k\Phi_{ik}(x_i)=\Phi_k\Phi_{jk}(x_j),$ i.e.
$\Phi_k(\Phi_{ik}(x_i)-\Phi_{jk}(x_j))=0.$ We can prove $\exists
h\geqslant k$ such that
$\Phi_{kh}(\Phi_{ik}(x_i)-\Phi_{jk}(x_j))=0.$ Then
$\psi_i(x_i)=\psi_j(x_j),$ thus $\psi$ is well-defined.

Let $(M_i, \Phi_{ji})_{i,j\in I}$ be an inverse system, set
$$M=\{(x_i)_{i\in I}\mid \Phi_{ji}(x_j)=x_i,\forall i\leqslant j\in I\}.$$
$\forall i\in I,$ define
\[ \xymatrix@R=0em{
   M \ar[r]^-{\Phi_i} & M_i   \\
   (x_j)_{j\in I} \ar@{|->}[r] & x_i }  \]
Then $M=\varprojlim\limits_{i\in I}M_i.$
\end{list}
\end{egs}

Let $\mathscr{C}$ be a category, let $(I, \leqslant)$ be a directed
set, and $(M_i, \varphi_{ij})_{i,j\in I}$ be a direct system.
Suppose $(M, \varphi_i)_{i\in I}$ is its direct limit, then
$$\varprojlim\limits_{i\in I}Hom(M_i, N)\cong
Hom(\varinjlim\limits_{i\in I}M, N).$$ Now let $(M_i,
\varphi_{ji})_{i,j\in I}$ be an inverse system, we suppose that $(M,
\varphi_i)_{i\in I}$ is its inverse limit, then
$$\varprojlim\limits_{i\in I}Hom(N, M_i)\cong
Hom(N, \varprojlim\limits_{i\in I}M).$$

\newpage

\section{Abelian category}

\begin{Def}
Let $\mathscr{C}$ be a category, we say that $\mathscr{C}$ is
additive if the following conditions are satisfied:
\enum
\item[(1)]$\forall A,B\in Obj\mathscr{C},$ the direct product
$A\times B$ exists in $\mathscr{C}.$
\item[(2)]$\forall A,B\in Obj\mathscr{C},$ $Hom(A,B)$ is an Abelian
group( noted additively ).
\item[(3)]$\forall A,B\in Obj\mathscr{C},$
\[ \xymatrix@R=0em{
   Hom{A,B}\times Hom(B,C) \ar[r] & Hom(A,C)  \\
   (f,g) \ar@{|->}[r] & g\circ f }  \]
is a group homomorphism with respect to each variable: $\forall f\in
Hom(A,B), g_1,g_2\in Hom(B,C), (g_1+g_2)\circ f=g_1\circ f+g_2\circ
f; \forall g\in Hom(B,C), f_1,f_2\in Hom(A,B), g\circ(f_1+f_2)
=g\circ f_1+g\circ f_2.$
\end{list}
\end{Def}
\begin{remark}
$\forall A,B\in Obj\mathscr{C},$ denote by $0_{AB}$ the zero element
in $Hom(A,B).$ Then $\forall f\in Hom(B,C),$ we have $f\circ
0_{AB}=0_{AC}.$ Likewise $\forall g\in Hom(C,A), 0_{AB}\circ
g=0_{CB}.$
\end{remark}
\begin{egs}\
\enum
\item[(1)]$Ab,$ the category of Abelian groups, is additive.
\item[(2)]Let $A$ be a ring, then $Mod(A)$ is an additive category.
\end{list}
\end{egs}
\begin{prop}
Let $\mathscr{C}$ be an additive category, and $A,B\in
Obj\mathscr{C}.$
\enum
\item[(1)]Let $(A\times B,p_1,p_2)$ be the direct product of $A$ and
$B,$ then $\exists ! k_1\in Hom(A,A\times B),\exists ! k_2\in
Hom(B,A\times B),$ such that
$p_1k_1=id_A,p_2k_2=id_B,p_1k_2=0,p_2k_1=0.$ Moreover,
$k_1p_1+k_2p_2=id_{A\times B}.$
\[ \xymatrix{
   A & & B                                                 \\
   & {A\times B} \ar"1,1"_-{p_1} \ar"1,3"^-{p_2}           \\
   A \ar[uu]^{id_A} \ar@{-->}"2,2"_-{\exists ! k_1} & & B
   \ar[uu]_{id_B} \ar@{-->}"2,2"^-{\exists ! k_2} }  \]
\item[(2)]Suppose $\exists P\in Obj\mathscr{C},$ and $p_1\in
Hom(P,A), p_2\in Hom(P,B), k_1\in Hom(A,P), k_2\in Hom(B,P),$ such
that $$p_1k_1=id_A, p_2k_2=id_B, k_1p_1+k_2p_2=id_P,$$ then
$(P,p_1,p_2)$ is the direct product of $A$ and $B;$ $(P,k_1,k_2)$ is
the direct sum of $A$ and $B.$
\end{list}
\end{prop}
\begin{proof}\
\enum
\item[(1)]
\item[(2)]
\end{list}
\end{proof}
\begin{Def}
Let $\mathscr{C}$ be a category, $A,B\in Obj\mathscr{C},f\in
Hom(A,B).$
\enum
\item[(1)]We call $f$ a monomorphism or injective if for any
$\alpha,\beta\in Hom(C,A)$ satisfying
$$f\alpha=f\beta\Longrightarrow \alpha=\beta.$$
\item[(2)]We call $f$ an epimorphism or surjective if for any
$\alpha,\beta\in Hom(C,A)$ satisfying
$$\alpha f=\beta f\Longrightarrow \alpha=\beta.$$
\item[(3)]We call $f$ a bijection or bijective if $f$ is injective
and surjective.
\end{list}
\end{Def}
\begin{remarks}\
\enum
\item[(1)]$f$ is injective in $\mathscr{C}$ iff $f$ is surjective in
$\mathscr{C}^{op}.$
\item[(2)]An isomorphism is a morphism with a two-sided inverse, so
an isomorphism if bijective. But the converse is false in general.
\end{list}
\end{remarks}
\begin{Def}
Let $\mathscr{C}$ be an additive category, $A,B\in Obj\mathscr{C},$
and $f\in Hom(A,B).$
\enum
\item[(1)]We call $i\in Hom(K,A)$ the kernel of $f$ if $f\circ i=0$
and $\forall i^{\prime}\in Hom(K^{\prime},A)$ such that $f\circ
i^{\prime}=0,$ then $\exists ! j\in Hom(K^{\prime},K)$ such that
$i^{\prime}=i\circ j.$ We often denote $K$ by $kerf$ and call it the
kernel of $f.$
\[ \xymatrix{
   K \ar[r]^i & A \ar[r]^f & B                                    \\
   {K^{\prime}} \ar[ur]_{i^{\prime}} \ar@{-->}[u]^{\exists! j} }  \]
\item[(2)]We call $p\in Hom(B,C)$ the cokernel of $f$ if $p\circ f=0$
and $\forall p^{\prime}\in Hom(B,C^{\prime})$ such that
$p^{\prime}\circ f=0,$ then $\exists ! j\in Hom(C,C^{\prime})$ such
that $p^{\prime}=j\circ p.$ We often denote $C$ by $cokerf$ and call
it the cokernel of $f.$
\[ \xymatrix{
   A \ar[r]^f & B \ar[r]^p \ar[d]_{p^{\prime}}& C \ar@{-->}[dl]^{\exists! j}\\
   & {C^{\prime}} }  \]
\item[(3)]We denote by $imf$ the kernel of the cokernel of $f.$
\item[(4)]We denote by $coimf$ the cokernel of the kernel of $f.$
\end{list}
\end{Def}
\begin{remarks}\
\enum
\item[(1)]$kerf$ is injective and unique up to isomorphism.
\item[(2)]$cokerf$ is surjective and unique up to isomorphism.
\item[(3)]There exists a canonical morphism
$$\bar{f}:coimf\rightarrow imf$$
which is induced by
\[ \xymatrix{
   K \ar[r]^i_{kerf} & A \ar[d]_{\bar{p}} \ar[r]^f & B
   \ar[r]^p_{cokerf} & C                                        \\
   & cokeri=coimf \ar@{-->}[r]_-{\exists ! \bar{f}}
   \ar@{-->}[ur]^-{\exists !f_1} & imf=kerp \ar[u]_{\bar{i}} }  \]
\end{list}
\end{remarks}
\begin{proof}\
\enum
\item[(1)]$\forall \alpha,\beta\in Hom(C,K),$ if
$i\circ\alpha=i\circ\beta,$ put $\varphi=i\circ\alpha=i\circ\beta.$
Then $f\circ\varphi=f\circ i\circ\alpha=0,$ thus $\exists!j\in
Hom(C,K)$ such that $\varphi=i\circ j.$ But
$\varphi=i\circ\alpha=i\circ\beta,$ so $\alpha=\beta=\varphi.$
\item[(2)]$\forall \alpha,\beta\in Hom(C,D),$ if $\alpha\circ
p=\beta\circ p,$ put $\varphi=\alpha\circ p=\beta\circ p.$ Then
$\varphi\circ f=\alpha\circ p\circ f=0,$ thus $\exists!j\in
Hom(C,C^{\prime}$ such that $\varphi=j\circ p.$ But
$\varphi=\alpha\circ p=\beta\circ p,$ so $\alpha=\beta=j.$
\item[(3)]$\mathit{1^{\circ}}$ We claim that we have the following commutative diagram
\[ \xymatrix{
   K \ar[r]^i & A \ar[r]^-{\bar{p}}_-{cokeri} \ar[d]_f & coimf
   \ar[dl]^-{\exists ! f_1}                               \\
   & B }  \]
for $f\circ i=0,$ $\bar{p}$ is the cokernel of $i,$ then $\exists
!f_1,$ such that $f=f_1\circ\bar{p}.$

$\mathit{2^{\circ}}$ The following diagram
\[ \xymatrix{
   imf \ar[r]^-{\bar{i}}_-{kerp} & B \ar[r]^p & C        \\
   coimf \ar@{-->}[u]^{\exists!\bar{f}} \ar[ur]_{f_1} }  \]
commutes, because $0=p\circ f=p\circ f_1\circ \bar{p}.$ $\bar{p}$ is
the cokernel of $i,$ thus is surjective. So $p\circ f_1=0.$ But
$\bar{i}$ is the kernel of $p,$ therefore $\exists!\bar{f}\in
Hom(coimf,imf)$ such that $f_1=\bar{i}\circ \bar{f}.$
\end{list}
\end{proof}
\begin{eg}
In the category of Abelian groups $Ab,$ take $f\in Hom(A,B),$ then
\[ \xymatrix@R=0em{
   kerf=\{a\in A\mid f(a)=0\}, & imf=\{f(a)\mid a\in A\}, \\
   cokerf=B/imf, & coimf=A/kerf. }  \]
and we have
\[ \xymatrix{
   kerf\ \ar@^{(->}[r]^i & A \ar[d] \ar[r]^f & B \ar[r]^-p & {B/imf=cokerf}\\
   & {A/kerf=coimf} \ar@{-->}[r]^-{\exists ! \bar{f}}
   & imf \ar[u] }  \]
\end{eg}
\begin{Def}
Let $\mathscr{C}$ be an additive category. A zero object $0$ in
$\mathscr{C}$ is by definition an object such that $Hom(0,0)=\{0\},$
i.e. the identity element of the object $0$ is equal to the zero
morphism.
\end{Def}
\begin{remarks}\
\enum
\item[(1)]$\forall X\in Obj\mathscr{C}, Hom(X,0)=\{0\},
Hom(0,X)=\{0\}.$ Indeed, $\forall f\in Hom(X,0), f=id_0\circ f=0.$
\item[(2)]Zero object in $\mathscr{C}$ are isomorphic: Let $0_1,
0_2$ be two zero objects, $Hom(0_1,0_2)=\{o_{12}\},
Hom(0_2,0_1)=\{0_{21}\}. 0_{21}\circ 0_{12}\in
Hom(0_1,0_1)=\{id_{0_1}\}, 0_{12}\circ 0_{21}\in
Hom(0_2,0_2)=\{id_{0_2}\}. 0_{21}\circ 0_{12}=id_{0_1}, 0_{12}\circ
0_{21}=id_{0_2},$ so $0_1\cong 0_2.$
\end{list}
\end{remarks}
\begin{Def}
Let $\mathscr{C}$ be an additive category with zero object. We say
that $\mathscr{C}$ is an Abelian category if $\forall A,B\in
Obj\mathscr{C},$ and $\forall f\in Hom(A,B),$ $kerf$ and $cokerf$ do
exist (and hence $imf$ and $coimf$ do exist), and the canonical
morphism $\bar{f}:coimf\rightarrow imf$ is an isomorphism.
\end{Def}
\begin{Def}
Let $\mathscr{C}$ be an Abelian category, $A,B\in Obj\mathscr{C},
f\in Hom(A,B).$
\enum
\item[(1)]If $f$ is a monomorphism, then we say that $A$ is a
sub-object of $B,$ noted $A\subseteq B.$
\item[(2)]If $f$ is an epimorphism, and $p: \rightarrow C$ is the
cokernel of $f,$ then we say that $C$ is the quotient of $B$ by $A,$
noted $A/B.$
\end{list}
\end{Def}
\begin{eg}
$Ab, Mod(A)$ are Abelian categories, where $A$ is a ring.
\end{eg}
\begin{prop}
Let $f: A\rightarrow B$ be a morphism in an Abelian category,
\enum
\item[(1)]$f$ injective iff $kerf=0.$ In this case, $imf=A.$
\item[(2)]$f$ is surjective iff $cokerf=0,$ iff $imf=B.$
\end{list}
\end{prop}
\begin{proof}
$\mathit{1^{\circ}}$ $f$ is injective iff $kerf=0:$

$\Longrightarrow:$ Suppose that $f$ is injective, $f\circ kerf=0,$
then $kerf=0.$

$\Longleftarrow:$
\[ \xymatrix{
   0 \ar[r]^-{kerf} & A \ar[r]^f & B             \\
   C \ar@{-->}[u]^{\exists!\varphi} \ar[ur]_j }  \]
Suppose that $kerf=0.$ If $f\circ j=0,$ then $\exists!\varphi: C
\rightarrow 0,$ such that $j=kerf\circ\varphi=0\circ\varphi=0.$
Therefore $f\circ j=0\Longrightarrow j=0,$ thus $f$ is injective.

$\mathit{2^{\circ}}$ $imf=A:$

\[ \xymatrix{
   0 \ar[r]^-{kerf} & A \ar[r]^-{id_A} \ar[d]_j & A
   \ar@{-->}[dl]^{\exists!\varphi=j}              \\
   & C }  \]
$id_A=coker(kerf)=coimf\cong imf,$ so $A=imf.$
\end{proof}
\begin{cor}
In an Abelian category, a bijective morphism is an isomorphism.
\end{cor}
\begin{proof}
Suppose $f: A\rightarrow B$ is bijective,
\[ \xymatrix{
   0 \ar[r]^i & A \ar[r]^f \ar[d]_{p_1} & B                \\
   & {A=coimf} \ar[r]^-{\bar{f}} & {imf=B} \ar[u]_{p_2} }  \]
$f=p_2\circ\bar{f}\circ p_1,$ $p_2,\bar{f},p_1$ are isomorphisms, so
$f$ is an isomorphism.
\end{proof}
\begin{Def}
Let $\mathscr{C}$ be an Abelian category,
\enum
\item[(1)]In $\mathscr{C},$ a sequence of morphisms
$A\stackrel{u}{\rightarrow} B\stackrel{v}{\rightarrow} C$ is called
exact if $v\circ u=0$ and the canonical morphism $coimu\rightarrow
kerv$ is an isomorphism, equivalently, $imu=kerv.$
\item[(2)]An exact sequence of the form $0\rightarrow A\rightarrow B\rightarrow C\rightarrow
0$ is called a short exact sequence. It is called split if it is
isomorphic to $0\rightarrow A\rightarrow A\oplus C\rightarrow
C\rightarrow 0,$ where $A\rightarrow A\oplus C,$ and $A\oplus
C=A\times C\rightarrow C$ are the canonical morphisms.
\end{list}
\end{Def}
\begin{remarks}\
\enum
\item[(1)]
\[ \xymatrix{
   keru \ar[r]^i & A \ar[r]^u \ar[d]_{p_1} & B \ar[r]^v & C         \\
   imu \ar[r]^-{\sim} & {coimu=cokeri} \ar@{-->}[r]_-{\exists!\varphi}
   \ar@{-->}[ur]^-{\exists!j} & kerv \ar[u]_{i^{\prime}} }  \]
$0=v\circ u=v\circ j\circ p_1,$ $p_1$ is surjective, so we have
$v\circ j=0,$ then $\exists!\varphi:cokeri\rightarrow kerv$ such
that $j=i^{\prime}\circ\varphi.$
\item[(2)]$0\rightarrow A\stackrel{f}{\rightarrow}B$ is exact iff
$f$ is injective.
\item[(3)]$A\stackrel{f}{\rightarrow}B\rightarrow 0$ is exact iff
$f$ is surjective.
\item[(4)]$0\rightarrow A\stackrel{f}{\rightarrow}B\stackrel{g}{\rightarrow}C\rightarrow
0$ is exact iff $f$ is injective, $imf=kerg,$ and $g$ is surjective.
In this case, $f=kerg,g=cokerf.$
\end{list}
\end{remarks}
\begin{prop}
Let $0\rightarrow A_1\stackrel{k_1}{\rightarrow} A
\stackrel{p_1}{\rightarrow}A_2\rightarrow 0$ be a short exact
sequence in an Abelian category. Then the followings are equivalent.
\enum
\item[(1)]The sequence is split;
\item[(2)]$\exists p_1\in Hom(A,A_1)$ such that $p_1k_1=id_{A_1};$
\item[(3)]$\exists k_2\in Hom(A_2,A)$ such that $p_2k_2=id_{A_2}.$
\end{list}
\end{prop}
\begin{proof}
$(1)\Longrightarrow(2),(3):$ $A$ is isomorphic to $A_1\otimes
A_2=A_1\times A_2,$ so the existence of $p_1$ and $k_2$ are obvious.

$(2)\Longrightarrow(1):$
\[ \xymatrix{
   0 \ar[r] & A_1 \ar@<.6ex>[r]^{k_1} & A \ar@<.6ex>[l]^{p_1} \ar[r]^{p_2}
   \ar[dr]_f & A_2 \ar[r] \ar@{-->}[d]^{\exists!k_2} & 0     \\
   & & & A }  \]
Put $f=id_A-k_1p_1\in Hom(A,A),$ then
$$f\circ k_1=k_1-k_1p_1k_1=k_1-k_1\circ id_{A_1}=k_1-k_1=0.$$
$p_2=cokerk_1,$ so $\exists! k_2:A_2\rightarrow A$ such that
$f=k_2p_2.$ Hence we have $id_A=k_1p_1+k_2p_2,$ then $id_{A_2}\circ
p_2=p_2=p_2\circ id_A=p_2k_2p_2.$ Since $p_2$ is surjective, then
$id_{A_2}=p_2k_2.$ Therefore $(A,k_1,k_2)$ gives $A_1\oplus A_2,$
$(A,p_1,p_2)$ gives $A_1\times A_2.$
\end{proof}
\begin{Def}
Let $\mathscr{C}$ and $\mathscr{D}$ be two Abelian categories, $F:
\mathscr{C}\rightarrow \mathscr{D}$ be a covariant functor.
\enum
\item[(1)]We say that $F$ is additive if $\forall A,B\in
Obj\mathscr{C},$ the mapping
\[ \xymatrix@R=0em{
   Hom_{\mathscr{C}}(A,B) \ar[r]^-F & Hom_{\mathscr{D}}(F(A),F(B))\\
   f \ar@{|->}[r] & F(f) }  \]
is a group homomorphism.
\item[(2)]We say that $F$ is exact if for any exact sequence
$A\stackrel{f}{\rightarrow} B\stackrel{g}{\rightarrow} C$ in
$\mathscr{C},$ the sequence $F(A)\stackrel{F(f)}{\rightarrow}
F(B)\stackrel{F(g)}{\rightarrow} F(C)$ is exact in $\mathscr{D}.$
\item[(3)]We say that $F$ is left exact if for any exact sequence
$0\rightarrow A\stackrel{f}{\rightarrow} B\stackrel{g}{\rightarrow}
C$ in $\mathscr{C},$ the sequence $0\rightarrow
F(A)\stackrel{F(f)}{\rightarrow} F(B)\stackrel{F(g)}{\rightarrow}
F(C)$ is exact in $\mathscr{D}.$
\item[(4)]We say that $F$ is right exact if for any exact sequence
$A\stackrel{f}{\rightarrow} B\stackrel{g}{\rightarrow} C\rightarrow
0$ in $\mathscr{C},$ the sequence $F(A)\stackrel{F(f)}{\rightarrow}
F(B)\stackrel{F(g)}{\rightarrow} F(C)\rightarrow 0$ is exact in
$\mathscr{D}.$
\end{list}
\end{Def}
\begin{prop}
Let $\mathscr{C}$ and $\mathscr{D}$ be two Abelian categories, $F:
\mathscr{C}\rightarrow \mathscr{D}$ be an additive covariant
functor.
\enum
\item[(1)]$\forall A,B\in Obj\mathscr{C}, F(A\otimes B)\cong
F(A)\otimes F(B).$
\item[(2)]If $0\rightarrow A\stackrel{f}{\rightarrow} B\stackrel{g}{\rightarrow} C\rightarrow
0$ is split in $\mathscr{C},$ so is $0\rightarrow
F(A)\stackrel{F(f)}{\rightarrow} F(B)\stackrel{F(g)}{\rightarrow}
F(C)\rightarrow 0$ in $\mathscr{D}.$
\end{list}
\end{prop}
\begin{proof}\
\enum
\item[(1)]
\[ \xymatrix{
   & {A\otimes B} \ar@<.6ex>[dl]^{p_1} \ar@<-.6ex>[dr]_{p_2}  \\
   A \ar@<.6ex>[ur]^{k_1} & & B \ar@<-.6ex>[ul]_{k_2} }  \]
$p_1k_1=id_A,p_2k_2=id_B,k_1p_1+k_2p_2=id_{A\otimes B}.$ Then we
have $$F(p_1)F(k_1)=F(p_1k_1)=F(id_A)=id_{F(A)},$$
$$F(p_2)F(k_2)=F(id_B)=id_{F(B)},$$
$$F(k_1)F(p_1)+F(k_2)F(p_2)=id_{F(A\otimes B)}.$$

Therefore $(F(A\otimes B),F(k_1),F(k_2))$ gives $F(A)\otimes F(B),$
so $F(A\otimes B)\cong F(A)\otimes F(B).$
\item[(2)]It's direct by $(1).$
\end{list}
\end{proof}
\begin{prop}
Let $\mathscr{C}$ be an additive category, $A,B\in Obj\mathscr{C},
f,g\in Hom(A,B).$
\enum
\item[(1)]If $\forall i\in Hom(C,A),$ we have $f\circ i=0$ iff
$g\circ i=0,$ then $kerf=kerg.$
\item[(2)]If $\forall p\in Hom(C,A),$ we have $p\circ f=0$ iff
$p\circ g=0,$ then $cokerf=cokerg.$
\end{list}
\end{prop}
\begin{proof}
\[ \xymatrix{
   kerf \ar[r]^-{i_f} & A \ar[r]^-f & B          \\
   kerg \ar[ur]_{i_g} \ar@{-->}[u]^{\exists!\psi} }
   \xymatrix{
   kerg \ar[r]^-{i_g} & A \ar[r]^-f & B             \\
   kerf \ar[ur]_{i_f} \ar@{-->}[u]^{\exists!\varphi} }
\]
$i_g=i_f\circ\psi, i_f=i_g\circ\varphi,$ then
$i_f\circ\psi\circ\varphi=i_g\circ\varphi=i_f=i_f\circ id_{kerf}.$
$i_f$ is injective, thus $\psi\circ\varphi=id_{kerf}.$
$i_g\circ\varphi\circ\psi=i_f\circ\psi=i_g=i_g\circ id_{kerg},$
$i_g$ is injective, so $\varphi\circ\psi=id_{kerg}.$ Therefore we
obtain that $kerf\cong kerg.$
\end{proof}
\begin{prop}
Let $\mathscr{C}$ and $\mathscr{D}$ be two Abelian categories,
$F:\mathscr{C}\rightarrow \mathscr{D}$ an additive covariant
functor. Then $F$ is exact iff $F$ sends every short exact sequence
in $\mathscr{C}$ to a short exact sequence in $\mathscr{D}.$
\end{prop}
\begin{proof}
$\Longrightarrow:$ is direct.

$\Longleftarrow:$ Suppose that $A\stackrel{f}{\rightarrow}
B\stackrel{g}{\rightarrow} C$ is exact in $\mathscr{C},$ then the
diagram
\[ \xymatrix{
   & & & coimg \ar[r]^-{\bar{g}} & img \ar@<.6ex>[r] \ar[d]^j & 0
   \ar@<.6ex>[l]                                               \\
   0 \ar[r] & kerf\ar[r]^-t & A \ar[r]^-f \ar[d] \ar[dr]^-p & B
   \ar[u] \ar[r]^-g \ar[ur]^-q & C \ar[r]^-s & cokerg \ar[r] &0 \\
   & & coimf\ar[r]_-{\bar{f}} & imf \ar[u]_i \ar@<.6ex>[r] & 0
   \ar@<.6ex>[l] }  \]
is commutative. Then we obtain the following three short exact
sequences:
$$0\rightarrow kerf\stackrel{t}{\rightarrow} A\stackrel{p}{\rightarrow} imf\rightarrow 0,$$
$$0\rightarrow kerg(=imf)\stackrel{i}{\rightarrow} B\stackrel{q}{\rightarrow} img\rightarrow 0,$$
$$0\rightarrow img\stackrel{j}{\rightarrow} C\stackrel{s}{\rightarrow} cokerg\rightarrow 0.$$
By hypothesis, $F$ sends the three short exact sequences in
$\mathscr{C}$ to an exact sequence in $\mathscr{D}:$
$$F(A)\rightarrow F(B)\rightarrow F(C).$$
\end{proof}
\begin{prop}
Let $\mathscr{C}$ be an Abelian category and $(E):0\rightarrow
A\stackrel{f}{\rightarrow} B \stackrel{g}{\rightarrow} C$ be a
sequence of morphisms in $\mathscr{C}.$ Then the followings are
equivalent:
\enum
\item[(1)]The sequence is exact.
\item[(2)]$\forall x\in Obj\mathscr{C},$ the sequence
$Hom_{\mathscr{C}}(X,E)$ is exact in $Ab:$
\[ \xymatrix@C=4em{
   0 \ar[r] & Hom_{\mathscr{C}}(X,A) \ar[r]^{Hom_{\mathscr{C}}(X,f)} &
   Hom_{\mathscr{C}}(X,B) \ar[r]^{Hom_{\mathscr{C}}(X,g)} &
   Hom_{\mathscr{C}}(X,C) }  \]
\end{list}
\end{prop}
\begin{proof}
$(1)\Longrightarrow(2):$

$\mathit{1^{\circ}}$ $Hom_{\mathscr{C}}(X,f)$ is surjective:

$\forall u\in Hom_{\mathscr{C}}(X,A),$ if
$Hom_{\mathscr{C}}(X,f)(u)=0,$ i.e. $f\circ u=0,$ since $f$ is
injective, so $u=0.$ Hence $Hom_{\mathscr{C}}(X,f)$ is injective.

$\mathit{2^{\circ}}$
$im(Hom_{\mathscr{C}}(X,f))=ker(Hom_{\mathscr{C}}(X,g)):$

$\forall v\in ker(Hom_{\mathscr{C}}(X,g)),$ we have
$0=Hom_{\mathscr{C}}(X,g)(v)=g\circ v.$ But $kerg=imf=f,$ then
$\exists u\in Hom_{\mathscr{C}}(X,A)$ such that $v=f\circ
u=Hom_{\mathscr{C}}(X,f)(u).$ Hence
$ker(Hom_{\mathscr{C}}(X,g))\subseteq im(Hom_{\mathscr{C}}(X,f)).$

Conversely, $Hom_{\mathscr{C}}(X,g)\circ Hom_{\mathscr{C}}(X,f) =
Hom_{\mathscr{C}}(X,g\circ f) = Hom_{\mathscr{C}}(X,0)=0,$ so we
obtain that $im(Hom_{\mathscr{C}}(X,f))\subseteq
ker(Hom_{\mathscr{C}}(X,g)).$

$(2)\Longrightarrow(1):$

$\mathit{1^{\circ}}$ $f$ is injective:

Take $X=kerf, X=kerf\stackrel{i}{\rightarrow}
A\stackrel{f}{\rightarrow} B\stackrel{g}{\rightarrow} C.$ Then
$0=f\circ i=Hom_{\mathscr{C}}(X,f)(i),$ since
$Hom_{\mathscr{C}}(X,f)$ is injective, so $i=o,$ therefore $f$ is
injective.

$\mathit{2^{\circ}}$ $f=kerg:$

To show that $f=kerg,$ we should show that $\forall u\in
Hom_{\mathscr{C}}(X,B)$ such that $g\circ u=0,$ then $\exists v\in
Hom_{\mathscr{C}}(X,A)$ such that $u=f\circ v.$ Indeed, $0=g\circ
u=Hom_{\mathscr{C}}(X,g)(u),$ so $u\in
ker(Hom_{\mathscr{C}}(X,g))=im(Hom_{\mathscr{C}}(X,f)).$ Then
$\exists v\in Hom_{\mathscr{C}}(X,A)$ such that
$u=Hom_{\mathscr{C}}(X,f)(v)=f\circ v.$
\end{proof}
\begin{remark}
$\forall X\in Obj\mathscr{C},$ the covariant functor
$Hom_{\mathscr{C}}(X,-)$ is left exact. In general, it is not exact.
Take, for example, $\mathscr{C}=Mod(A),$ with $A=\mathbb{Z},
B=\mathbb{Z}, C=\mathbb{Z}/2\mathbb{Z}, X=\mathbb{Z}/2\mathbb{Z},$
the canonical ring homomorphisms $0 \rightarrow A\rightarrow
B\rightarrow C\rightarrow 0$ is exact, but $0 \rightarrow
Hom_{\mathscr{C}}(X,A)\rightarrow Hom_{\mathscr{C}}(X,B)\rightarrow
Hom_{\mathscr{C}}(X,C)\rightarrow 0$ can't be exact, because
$Hom_{\mathscr{C}}(X,B)=0,$ while $Hom_{\mathscr{C}}(X,C)$ includes
at least $id_{\mathbb{Z}/2\mathbb{Z}}.$
\end{remark}
\begin{prop}
Let $\mathscr{C}$ be an Abelian category and $(E):0\rightarrow
A\stackrel{f}{\rightarrow} B \stackrel{g}{\rightarrow} C\rightarrow
0$ be a sequence of morphisms in $\mathscr{C}.$ Then the followings
are equivalent:
\enum
\item[(1)]The sequence $(E)$ is split.
\item[(2)]$\forall X\in Obj\mathscr{C},$ the sequence
$Hom_{\mathscr{C}}(X,E)$ is exact:
\[ \xymatrix@C=3em{
   0 \ar[r] & Hom_{\mathscr{C}}(X,A) \ar[r]^{Hom_{\mathscr{C}}(X,f)} &
   Hom_{\mathscr{C}}(X,B) \ar[r]^{Hom_{\mathscr{C}}(X,g)} &
   Hom_{\mathscr{C}}(X,C) \ar[r] & 0 }  \]
\end{list}
\end{prop}
\begin{proof}
$(1)\Longrightarrow(2):$ comes from the fact that
$Hom_{\mathscr{C}}(X,-)$ is covariant and additive.

$(2)\Longrightarrow(1):$ Put $X=C,$ since $Hom_{\mathscr{C}}(C,g)$
is surjective, then for $id_C\in Hom_{\mathscr{C}}(C,C), \exists
v\in Hom_{\mathscr{C}}(C,B)$ such that
$id_C=Hom_{\mathscr{C}}(C,g)(v)=g\circ v.$ Then $(B,f,v)$ gives
$A\otimes C,$ so $(E)$ is split.
\end{proof}
\begin{prop}
Let $\mathscr{C}$ be an Abelian category and $(E):
A\stackrel{f}{\rightarrow} B \stackrel{g}{\rightarrow} C\rightarrow
0$ be a sequence of morphisms in $\mathscr{C}.$ Then the followings
are equivalent:
\enum
\item[(1)]The sequence $(E)$ is exact.
\item[(2)]$\forall X\in Obj\mathscr{C},$ the sequence
$Hom_{\mathscr{C}}(E,X)$ is exact:
\[ \xymatrix@C=4em{
   0 \ar[r] & Hom_{\mathscr{C}}(C,X) \ar[r]^{Hom_{\mathscr{C}}(g,X)} &
   Hom_{\mathscr{C}}(B,X) \ar[r]^{Hom_{\mathscr{C}}(f,X)} &
   Hom_{\mathscr{C}}(A,X) }  \]
\end{list}
\end{prop}
\begin{remark}
$\forall X\in Obj\mathscr{C},$ the additive contravariant functor
$Hom_{\mathscr{C}}(-,X)$ is left exact.
\end{remark}
\begin{prop}
Let $\mathscr{C}$ be an Abelian category and $(E):0\rightarrow
A\stackrel{f}{\rightarrow} B \stackrel{g}{\rightarrow} C\rightarrow
0$ be a sequence of morphisms in $\mathscr{C}.$ Then the followings
are equivalent:
\enum
\item[(1)]The sequence $(E)$ is split.
\item[(2)]$\forall X\in Obj\mathscr{C},$ the sequence
$Hom_{\mathscr{C}}(E,X)$ is exact:
\[ \xymatrix@C=3em{
   0 \ar[r] & Hom_{\mathscr{C}}(C,X) \ar[r]^{Hom_{\mathscr{C}}(g,X)} &
   Hom_{\mathscr{C}}(B,X) \ar[r]^{Hom_{\mathscr{C}}(f,X)} &
   Hom_{\mathscr{C}}(A,X) \ar[r] & 0 }  \]
\end{list}
\end{prop}
\begin{Def}
Let $\mathscr{C}$ and $\mathscr{D}$ be two categories,
$F:\mathscr{C}\rightarrow \mathscr{D}, G:\mathscr{D}\rightarrow
\mathscr{C}$ be two covariant functors. We say that $F$ is a left
adjoint of $G$ ($G$ is a right adjoint of $F$), if $\forall C\in
Obj\mathscr{C}, Y\in Obj\mathscr{D},$ there exists a bijection
$\Phi_{XY}: Hom_{\mathscr{D}}(F(X),Y)\rightarrow
Hom_{\mathscr{C}}(X,G(Y)),$ which is functorial in both $X$ and $Y.$
i.e. $\forall f\in Hom_{\mathscr{C}}(X,X^{\prime}), \forall g\in
Hom_{\mathscr{D}}(Y,Y^{\prime}),$ we have the following commutative
diagram
\[ \xymatrix{
   Hom_{\mathscr{D}}(F(X^{\prime}),Y) \ar[r]^-{\Phi_{X^{\prime}Y}}
   \ar[d]_{Hom_{\mathscr{D}}(F(f),Y)} &
   Hom_{\mathscr{C}}(X^{\prime},G(Y)) \ar[d]^{Hom_{\mathscr{C}}(f,G(Y))}\\
   Hom_{\mathscr{D}}(F(X),Y) \ar[r]^-{\Phi_{XY}}
   \ar[d]_{Hom_{\mathscr{D}}(F(X),g)} & Hom_{\mathscr{C}}(X,G(Y))
   \ar[d]^{Hom_{\mathscr{C}}(X,G(g))}                           \\
   Hom_{\mathscr{D}}(F(X),Y^{\prime}) \ar[r]^-{\Phi_{XY^{\prime}}} &
   Hom_{\mathscr{C}}(X,G(Y^{\prime})) }  \]
\end{Def}
\begin{remark}
Often, we say that $F$ is a left adjoint functor, and $G$ is a right
adjoint functor.
\end{remark}
\begin{prop}
Let $\mathscr{C}$ and $\mathscr{D}$ be two categories,
$F:\mathscr{C}\rightarrow \mathscr{D}, G:\mathscr{D}\rightarrow
\mathscr{C}$ be two covariant functors such that $F$ is a left
adjoint of $G.$ Then $F$ is right exact and $G$ is left exact.
\end{prop}
\begin{proof}
Let $(E): A\stackrel{f}{\rightarrow} B \stackrel{g}{\rightarrow}
C\rightarrow 0$ be an exact sequence in $\mathscr{C},$ then $\forall
Y\in \mathscr{D},$ the sequence $Hom_{\mathscr{C}}(E,G(Y))$ is
exact, and therefore $Hom_{\mathscr{D}}(F(E),Y)$ is exact, so $F(E)$
is exact. Hence $F$ is right exact.
\end{proof}
\begin{remark}
In Abelian categories, a left adjoint functor is right exact, a
right adjoint functor is left exact.
\end{remark}
\begin{prop}
Let $A$ be a ring, and $M\in ObjMod(A).$ Define
$$T_M(N)=M\otimes_A N, \forall N\in ObjMod(A);$$
$$T_M(f)=id_M\otimes_A f, \forall f\in Hom_A(N,P).$$
Then $T_M$ is covariant and is the left adjoint of $Hom_A(M,-),$
thus $T_M$ is right exact: $N_1\rightarrow N_2\rightarrow
N_3\rightarrow 0$ is exact $\Longrightarrow$ $M\otimes_A
N_1\rightarrow M\otimes_A N_2\rightarrow M\otimes_A N_3\rightarrow
0$ is exact.
\end{prop}
\begin{proof}
We have the canonical isomorphisms:
$$Hom(M\otimes N,P)\cong Hom(m,n;p)\cong Hom(N,Hom(M,P)),$$
defined by
\[ \xymatrix{
   M\times N \ar[d]_{j_{M,N}} \ar[r]^-f & P \\
   M\otimes N \ar@{-->}[ur]_-{\exists!\bar{f}} }  \]
where $f$ is any bilinear homomorphism. Define $f_N: N\rightarrow
Hom(M,P)$ by putting $f_N(n)=f(-,m), \forall n\in N.$ Then
\[ \xymatrix@R=0em{
   {\Phi_{NP}: Hom(T_M(N),P)} \ar[r] & {Hom(N,Hom(M,P))} \\
   \bar{f} \ar@{|->}[r] & f_N }  \]
is an isomorphism and functorial in $N$ and $P.$
\end{proof}
\begin{Def}
Let $A$ be a ring and $M\in ObjMod(A).$ \enum
\item[(1)]We say that $M$ is flat if $T_M$ is exact.
\item[(2)]We say that $M$ is faithful flat, if the exactness of
$(E): N_1\rightarrow N_2\rightarrow N_3$ is equivalent to the
exactness of $(T_M(E)).$
\item[(3)]We say that $M$ is projective if $Hom_A(M,-)$ is exact.
\item[(4)]We say that $M$ is injective if $Hom_A(-,M)$ is exact.
\end{list}
\end{Def}
\begin{remark}
$A$ is a faithful flat $A$-module.
\end{remark}
\begin{prop}
Let $A$ be a ring and $S\subseteq A$ be a multiplicatively closed
subset of $A.$ Define $I_S=T_{S^{-1}A},$ then $I_S$ is exact.
\end{prop}
\begin{proof}
$$\forall M\in ObjMod(A), I_S(M)=S^{-1}A\otimes_A M=S^{-1}M;$$
$$\forall f\in Hom_A(N,P), I_S(f)=id_{S^{-1}A}\otimes_A f=S^{-1}f.$$
Let $N_1\stackrel{f}{\rightarrow}N_2\stackrel{g}{\rightarrow}N_3$ be
exact, $g\circ f=0, imf=kerg.$ Then $S^{-1}g\circ
S^{-1}f=S^{-1}g\circ f=0,$ thus $im(S^{-1}f)\subseteq ker(S^{-1}g).$
$forall y\in ker(S^{-1}g),$ write $y=\frac{a}{s}$ with $a\in N_2,
s\in S.$ Since $0=S^{-1}g(y)=S^{-1}g(\frac{a}{s})=\frac{g(a)}{s},$
so $\exists t\in S$ such that $g(a)t=0,$ i.e. $g(at)=0.$ Hence
$at\in kerg=imf,$ then $\exists b\in N_1$ such that $at=f(b),$ then
$y=\frac{a}{s}=\frac{at}{st}=\frac{f(b)}{st}=S^{-1}f(\frac{b}{st}).$
Set $x=\frac{b}{st},$ then $x\in S^{-1}N_1, y=S^{-1}f(x),$ thus
$im(S^{-1}f)\supseteq ker(S^{-1}g).$ Hence
$im(S^{-1}f)=ker(S^{-1}g).$ Then $S^{-1}N_1
\stackrel{S^{-1}f}{\rightarrow}S^{-1}
N_2\stackrel{S^{-1}g}{\rightarrow}S^{-1} N_3$ is exact.
\end{proof}
\begin{lemma}[Snake lemma]
Let
\[ \xymatrix{
   & A \ar[d]_u \ar[r]^f & B \ar[d]_v \ar[r]^g & C \ar[d]_w \ar[r]&
   0\\
   0 \ar[r] & A^{\prime} \ar[r]^{f^{\prime}} & B^{\prime}
   \ar[r]^{g^{\prime}} & C^{\prime} }  \]
be a commutative diagram in an Abelian category $\mathscr{C}$ such
that the two rows are exact. Then $\exists \delta\in
Hom_{\mathscr{C}}(kerw,cokeru)$ such that
$$keru\rightarrow kerv\rightarrow kerw\rightarrow cokeru\rightarrow cokerv\rightarrow cokerw$$
is exact.
\end{lemma}
\begin{lemma}[Five lemma]
Let
\[ \xymatrix{
   A_1 \ar[d]_{f_1} \ar[r] & A_2 \ar[d]_{f_2} \ar[r] & A_3 \ar[d]_{f_3}
   \ar[r] & A_4 \ar[d]_{f_4} \ar[r] & A_5 \ar[d]_{f_5}         \\
   B_1 \ar[r] & B_2 \ar[r] & B_3 \ar[r] & B_4 \ar[r] & B_5 }  \]
be a commutative diagram in an Abelian category $\mathscr{C}$ such
that the two rows are exact. If $f_1,f_2,f_4,f_5$ are isomorphisms,
so is $f_3.$
\end{lemma}
