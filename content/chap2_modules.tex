%
\chapter{Modules}

\newpage

\section{Basic notions about modules}

\begin{Def}
Let $A$ be a ring. An $A$-module $M$ is an Abelian group on which
$A$ acts linearly:
\[ \xymatrix@R=0em{
   \mu: A\times M \ar[r] & M \\
   (a,m) \ar@{|->}[r] & \mu(a,m):=am, }  \]
satisfying: $a(m_1+m_2)=am_1+am_2, a(\lambda m)=(a\lambda)m,
(a+b)m=am+bm, 1\cdot m=m,$ where $a,b,\lambda\in A; m_1,m_2\in M.$
\end{Def}
\begin{egs}\
\enum
\item[(1)]An ideal of $A$ is an $A$-module. In particular, $A$ is an
$A$-module.
\item[(2)]If $A$ is a field $k,$ then $A$-modules $=$ $k-$vector
spaces.
\item[(3)]If $A=\mathbb{Z},$ then $\mathbb{Z}$-modules $=$ Abelian
groups.
\item[(4)]If $A=k[x]$ with $k$ a field, then $A$-module $=$
$k$-vector space with a linear transformation.
\item[(5)]If $G$ is a finite group, $A=k[G]$ the group-algebra of
$G$ over the field $k,$ then $A$-module $=$ $k$-representation of
$G.$
\end{list}
\end{egs}
\begin{Def}[$A$-module morphism]
$f: M\rightarrow M$ satisfying $f(\lambda x+\mu y)=\lambda f(x)+\mu
f(y), \forall \lambda,\mu\in A.$
\end{Def}
\begin{Def}
Let $M$ be an $A$-module, $M^{\prime}$ be a subgroup of $M$ such
that $M^{\prime}$ is an $A$-module, then $M^{\prime}$ is called an
$A$-submodule of $M.$
\end{Def}
\begin{Def}
Let $M^{\prime}$ be a submodule of $M,$ then $M/M^{\prime} =
\{m+M^{\prime}\mid m\in M\}$ is called a quotient module. We denote
$m+M^{\prime}$ by $\bar{m},$ and $\bar{m}_1+\bar{m}_2 =
\overline{m_1+m_2}, \lambda \bar{m} = \overline{\lambda m}.$
\end{Def}

\newpage

\section{Operations on submodules}

Sum: Let $M_i(i\in I)$ be $A$-modules,
$$\sum\limits_{i\in I}M_i:=\{\sum\limits_{i\in I}x_j\mid J\subseteq I, J\text{ a finite set, } x_j\in M_j\}.$$

Intersection: $\bigcap\limits_{i\in I}M_i.$

Product between an ideal and a module: Let $I\subseteq A$ be an
ideal, and $M$ be an $A$-module.
$$I\cdot M: = \{\sum\limits{i<\infty}a_im_i\mid a_i\in I, m_i\in
M\}.$$

Module quotient: Let $N,P\subseteq M$ be submodules, $(N:P):= \{a\in
A\mid aP\subseteq N\}$ is an ideal of $A.$
\begin{remark}
If $M$ is an $A$-module, then $M$ is $A/Ann(M)$-faithful.
\end{remark}

Direct product: Let $M_i(i\in I)$ be $A$-modules, then their direct
product is defined by
$$\prod\limits_{i\in I}M_i:= \{(x_i)_{i\in I}\mid x_i\in M_i\}.$$

Direct sum: Let $M_i(i\in I)$ be $A$-modules, then their direct
product is defined by
$$\bigoplus\limits_{i\in I}M_i:= \{(x_i)_{i\in I}\in\prod\limits_{i\in I}M_i
\mid x_i=0 \text{ except for a finite number of }i\}.$$

\begin{Def}[Free modules generated by a nonempty set]
Let $I$ be a nonempty set, put
$$A^I:=\{(x_i)_{i\in I}\mid x_i\in A_i\},$$
$$A^{(I)}:=\{(x_i)_{i\in I}\in A^I \mid x_i=0 \text{ except for a finite number
of }i\}.$$ By definition, an $A$-module $M$ is said to be free, if
there exists a nonempty $I$ such that $M\cong A^{(I)},$ or $M$ has
an $A$-module basis.
\end{Def}
\begin{prop}\
\enum
\item[(1)]Let $L\supseteq M\supseteq N$ be three modules, then
$(L/N)/(M/N)\cong L/M.$
\item[(2)]Let $M_1,M_2\subseteq M$ be modules, then
$(M_1+M_2)/M_1\cong M_2/(M_1\cap M_2).$
\end{list}
\end{prop}
\begin{proof}\
\enum
\item[(1)]We have a commutative diagram
\[ \xymatrix{
   L \ar[r]^-{f_1} \ar[d]_{f_2} & L/M  \\
   L/N \ar[ur]_{f_3} }  \]
where $f_1,f_2,f_3$ are defined respectively by $x\mapsto x+M,
x\mapsto x+N, x+N\mapsto x+M.$ Since $kerf_2=M/N,$ then
$(L/N)/(M/N)\cong L/M.$
\item[(2)]Let $M_2\stackrel{f_1}{\rightarrow}
M_1+M_2\stackrel{f_2}{\rightarrow} (M_1+M_2)/M_1$ be defined by
$f_1: x_2\mapsto x_2, f_2: x\mapsto x+M_1.$ Then $kerf_2f_1=M_1\cap
M_2,$ hence $(M_1+M_2)/M_1\cong M_2/(M_1\cap M_2).$
\end{list}
\end{proof}

\newpage

\section{Finitely generated module}

\begin{Def}
An $A$-module is finitely generated if $\exists e_1,\cdots,e_n\in M$
such that $x\in M,$ we have $x=\sum\limits_{i=1}^nx_ie_i(x_i\in A).$
\end{Def}
\begin{prop}
$M$ is finitely generated iff $M$ is isomorphic to a quotient of
$A^n$ for some $n\geqslant0.$
\end{prop}
\begin{proof}
$\Longrightarrow:$ Put
\[ \xymatrix@R=0em{
   f: A^n \ar[r] & M \\
   (x_1,\cdots,x_n) \ar@{|->}[r] & \sum\limits_{i=1}^nx_ie_i }  \]
Then we have $A^n/kerf\cong M.$

$\Longleftarrow:$ Let $A^n$ be generated by $e_1,\cdots,e_n.$ If $N$
is a submodule of $A^n$ such that $M\cong A^n/N,$ then $A^n/N$ is
finitely generated by $\bar{e}_1,\cdots,\bar{e}_n.$
\end{proof}
\begin{prop}
Let $M$ be a finitely generated $A$-module, let $I$ be an ideal of
$A,$ and $\varphi: M\rightarrow M$ be an $A$-module morphism such
that $\varphi(M)\subseteq IM,$ then $\exists a_0,\cdots,a_{n-1}\in
I,$ such that $\varphi^n+\sum\limits_{j=0}^{n-1}a_j\varphi^j=0.$
\end{prop}
\begin{proof}
Let $e_1,\cdots,e_n$ be a set of generators of $M,$ then $\forall
i(1\leqslant i\leqslant n),$ since $\varphi(e_i)\in IM,$ we can
write $\varphi(e_i)=\sum\limits_{j=1}^n a_{ij}e_j$ with $a_{ij}\in
I(1\leqslant j\leqslant n).$ So
$\sum\limits_{j=1}^n(\delta_{ij}\varphi-a_{ij})(e_j)=0.$ Set
$D=(a_{ij})_{1\leqslant i,j\leqslant n},$ let $E_n$ be the unit
matrix of size $n,$ then we can write $(\varphi
E_n-D)(e_1,\cdots,e_n)^T=0.$ Then we have
$$(\varphi E_n-D)^{\ast}(\varphi E_n-D)(e_1,\cdots,e_n)^T=0,$$
i.e. $\det(\varphi E_n-D)(e_1,\cdots,e_n)^T=0.$ So
$$(\varphi^n+\sum\limits_{j=0}^{n-1}a_j\varphi^j)(e_1,\cdots,e_n)=0,$$
i.e. $\forall 1\leqslant i\leqslant n,$ we have
$$\varphi^n(e_i)+\sum\limits_{j=0}^{n-1}a_j\varphi^j(e_i)=0.$$
Since $e_1,\cdots,e_n$ generate $M,$ then $\forall x\in M,$ we have
$$\varphi^n(x)+\sum\limits_{j=0}^{n-1}a_j\varphi^j(x)=0.$$ i.e.
$\varphi^n+\sum\limits_{j=0}^{n-1}a_j\varphi^j=0.$
\end{proof}
\begin{cor}
Let $M$ be finitely generated, let $I$ be an ideal of $A,$ such that
$IM=M,$ then $\exists x\in A$ such that $x\equiv 1(mod\,I)$ and
$xM=0.$
\end{cor}
\begin{proof}
Take $\varphi=id_M,$ then $\varphi(M)=M=IM,$ thus $\exists a_0,
\cdots, a_{n-1}\in I$ such that
$\varphi^n+\sum\limits_{j=0}^{n-1}a_j\varphi^j=0.$ Set $x =
1+\sum\limits_{j=0}^{n-1}a_j\equiv 1(mod\,I),$ and $\forall m\in M,$
we have
\begin{eqnarray*}
0 & = & (\varphi^n+\sum\limits_{j=0}^{n-1}a_j\varphi^j)(m) =
\varphi^n(m)+\sum\limits_{j=0}^{n-1}a_j\varphi^j(m) =
m+\sum\limits_{j=0}^{n-1}a_jm \\
& = & (1+\sum\limits_{j=0}^{n-1}a_j)(m) = xm.
\end{eqnarray*}
So $xM=0.$
\end{proof}
\begin{cor}[Nakayama's lemma]
Let $M$ be finitely generated, $I\subseteq \mathfrak{R}_A$ be an
ideal of $A.$ If $IM=M,$ then $M=0.$
\end{cor}
\begin{proof}
Since $IM=M,$ we can find $x\in A$ such that $xM=0, x\equiv
1(mod\,I).$ Thus $y=x-1\in I\subseteq \mathfrak{R}_A,$ so $x=1+y\in
A^{\times}.$ Then $M=x^{-1}xM=0.$
\end{proof}
\begin{cor}
Let $M$ be finitely generated, $N\subseteq M$ be a submodule, $I
\subseteq \mathfrak{R}_A$ be an ideal of $A.$ If $M=IM+N,$ then
$M=N.$
\end{cor}
\begin{proof}
Note that $M=N\Longleftrightarrow M/N=0.$ Set $\overline{M}=M/N.$
Since $M$ is finitely generated, then $\overline{M}$ is finitely
generated as well.
$$\overline{M} = M/N = (IM+N)/N = I\cdot M/N = I\cdot \overline{M}.$$
By the previous corollary, we have $\overline{M}=0,$ hence $M=N.$
\end{proof}
\begin{prop}
Let $A$ be a local ring, $\mathfrak{M}$ be its maximal ideal, and
$k=A/\mathfrak{M}$ a field. Let $M$ be a finitely generated
$A$-module, then $M/\mathfrak{M}M$ is a $k$-vector space. If
$e_1,\cdots,e_n\in M$ such that their images in $M/\mathfrak{M}M$
form a $k$-basis, then $e_1,\cdots,e_n$ is a set of generators of
$M.$
\end{prop}
\begin{proof}
$\mathit{1^{\circ}}$ Set $k=A/\mathfrak{M},$ then we have that
$M/\mathfrak{M}M$ is an $A/\mathfrak{M}$-module, i.e.
$M/\mathfrak{M}M$ is a $k$-vector space.

$\mathit{2^{\circ}}$ Set $N = <e_1,\cdots,e_n> = \sum\limits_{i=1}^n
Ae_i\subseteq M.$ Consider
\[ \xymatrix@R=0em{
   f: N \ar[r] & M/\mathfrak{M}M  \\
   x \ar@{|->}[r] & \bar{x}=x+\mathfrak{M}M }  \]
then $imf=f(N)=(N+\mathfrak{M}M)/\mathfrak{M}M.$
$\bar{e}_1,\cdots,\bar{e}_n$ form a $k$-basis for $M/\mathfrak{M}M,$
so $f(N)=M/\mathfrak{M}M,$ i.e. $(N+\mathfrak{M}M)/\mathfrak{M}M =
M/\mathfrak{M}M.$ Thus
$$(N+\mathfrak{M}M)/M\cong
(N+\mathfrak{M}M/\mathfrak{M}M)/(M/\mathfrak{M}M) = 0.$$ Hence we
have $N+\mathfrak{M}M=M,$ then $N=M.$
\end{proof}

\newpage

\section{Tensor product of modules}

Set $I=M\times N,$
\begin{eqnarray*}
C=A^{(I)} & = & \{\sum\limits_{j\in J}a_j\cdot j\mid a_j\in A,
J\subseteq I \text{ a finite set}\}                         \\
& = & \{\sum\limits_{i< \infty}a_ie_{(x_i,y_i)}\mid a_i\in A, x_i\in
M, y_i\in N\}.
\end{eqnarray*}
\begin{eqnarray*}
E & = & \{e_{(x+x^{\prime},y)}-e_{(x,y)}-e_{(x^{\prime},y)},
e_{(x,y+y^{\prime})}-e_{(x,y)}-e_{(x,y^{\prime})},   \\
& & e_{(ax,y)}-ae_{(x,y)}, e_{(x,ay)}-ae_{(x,y)} \mid a\in A,
x,x^{\prime}\in M, y,y^{\prime}\in N\},
\end{eqnarray*}
$D=<E>\subseteq C.$ Define $M\otimes_A N=C/D, \overline{e_{(x,y)}} =
e_{(x,y)}+D,$ we denote $\overline{e_{(x,y)}}$ by $x\otimes y.$
$M\otimes_A N$ is called tensor product, satisfying
\enum
\item[(i)]$(x+x^{\prime})\otimes y = x\otimes y + x^{\prime}\otimes
y,$
\item[(ii)]$x\otimes (y+y^{\prime}) = x\otimes y + x\otimes
y^{\prime},$
\item[(iii)]$(ax)\otimes y = a(x\otimes y),$
\item[(iv)]$x\otimes (ay) = a(x\otimes y).$
\end{list}
And there is a bilinear homomorphism:
\[ \xymatrix@R=0em{
   M\times N \ar[r]^-{j_{M,N}} & M\otimes_A N  \\
   (m,n) \ar@{|->}[r] & m\otimes n }  \]
\begin{thm}
Let $M,N$ be two $A$-modules, then there exists a pair $(M\otimes_A
N, j_{M,N}),$ with $M\otimes_A N$ an $A$-module and $j_{M,N}:
M\times N\rightarrow M\otimes_A N$ bilinear, such that for any
bilinear $A$-module homomorphism $f: M\times N\rightarrow P$ holds
\[ \xymatrix{
   M\times N \ar[r]^-f \ar[d]_{j_{M,N}} & P  \\
   M\otimes_A N \ar@{-->}[ur]_{\exists !\tilde{f}} }  \]
i.e. $f$ factors through $j_{M,N}$ that $f=\tilde{f}\circ j_{M,N}.$

If there exists another pair $(T,j)$ that satisfies the same
property, then there exists a unique isomorphism $\varphi:
M\otimes_A N\rightarrow T$ such that $j=\varphi\circ j_{M,N}.$
\end{thm}
\begin{proof}
$\mathit{1^{\circ}}$

$\mathit{2^{\circ}}$
\end{proof}
\begin{remark}
For $M_1,\cdots,M_r$ which are $A$-modules, we can consider
$r$-linear homomorphism and define $M_1\otimes M_2\otimes \cdots
\otimes M_r$ by $((M_1\otimes M_2)\otimes M_3)\otimes M_4\cdots$
$\otimes$ is associative.
\end{remark}
\begin{prop}
Let $M,N,P$ be three $A$-modules, then there are unique isomorphisms
\enum
\item[(1)]$M\otimes N\rightarrow N\otimes M,$
\item[(2)]$(M\otimes N)\otimes P\rightarrow M\otimes (N\otimes
P)\rightarrow M\otimes N\otimes P,$
\item[(3)]$(M\oplus N)\otimes P\rightarrow (M\otimes P)\oplus
(N\otimes P),$
\item[(4)]$A\otimes M\rightarrow M$
\end{list}
such that respectively
\enum
\item[(1)]$x\otimes y\mapsto y\otimes x,$
\item[(2)]$(x\otimes y)\otimes z\mapsto x\otimes (y\otimes z)\mapsto
x\otimes y\otimes z,$
\item[(3)]$(x,y)\otimes z\mapsto (x\otimes z, y\otimes z),$
\item[(4)]$a\otimes x\mapsto ax.$
\end{list}
\end{prop}
\begin{proof}

\end{proof}
Tensor product of homomorphisms: Let
$$M\stackrel{f}{\rightarrow}
M^{\prime}\stackrel{f^{\prime}}{\rightarrow} M^{\prime\prime},$$
$$N\stackrel{g}{\rightarrow}
N^{\prime}\stackrel{g^{\prime}}{\rightarrow} N^{\prime\prime}$$ be
homomorphisms of $A$-modules, then there exists a unique $A$-module
homomorphism $f\otimes g: M\otimes N\rightarrow M^{\prime}\otimes
N^{\prime}$ and we have
$$(f^{\prime}\circ f)\otimes (g^{\prime}\circ g) = (f^{\prime}\otimes
g^{\prime})\circ (f\otimes g).$$
\begin{Def}
Let $f:A\rightarrow B$ be a homomorphism of rings, let $N$ be a
$B$-module. $\forall a\in A, x\in N,$ define $ax=f(a)x.$ Then $N$
becomes an $A$-module, and we say that $N$ is obtained from $N$ by
restriction of scalars.
\end{Def}
\begin{remark}
By restriction of scalars, $B$ becomes an $A$-module. Then we say
that $B$ is an $A$-algebra. Indeed, $B=A[x_i]_{i\in I}.$ When $I$ is
finite, we say that $B$ is a finitely generated $A$-algebra.
\end{remark}
\begin{prop}
If $N$ is finitely generated as a $B$-module (resp. $B$-algebra),
and $B$ is finitely generated as an $A$-module (resp. $A$-algebra),
then $N$ is finitely generated as an $A$-module (resp. $A$-algebra).
\end{prop}
\begin{proof}

\end{proof}
\begin{Def}[Extension of scalars]
Let $M$ be an $A$-module, put $M_B=B\otimes_A M.$ We want to define
$B$-module structure over $M_B:$ $\forall b\in B, \forall z\in M_B,$
with $z=\sum\limits_{i=1}^l b_i\otimes m_i,$ set $b\cdot z:=
\sum\limits_{i=1}^l (bb_i)\otimes m_i.$
\end{Def}
\begin{prop}
If $M$ is finitely generated as an $A$-module, then $M_B$ is
finitely generated as a $B$-module.
\end{prop}
\begin{proof}
By hypothesis, we can write $M=\sum\limits_{i=1}^n Ae_i,$ then $M_B
= B\otimes_A M = \sum\limits_{i=1}^n B(1\otimes e_i),$ because
$\forall x\in B,$
\begin{eqnarray*}
x & = & \sum b_j\otimes x_j = \sum b_j\otimes(\sum x_{ij}e_i) \\
& = & \sum (b_jx_{ij})\otimes e_i = \sum b_jx_{ij}(1\otimes e_i).
\end{eqnarray*}
\end{proof}
Let $A\stackrel{f}{\rightarrow} B, A\stackrel{g}{\rightarrow} C$ be
homomorphism of rings. Set $D=B\otimes_A C,$ $D$ is an $A$-module.
We shall endow $D$ with an $A$-algebra structure:
\[ \xymatrix@R=0em{
   \mu: D\times D \ar[r] & D  \\
   (x\times y, x^{\prime}\times y^{\prime}) \ar@{|->}[r] &
   (xx^{\prime})\otimes (yy^{\prime}) }  \]
which is defined by
\[ \xymatrix{
   (B\otimes C)\times (B\otimes C) \ar@{-->}[r]^-{\mu} & (B\otimes
   C) \\
   (B\times C)\times (B\times C) \ar[u] \ar[ur] }  \]

\newpage

\section{Modules of factions}

Let $S\subseteq A$ be a multiplicatively closed set, $M$ be an
$A$-algebra. Set $\Omega=M\times S.$
\begin{egs}\
\enum
\item[(1)]
\item[(2)]
\end{list}
\end{egs}
\begin{prop}
Let $N,P\subseteq M$ be $A$-modules, then
\enum
\item[(1)]$S^{-1}(N+P) = S^{-1}N+S^{-1}P,$
\item[(2)]$S^{-1}(N\cap P) = S^{-1}N\cap S^{-1}P,$
\item[(3)]$S^{-1}(M/N) = S^{-1}M/S^{-1}N.$
\end{list}
\end{prop}
\begin{proof}

\end{proof}
\begin{prop}
Let $M$ be an $A$-module, then there is a canonical isomorphism
$$S^{-1}A\otimes_A M\cong S^{-1}M$$ defined by $\frac{a}{s}\otimes
m\mapsto \frac{am}{s}.$
\end{prop}
\begin{proof}

\end{proof}
\begin{prop}
Let $M,N$ be $A$-modules, then there is a canonical isomorphism
$S^{-1}M\otimes_{S^{-1}A} S^{-1}N\cong S^{-1}(M\otimes_A N).$ In
particular, if $\mathfrak{p}$ is a prime ideal of $A,$ then
$M_{\mathfrak{p}}\otimes_{A_{\mathfrak{p}}} N_{\mathfrak{p}}\cong
(M\otimes_A N)_{\mathfrak{p}}.$
\end{prop}
\begin{proof}
We have
\begin{eqnarray*}
& & S^{-1}M\otimes_{S^{-1}A} S^{-1}N    \\
& \cong & (M\otimes_A S^{-1}A) \otimes_{S^{-1}A} (S^{-1}A\otimes_A
N)\\
& \cong & M\otimes_A N\otimes_A S^{-1}A   \\
& \cong & S^{-1}A\otimes_A (M\otimes_A N) \\
& \cong & S^{-1}(M\otimes_A N).
\end{eqnarray*}
\end{proof}
\begin{prop}
Let $M$ be an $A$-module, then the following are equivalent:
\enum
\item[(1)]$M=0;$
\item[(2)]$M_{\mathfrak{p}}=0$ for each prime ideal $\mathfrak{p}$
of $A;$
\item[(3)]$M_{\mathfrak{M}}=0$ for each maximal ideal $\mathfrak{M}$
of $A.$
\end{list}
\end{prop}
\begin{proof}

\end{proof}
\begin{prop}
Let $\varphi: M\rightarrow N$ be an $A$-module morphism, then the
following are equivalent:
\enum
\item[(1)]$\varphi$ is injective;
\item[(2)]$\varphi_{\mathfrak{p}}:M_{\mathfrak{p}}\rightarrow
N_{\mathfrak{p}}$ is injective for each prime ideal $\mathfrak{p}$
of $A;$
\item[(3)]$\varphi_{\mathfrak{M}}:M_{\mathfrak{M}}\rightarrow
N_{\mathfrak{M}}$ is injective for each maximal ideal $\mathfrak{M}$
of $A.$
\end{list}
\end{prop}
\begin{remark}
"injective" can be all changed into "surjective.
\end{remark}
\begin{proof}

\end{proof}

\newpage

\section{Integral dependence}

\begin{Def}
Let $B$ be a ring, $A$ be a subring of $B,$ and $x\in B.$ We say
that $x$ is integral over $A$ if $\exists a_0,a_1,\cdots,a_{n-1}\in
A$ such that $x^n+\sum\limits_{j=0}^{n-1}a_jx^j=0.$
\end{Def}
\begin{prop}
The following are equivalent:
\enum
\item[(1)]$x\in B$ is integral over $A;$
\item[(2)]$A[x]$ is a finitely generated $A$-module;
\item[(3)]$A[x]$ is contained in a subring $C$ of $B$ such that $C$
is a finitely generated $A$-module.
\item[(4)]There exists a faithful $A[x]$-module $M$ which is
finitely generated as an $A$-module.
\end{list}
\end{prop}
\begin{proof}

\end{proof}
\begin{cor}
Let $x_1,\cdots,x_n\in B,$ if each of them is integral over $A,$
then $A[x_1,\cdots,x_n]$ is a finitely generated $A$-module.
\end{cor}
\begin{proof}

\end{proof}
\begin{cor}
Set $\bar{A}^B = \{x\in B\mid x \text{ is integral over }A\}.$ Then
$\bar{A}^B\supseteq A$ is a ring, called the integral closure of $A$
in $B:$
\enum
\item[(1)]if $\bar{A}^B=A,$ we say that $A$ is integrally closed in
$B;$
\item[(2)]if $\bar{A}^B=B,$ we say that $B$ is integral over $A.$
\end{list}
\end{cor}
\begin{proof}

\end{proof}
\begin{cor}[transitivity]
Let $A\subseteq B\subseteq C$ be rings. If $C$ is integral over $B,$
$B$ is integral over $A,$ then $C$ is integral over $A.$
\end{cor}
\begin{proof}
Take $x\in C,$ which is integral over $B,$ i.e.
$x^n+\sum\limits_{j=0}^{n-1}b_jx^j=0,$ with $b_j\in B, n\geqslant
1.$ Set $B_0=B[b_0,\cdots,b_{n-1}],$ then $x$ is integral over
$B_0,$ thus $B_0[x]$ is a finitely generated $B_0$-module. Since $B$
is integral over $A,$ then $b_0,\cdots,b_{n-1}$ are integral over
$A,$ hence $B_0$ is a finitely generated $A$-module. Then $B_0[x]$
is a finitely generated $A$-module. Thus $A[x]$ is a finitely
generated $A$-module. Therefore $x$ is integral over $A.$
\end{proof}
\begin{cor}
Let $A\subseteq B$ be rings and let $C=\bar{A}^B$ be the integral
closure of $A$ in $B,$ then $C$ is integrally closed in $B.$
\end{cor}
\begin{proof}
$\forall x\in B,$ if $x$ is integral over $C,$ then $x$ is integral
over $A,$ for $C$ is integral over $A.$ Thus $x\in C,$ so $C$ is
integrally closed in $B.$
\end{proof}
\begin{prop}
Let $A\subseteq B$ be rings such that $B$ is integral over $A,$
\enum
\item[(1)]If $J$ is an ideal of $B,$ and $I\subseteq J^c = J\cap A,$
then $B/J$ is integral over $A/I.$
\item[(2)]If $S\subseteq A$ is multiplicatively closed, then
$S^{-1}B$ is integral over $S^{-1}A.$
\end{list}
\end{prop}
\begin{proof}\
\enum
\item[(1)]We have a commutative diagram
\[ \xymatrix{
   A \ar@^{(->}[r] \ar[d] & B \ar[r] & B/J \\
   A/I \ar@^{(->}"1,3"_j }  \]
where $j$ is defined by $a+I\mapsto a+J.$ $\forall x\in B,$ set
$\bar{x}=x+J.$ Since $x$ is integral over $A,$ then $\exists
a_0,a_1,\cdots,a_{n-1}\in A$ such that
$x^n+\sum\limits_{j=0}^{n-1}a_jx^j=0.$ Then we have
$\bar{x}^n+\sum\limits_{j=0}^{n-1}\bar{a}_j\bar{x}^j=0,$ so
$\bar{x}$ is integral over $A/I.$ Hence $B/J$ is integral over
$A/I.$
\item[(2)]Take $z\in S^{-1}B,$ write $z=\frac{x}{s}$ with $x\in B,
s\in S.$ $\forall x\in B,$ if $x$ is integral over $A,$ then
$\exists a_0,a_1,\cdots,a_{n-1}\in A$ such that
$x^n+\sum\limits_{j=0}^{n-1}a_jx^j=0.$ Hence $(\frac{x}{s})^n +
\sum\limits_{j=0}^{n-1}(\frac{a_j}{s^{n-j}})(\frac{x}{j})^j=0.$ Then
$\frac{x}{j}$ is integral over $S^{-1}A,$ so $S^{-1}B$ is integral
over $S^{-1}A.$
\end{list}
\end{proof}
\begin{Def}
Let $A$ be an integral domain. If $A$ is integrally closed in its
field of fractions, then we say that $A$ is integrally closed.
\end{Def}
\begin{prop}
Let $A\subseteq B$ be rings, and $S\subseteq A$ be multiplicatively
closed. Then $S^{-1}(\bar{A}^B)=\overline{S^{-1}A}^{S^{-1}B}.$
\end{prop}
\begin{proof}
$\mathit{1^{\circ}}$ $S^{-1}(\bar{A}^B)\subseteq
\overline{S^{-1}A}^{S^{-1}B}:$

Put $C=\bar{A}^B\subseteq B.$ $C$ is integral over $A,$ thus
$S^{-1}C\subseteq S^{-1}B$ is integral over $S^{-1}A,$ so we have
that $S^{-1}(\bar{A}^B)\subseteq \overline{S^{-1}A}^{S^{-1}B}.$

$\mathit{2^{\circ}}$ $S^{-1}(\bar{A}^B)\supseteq
\overline{S^{-1}A}^{S^{-1}B}:$

$\forall z\in \overline{S^{-1}A}^{S^{-1}B},$ we can find
$\frac{a_i}{s_i}\in S^{-1}A(0\leqslant i\leqslant n-1),$ such that
$z^n+\sum\limits_{i=0}^{n-1}(\frac{a_i}{s_i})z^i=0.$ Write
$z=\frac{b}{s}$ with $b\in B, s\in S.$ Then $(\frac{b}{s})^n +
\sum\limits_{i=0}^{n-1}(\frac{a_i}{s_i})(\frac{b}{s})^i=0.$ Set
$t=\prod\limits_{i=0}{n-1}s_i,$ then
$$0 = (ts)^n[(\frac{b}{s})^n +
\sum\limits_{i=0}^{n-1}(\frac{a_i}{s_i})(\frac{b}{s})^i],$$ i.e.
$$\frac{(tb)^n}{1} +
\sum\limits_{i=0}^{n-1}(\frac{t^n}{s_i})a_ib^is^{n-i}=0.$$ Denote
$\frac{t^n}{s_i} = \frac{s_0^ns_1^n\cdots s_i^{n-1}\cdots
s_{n-1}^n}{1}$ by $\frac{c_i}{1}.$ Then $\exists u\in S$ such that
$u[(tb)^n + \sum\limits_{i=0}^{n-1}c_ia_ib^is^{n-i}]=0.$ Thus
$$(utb)^n + \sum\limits_{i=0}^{n-1}u^{n-i}t^{n-i-1}a_is_i(utb)^i=0$$
with $u^{n-i}t^{n-i-1}a_is_i\in A.$ Thus $utb$ is integral over $A.$
$z=\frac{b}{s}=\frac{utb}{uts}\in S^{-1}(\bar{A}^B),$ so
$S^{-1}(\bar{A}^B)\supseteq \overline{S^{-1}A}^{S^{-1}B}.$
\end{proof}
\begin{prop}
Let $A$ be an integral domain, then the following are equivalent:
\enum
\item[(1)]$A$ is integrally closed;
\item[(2)]$A_{\mathfrak{p}}$ is integrally closed for each prime
ideal $\mathfrak{p}$ of $A;$
\item[(3)]$A_{\mathfrak{M}}$ is integrally closed for each maximal
ideal $\mathfrak{M}$ of $A.$
\end{list}
\end{prop}
\begin{proof}
$(1)\Longrightarrow(2)$ and $(2)\Longrightarrow(3)$ are trivial.

$(3)\Longrightarrow(1):$ Set $F=(A^{\times})^{-1}A,$ the field of
fractions of $A.$ Put $C=\bar{A}^F.$ For any maximal ideal
$\mathfrak{M}$ of $A,$ set $S=A\setminus \mathfrak{M},$ then
$$C_{\mathfrak{M}} = S^{-1}(\bar{A}^F) = \overline{S^{-1}A}^{S^{-1}F}
= \overline{S^{-1}A}^F = \bar{A}_{\mathfrak{M}}^F =
A_{\mathfrak{M}}.$$ Therefore $0 =C_{\mathfrak{M}}/A_{\mathfrak{M}}
= (C/A)_{\mathfrak{M}},$ then $C/A=0,$ i.e. $A=\bar{A}^F.$
\end{proof}
\begin{prop}
Let $A\subseteq B$ be two integral domain such that $B$ is integral
over $A,$ then $B$ is a field iff $A$ is a field.
\end{prop}
\begin{proof}
$\Longrightarrow:$ Assume that $B$ is a field. $\forall x\in
A^{\times}=A\setminus \{0\},$ we have $x\in B,$ thus $x^{-1}\in B,$
and is integral over $A.$ So we can find $a_0,a_1,\cdots,a_{n-1}\in
A$ such that $(x^{-1})^n+\sum\limits_{j=0}^{n-1}a_j(x^{-1})^j=0.$
Hence $x^{-1}=-\sum\limits_{j=0}^{n-1}a_jx^{n-j-1}\in A.$ Therefore
$A$ is a field.

$\Longleftarrow:$ Suppose that $A$ is a field. $\forall x\in
B^{\times},$ $x$ is integral over $A.$ So we can find a smallest
$n\geqslant 1$ and $a_0,a_1,\cdots,a_{n-1}\in A$ such that
$x^n+\sum\limits_{j=0}^{n-1}a_jx^j=0,$ necessarily $a_0\neq 0.$ Thus
$-(x^{n-1}+\sum\limits_{j=1}^{n-1}a_jx^{j-1})x=a_0,$ then
$$x^{-1}=-a_0^{-1}(x^{n-1}+\sum\limits_{j=1}^{n-1}a_jx^{j-1})\in
B.$$
So $B$ is a field.
\end{proof}
\begin{prop}
Let $A\subseteq B$ be rings such that $B$ is integral over $A.$ Let
$q$ be a prime ideal of $B.$ Then $\mathfrak{p}=q^c=A\cap q$ is
maximal in $A$ iff $q$ is maximal in $B.$
\end{prop}
\begin{proof}
$B/q$ is integral over $A/\mathfrak{p},$ because $B$ is integral
over $A.$ Since $q$ is a prime ideal of $B,$ and $\mathfrak{p}=q^c,$
hence $\mathfrak{p}$ is prime in $A.$ So $B/q$ and $A/\mathfrak{p}$
are integral domains. By the previous corollary, $B/q$ is a field
iff $A/\mathfrak{p}$ is a field. Therefore $\mathfrak{p}=q^c=A\cap
q$ is maximal in $A$ iff $q$ is maximal in $B.$
\end{proof}
\begin{cor}
Let $A\subseteq B$ be rings such that $B$ is integral over $A.$ Let
$q_1,q_2$ be two prime ideals of $B$ such that $q_1\subseteq q_2$
and $q_1^c=q_2^c=\mathfrak{p},$ then $q_1=q_2.$
\end{cor}
\begin{proof}
Set $S=A\setminus \mathfrak{p},$ then $A_{\mathfrak{p}} = S^{-1}A,
B_{\mathfrak{p}} = S^{-1}B,$ and $B_{\mathfrak{p}}$ is integral over
$A_{\mathfrak{p}}.$ Set $\mathfrak{M}=S^{-1}\mathfrak{p},$ then
$\mathfrak{M}$ is the unique maximal ideal of $A_{\mathfrak{p}}.$
Set $N_1=S^{-1}q_1, N_2=S^{-1}q_2.$
$$N_1\cap A_{\mathfrak{p}} = S^{-1}q_1\cap S^{-1}A = S^{-1}(q_1\cap
A) = S^{-1}q_1^c = S^{-1}\mathfrak{p} = \mathfrak{M},$$ we can prove
$N_2\cap A_{\mathfrak{p}} = S^{-1}\mathfrak{p} = \mathfrak{M}$
likewise. Then $N_1$ and $N_2$ are maximal in $B_{\mathfrak{p}}.$
But $N_1\subseteq N_2,$ so $N_1=N_2.$ Since $S\cap q_1\subseteq A,$
so
$$S\cap q_1 = S\cap q_1\cap A = S\cap (q_1\cap A) = S\cap
\mathfrak{p} = \emptyset.$$ We can prove $S\cap q_2=\emptyset$
likewise. Hence $q_1=q_2.$
\end{proof}
\begin{thm}
Let $A\subseteq B$ be rings such that $B$ is integral over $A.$ Let
$\mathfrak{p}$ be a prime ideal of $A.$ Then there exists a prime
ideal $q$ of $B$ such that $q\cap A=\mathfrak{p}.$
\end{thm}
\begin{proof}
Set $S=A\setminus \mathfrak{p},$ then $B_{\mathfrak{p}}$ is integral
over $A_{\mathfrak{p}},$ and
\[ \xymatrix{
   A \ar[d]_{\alpha} \ar@^{(->}[r]^-i & B \ar[d]^{\beta} \\
   S^{-1}A=A_{\mathfrak{p}} \ar[r]^-{S^{-1}i} &
   S^{-1}B=B_{\mathfrak{p}} }  \]
is commutative. $B_{\mathfrak{p}}\neq 0,$ thus we can find a maximal
ideal $\mathfrak{M}$ of $B_{\mathfrak{p}}.$ Then $N =
\mathfrak{M}\cap A_{\mathfrak{p}}$ is maximal in $A_{\mathfrak{p}},$
thus $N=S^{-1}\mathfrak{p}.$ Set $q=\beta^{-1}(\mathfrak{M}),$ then
$q$ is prime in $B,$ and
$$q\cap A = i^{-1}(q) = i^{-1}\circ\beta^{-1}(\mathfrak{M}) =
\alpha^{-1}\circ j^{-1}(\mathfrak{M}) = \alpha^{-1}(N) =
\alpha^{-1}(S^{-1}\mathfrak{p}) = \mathfrak{p}.$$
\end{proof}

\newpage

\section{Noetherian modules and Noetherian rings}

\begin{prop}
Let $(\Sigma, \leqslant)$ be a partially ordered set, then the
following are equivalent:
\enum
\item[(i)]Every increasing sequence $x_1\leqslant x_2\leqslant
\cdots$ in $\Sigma$ is stationary;
\item[(ii)]Every nonempty subset of $\Sigma$ has a maximal element.
\end{list}
\end{prop}
\begin{proof}
$(i)\Longrightarrow(ii):$ An increasing sequence
$\tilde{x}_1\leqslant \tilde{x}_2\leqslant \cdots$ in a nonempty
subset $A$ of $\Sigma$ is an increasing sequence in $\Sigma,$ hence
it is stationary. According to Zorn's lemma, $A$ has a maximal
element.

$(ii)\Longrightarrow(i):$ By contradiction, if $\exists x_1\leqslant
x_2\leqslant \cdots$ in $\Sigma$ is not stationary, then
$\{x_1,\cdots,x_n,\cdots\}$ has no maximal element.
\end{proof}
\begin{Def}
Let $M$ be an $A$-module, set $\Sigma=\{N\mid N \text{ is a
submodule of }M\}.$
\enum
\item[(1)]If $(\Sigma, \subseteq)$ satisfies $(i),$ then we say that
$M$ is Noetherian, and $(i)$ is called a.c.c, $(ii)$ is called
maximal condition.
\item[(2)]If $(\Sigma, \supseteq)$ satisfies $(i),$ then we say that
$M$ is Artinian, and $(i)$ is called d.c.c, $(ii)$ is called minimal
condition.
\end{list}
\end{Def}
\begin{egs}\
\enum
\item[(1)]A finite Abelian group ($\mathbb{Z}$-module) satisfies
a.c.c and d.c.c.
\item[(2)]$\mathbb{Z}$ (as a $\mathbb{Z}$-module) satisfies a.c.c
but not d.c.c.
\item[(3)]$k[X]$ ($k$ is a field) satisfies a.c.c but not d.c.c.
\item[(4)]Set $A=k[X_1,\cdots,X_n]$ with $k$ a field, then $A$
satisfies neither a.c.c nor d.c.c:
$$(X_1)\subset (X_1,X_2)\subset (X_1,X_2,X_3)\subset \cdots$$
$$(X_1)\supset (X_1^2)\supset (X_1^3)\supset \cdots$$
\end{list}
\end{egs}
\begin{prop}
An $A$-module $M$ is Noetherian iff every submodule of $M$ is
finitely generated.
\end{prop}
\begin{proof}
$\Longrightarrow:$ Take $N\subseteq M$ a submodule, put
$$\Sigma=\{L\mid L\subseteq N \text{ is a finitely generated }
A-\text{module}\}.$$ At least $0\in \Sigma,$ so $\Sigma\neq
\emptyset.$ $\Sigma$ contains a maximal element $T,$ then $T=N,$
otherwise, $\exists x\in N\setminus T,$ then $T\subset T+Ax\in
\Sigma.$

$\Longleftarrow:$ Let
$$N_1\subseteq N_2\subseteq N_3\subseteq \cdots\subseteq
N_n\subseteq \cdots$$ be submodules of $M.$ Put
$N=\bigcup\limits_{i=1}^{\infty}N_i,$ then $N$ is an $A$-module, and
a submodule of $M.$ By hypothesis, $\exists x_1,\cdots,x_t\in N$
such that $N=(x_1,\cdots,x_t).$ Therefore $\exists n\geqslant 1,
x_1,\cdots,x_t\in N_n,$ hence $N=N_n.$ So $M$ is Noetherian.
\end{proof}
\begin{prop}
Let $N\subseteq M$ be two modules,
\enum
\item[(1)]$M$ is Noetherian iff $N$ and $M/N$ are Noetherian.
\item[(2)]$M$ is Artinian iff $N$ and $M/N$ are Artinian.
\end{list}
\end{prop}
\begin{proof}\
\enum
\item[(1)]$\Longrightarrow$ is trivial.

$\Longleftarrow:$ Let
$$T_1\subseteq T_2\subseteq T_3\subseteq \cdots\subseteq
T_n\subseteq \cdots$$ be submodules of $M.$ Then $\{T_i\cap
N\}_{i\geqslant 1}$ is stationary, i.e. $\exists n\geqslant 1$ such
that $\forall i\geqslant n, T_{i+1}\cap N=T_i\cap N.$
$\{\overline{T}_i\}_{i\geqslant 1}$ is stationary in $M/N,$ so
$\exists m\geqslant1$ such that $\forall i\geqslant m,
\overline{T}_{i+1} = \overline{T}_i.$ But $\overline{T}_i =
T_i+N/N\cong T_i/T_i\cap N.$ $\forall i\geqslant m+n,$ we have
$\overline{T}_{i+1} = \overline{T}_i$ and $T_{i+1}\cap N = T_i\cap
N,$ so
$$T_i/T_i\cap N = T_i/T_{i+1}\cap N = T_{i+1}/T_{i+1}\cap N,$$
thus $T_i=T_{i+1}.$
\item[(2)]We can prove the statement likewise.
\end{list}
\end{proof}
\begin{prop}
If $M_1,\cdots,M_n$ are Noetherian (resp. Artinian) modules, so is
$\bigoplus\limits_{i=1}^n M_i.$
\end{prop}
\begin{proof}
We prove by induction on $n:$

$n=1$ is trivial.

Suppose the proposition holds for $n-1(n\geqslant 2),$ since
$\bigoplus\limits_{i=2}^n M_i\cong \bigoplus\limits_{i=1}^n
M_i/M_1,$ so $\bigoplus\limits_{i=1}^n M_i$ is Noetherian.
\end{proof}
\begin{Def}
A ring $A$ is Noetherian (resp. Artinian) if $A$ is Noetherian
(resp. Artinian) as an $A$-module.
\end{Def}
\begin{remark}
$A$ is a Noetherian ring $\Longleftrightarrow$ a.c.c holds for
ideals of $A;$\\
$A$ is a Artinian ring $\Longleftrightarrow$ d.c.c holds for ideals
of $A.$
\end{remark}
\begin{prop}
A ring $A$ is Noetherian iff each ideal of $A$ is finitely
generated.
\end{prop}
\begin{proof}
$\Longrightarrow:$ Let $A$ be a Noetherian ring, then $A$ is
Noetherian as an $A$-module by definition. Since each ideal of $A$
is an $A$-module, hence is finitely generated.

$\Longleftarrow:$ Since every submodule $N$ of a ring $A$ as an
$A$-module is an ideal of $A,$ then it is finitely generated by
hypothesis, hence $A$ is a Noetherian ring.
\end{proof}
\begin{egs}\
\enum
\item[(1)]Any field is Artinian and Noetherian. So is
$\mathbb{Z}/n\mathbb{Z}.$
\item[(2)]PIDs are Noetherian.
\item[(3)]$k[X_1,\cdots,X_n]$ is not Noetherian, but it is an
integral domain, thus has a field of fractions which is Noetherian.
\item[(4)]Let $X$ be a compact infinite Hausdorff space. We denote
by $\mathscr{C}(X)$ the ring of continuous real valued functions
defined on $X.$ Take $F_1\supset F_2\supset F_3\supset \cdots$ of
closed sets in $X.$ $\forall n\geqslant 1,$ set
$$I_n = \{f\in\mathscr{C}(X)\mid f(x), \forall x\in F_n\}.$$
Then $I_1\subset I_2\subset I_3\subset \cdots$ are ideals in
$\mathscr{C}(X).$ Hence $\mathscr{C}(X)$ is not Noetherian.
\end{list}
\end{egs}
\begin{prop}
Let $A$ be a Noetherian (resp. Artinian) ring, $M$ be a finitely
generated $A$-module, then $M$ is Noetherian (resp. Artinian).
\end{prop}
\begin{proof}
Suppose $M=(x_1,\cdots,x_n)_A.$ Consider
$A^n=\bigoplus\limits_{i=1}^nA,$ $A^n$ is a Noetherian $A$-module.
Define $f: A^n\rightarrow M$ by $e_i\mapsto x_i,$ then $M\cong
A^n/kerf,$ thus is Noetherian.
\end{proof}
\begin{prop}
Let $A$ be a Noetherian (resp. Artinian) ring, $I$ be an ideal of
$A.$ Then $A/I$ is a Noetherian (resp. Artinian) ring.
\end{prop}
\begin{proof}
There is a one-to-one correspondence between ideals of $A/I$ and
ideals of $A$ which contains $I.$ i.e.
\[ \xymatrix{
   I\subseteq N, N \text{ an ideal of }A \ar@{<->}[r]^-{1-1} &
   J \text{ an ideal of } A/I, J=N/I. }  \]
Since $A$ is Noetherian, then each ideal $N$ of $A$ is finitely
generated, thus $J$ is finitely generated. Hence $A/I$ is a
Noetherian ring.
\end{proof}
\begin{prop}
Let  $A$ be a Noetherian ring and $S\subseteq A$ a multiplicatively
closed subset. Then $S^{-1}A$ is a Noetherian ring.
\end{prop}
\begin{proof}
There is a one-to-one correspondence:
\[ \xymatrix{
   I \text{ an ideal of }A, I\cap S=\emptyset \ar@{<->}[r]^-{1-1} &
   J \text{ an ideal of } S^{-1}A, J=S^{-1}I. }  \]
$A$ is a Noetherian ring, then each ideal $I$ of $A$ is finitely
generated, thus $J$ is finitely generated. Therefore $S^{-1}A$ is a
Noetherian ring.
\end{proof}
\begin{thm}[Hilbert's Basis Theorem]
If $A$ is a Noetherian ring, then $A[X]$ is also Noetherian.
\end{thm}
\begin{proof}
Let $J$ be an ideal of $A[X],$ $I$ be the set of the leading
coefficients of $f\in J.$ Then $I$ is an ideal of $A.$ Thus we can
find $a_1,\cdots,a_n\in A$ such that $I=(a_1,\cdots,a_n)_A.$ For
each $i(1\leqslant i\leqslant n),$ there exists $f_i\in J$ such that
$$f_i=a_iX^{r_i}+\cdots,$$
where $r_i:=\deg f_i.$ Set $r=\max\limits_{1\leqslant i\leqslant n}
r_i,$ $J^{\prime}=(f_1,\cdots,f_n)_{A[X]}\subseteq J.$ $\forall f\in
J,$ we can write $f=ax^m+\sum\limits_{j=0}^m b_jx^j$ with $a, b_j\in
A.$ Then $a\in I.$ If $m=\deg f\geqslant r$ with
$a=\sum\limits_{i=1}^n u_ia_i(u_i\in A),$ then
$f-\sum\limits_{i=1}^n u_if_ix^{m-r_i}\in J.$ In this way we can
write $f=g+h$ with $h\in J^{\prime}$ and $\deg g<r.$

Set $M=\sum\limits_{i=0}^{n-1} Ax^i,$ then $J=(J\cap M)+J^{\prime}.$
$M$ is a finitely generated $A$-module, then $M$ is a Noetherian
$A$-module. So $J\cap M$ is also a Noetherian $A$-module, thus is
finitely generated. Write $J\cap M=(g_1,\cdots,g_t)_A$ with
$g_1,\cdots,g_t\in J\cap M\subseteq A[X].$ Then
$J=(f_1,\cdots,f_n,g_1,\cdots,g_t)_{A[X]}.$ So $A[X]$ is a
Noetherian ring.
\end{proof}
\begin{cor}
If $A$ is a Noetherian ring, so is $A[X_1,\cdots,X_n].$
\end{cor}
\begin{proof}
$$A[X_1,\cdots,X_n]=A[X_1,\cdots,X_{n-1}][X_n],$$
so $A[X_1,\cdots,X_n]$ is Noetherian if $A$ is a Noetherian ring
according to Hilbert's Basis Theorem.
\end{proof}
\begin{cor}
Let $B$ be a finitely generated $A$-algebra. If $A$ is Noetherian,
so is $B.$ In particular, every finitely generated ring, and every
finitely generated algebra over a field is Noetherian.
\end{cor}
\begin{proof}
$B$ can be written $A[X_1,\cdots,X_n]/I$ with $I$ an ideal of
$A[X_1,\cdots,X_n].$ So, if $A$ is Noetherian, then
$B=A[X_1,\cdots,X_n]/I$ is Noetherian as well.

Finitely generated ring $=$ $\mathbb{Z}[x_1,\cdots,x_n].$ A field is
Noetherian, so is $\mathbb{Z}.$ Thus every finitely generated ring,
and every finitely generated algebra over a field is Noetherian.
\end{proof}
\begin{prop}
Let $A\subseteq B\subseteq C$ be rings. If $A$ is Noetherian, $C$ is
a finitely generated $A$-algebra such that $C$ is either
\enum
\item[(1)]finitely generated as a $B$-module,
\item[(2)]integral over $B,$
\end{list}
then $B$ is a finitely generated $A$-algebra.
\end{prop}
\begin{proof}
Since $(1)$ and $(2)$ are equivalent, so we only consider $(1).$ $C$
can be written $C=A[x_1,\cdots,x_n]=B[x_1,\cdots,x_n],$ for $C$ is a
finitely generated $A$-algebra. From $(1)$ we know that
$C=\sum\limits_{i=1}^n By_i$ with $y_i\in C(1\leqslant i\leqslant
n).$ Then $x_l=\sum\limits_{j=1}^n b_{lj}y_j$ with $1\leqslant
l\leqslant n, b_{lj}\in B.$ $y_iy_j=\sum\limits_{h=1}^n b_{ijh}y_h,
\forall 1\leqslant i,j\leqslant n.$

Set $B_0=A[b_{lj},b_{ijh}]_{1\leqslant l,i,j,h\leqslant n},$ then
$B_0$ is Noetherian and $A\subseteq B_0\subseteq B.$ $\forall f\in
C, f=\sum a_{i_1\cdots i_n}x_1^{i_1}\cdots x_n^{i_n}$ with
$a_{i_1\cdots i_n}\in A.$ Since $x_l=\sum\limits_{j=1}^n b_{lj}y_j,
1\leqslant l\leqslant n,$ then $f\in \sum\limits_{j=1}^nB_0y_j,$
which is a finitely generated $B_0$-module. Then $C\subseteq
\sum\limits_{j=1}^n B_0y_j$ is a Noetherian $B_0$-module.
$B_0\subseteq B\subseteq C,$ thus $B$ is a $B_0$-module and a
submodule of $C,$ hence $B$ is a finitely generated $B_0$-module. So
$B=\sum\limits_{j=1}^m B_0u_i$ with $m\geqslant 1, u_i\in B.$ Since
$B_0=A[b_{lj},b_{ijh}]_{1\leqslant l,i,j,h\leqslant n},$ then
$B=A[b_{lj},b_{ijh},u_s]_{1\leqslant s\leqslant m \atop 1\leqslant
l,i,j,h\leqslant n },$ hence $B$ is a finitely generated
$A$-algebra.
\end{proof}
\begin{prop}
Let $k$ be a field, $E$ be a finitely generated $k$-algebra. If $E$
is a field, then $E$ is a finite algebraic extension of $k.$
\end{prop}
\begin{proof}
If $E$ is a field, write $E=k[x_1,\cdots,x_n]=k(x_1,\cdots,x_n).$ By
contradiction, we suppose that $E$ is not an algebraic extension of
$k,$ we may assume $x_1,\cdots,x_r$ are algebraically independent
over $k.$ Set $F=k(x_1,\cdots,x_r),
E=k[x_1,\cdots,x_r,x_{r+1},\cdots,x_n],$ and $x_{r+1},\cdots,x_n$
are algebraic over $F.$

Then $k\subseteq F\subseteq E$ are rings such that $k$ is Noetherian
(for $k$ is a field), $E$ is a finitely generated $k$-algebra and
integral over $F.$ According to the previous proposition, $F$ is a
finitely generated $k$-algebra, thus can be written
$F=k[y_1,\cdots,y_s]$ with $s\geqslant 1, y_1,\cdots,y_s\in F.$
\[ \xymatrix{
   E \ar@{-}[dd] & \text{a finitely generated $k$ algebra} \\
   & \text{and integral over $F$ }      \\
   F \ar[r]^-{\text{thus}} \ar@{-}[d] & \text{a finitely generated
   $k$-algebra}             \\
   k & \text{a field, thus is Noetherian} }  \]
$\forall 1\leqslant i\leqslant s,$ write $y_i=\frac{f_i}{g_i}$ with
$f_i,g_i\in k[x_1,\cdots,x_r].$ There exists an $h\in
k[x_1,\cdots,x_r]\subseteq F$ such that $h$ is irreducible and
$h\nmid g_i(\forall 1\leqslant i\leqslant s).$ Since $\frac{1}{h}\in
F,$ then $\frac{1}{h}=\frac{f}{g_1^{k_1}\cdots g_s^{k_s}}$ with
$f\in k[x_1,\cdots,x_r],$ i.e. $hf=g_1^{k_1}\cdots g_s^{k_s},$
absurd.
\end{proof}
\begin{cor}
Let $k$ be a field and $A$ be a finitely generated $k$-algebra. Let
$\mathfrak{M}$ be a maximal ideal of $A,$ then $A/\mathfrak{M}$ is a
finite algebraic extension of $k.$ In particular, if $k$ is
algebraically closed, then $A/\mathfrak{M}\cong k.$
\end{cor}
\begin{proof}
$A=k[x_1,\cdots,x_n]$ is a finitely generated $k$-algebra, thus
$A/\mathfrak{M}$ is a finitely generated $k$-algebra. Consider
$$k\stackrel{f}{\hookrightarrow}
A=k[x_1,\cdots,x_n]\stackrel{g}{\rightarrow}
A/\mathfrak{M}=k[\bar{x}_1,\cdots,\bar{x}_n],$$ $g\circ f$ is
injective, so $A/\mathfrak{M}$ is a field and a finitely generated
$k$-algebra, thus is a finite extension of $k.$
\end{proof}
\begin{cor}
Let $k$ be an algebraically closed field. Set $A=k[X_1,\cdots,X_n].$
An ideal $\mathfrak{M}$ is maximal iff $\exists x_1,\cdots,x_n\in k$
such that $\mathfrak{M}=(X_1-x_1,\cdots,X_n-x_n).$
\end{cor}
\begin{proof}
$\Longleftarrow:$ $A/\mathfrak{M}=k,$ so $\mathfrak{M}$ is maximal.

$\Longrightarrow:$ Set $x_i=X_i+\mathfrak{M}\in A/\mathfrak{M}=k,$
$x_i\equiv X_i(mod\,\mathfrak{M}),$ so $X_i-x_i\in \mathfrak{M}.$
Then $(X_1-x_1,\cdots,X_n-x_n)\subseteq \mathfrak{M}.$ However,
$(X_1-x_1,\cdots,X_n-x_n)$ is maximal, hence
$M=(X_1-x_1,\cdots,X_n-x_n).$
\end{proof}
\begin{cor}
Let $k$ be an algebraically closed field and $A$ be finitely
generated $k$-algebra. If $\mathfrak{p}$ is a prime ideal of $A$
which is not maximal, then
$$\mathfrak{p}=\bigcap\limits_{\mathfrak{p}\subseteq q, q\text{
prime}} q.$$
\end{cor}
\begin{proof}
$\mathit{1^{\circ}}$ $\mathfrak{p}\subseteq
\bigcap\limits_{\mathfrak{p}\subseteq q, q\text{ prime}} q$ is
obvious.

$\mathit{2^{\circ}}$ We show that $\forall f\in A\setminus
\mathfrak{p},$ there exists a prime ideal $q$ of $A$ such that
$f\not\in q$ and $\mathfrak{p}\subset q.$

Put $B=A/\mathfrak{p}$ an integral domain but not a field. Set
$g=f+\mathfrak{p},$ then $g\in B, g\neq 0.$ Put $S=\{g^n\mid
n\geqslant 0\}, B_g=S^{-1}B=B[\frac{1}{g}]$ a finitely generated
$B$-algebra. $A$ is a finitely generated $k$-algebra, so
$B=A/\mathfrak{p},$ and then $B_g,$ are finitely generated
$k$-algebras. $B_g$ is not a field, otherwise $B_g$ is a finite
algebraic extension over $k,$ then $B$ is a field, absurd.

Take $\mathfrak{M}$ a maximal ideal in $B_g,$ $g\not\in
\mathfrak{M}.$ Then $N=\mathfrak{M}\cap B$ is a prime ideal of $B.$
Indeed, $\mathfrak{M}= S^{-1}N.$ Thus there exists a prime ideal $q$
of $A$ such that $N=q/\mathfrak{p}.$ $\mathfrak{M}\neq 0$ for $B_g$
is not a field, so $N\neq 0,$ and therefore $\mathfrak{p}\subset q.$
Since $g\not\in \mathfrak{M},$ then $g\not\in N,$ hence $f\not\in
q.$ So $\mathfrak{p}\supseteq \bigcap\limits_{\mathfrak{p}\subseteq
q, q\text{ prime}} q.$
\end{proof}
\begin{cor}
Let $k$ be an algebraically closed field and $A$ be finitely
generated $k$-algebra. Then $A$ is a Jacobson ring, i.e. every prime
ideal $\mathfrak{p}$ of $A$ holds
$$\mathfrak{p}=\bigcap\limits_{\mathfrak{p}\subseteq \mathfrak{M},
\mathfrak{M}\text{ maximal}} \mathfrak{M}.$$
\end{cor}
\begin{proof}
Fix $\mathfrak{p}$ a prime ideal of $A.$ Note that
$$\mathfrak{p}=\bigcap\limits_{\mathfrak{p}\subseteq \mathfrak{M},
\mathfrak{M}\text{ maximal}} \mathfrak{M} \Longleftrightarrow
B=A/\mathfrak{p}, \mathfrak{R}_B=0.$$ $B$ is a finitely generated
$k$-algebra and an integral domain. We prove that
$\mathfrak{R}_B=0.$

By contradiction, suppose that $\mathfrak{R}_B\neq 0.$ Then $0$ is
not a maximal ideal in $B,$ thus $B$ is not a field. Take $f\in
\mathfrak{R}_B\setminus \{0\},$ put $S=\{f^n\mid n\geqslant 0\},
B_f=S^{-1}B=B[\frac{1}{f}].$ $B_f$ is not a field. Take
$\mathfrak{M}\neq 0,$ a maximal ideal of $B_f.$ Set
$N=\mathfrak{M}\cap B$ a prime ideal of $B,$ then
$\mathfrak{M}=S^{-1}N, f\not\in N.$ There is a one-to-one
correspondence of prime ideals:
\[ \xymatrix{
   Q, \text{ prime in } B_f \ar[r]^-{1-1} & q, \text{ prime in }B,
   \text{ and } q\cap S=\emptyset }  \]

Let $q$ be a prime ideal of $B$ such that $q\cap S=\emptyset,$ if
$q\supset N,$ then $f\in q,$ otherwise $\mathfrak{M}\subset
\bar{q}.$ $N$ is not maximal, because $f\not\in N$ and $f\in
\mathfrak{R}_B.$ Therefore $N\neq \bigcap\limits_{N\subset q,
q\text{ prime}} q,$ absurd.
\end{proof}

\newpage

\section{Affine variety}

\begin{Def}
Fix $k$ an algebraically closed field, $A=k[X_1,\cdots,X_n].$
$\forall E\subseteq A,$ define
$$V(E)=\{x\in k^n\mid f(x)=0, \forall f\in E\},$$ called affine
algebraic set.
\end{Def}

Basic properties:
\enum
\item[(1)]$V(E)=V(<E>).$
\item[(2)]$V(0)=k^n, V((1))=\emptyset.$
\item[(3)]Let $I,J$ be two ideals of $A,$ then $V(I)\cap
V(J)=V(I+J), V(I)\cup V(J)=V(I\cap J), \bigcap\limits_i
V(I_i)=V(\sum\limits_i I_i).$
\item[(4)]
\item[(5)]
\item[(6)]
\item[(7)]
\end{list}
